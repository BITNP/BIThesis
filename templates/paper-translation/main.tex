%%
% BIThesis 本科毕业设计论文翻译模板 —— 使用 XeLaTeX 编译 The BIThesis Template for Bachelor Paper Translation
% This file has no copyright assigned and is placed in the Public Domain.
% Compile with: xelatex -> biber -> xelatex -> xelatex
%%

% 请勿删除下面两行注释,以免影响编译。
% !TeX program = xelatex
% !BIB program = biber


\documentclass[type=bachelor_translation]{bithesis}

% 此处仅列出常用的配置。全部配置用法请见「bithesis.pdf」手册。
\BITSetup{
  cover = {
    % 封面需要「北京理工大学」字样图片,如无必要请勿修改该项。
    headerImage = images/header.png,
    % 封面标题需要“华文细黑”,如无必要请勿修改该项。
    xiheiFont = STXIHEI.TTF,
    % 官方模板采用了固定的下划线宽度。我们采用以下两个选项来达成这个效果。
    % 如果你想要使用自动计算的下划线宽度,也可以删去以下两个选项。
    autoWidth = false,
    valueMaxWidth = 20em,
  },
  info = {
    title = 北京理工大学本科生毕业设计(论文)题目,
    titleEn = {The Subject of Undergraduate Graduation Project (Thesis) of Beijing Institute of Technology},
    % 注意,这里要写的是毕设的大标题,不是你要翻译的文献的标题。
    %
    % 想要删除某项封面信息,直接删除该项即可。
    % 想要让某项封面信息留空(但是保留下划线),请传入空白符组成的字符串,如"{~}"。
    % 如需要换行,则用 “\\” 符号分割。
    school = 计算机学院,
    major = 计算机科学与技术,
    author = 惠计算,
    class = 0596xxxx,
    studentId = 11xxxxxxxx,
    supervisor = 张哈希,
    translationTitle = 填写你的外文翻译中文题目,
    translationOriginTitle = 填写你的外文翻译原英文题目,
    keywords = {北京理工大学;本科生;毕业设计(外文翻译)——请在“main.tex”开头设置},
    % 如果你的毕设为校外毕设,请将下面这一行语句解除注释(删除第一个百分号字符)并填写你的校外毕设导师名字
    % externalSupervisor = 左偏树,
  },
  style = {
    % 页眉若要按信息与电子学院的要求,删掉“外文翻译”几字,请解除下一行的注释。
    % head = {北京理工大学本科生毕业设计(论文)},
    %
    % 开启该选项后,将用 Times New Roman 的开源字体 TeX Gyre Termes 作为正文字体。
    % 这个选项适用于以下情况:
    % 1. 不想在系统中安装 Times New Roman。
    % 2. 在 Linux/macOS 下遇到 `\textsc` 无法正常显示的问题。
    % betterTimesNewRoman = true,
  },
  misc = {
    % 微调表格行间距
    tabularRowSeparation = 1.25,
  },
}

% 大部分关于参考文献样式的修改,都可以通过此处的选项进行配置。
% 详情请搜索「biblatex-gb7714-2015 文档」进行阅读。
\usepackage[
  backend=biber,
  style=gb7714-2015,
  gbalign=gb7714-2015,
  gbnamefmt=lowercase,
  gbpub=false,
  doi=false,
  url=false,
  eprint=false,
  isbn=false,
]{biblatex}

% 参考文献引用文件位于 misc/ref.bib
\addbibresource{misc/ref.bib}

% 若要按计算机学院要求,添加“北京理工大学”水印,请参考
% https://bithesis.bitnp.net/faq/watermark.html

% 文档开始
\begin{document}

% 标题页面:如无特殊需要,本部分无需改动
\MakeCover

% 前置内容定义
\frontmatter
% 摘要:在相应 TeX 文件撰写
%%
% BIThesis 本科毕业设计论文模板(全英文) —— 使用 XeLaTeX 编译 The BIThesis Template for Undergraduate Thesis
% This file has no copyright assigned and is placed in the Public Domain.
%%

% 摘要若要按经管学院的要求,先英文再中文,请调换以下 abstract、abstractEn 的顺序。

\begin{abstract}
  Conventional  product  development  employs  a  design-build-test  philosophy.
  The sequentially  executed  development  process  often  results  in  prolonged
  lead  times  and elevated product costs. The proposed e-Design paradigm employs
  IT-enabled technology for product design, including virtual prototyping (VP) to
  support a cross-functional team in analyzing  product  performance,  reliability,
  and  manufacturing costs  early  in  product development, and in making quantitative
  trade-offs for design decision making. Physical prototypes  of  the  product  design
  are  then  produced  using  the  rapid  prototyping  (RP) technique  and  computer
  numerical  control  (CNC)  to  support  design  verification  and functional prototyping, respectively.
\end{abstract}

\begin{abstractEn}
  Conventional  product  development  employs  a  design-build-test  philosophy.
  The sequentially  executed  development  process  often  results  in  prolonged
  lead  times  and elevated product costs. The proposed e-Design paradigm employs
  IT-enabled technology for product design, including virtual prototyping (VP) to
  support a cross-functional team in analyzing  product  performance,  reliability,
  and  manufacturing costs  early  in  product development, and in making quantitative
  trade-offs for design decision making. Physical prototypes  of  the  product  design
  are  then  produced  using  the  rapid  prototyping  (RP) technique  and  computer
  numerical  control  (CNC)  to  support  design  verification  and functional prototyping, respectively.
\end{abstractEn}


\MakeTOC

% 正文开始
\mainmatter

% 在这里引用各章 TeX 文件,按需添加
%%
% The BIThesis Template for Bachelor Graduation Thesis
%
% 北京理工大学毕业设计(论文)第一章节 —— 使用 XeLaTeX 编译
%
% Copyright 2020-2021 BITNP
%
% This work may be distributed and/or modified under the
% conditions of the LaTeX Project Public License, either version 1.3
% of this license or (at your option) any later version.
% The latest version of this license is in
%   http://www.latex-project.org/lppl.txt
% and version 1.3 or later is part of all distributions of LaTeX
% version 2005/12/01 or later.
%
% This work has the LPPL maintenance status `maintained'.
%
% The Current Maintainer of this work is Feng Kaiyu.
%
% 第一章节

\chapter{一级题目}

\section{二级题目}
% 这里插入一个参考文献,仅作参考
正文……\cite{yuFeiJiZongTiDuoXueKeSheJiYouHuaDeXianZhuangYuFaZhanFangXiang2008}

\subsection{三级题目}

正文……\cite{Hajela2012Application}

\textcolor{blue}{正文部分:宋体、小四;正文行距:22磅;间距段前段后均为0行。阅后删除此段。}

\textcolor{blue}{图、表居中,图注标在图下方,表头标在表上方,宋体、五号、居中,1.25倍行距,间距段前段后均为0行,图表与上下文之间各空一行。阅后删除此段。}

\textcolor{blue}{\underline{\underline{图-示例:(阅后删除此段)}}}

\begin{figure}[htbp]
  \vspace{13pt} % 调整图片与上文的垂直距离
  \centering
  \includegraphics[]{images/bit_logo.png}
  \caption{标题序号}\label{标题序号} % label 用来在文中索引
\end{figure}

\textcolor{blue}{\underline{\underline{表-示例:(阅后删除此段)}}}

\begin{table}[htbp]
  \linespread{1.5}
  \zihao{5}
  \centering
  \caption{统计表}\label{统计表}
  \begin{tabular}{*{5}{>{\centering\arraybackslash}p{2cm}}}
    \hline
    项目    & 产量    & 销量    & 产值   & 比重    \\ \hline
    手机    & 1000  & 10000 & 500  & 50\%  \\
    计算机   & 5500  & 5000  & 220  & 22\%  \\
    笔记本电脑 & 1100  & 1000  & 280  & 28\%  \\ \hline
    合计    & 17600 & 16000 & 1000 & 100\% \\ \hline
    \end{tabular}
\end{table}

\textcolor{blue}{公式标注应于该公式所在行的最右侧。对于较长的公式只可在符号处(+、-、*、/、$\leqslant$ $\geqslant$ 等)转行。在文中引用公式时,在标号前加“式”,如式(1-2)。阅后删除此
段。}

\textcolor{blue}{公式-示例:(阅后删除此段)}
% 公式上下不要空行,置于同一个段落下即可,否则上下距离会出现高度不一致的问题
\begin{equation}
    LRI=1\ ∕\ \sqrt{1+{\left(\frac{{\mu }_{R}}{{\mu }_{s}}\right)}^{2}{\left(\frac{{\delta }_{R}}{{\delta }_{s}}\right)}^{2}}
\end{equation}

\subsubsection{生僻字}

% 一个可能无法正常显示的生僻字
一个可能无法正常显示的生僻字: 彧。下文注释中,介绍了如何通过自定义字体来显示生僻字。

% 定义一个提供了生僻字的字体,注意要确保你的系统存在该字体
% \setCJKfamilyfont{custom-font}{Noto Serif CJK SC}

% 使用自己定义的字体
% 使用提供了相应字型的字体:\CJKfamily{custom-font}{彧}。



% 后置内容
\backmatter

% 结论:在相应 TeX 文件撰写
%%
% The BIThesis Template for Reading Report
%
% 北京理工大学读书报告结论 —— 使用 XeLaTeX 编译
%
% Copyright 2020-2023 BITNP
%
% This work may be distributed and/or modified under the
% conditions of the LaTeX Project Public License, either version 1.3
% of this license or (at your option) any later version.
% The latest version of this license is in
%   https://www.latex-project.org/lppl.txt
% and version 1.3 or later is part of all distributions of LaTeX
% version 2005/12/01 or later.
%
% This work has the LPPL maintenance status `maintained'.
%
% The Current Maintainer of this work is Feng Kaiyu.
%
% Compile with: xelatex -> biber -> xelatex -> xelatex

\begin{conclusion}
  % 结论部分尽量不使用 \subsection 二级标题,只使用 \section 一级标题

  % 这里插入一个参考文献,仅作参考
  本文结论……\cite{张伯伟2002全唐五代诗格会考}。

  \textcolor{blue}{结论作为正文的最后部分单独排写,但不加章号。阅后删除此段。}

  \textcolor{blue}{结论正文样式与文章正文相同:宋体、小四;行距:22磅;间距段前段后均为0行。阅后删除此段。}
\end{conclusion}

% 参考文献:
% 添加文献时,请按 BibTeX 格式添加至 misc/ref.bib,并在正文所需位置使用 \cite{…} 引用。
% 如无特殊需要,无需改动相应 TeX 文件。
%%
% The BIThesis Template for Graduate Thesis
%
% Copyright 2020-2023 Yang Yating, BITNP
%
% This work may be distributed and/or modified under the
% conditions of the LaTeX Project Public License, either version 1.3
% of this license or (at your option) any later version.
% The latest version of this license is in
%   https://www.latex-project.org/lppl.txt
% and version 1.3 or later is part of all distributions of LaTeX
% version 2005/12/01 or later.
%
% This work has the LPPL maintenance status `maintained'.
%
% The Current Maintainer of this work is Feng Kaiyu.

%
% 如无特殊需要,本页面无需更改。
%
% **注意:如果发现渲染出来的文献编号不正确,请同时使用以下两个方式解决:**
% 1. 清除缓存后重新编译(比如使用 `latexmk -c`)。
% 2. 请确保无编译错误。


\begin{bibprint}
  \printbibliography[heading=none,notcategory=mypub,resetnumbers=true]
\end{bibprint}

% 附录:在相应 TeX 文件撰写;不需要时可删除
\input{misc/3_appendix.tex}

\end{document}
