%%
% The BIThesis Template for Bachelor Paper Translation
%
% 北京理工大学毕业设计(论文) —— 使用 XeLaTeX 编译
%
% Copyright 2020-2023 BITNP
%
% This work may be distributed and/or modified under the
% conditions of the LaTeX Project Public License, either version 1.3
% of this license or (at your option) any later version.
% The latest version of this license is in
%   https://www.latex-project.org/lppl.txt
% and version 1.3 or later is part of all distributions of LaTeX
% version 2005/12/01 or later.
%
% This work has the LPPL maintenance status `maintained'.
%
% The Current Maintainer of this work is Feng Kaiyu.
%
% Compile with: xelatex -> biber -> xelatex -> xelatex

% !TeX program = xelatex
% !BIB program = biber


\documentclass[type=bachelor_translation]{bithesis}

% 此处仅列出常用的配置。全部配置用法请见「bithesis.pdf」手册。
\BITSetup{
  cover = {
    % 封面需要「北京理工大学」字样图片,如无必要请勿修改该项。
    headerImage = images/header.png,
    % 封面标题需要“华文细黑”,如无必要请勿修改该项。
    xiheiFont = STXIHEI.TTF,
    % 官方模板采用了固定的下划线宽度。我们采用以下两个选项来达成这个效果。
    % 如果你想要使用自动计算的下划线宽度,也可以删去以下两个选项。
    autoWidth = false,
    valueMaxWidth = 20em,
  },
  info = {
    title = 北京理工大学本科生毕业设计(论文)题目,
    titleEn = {The Subject of Undergraduate Graduation Project (Thesis) of Beijing Institute of Technology},
    % 注意,这里要写的是毕设的大标题,不是你要翻译的文献的标题。
    %
    % 想要删除某项封面信息,直接删除该项即可。
    % 想要让某项封面信息留空(但是保留下划线),请传入空白符组成的字符串,如"{~}"。
    % 如需要换行,则用 “\\” 符号分割。
    school = 计算机学院,
    major = 计算机科学与技术,
    author = 惠计算,
    class = 0596xxxx,
    studentId = 11xxxxxxxx,
    supervisor = 张哈希,
    translationTitle = 填写你的外文翻译中文题目,
    translationOriginTitle = 填写你的外文翻译原英文题目,
    keywords = {北京理工大学;本科生;毕业设计(外文翻译)——请在“main.tex”开头设置},
    % 如果你的毕设为校外毕设,请将下面这一行语句解除注释(删除第一个百分号字符)并填写你的校外毕设导师名字
    % externalSupervisor = 左偏树,
  },
  style = {
    % 页眉若要按信息与电子学院的要求,删掉“外文翻译”几字,请解除下一行的注释。
    % head = {北京理工大学本科生毕业设计(论文)},
    %
    % 开启该选项后,将用 Times New Roman 的开源字体 TeX Gyre Termes 作为正文字体。
    % 这个选项适用于以下情况:
    % 1. 不想在系统中安装 Times New Roman。
    % 2. 在 Linux/macOS 下遇到 `\textsc` 无法正常显示的问题。
    % betterTimesNewRoman = true,
  },
  misc = {
    % 微调表格行间距
    tabularRowSeparation = 1.25,
  },
}

% 大部分关于参考文献样式的修改,都可以通过此处的选项进行配置。
% 详情请搜索「biblatex-gb7714-2015 文档」进行阅读。
\usepackage[
  backend=biber,
  style=gb7714-2015,
  gbalign=gb7714-2015,
  gbnamefmt=lowercase,
  gbpub=false,
  doi=false,
  url=false,
  eprint=false,
  isbn=false,
]{biblatex}

% 参考文献引用文件位于 misc/ref.bib
\addbibresource{misc/ref.bib}

% 如果要按照计算机学院的要求,带有“北京理工大学”水印,请使用此 issue 提供的方法:
% https://github.com/BITNP/BIThesis/issues/350#issuecomment-1565974141

% 文档开始
\begin{document}

% 标题页面:如无特殊需要,本部分无需改动
\MakeCover

% 前置内容定义
\frontmatter
% 摘要:在相应 TeX 文件撰写
%%
% The BIThesis Template for Bachelor Graduation Thesis
%
% 北京理工大学毕业设计(论文)中英文摘要 —— 使用 XeLaTeX 编译
%
% Copyright 2020-2023 BITNP
%
% This work may be distributed and/or modified under the
% conditions of the LaTeX Project Public License, either version 1.3
% of this license or (at your option) any later version.
% The latest version of this license is in
%   http://www.latex-project.org/lppl.txt
% and version 1.3 or later is part of all distributions of LaTeX
% version 2005/12/01 or later.
%
% This work has the LPPL maintenance status `maintained'.
%
% The Current Maintainer of this work is Feng Kaiyu.

% 中英文摘要章节
\begin{abstract}
% 中文摘要正文从这里开始
本文……。

\textcolor{blue}{摘要正文选用模板中的样式所定义的“正文”,每段落首行缩进 2 个字符;或者手动设置成每段落首行缩进 2 个汉字,字体:宋体,字号:小四,行距:固定值 22 磅,间距:段前、段后均为 0 行。阅后删除此段。}

\textcolor{blue}{摘要是一篇具有独立性和完整性的短文,应概括而扼要地反映出本论文的主要内容。包括研究目的、研究方法、研究结果和结论等,特别要突出研究结果和结论。中文摘要力求语言精炼准确,本科生毕业设计(论文)摘要建议 300-500 字。摘要中不可出现参考文献、图、表、化学结构式、非公知公用的符号和术语。英文摘要与中文摘要的内容应一致。阅后删除此段。}

\end{abstract}

% 一般情况下,超出该行宽度的最后一个单词会被排版引擎尝试(在适合的地方)插入连字符(hyphen)
% 换行。然而,当单词本身比较生僻的情况下,LaTeX 可能会保留完整单词,从而导致该行宽度
% 绕过限制。以“aaaaaabbb” 为例,你可以在以下两个方法中的一种来指定插入连字符的位置:
% 1. 在此处使用 \hyphenation{aaaaaa-bbb}。
% 2. 在文章中使用 aaaaaa\-bbb。
% 以上两种方法都能让引擎在 ab 交界处插入连字符,从而正常换行。

% 英文摘要章节
\begin{abstractEn}
% 英文摘要正文从这里开始
In order to study……

\textcolor{blue}{Abstract 正文设置成每段落首行缩进 2 字符,字体:Times New Roman,字号:小四,行距:固定值 22 磅,间距:段前、段后均为 0 行。阅后删除此段。}
\end{abstractEn}


\MakeTOC

% 正文开始
\mainmatter

% 在这里引用各章 TeX 文件,按需添加
%%
% The BIThesis Template for Bachelor Graduation Thesis
%
% 北京理工大学毕业设计(论文)第一章节 —— 使用 XeLaTeX 编译
%
% Copyright 2020-2023 BITNP
%
% This work may be distributed and/or modified under the
% conditions of the LaTeX Project Public License, either version 1.3
% of this license or (at your option) any later version.
% The latest version of this license is in
%   https://www.latex-project.org/lppl.txt
% and version 1.3 or later is part of all distributions of LaTeX
% version 2005/12/01 or later.
%
% This work has the LPPL maintenance status `maintained'.
%
% The Current Maintainer of this work is Feng Kaiyu.
%
% 第一章节

\chapter{一级题目}

\section{二级题目}
% 这里插入一个参考文献,仅作参考

\subsection{三级题目}

正文……\parencite{yuFeiJiZongTiDuoXueKeSheJiYouHuaDeXianZhuangYuFaZhanFangXiang2008}……\cite{Hajela2012Application}

\textcolor{blue}{正文部分:宋体、小四;正文行距:22磅;间距段前段后均为0行。阅后删除此段。}

\textcolor{blue}{图、表居中,图注标在图下方,表头标在表上方,宋体、五号、居中,1.25倍行距,间距段前段后均为0行,图表与上下文之间各空一行。阅后删除此段。}

\textcolor{blue}{\underline{\underline{图-示例:(阅后删除此段)}}}


\begin{figure}[htbp]
  \centering
  \includegraphics[]{images/bit_logo.png}
  \caption{标题序号}\label{标题序号} % label 用来在文中索引
\end{figure}

\textcolor{blue}{\underline{\underline{表-示例:(阅后删除此段)}}}
% 三线表
\begin{table}[htbp]
  \centering
  \caption{统计表}\label{统计表}
  \begin{tabular}{*{5}{>{\centering\arraybackslash}p{2cm}}} \toprule
    项目    & 产量    & 销量    & 产值   & 比重    \\ \midrule
    手机    & 1000  & 10000 & 500  & 50\%  \\
    计算机   & 5500  & 5000  & 220  & 22\%  \\
    笔记本电脑 & 1100  & 1000  & 280  & 28\%  \\ \midrule
    合计    & 17600 & 16000 & 1000 & 100\% \\ \bottomrule
    \end{tabular}
\end{table}

\textcolor{blue}{公式标注应于该公式所在行的最右侧。对于较长的公式只可在符号处(+、-、*、/、$\leqslant$ $\geqslant$ 等)转行。在文中引用公式时,在标号前加“式”,如式(1-2)。阅后删除此
段。}

\textcolor{blue}{公式-示例:(阅后删除此段)}
% 公式上下不要空行,置于同一个段落下即可,否则上下距离会出现高度不一致的问题
\begin{equation}
    LRI=1\ ∕\ \sqrt{1+{\left(\frac{{\mu }_{R}}{{\mu }_{s}}\right)}^{2}{\left(\frac{{\delta }_{R}}{{\delta }_{s}}\right)}^{2}}
\end{equation}


\section{字体效果表格}

% 列:Regular、Italic、Bold、Bold Italic
% 行:宋体、黑体、楷体、Serif、Sans Serif、Typewriter、Math

\begin{table}[htb]
    \centering
    \caption{字体效果表格}
    \begin{tabular}{@{}lllll@{}}
    \toprule
               & Regular & Bold & Italic & Bold Italic \\ \midrule
      宋体       & 宋体      & \colorbox{orange}{\textbf{宋体粗体}} & \textit{楷体}     &   \colorbox{gray}{\textbf{\textit{楷书粗斜体}}}  \\
      黑体         & {\heiti{}黑体}      & \textbf{\heiti{}黑体粗体} &   \textit{\heiti{}黑体斜体}     & \colorbox{gray}{\textit{\textbf{\heiti{}黑体粗斜体}}}   \\
      楷体         & {\kaishu{}楷书}      & \textbf{\kaishu{}楷书粗体} & \textit{\kaishu{}斜体楷体} &  \colorbox{gray}{\textbf{\textit{\kaishu{}楷书粗斜体}}}    \\
    Serif(Roman/Normal)      &    Regular    &  \textbf{Bold}  &    \textit{Italic}    &     \textbf{\textit{Bold Italic}}    \\
    Sans Serif &  \textsf{Regular}       &  \textbf{\textsf{Bold}}    &  \textit{\textsf{Bold}}   &    \textbf{\textit{\textsf{Bold}}}   \\
    % 有些字体缺少特定变体,LaTeX会抛出警告,同时尽量替换为相近变体。
    % 若编译结果能接受,可忽略之。
    % 例如此处Typewriter代表的Latin Modern Sans Typewriter(lmtt)字体没有bold extended italic(bx/it)变体,
    % 所以`\textbf{\textit{\texttt{…}}}`会被替换为bold slant(b/sl)变体,并引发如下警告。(其中TU指字体编码)
    % Font shape `TU/lmtt/bx/it' in size <…> not available. Font shape `TU/lmtt/b/sl' tried instead.
    Typewriter &  \texttt{Regular}       &  \textbf{\texttt{Bold}}    &  \textit{\texttt{Bold}}   &    \textbf{\textit{\texttt{Bold}}}   \\
    Math       &   $\mathnormal{Regular} \mathrm{Roman}$  & $\mathbf{Bold}$   &    $\mathit{Italic}$    &  $\mathbf{\mathit{Bold Italic}}$    \\ \bottomrule
    \end{tabular}
\end{table}

\begin{itemize}[nosep]
  \item \colorbox{orange}{宋体粗体}在 Windows 下会成为黑体。这是因为 Windows 的中易宋体没有粗体字重而进行的妥协。
    如果想要获得宋体粗体的样式,请在配置中开启伪粗体选项。
  \item \colorbox{gray}{粗斜体}的效果是因操作系统字体而定的,中文写作中不会使用这种字形,可以忽略。
\end{itemize}

\textit{有关公式与上下文间距的一些注意事项:请保证源码中的公式的环境(如}
\\ \verb|\begin{equation}|
  \textit{)与上一段落不要有空行。否则,公式和上文段落之间会有额外的空白。}


\section{常见问题和疑难解答}

如果您遇到\href{https://bithesis.bitnp.net/faq/char-missing.html}{生僻字无法显示}、
\href{https://bithesis.bitnp.net/faq/enumitem-nosep.html}{列表项间距过大}、
\href{https://bithesis.bitnp.net/faq/longtable.html}{三线表需要跨页}等问题,
请参考\textcolor{magenta}{\href{https://bithesis.bitnp.net/faq/}{在线文档的「疑难杂症」部分}}。


% 后置内容
\backmatter

% 结论:在相应 TeX 文件撰写
%%
% The BIThesis Template for Reading Report
%
% 北京理工大学读书报告结论 —— 使用 XeLaTeX 编译
%
% Copyright 2020-2023 BITNP
%
% This work may be distributed and/or modified under the
% conditions of the LaTeX Project Public License, either version 1.3
% of this license or (at your option) any later version.
% The latest version of this license is in
%   http://www.latex-project.org/lppl.txt
% and version 1.3 or later is part of all distributions of LaTeX
% version 2005/12/01 or later.
%
% This work has the LPPL maintenance status `maintained'.
%
% The Current Maintainer of this work is Feng Kaiyu.
%
% Compile with: xelatex -> biber -> xelatex -> xelatex

\begin{conclusion}
  % 结论部分尽量不使用 \subsection 二级标题,只使用 \section 一级标题

  % 这里插入一个参考文献,仅作参考
  本文结论……\cite{张伯伟2002全唐五代诗格会考}。

  \textcolor{blue}{结论作为正文的最后部分单独排写,但不加章号。阅后删除此段。}

  \textcolor{blue}{结论正文样式与文章正文相同:宋体、小四;行距:22磅;间距段前段后均为0行。阅后删除此段。}
\end{conclusion}

% 参考文献:
% 添加文献时,请按 BibTeX 格式添加至 misc/ref.bib,并在正文所需位置使用 \cite{…} 引用。
% 如无特殊需要,无需改动相应 TeX 文件。
%%
% BIThesis 读书报告模板 —— 使用 XeLaTeX 编译 The BIThesis Template for Reading Report
% This file has no copyright assigned and is placed in the Public Domain.
%%

\begin{bibprint}

% \printbibliography[heading=none]
% 正式使用时,请启用上方语句以输出所有的参考文献,并删除/注释下方示例内容。
% 删除后仍可参考 https://bithesis.bitnp.net/faq/bib-entry.html 的在线版本。

% -------------------------------- 示例内容(正式使用时请删除) ------------------------------------- %

% 抑制多次调用 \printbibliography 的 warning,只有示例代码会需要此语句。
\BiblatexSplitbibDefernumbersWarningOff

\textcolor{blue}{参考文献书写规范}

\textcolor{blue}{参考国家标准《信息与文献参考文献著录规则》【GB/T 7714—2015】,参考文献书写规范如下:}

\textcolor{blue}{\textbf{1. 文献类型和标识代码}}

\textcolor{blue}{普通图书:M}\qquad\textcolor{blue}{会议录:C}\qquad\textcolor{blue}{汇编:G}\qquad\textcolor{blue}{报纸:N}

\textcolor{blue}{期刊:J}\qquad\textcolor{blue}{学位论文:D}\qquad\textcolor{blue}{报告:R}\qquad\textcolor{blue}{标准:S}

\textcolor{blue}{专利:P}\qquad\textcolor{blue}{数据库:DB}\qquad\textcolor{blue}{计算机程序:CP}\qquad\textcolor{blue}{电子公告:EB}

\textcolor{blue}{档案:A}\qquad\textcolor{blue}{舆图:CM}\qquad\textcolor{blue}{数据集:DS}\qquad\textcolor{blue}{其他:Z}

\textcolor{blue}{\textbf{2. 不同类别文献书写规范要求}}

\textcolor{blue}{\textbf{期刊}}

\noindent\textcolor{blue}{[序号] 主要责任者. 文献题名[J]. 刊名, 出版年份, 卷号(期号): 起止页码. }
\cite{yuFeiJiZongTiDuoXueKeSheJiYouHuaDeXianZhuangYuFaZhanFangXiang2008, Hajela2012Application}

\printbibliography [type=article,heading=none]

\textcolor{blue}{\textbf{普通图书}}

\noindent\textcolor{blue}{[序号] 主要责任者. 文献题名[M]. 出版地: 出版者, 出版年: 起止页码. }
\cite{张伯伟2002全唐五代诗格会考, OBRIEN1994Aircraft}

\printbibliography [keyword={book},heading=none]

\textcolor{blue}{\textbf{会议论文集}}

\noindent\textcolor{blue}{[序号] 主要责任者.题名:其他题名信息[C]. 出版地: 出版者, 出版年. }
\cite{雷光春2012}

\printbibliography [type=proceedings,heading=none]

\textcolor{blue}{\textbf{专著中析出的文献}}

\noindent\textcolor{blue}{[序号] 析出文献主要责任者. 析出题名[M]//专著主要责任者. 专著题名. 出版地: 出版者, 出版年: 起止页码. }
\cite{白书农}

\printbibliography [type=inbook,heading=none]

\textcolor{blue}{\textbf{学位论文}}

\noindent\textcolor{blue}{[序号] 主要责任者. 文献题名[D]. 保存地: 保存单位, 年份. }
\cite{zhanghesheng, Sobieski}

\printbibliography [keyword={thesis},heading=none]

\textcolor{blue}{\textbf{报告}}

\noindent\textcolor{blue}{[序号] 主要责任者. 文献题名[R]. 报告地: 报告会主办单位, 年份. }
\cite{fengxiqiao, Sobieszczanski}

\printbibliography [keyword={techreport},heading=none]

\textcolor{blue}{\textbf{专利文献}}

\noindent\textcolor{blue}{[序号] 专利所有者. 专利题名:专利号[P]. 公告日期或公开日期[引用日期]. 获取和访问路径. 数字对象唯一标识符.}
\cite{jiangxizhou}

\printbibliography [type=patent,heading=none]

\textcolor{blue}{\textbf{国际、国家标准}}

\noindent\textcolor{blue}{[序号] 主要责任人. 题名: 其他题名信息[S]. 出版地: 出版者, 出版年: 引文页码.}
\cite{GB/T3792.4-2009}

\printbibliography [keyword={standard},heading=none]

\textcolor{blue}{\textbf{报纸文章}}

\noindent\textcolor{blue}{[序号] 主要责任者. 文献题名[N]. 报纸名, 年(期): 页码. }
\cite{xiexide}

\printbibliography [keyword={newspaper},heading=none]

\textcolor{blue}{\textbf{电子文献}}

\noindent\textcolor{blue}{[序号] 主要责任者. 电子文献题名[文献类型/载体类型]. (发表或更新日期) [引用日期]. 获取和访问路径. 数字对象唯一标识符. }
\cite{yaoboyuan}

\printbibliography [keyword={online},heading=none]

\textcolor{blue}{关于参考文献的未尽事项可参考国家标准《信息与文献参考文献著录规则》(GB/T 7714—2015)}

\end{bibprint}

% 附录:在相应 TeX 文件撰写;不需要时可删除
%%
% The BIThesis Template for Bachelor Paper Translation
%
% 北京理工大学毕业设计(论文) —— 使用 XeLaTeX 编译
%
% Copyright 2020-2023 BITNP
%
% This work may be distributed and/or modified under the
% conditions of the LaTeX Project Public License, either version 1.3
% of this license or (at your option) any later version.
% The latest version of this license is in
%   https://www.latex-project.org/lppl.txt
% and version 1.3 or later is part of all distributions of LaTeX
% version 2005/12/01 or later.
%
% This work has the LPPL maintenance status `maintained'.
%
% The Current Maintainer of this work is Feng Kaiyu.
%
% Compile with: xelatex -> biber -> xelatex -> xelatex
%%

\begin{appendices}
  附录相关内容…

  % 这里示范一下添加多个附录的方法:

  \section{\LaTeX 环境的安装}
  \LaTeX 环境的安装。

  \section{BIThesis 使用说明}
  BIThesis 使用说明。

  \textcolor{blue}{附录是毕业设计(论文)主体的补充项目,为了体现整篇文章的完整性,写入正文又可能有损于论文的条理性、逻辑性和精炼性,这些材料可以写入附录段,但对于每一篇文章并不是必须的。附录依次用大写正体英文字母 A、B、C……编序号,如附录 A、附录 B。阅后删除此段。}

  \textcolor{blue}{附录正文样式与文章正文相同:宋体、小四;行距:22 磅;间距段前段后均为 0 行。阅后删除此段。}

\end{appendices}


\end{document}
