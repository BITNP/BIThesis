%%
% BIThesis 本科毕业设计论文翻译模板 —— 使用 XeLaTeX 编译 The BIThesis Template for Bachelor Paper Translation
% This file has no copyright assigned and is placed in the Public Domain.
%%

% 第一章节

\chapter{一级题目}

\section{二级题目}
% 这里插入一个参考文献,仅作参考
正文……\cite{yuFeiJiZongTiDuoXueKeSheJiYouHuaDeXianZhuangYuFaZhanFangXiang2008,Sobieszczanski}

\subsection{三级题目}

正文……\cite{Hajela2012Application,fengxiqiao}

\textcolor{blue}{正文部分:宋体、小四;正文行距:22磅;间距段前段后均为0行。阅后删除此段。}

\textcolor{blue}{图、表居中,图注标在图下方,表头标在表上方,宋体、五号、居中,1.25倍行距,间距段前段后均为0行,图表与上下文之间各空一行。阅后删除此段。}

\textcolor{blue}{\underline{\underline{图-示例:(阅后删除此段)}}}

\begin{figure}[htbp]
  \centering
  \includegraphics[]{images/bit_logo.png}
  \caption{标题序号}\label{标题序号} % label 用来在文中索引
\end{figure}

\textcolor{blue}{\underline{\underline{表-示例:(阅后删除此段)}}}
% 三线表
\begin{table}[htbp]
  \centering
  \caption{统计表}\label{统计表}
  \begin{tabular}{*{5}{>{\centering\arraybackslash}p{2cm}}} \toprule
    项目    & 产量    & 销量    & 产值   & 比重    \\ \midrule
    手机    & 1000  & 10000 & 500  & 50\%  \\
    计算机   & 5500  & 5000  & 220  & 22\%  \\
    笔记本电脑 & 1100  & 1000  & 280  & 28\%  \\ \midrule
    合计    & 17600 & 16000 & 1000 & 100\% \\ \bottomrule
    \end{tabular}
\end{table}

\textcolor{blue}{公式标注应于该公式所在行的最右侧。对于较长的公式只可在符号处(+、-、*、/、$\leqslant$ $\geqslant$ 等)转行。在文中引用公式时,在标号前加“式”,如式(1-2)。阅后删除此
段。}

\textcolor{blue}{公式-示例:(阅后删除此段)}
% 公式上下不要空行,置于同一个段落下即可,否则上下距离会出现高度不一致的问题
\begin{equation}
    LRI=1\ ∕\ \sqrt{1+{\left(\frac{{\mu }_{R}}{{\mu }_{s}}\right)}^{2}{\left(\frac{{\delta }_{R}}{{\delta }_{s}}\right)}^{2}}
\end{equation}
