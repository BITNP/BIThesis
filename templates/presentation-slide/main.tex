%%
% The BIThesis Template for Presentation
%
% 北京理工大学毕业设计(论文) —— 使用 XeLaTeX 编译
%
% Copyright 2021-2023 BITNP
%
% This work may be distributed and/or modified under the
% conditions of the LaTeX Project Public License, either version 1.3
% of this license or (at your option) any later version.
% The latest version of this license is in
%   http://www.latex-project.org/lppl.txt
% and version 1.3 or later is part of all distributions of LaTeX
% version 2005/12/01 or later.
%
% This work has the LPPL maintenance status `maintained'.
%
% The Current Maintainer of this work is Feng Kaiyu.
%
% Compile with: xelatex -> biber -> xelatex -> xelatex
%%

% !TeX program = xelatex
% !BIB program = biber

\documentclass[
  %% ====== 这一部分选项将传递给 ctexbeamer,你也可以添加其他的选项 ====
  % 修改 `aspectratio` 可以修改画布比例 比如 `43` 或者 `169`
  aspectratio=169,
  presentation,
  %% ============================================================
  % 设置封面的图片
  titlegraphic=./images/bit.png,
  % 设置每页标题的 logo
  framelogo=./images/bit.png
]{bitbeamer}

%% --> 基本信息设置
\title{键入你的标题}
\author{Feng Kaiyu \and Jiang Yingqi}
\institute{北京理工大学}
\date{\zhtoday}

\usefonttheme{serif}              % 使用衬线字体
\usefonttheme{professionalfonts}  % 数学公式字体

% 引入你需要使用的包
\usepackage{caption}
\usepackage{subcaption}
\usepackage{color}

\setbeamertemplate{caption}[numbered]

% 使用 bibtex
\usepackage[
  backend=biber,
  style=gb7714-2015,
  gbalign=gb7714-2015,
  gbnamefmt=lowercase,
  gbpub=false,
  doi=false,
  url=false,
  eprint=false,
  isbn=false,
]{biblatex}

\addbibresource{ref.bib}

% 设置段落间距
\parindent2em

\begin{document}

% 封面
\frame{\titlepage}

%
%% --> 目录结构
%
\begin{frame}{目录}
  \tableofcontents[hideallsubsections]
\end{frame}

%% 每一节开头显示目录,并高亮当前节的主题
\AtBeginSection[]{\frame{\tableofcontents[currentsection,hideallsubsections]}}

%% --> 正式内容开始
%
\section{使用示例}    % 第 1 节

\begin{frame}[t] % 第一页
  \frametitle{使用示例}

  % 关于 `\CJKunderline{}` `\CJKunderline*{}` 等更多用法,请参见 http://mirrors.ibiblio.org/CTAN/macros/xetex/latex/xecjk/xeCJK.pdf 的 xeCJKfntef 用法说明 一节

  苏子愀然,\textbf{正襟危坐},而问客曰:“\CJKunderline{何为其然也}?”客曰:“‘月明星稀,乌鹊南飞。’此非曹孟德之诗乎?西望夏口,东望武昌,山川相缪,\CJKunderdot[textformat=\bfseries]{郁乎苍苍},此非孟德之困于周郎者乎?方其破荆州,下江陵,顺流而东也,舳舻千里,旌旗蔽空,酾酒临江,横槊赋诗,固一世之雄也,而今安在哉?况吾与子渔樵于江渚之上,侣鱼虾而友麋鹿,驾一叶之扁舟,举匏樽以相属。\textsl{寄蜉蝣于天地,渺沧海之一粟。}哀吾生之\footnote{之:助词,取独。}须臾,羡长江之无穷。挟飞仙以遨游,抱明月而长终。\CJKhl{yellow}{知不可乎骤得,托遗响于悲风。}”

\end{frame}

\begin{frame}[t]
  \frametitle{使用示例}
  \framesubtitle{定义引理并引用}

\begin{lemma}\label{lemma:1}
在不诚实节点最多仅能领先一块的情况下,不诚实节点利用抢先广播,可以获得不少于原有策略的收益。
\end{lemma}

引理\ref{lemma:1}。引用\cite{eyal_majority_2013}。
\end{frame}

\begin{frame}[t]
  \frametitle{使用示例}
  \framesubtitle{插入图片}

\begin{figure}
    \centering
    \begin{subfigure}{0.47\textwidth}
      \includegraphics[width=\textwidth]{example-image-a}
      \caption{Image A}\label{fig:1-1}
  \end{subfigure}
  \begin{subfigure}{0.47\textwidth}
      \includegraphics[width=\textwidth]{example-image-b}
      \caption{Image B}\label{fig:1-2}
  \end{subfigure}
    \caption{No sea takimata sanctus est Lorem ipsum dolor sit amet.  }%
    \label{fig:1}
  \end{figure}

\end{frame}

\section{总结}

\begin{frame}[t]
  \frametitle{参考文献}

  \printbibliography[heading=none]
\end{frame}

\begin{frame}[c]

  谢谢!

\end{frame}

\end{document}
