%%
% The BIThesis Template for Bachelor Graduation Thesis
%
% 北京理工大学毕业设计(论文) —— 使用 XeLaTeX 编译
%
% Copyright 2021 BITNP
%
% This work may be distributed and/or modified under the
% conditions of the LaTeX Project Public License, either version 1.3
% of this license or (at your option) any later version.
% The latest version of this license is in
%   http://www.latex-project.org/lppl.txt
% and version 1.3 or later is part of all distributions of LaTeX
% version 2005/12/01 or later.
%
% This work has the LPPL maintenance status `maintained'.
%
% The Current Maintainer of this work is Feng Kaiyu.
%
% Compile with: xelatex -> biber -> xelatex -> xelatex

% 修改 `aspectratio` 可以修改画布比例 比如 `43` 或者 `169`
\documentclass[UTF8,aspectratio=43,presentation]{ctexbeamer}

\usetheme{Madrid}
\usefonttheme{serif}              % 使用衬线字体
\usefonttheme{professionalfonts}  % 数学公式字体

\usepackage{zhnumber}
\usepackage{soul}
\usepackage{amsmath}
\usepackage{hyperref}
\usepackage[overload]{empheq}
\usepackage{tikz}
\usepackage{subcaption}
\usepackage{caption}
\usepackage{booktabs}
\usepackage{amsmath}
\usepackage{xspace}
\usepackage{color}
\usepackage[ruled,lined,noend]{algorithm2e}
\usepackage{fontspec}
\usepackage{xeCJKfntef}

\setbeamertemplate{caption}[numbered]

% 设置主题色
\colorlet{beamer@blendedblue}{green!40!black}

% BIT Logo
\titlegraphic{
    \includegraphics[width=2cm]{images/bit.png}
}

%\newtheorem{lemma}{引理}

% 以下内容定义了 `\cjkhl` 宏。
\newcommand*{\cjkhl}[2]{{\def\xcjklhcolorbox{\colorbox{#1}}\xcjkhl#2\relax}}

\makeatletter

%look ahhead for the next character
\def\xcjkhl{\futurelet\tmp\xxcjkhl}

%helper macro to make leaders with a highlight box.
\def\xcjkhlleaders{\leavevmode\leaders\hbox{%
  \fboxsep\z@\xcjklhcolorbox{\strut\kern.1pt%
  \ifx\cjkhlbleeda\relax\else\kern\cjkhlbleeda\fi\relax}%
\ifx\cjkhlbleeda\relax\else\kern-\cjkhlbleeda\fi\relax}}

% leaders with a 1pt of stretch/shrink to put before or after punctuation
\def\@@yhlstretch{\leavevmode\xcjkhlleaders\hskip\z@\@plus.1em \@minus.1em }

\def\@chkhlpar#1\fi\fi#2{\par\noindent\xcjkhl}

\def\@cjkhl@beforeafter#1{%
  \ifx\tmp#1%
    \@@yhlstretch
    \let\@chkhlstretch\@@yhlstretch
    \let\cjkhlpenalty\@highpenalty
  \fi}

\def\@cjkhl@after#1{%
  \ifx\tmp#1%
    \let\@chkhlstretch\@@yhlstretch
    \let\cjkhlpenalty\@highpenalty
  \fi}

\def\@cjkhl@before#1{%
  \ifx\tmp#1%
    \@@yhlstretch
    \let\cjkhlpenalty\@highpenalty
  \fi}

\def\xxcjkhl{%
%look for a \par (from \obeylines)
\ifx\tmp\par\expandafter\@chkhlpar\fi
%look for a \relax to finish
\ifx\tmp\relax
\else
%by default do no stretch leaders after the character
\let\@chkhlstretch\relax
\let\cjkhlpenalty\z@
%these stretch before and after
\@cjkhl@beforeafter?%
\@cjkhl@beforeafter;%
% these stretch after
\@cjkhl@after,%
\@cjkhl@after。%
\@cjkhl@after)%
\@cjkhl@after》%
\@cjkhl@after”%
%these stretch before
\@cjkhl@before(%
\@cjkhl@before《%
\@cjkhl@before“%
% look for a space
\ifx\tmp\@sptoken
\xxxcjkhlsp
\else
%default case stick the current character in a box
\xxxcjkhl
\fi\fi}

%make a highlight leaders stretch/shrink as much as a normal word space.
\def\xxxcjkhlsp#1\fi\fi#2{%
\fi\fi
\xcjkhlleaders\hskip \fontdimen2\font  plus \fontdimen3\font minus \fontdimen4\font\relax
\xcjkhl#2}

%get out of a double \if test
\def\xxxcjkhl\fi\fi{%
\fi\fi
\@chkhl}

%The simple case box the current character and start looking for the next.
% bleed slightly on the right to avoid gaps showing
\let\cjkhlbleeda\relax
\def\cjkhlbleeda{.07pt}
\def\@chkhl#1{{%
\fboxsep\z@
\leavevmode\penalty\cjkhlpenalty
  \xcjklhcolorbox{%
   \strut#1\ifx\cjkhlbleeda\relax\else\kern\cjkhlbleeda\fi}}%
\ifx\cjkhlbleeda\relax\else\kern-\cjkhlbleeda\fi\relax
\@chkhlstretch
\xcjkhl}%

\makeatother


% 以下内容定义了 `\hl` 的宏。
\makeatletter
\let\HL\hl
\renewcommand\hl{%
  \let\set@color\beamerorig@set@color
  \let\reset@color\beamerorig@reset@color
  \HL}
\makeatother

\usepackage[
  backend=biber,
  style=gb7714-2015,
  gbalign=gb7714-2015,
  gbnamefmt=lowercase,
  gbpub=false,
  doi=false,
  url=false,
  eprint=false,
  isbn=false,
]{biblatex}

\usetheme{default}

\addbibresource{ref.bib}

\parindent2em

\begin{document}

%% --> 导言页
\title{键入你的标题}
\author{Feng Kaiyu}
\institute{北京理工大学}
\date{\zhtoday}
\frame{\titlepage}



\addtobeamertemplate{frametitle}{}{%
\begin{tikzpicture}[remember picture,overlay]
\node[anchor=north east,yshift=2pt] at (current page.north east) {\includegraphics[height=0.8cm]{images/bit.png}};
\end{tikzpicture}}

%% --> 目录结构
%
\begin{frame}{目录}
  \tableofcontents[hideallsubsections]
\end{frame}

%% 每一节开头显示目录,并高亮当前节的主题
\AtBeginSection[]{\frame{\tableofcontents[currentsection,hideallsubsections]}}

%% --> 正式内容开始
%
\section{使用示例}    % 第 1 节

\begin{frame}[t]
  \frametitle{使用示例}

 苏子愀然,\textbf{正襟危坐},而问客曰:“\CJKunderline{何为其然也}?”客曰:“‘月明星稀,乌鹊南飞。’此非曹孟德之诗乎?西望夏口,东望武昌,山川相缪,郁乎苍苍,此非孟德之困于周郎者乎?方其破荆州,下江陵,顺流而东也,舳舻千里,旌旗蔽空,酾酒临江,横槊赋诗,固一世之雄也,而今安在哉?况吾与子渔樵于江渚之上,侣鱼虾而友麋鹿,驾一叶之扁舟,举匏樽以相属。\textsl{寄蜉蝣于天地,渺沧海之一粟。}哀吾生之\footnote{之:助词,取独。}须臾,羡长江之无穷。挟飞仙以遨游,抱明月而长终。\cjkhl{yellow}{知不可乎骤得,托遗响于悲风。}” 

\end{frame}

\begin{frame}[t]
  \frametitle{使用示例}
  \framesubtitle{定义引理并引用}
  
\begin{lemma}
  \label{lemma:1}
在不诚实节点最多仅能领先一块的情况下,不诚实节点利用抢先广播,可以获得不少于原有策略的收益。
\end{lemma}

引理\ref{lemma:1}。引用\cite{eyal_majority_2013}。
\end{frame}

\begin{frame}[t]
  \frametitle{使用示例}
  \framesubtitle{引理1的策略证明}

\begin{figure}
    \centering
    \begin{subfigure}{0.47\textwidth}
      \includegraphics[width=\textwidth]{example-image-a}
      \caption{Image A}
      \label{fig:1-1}
  \end{subfigure}
  \begin{subfigure}{0.47\textwidth}
      \includegraphics[width=\textwidth]{example-image-b}
      \caption{Image B}
      \label{fig:1-2}
  \end{subfigure}
    \caption{No sea takimata sanctus est Lorem ipsum dolor sit amet.  }%
    \label{fig:1}
  \end{figure}
  
\end{frame}

\section{空白章节}

\begin{frame}[t]
  \frametitle{参考文献}

\printbibliography[heading=none]
  
\end{frame}

\begin{frame}[c]

  谢谢!
  
\end{frame}

\end{document}
