%%
% The BIThesis Template for Bachelor Graduation Thesis
%
% 北京理工大学毕业设计(论文)第一章节 —— 使用 XeLaTeX 编译
%
% Copyright 2020-2023 BITNP
%
% This work may be distributed and/or modified under the
% conditions of the LaTeX Project Public License, either version 1.3
% of this license or (at your option) any later version.
% The latest version of this license is in
%   https://www.latex-project.org/lppl.txt
% and version 1.3 or later is part of all distributions of LaTeX
% version 2005/12/01 or later.
%
% This work has the LPPL maintenance status `maintained'.
%
% The Current Maintainer of this work is Feng Kaiyu.
%
% 第一章节

\chapter{一级题目}

\section{二级题目}
% 这里插入一个参考文献,仅作参考

\subsection{三级题目}

正文……\parencite{yuFeiJiZongTiDuoXueKeSheJiYouHuaDeXianZhuangYuFaZhanFangXiang2008}……\cite{Hajela2012Application}

\textcolor{blue}{正文部分:宋体、小四;正文行距:22磅;间距段前段后均为0行。阅后删除此段。}

\textcolor{blue}{图、表居中,图注标在图下方,表头标在表上方,宋体、五号、居中,1.25倍行距,间距段前段后均为0行,图表与上下文之间各空一行。阅后删除此段。}

\textcolor{blue}{\underline{\underline{图-示例:(阅后删除此段)}}}


\begin{figure}[htbp]
  \centering
  \includegraphics[]{images/bit_logo.png}
  \caption{标题序号}\label{标题序号} % label 用来在文中索引
\end{figure}

\textcolor{blue}{\underline{\underline{表-示例:(阅后删除此段)}}}
% 三线表
\begin{table}[htbp]
  \centering
  \caption{统计表}\label{统计表}
  \begin{tabular}{*{5}{>{\centering\arraybackslash}p{2cm}}} \toprule
    项目    & 产量    & 销量    & 产值   & 比重    \\ \midrule
    手机    & 1000  & 10000 & 500  & 50\%  \\
    计算机   & 5500  & 5000  & 220  & 22\%  \\
    笔记本电脑 & 1100  & 1000  & 280  & 28\%  \\ \midrule
    合计    & 17600 & 16000 & 1000 & 100\% \\ \bottomrule
    \end{tabular}
\end{table}

\textcolor{blue}{公式标注应于该公式所在行的最右侧。对于较长的公式只可在符号处(+、-、*、/、$\leqslant$ $\geqslant$ 等)转行。在文中引用公式时,在标号前加“式”,如式(1-2)。阅后删除此
段。}

\textcolor{blue}{公式-示例:(阅后删除此段)}
% 公式上下不要空行,置于同一个段落下即可,否则上下距离会出现高度不一致的问题
\begin{equation}
    LRI=1\ ∕\ \sqrt{1+{\left(\frac{{\mu }_{R}}{{\mu }_{s}}\right)}^{2}{\left(\frac{{\delta }_{R}}{{\delta }_{s}}\right)}^{2}}
\end{equation}

\subsubsection{生僻字}

% 一个可能无法正常显示的生僻字: 彧。下文注释中,介绍了如何通过自定义字体来显示生僻字。

% 定义一个提供了生僻字的字体,注意要确保你的系统存在该字体
% \setCJKfamilyfont{custom-font}{Noto Serif CJK SC}

% 使用自己定义的字体
% 使用提供了相应字形的字体:{\CJKfamily{custom-font} 彧}。
% \CJKfamily 会切换字体,影响之后所有内容。故另套“{}”来分组,限制其作用范围。


\section{字体效果表格}

% 列:Regular、Italic、Bold、Bold Italic
% 行:宋体、黑体、楷体、Serif、Sans Serif、Typewriter、Math

\begin{table}[htb]
    \centering
    \caption{字体效果表格}
    \begin{tabular}{@{}lllll@{}}
    \toprule
               & Regular & Bold & Italic & Bold Italic \\ \midrule
      宋体       & 宋体      & \colorbox{orange}{\textbf{宋体粗体}} & \textit{楷体}     &   \colorbox{gray}{\textbf{\textit{楷书粗斜体}}}  \\
      黑体         & {\heiti{}黑体}      & \textbf{\heiti{}黑体粗体} &   \textit{\heiti{}黑体斜体}     & \colorbox{gray}{\textit{\textbf{\heiti{}黑体粗斜体}}}   \\
      楷体         & {\kaishu{}楷书}      & \textbf{\kaishu{}楷书粗体} & \textit{\kaishu{}斜体楷体} &  \colorbox{gray}{\textbf{\textit{\kaishu{}楷书粗斜体}}}    \\
    Serif(Roman/Normal)      &    Regular    &  \textbf{Bold}  &    \textit{Italic}    &     \textbf{\textit{Bold Italic}}    \\
    Sans Serif &  \textsf{Regular}       &  \textbf{\textsf{Bold}}    &  \textit{\textsf{Bold}}   &    \textbf{\textit{\textsf{Bold}}}   \\
    % 有些字体缺少特定变体,LaTeX会抛出警告,同时尽量替换为相近变体。
    % 若编译结果能接受,可忽略之。
    % 例如此处Typewriter代表的Latin Modern Sans Typewriter(lmtt)字体没有bold extended italic(bx/it)变体,
    % 所以`\textbf{\textit{\texttt{…}}}`会被替换为bold slant(b/sl)变体,并引发如下警告。(其中TU指字体编码)
    % Font shape `TU/lmtt/bx/it' in size <…> not available. Font shape `TU/lmtt/b/sl' tried instead.
    Typewriter &  \texttt{Regular}       &  \textbf{\texttt{Bold}}    &  \textit{\texttt{Bold}}   &    \textbf{\textit{\texttt{Bold}}}   \\
    Math       &   $\mathnormal{Regular} \mathrm{Roman}$  & $\mathbf{Bold}$   &    $\mathit{Italic}$    &  $\mathbf{\mathit{Bold Italic}}$    \\ \bottomrule
    \end{tabular}
\end{table}

\begin{itemize}[nosep]
  \item \colorbox{orange}{宋体粗体}在 Windows 下会成为黑体。这是因为 Windows 的中易宋体没有粗体字重而进行的妥协。
    如果想要获得宋体粗体的样式,请在配置中开启伪粗体选项。
  \item \colorbox{gray}{粗斜体}的效果是因操作系统字体而定的,中文写作中不会使用这种字形,可以忽略。
\end{itemize}

\textit{有关公式与上下文间距的一些注意事项:请保证源码中的公式的环境(如}
\\ \verb|\begin{equation}|
  \textit{)与上一段落不要有空行。否则,公式和上文段落之间会有额外的空白。}
