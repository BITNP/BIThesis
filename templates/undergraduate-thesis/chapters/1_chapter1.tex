%%
% The BIThesis Template for Bachelor Graduation Thesis
%
% 北京理工大学毕业设计(论文)第一章节 —— 使用 XeLaTeX 编译
%
% Copyright 2020-2022 BITNP
%
% This work may be distributed and/or modified under the
% conditions of the LaTeX Project Public License, either version 1.3
% of this license or (at your option) any later version.
% The latest version of this license is in
%   http://www.latex-project.org/lppl.txt
% and version 1.3 or later is part of all distributions of LaTeX
% version 2005/12/01 or later.
%
% This work has the LPPL maintenance status `maintained'.
%
% The Current Maintainer of this work is Feng Kaiyu.
%
% 第一章节

\chapter{一级题目}

\section{二级题目}
% 这里插入一个参考文献,仅作参考
正文……\cite{yuFeiJiZongTiDuoXueKeSheJiYouHuaDeXianZhuangYuFaZhanFangXiang2008}

\subsection{三级题目}

正文……\cite{Hajela2012Application}

\textcolor{blue}{正文部分:宋体、小四;正文行距:22磅;间距段前段后均为0行。阅后删除此段。}

\textcolor{blue}{图、表居中,图注标在图下方,表头标在表上方,宋体、五号、居中,1.25倍行距,间距段前段后均为0行,图表与上下文之间各空一行。阅后删除此段。}

\textcolor{blue}{\underline{\underline{图-示例:(阅后删除此段)}}}

\begin{figure}[htbp]
  \vspace{13pt} % 调整图片与上文的垂直距离
  \centering
  \includegraphics[]{images/bit_logo.png}
  \caption{标题序号}\label{标题序号} % label 用来在文中索引
\end{figure}

\textcolor{blue}{\underline{\underline{表-示例:(阅后删除此段)}}}

% \begin{table}[htbp]
%   \linespread{1.5}
%   \zihao{5}
%   \centering
%   \caption{统计表}\label{统计表}
%   \begin{tabular}{*{5}{>{\centering\arraybackslash}p{2cm}}}
%     \hline
%     项目    & 产量    & 销量    & 产值   & 比重    \\ \hline
%     手机    & 1000  & 10000 & 500  & 50\%  \\
%     计算机   & 5500  & 5000  & 220  & 22\%  \\
%     笔记本电脑 & 1100  & 1000  & 280  & 28\%  \\ \hline
%     合计    & 17600 & 16000 & 1000 & 100\% \\ \hline
%     \end{tabular}
% \end{table}

\begin{table}[htbp]
  \linespread{1.5}
  \zihao{5}
  \centering
  \caption{统计表}\label{统计表}
  \begin{tabular}{*{5}{>{\centering\arraybackslash}p{2cm}}}
    \toprule
    项目    & 产量    & 销量    & 产值   & 比重    \\ \hline
    手机    & 1000  & 10000 & 500  & 50\%  \\
    计算机   & 5500  & 5000  & 220  & 22\%  \\
    笔记本电脑 & 1100  & 1000  & 280  & 28\%  \\
    合计    & 17600 & 16000 & 1000 & 100\% \\ \bottomrule
    \end{tabular}
\end{table}

\textcolor{blue}{公式标注应于该公式所在行的最右侧。对于较长的公式只可在符号处(+、-、*、/、$\leqslant$ $\geqslant$ 等)转行。在文中引用公式时,在标号前加“式”,如式(1-2)。阅后删除此
段。}

\textcolor{blue}{公式-示例:(阅后删除此段)}
% 公式上下不要空行,置于同一个段落下即可,否则上下距离会出现高度不一致的问题
\begin{equation}
    LRI=1\ ∕\ \sqrt{1+{\left(\frac{{\mu }_{R}}{{\mu }_{s}}\right)}^{2}{\left(\frac{{\delta }_{R}}{{\delta }_{s}}\right)}^{2}}
\end{equation}

\subsubsection{生僻字}

% 一个可能无法正常显示的生僻字
一个可能无法正常显示的生僻字: 彧。下文注释中,介绍了如何通过自定义字体来显示生僻字。

% 定义一个提供了生僻字的字体,注意要确保你的系统存在该字体
% \setCJKfamilyfont{custom-font}{Noto Serif CJK SC}

% 使用自己定义的字体
% 使用提供了相应字型的字体:\CJKfamily{custom-font}{彧}。

