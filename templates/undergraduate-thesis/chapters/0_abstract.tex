%%
% The BIThesis Template for Bachelor Graduation Thesis
%
% 北京理工大学毕业设计(论文)中英文摘要 —— 使用 XeLaTeX 编译
%
% Copyright 2020-2023 BITNP
%
% This work may be distributed and/or modified under the
% conditions of the LaTeX Project Public License, either version 1.3
% of this license or (at your option) any later version.
% The latest version of this license is in
%   https://www.latex-project.org/lppl.txt
% and version 1.3 or later is part of all distributions of LaTeX
% version 2005/12/01 or later.
%
% This work has the LPPL maintenance status `maintained'.
%
% The Current Maintainer of this work is Feng Kaiyu.

% 中英文摘要章节
\begin{abstract}
% 中文摘要正文从这里开始
本文……。

\textcolor{blue}{摘要正文选用模板中的样式所定义的“正文”,每段落首行缩进 2 个字符;或者手动设置成每段落首行缩进 2 个汉字,字体:宋体,字号:小四,行距:固定值 22 磅,间距:段前、段后均为 0 行。阅后删除此段。}

\textcolor{blue}{摘要是一篇具有独立性和完整性的短文,应概括而扼要地反映出本论文的主要内容。包括研究目的、研究方法、研究结果和结论等,特别要突出研究结果和结论。中文摘要力求语言精炼准确,本科生毕业设计(论文)摘要建议 300-500 字。摘要中不可出现参考文献、图、表、化学结构式、非公知公用的符号和术语。英文摘要与中文摘要的内容应一致。阅后删除此段。}

\end{abstract}

% 一般情况下,超出该行宽度的最后一个单词会被排版引擎尝试(在适合的地方)插入连字符(hyphen)
% 换行。然而,当单词本身比较生僻的情况下,LaTeX 可能会保留完整单词,从而导致该行宽度
% 绕过限制。以“aaaaaabbb” 为例,你可以在以下两个方法中的一种来指定插入连字符的位置:
% 1. 在此处使用 \hyphenation{aaaaaa-bbb}。
% 2. 在文章中使用 aaaaaa\-bbb。
% 以上两种方法都能让引擎在 ab 交界处插入连字符,从而正常换行。

% 英文摘要章节
\begin{abstractEn}
% 英文摘要正文从这里开始
In order to study……

\textcolor{blue}{Abstract 正文设置成每段落首行缩进 2 字符,字体:Times New Roman,字号:小四,行距:固定值 22 磅,间距:段前、段后均为 0 行。阅后删除此段。}
\end{abstractEn}
