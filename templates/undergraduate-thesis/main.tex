%%
% The BIThesis Template for Bachelor Graduation Thesis
%
% 北京理工大学毕业设计(论文) —— 使用 XeLaTeX 编译
%
% Copyright 2021-2023 BITNP
%
% This work may be distributed and/or modified under the
% conditions of the LaTeX Project Public License, either version 1.3
% of this license or (at your option) any later version.
% The latest version of this license is in
%   http://www.latex-project.org/lppl.txt
% and version 1.3 or later is part of all distributions of LaTeX
% version 2005/12/01 or later.
%
% This work has the LPPL maintenance status `maintained'.
%
% The Current Maintainer of this work is Feng Kaiyu.
%
% Compile with: xelatex -> biber -> xelatex -> xelatex

% !TeX program = xelatex
% !BIB program = biber


% 开启盲审格式 blindPeerReview=true (如:[type=bachelor,blindPeerReview=true])

\documentclass[type=bachelor]{bithesis}

% 此处仅列出常用的配置。全部配置用法请见「bithesis.pdf」手册。
\BITSetup{
  cover = {
    % 在封面中载入有「北京理工大学」字样的图片,如无必要请勿改动。
    headerImage = images/header.png,
    % 在封面标题中使用思源黑体,使用此选项可以保证与 Word 封面标题的字体一致。
    xiheiFont = STXIHEI.TTF,
    %% 使用以下参数来自定义封面日期
    % date = 2022年6月,
    % 本科生盲审要求删去封面,而不是隐藏封面信息。
    hideCoverInPeerReview = true,
  },
  info = {
    % 想要删除某项封面信息,直接删除该项即可。
    % 想要让某项封面信息留空(但是保留下划线),请传入空白符组成的字符串,如"{~}"。
    % 如需要换行,则用 “\\” 符号分割。
    title = 北京理工大学本科生毕业设计(论文)题目,
    titleEn = {The Subject of Undergraduate Graduation Project (Thesis) of Beijing Institute of Technology},
    school = 计算机学院,
    major = 计算机科学与技术,
    class = 0561xxxx,
    author = 惠计算,
    studentId = 11xxxxxxxx,
    supervisor = 张哈希,
    keywords = {北京理工大学;本科生;毕业设计(论文)},
    keywordsEn = {BIT; Undergraduate; Graduation Project (Thesis)},
    % 如果你的毕设为校外毕设,请将下面这一行语句解除注释(删除第一个百分号字符)并填写你的校外毕设导师名字
    % externalSupervisor = 左偏树,
  },
  style = {
    % 开启 Windows 平台下的中易宋体伪粗体。
    % windowsSimSunFakeBold = true,
  }
}

% 使用 listings 宏包进行代码块使用,并使用了预定义的样式,
% 你也可以选用自己的喜欢的其他宏包,如 minted;
% 然而由于 minted 依赖 Python 的 Pygments 库作为外部依赖,因此出于模板的简洁程度考虑,我们没有提供 minted 进行代码块书写的示例。
\usepackage{listings}


% 大部分关于参考文献样式的修改,都可以通过此处的选项进行配置。
% 详情请搜索「biblatex-gb7714-2015 文档」进行阅读。
\usepackage[
  backend=biber,
  style=gb7714-2015,
  gbalign=gb7714-2015,
  gbnamefmt=lowercase,
  gbpub=false,
  doi=false,
  url=false,
  eprint=false,
  isbn=false,
]{biblatex}

% 参考文献引用文件位于 misc/ref.bib
\addbibresource{misc/ref.bib}

% 如果要按照计算机学院的要求,
% 在外文翻译报告中使用带有“北京理工大学”水印
% 请使用此 issue 提供的方法:
% https://github.com/BITNP/BIThesis/issues/350#issuecomment-1565974141

% 文档开始
\begin{document}

% 标题页面:如无特殊需要,本部分无需改动
% %%
% The BIThesis Template for Bachelor Graduation Thesis
%
% 北京理工大学毕业设计(论文)封面页 —— 使用 XeLaTeX 编译
%
% Copyright 2020-2021 BITNP
%
% This work may be distributed and/or modified under the
% conditions of the LaTeX Project Public License, either version 1.3
% of this license or (at your option) any later version.
% The latest version of this license is in
%   http://www.latex-project.org/lppl.txt
% and version 1.3 or later is part of all distributions of LaTeX
% version 2005/12/01 or later.
%
% This work has the LPPL maintenance status `maintained'.
%
% The Current Maintainer of this work is Feng Kaiyu.
%
% 封面
%
% 如无特殊需要,本页面无需更改

% Underline new command for student information
% Usage: \dunderline[<offset>]{<line_thickness>}
\newcommand\dunderline[3][-1pt]{{%
  \setbox0=\hbox{#3}
  \ooalign{\copy0\cr\rule[\dimexpr#1-#2\relax]{\wd0}{#2}}}}

% Cover Page
\begin{titlepage}
  \makeatletter
  \@ifundefined{externalMentorName}{
    % 校内毕设封面顶部间距
    \vspace*{19mm}
  }{
    % 校外毕设封面顶部间距
    \vspace*{13mm}
  }
  \centering

  \includegraphics[width=9.87cm]{images/header.png}

  \vspace*{-3mm}

  \zihao{-0}\textbf{\ziju{0.12}\songti{本科生毕业设计(论文)}}

  \vspace{16mm}

  \zihao{2}\textbf{\xihei\thesisTitle}

  \vspace{3mm}

  \begin{spacing}{1.2}
    \zihao{3}\selectfont{\textbf{\thesisTitleEN}}
  \end{spacing}

  \vspace{15mm}

  \flushleft

  \makeatletter
  \@ifundefined{externalMentorName}{
    % 生成校内毕设封面字段
    \makeatother
    \begin{spacing}{1.8}
      \hspace{27mm}\songti\zihao{3}\selectfont{学\hspace{11mm}院:\dunderline[-10pt]{1pt}{\makebox[78mm][c]{\deptName}}}

      \hspace{27mm}\songti\zihao{3}\selectfont{专\hspace{11mm}业:\dunderline[-10pt]{1pt}{\makebox[78mm][c]{\majorName}}}

      \hspace{27mm}\songti\zihao{3}\selectfont{学生姓名:\dunderline[-10pt]{1pt}{\makebox[78mm][c]{\yourName}}}

      \hspace{27mm}\songti\zihao{3}\selectfont{学\hspace{11mm}号:\dunderline[-10pt]{1pt}{\makebox[78mm][c]{\yourStudentID}}}

      \hspace{27mm}\songti\zihao{3}\selectfont{指导教师:\dunderline[-10pt]{1pt}{\makebox[78mm][c]{\mentorName}}}
    \end{spacing}
  }{
    % 生成校外毕设封面字段
    \makeatother
    \begin{spacing}{1.8}
      \hspace{19.4mm}\songti\zihao{3}\selectfont{学\hspace{19.6mm}院\hspace{3mm}:\dunderline[-10pt]{1pt}{\makebox[77.4mm][c]{\deptName}}}

      \hspace{19.4mm}\songti\zihao{3}\selectfont{专\hspace{19.6mm}业\hspace{3mm}:\dunderline[-10pt]{1pt}{\makebox[77.4mm][c]{\majorName}}}

      \hspace{19.4mm}\songti\zihao{3}\selectfont{学\hspace{2.8mm}生\hspace{2.8mm}姓\hspace{2.8mm}名\hspace{3mm}:\dunderline[-10pt]{1pt}{\makebox[77.4mm][c]{\yourName}}}

      \hspace{19.4mm}\songti\zihao{3}\selectfont{学\hspace{19.6mm}号\hspace{3mm}:\dunderline[-10pt]{1pt}{\makebox[77.4mm][c]{\yourStudentID}}}

      \hspace{19.4mm}\songti\zihao{3}\selectfont{指\hspace{2.8mm}导\hspace{2.8mm}教\hspace{2.8mm}师\hspace{3mm}:\dunderline[-10pt]{1pt}{\makebox[77.4mm][c]{\mentorName}}}

      \hspace{19.4mm}\songti\zihao{3}\selectfont{校外指导教师:\dunderline[-10pt]{1pt}{\makebox[77.4mm][c]{\externalMentorName}}}
    \end{spacing}
  }

  \vspace*{\fill}
  \centering
  \zihao{3}\ziju{0.5}\songti{\today}
\end{titlepage}

\MakeCover

% 原创性声明:如无特殊需要,本部分无需改动
% 更改为 PDF 页面插入,如需要添加内容,可考虑先用 Word 制作再覆盖 misc/1_originality.pdf
% ====== 原创性声明(PDF 格式)======
\begin{blindPeerReview}
  \includepdf{misc/1_originality.pdf}\newpage
\end{blindPeerReview}
% ====== 原创性声明(PDF 格式)======
% ====== 原创性声明(LaTeX 格式)======
% %%
% The BIThesis Template for Bachelor Graduation Thesis
%
% 北京理工大学毕业设计(论文)原创性声明页 —— 使用 XeLaTeX 编译
%
% Copyright 2020 Spencer Woo
%
% This work may be distributed and/or modified under the
% conditions of the LaTeX Project Public License, either version 1.3
% of this license or (at your option) any later version.
% The latest version of this license is in
%   http://www.latex-project.org/lppl.txt
% and version 1.3 or later is part of all distributions of LaTeX
% version 2005/12/01 or later.
%
% This work has the LPPL maintenance status `maintained'.
%
% The Current Maintainer of this work is Spencer Woo.
%
% 如无特殊需要,本页面无需更改

% 原创性声明页无页码页面格式
\fancypagestyle{originality}{
  % 页眉高度
  \setlength{\headheight}{20pt}

  % 页眉和页脚(页码)的格式设定
  \fancyhf{}
  \fancyhead[C]{\zihao{4}\ziju{0.08}\songti{北京理工大学本科生毕业设计(论文)}}

  % 页眉分割线稍微粗一些
  \renewcommand{\headrulewidth}{0.6pt}
}

\pagestyle{originality}
\topskip=0pt

% 圆形数字编号定义
\newcommand{\circled}[2][]{\tikz[baseline=(char.base)]
  {\node[shape = circle, draw, inner sep = 1pt]
  (char) {\phantom{\ifblank{#1}{#2}{#1}}};
  \node at (char.center) {\makebox[0pt][c]{#2}};}}
\robustify{\circled}

% 设置行间距
\setlength{\parskip}{0.4em}
\renewcommand{\baselinestretch}{1.41}

% 顶部空白
\vspace*{-6mm}

% 原创性声明部分
\begin{center}
  \heiti\zihao{2}\textbf{原创性声明}
\end{center}

% 本部分字号为小三
\zihao{-3}

本人郑重声明:所呈交的毕业设计(论文),是本人在指导老师的指导下独立进行研究所取得的成果。除文中已经注明引用的内容外,本文不包含任何其他个人或集体已经发表或撰写过的研究成果。对本文的研究做出重要贡献的个人和集体,均已在文中以明确方式标明。

特此申明。

\vspace{13mm}

\begin{flushright}
  本人签名:\hspace{40mm}日\hspace{2.5mm}期:\hspace{13mm}年\hspace{8mm}月\hspace{8mm}日
\end{flushright}

\vspace{17mm}

% 使用授权声明部分
\begin{center}
  \heiti\zihao{2}\textbf{关于使用授权的声明}
\end{center}

本人完全了解北京理工大学有关保管、使用毕业设计(论文)的规定,其中包括:\circled{1}学校有权保管、并向有关部门送交本毕业设计(论文)的原件与复印件;\circled{2}学校可以采用影印、缩印或其它复制手段复制并保存本毕业设计(论文);\circled{3}学校可允许本毕业设计(论文)被查阅或借阅;\circled{4}学校可以学术交流为目的,复制赠送和交换本毕业设计(论文);\circled{5}学校可以公布本毕业设计(论文)的全部或部分内容。

\vspace*{1mm}

\begin{flushright}
  \begin{spacing}{1.65}
    \zihao{-3}
    本人签名:\hspace{40mm}日\hspace{2.5mm}期:\hspace{13mm}年\hspace{8mm}月\hspace{8mm}日\\
    指导老师签名:\hspace{40mm}日\hspace{2.5mm}期:\hspace{13mm}年\hspace{8mm}月\hspace{8mm}日
  \end{spacing}
\end{flushright}

\newpage

% ====== 原创性声明(LaTeX 格式)======

% 前置页面定义
\frontmatter
% 摘要:在摘要相应的 TeX 文件处进行摘要部分的撰写
%%
% BIThesis 本科毕业设计论文模板(全英文) —— 使用 XeLaTeX 编译 The BIThesis Template for Undergraduate Thesis
% This file has no copyright assigned and is placed in the Public Domain.
%%

% 摘要若要按经管学院的要求,先英文再中文,请调换以下 abstract、abstractEn 的顺序。

\begin{abstract}
  Conventional  product  development  employs  a  design-build-test  philosophy.
  The sequentially  executed  development  process  often  results  in  prolonged
  lead  times  and elevated product costs. The proposed e-Design paradigm employs
  IT-enabled technology for product design, including virtual prototyping (VP) to
  support a cross-functional team in analyzing  product  performance,  reliability,
  and  manufacturing costs  early  in  product development, and in making quantitative
  trade-offs for design decision making. Physical prototypes  of  the  product  design
  are  then  produced  using  the  rapid  prototyping  (RP) technique  and  computer
  numerical  control  (CNC)  to  support  design  verification  and functional prototyping, respectively.
\end{abstract}

\begin{abstractEn}
  Conventional  product  development  employs  a  design-build-test  philosophy.
  The sequentially  executed  development  process  often  results  in  prolonged
  lead  times  and elevated product costs. The proposed e-Design paradigm employs
  IT-enabled technology for product design, including virtual prototyping (VP) to
  support a cross-functional team in analyzing  product  performance,  reliability,
  and  manufacturing costs  early  in  product development, and in making quantitative
  trade-offs for design decision making. Physical prototypes  of  the  product  design
  are  then  produced  using  the  rapid  prototyping  (RP) technique  and  computer
  numerical  control  (CNC)  to  support  design  verification  and functional prototyping, respectively.
\end{abstractEn}


\MakeTOC

% 正文开始
\mainmatter

% 第一章
%%
% The BIThesis Template for Bachelor Graduation Thesis
%
% 北京理工大学毕业设计(论文)第一章节 —— 使用 XeLaTeX 编译
%
% Copyright 2020-2021 BITNP
%
% This work may be distributed and/or modified under the
% conditions of the LaTeX Project Public License, either version 1.3
% of this license or (at your option) any later version.
% The latest version of this license is in
%   http://www.latex-project.org/lppl.txt
% and version 1.3 or later is part of all distributions of LaTeX
% version 2005/12/01 or later.
%
% This work has the LPPL maintenance status `maintained'.
%
% The Current Maintainer of this work is Feng Kaiyu.
%
% 第一章节

\chapter{一级题目}

\section{二级题目}
% 这里插入一个参考文献,仅作参考
正文……\cite{yuFeiJiZongTiDuoXueKeSheJiYouHuaDeXianZhuangYuFaZhanFangXiang2008}

\subsection{三级题目}

正文……\cite{Hajela2012Application}

\textcolor{blue}{正文部分:宋体、小四;正文行距:22磅;间距段前段后均为0行。阅后删除此段。}

\textcolor{blue}{图、表居中,图注标在图下方,表头标在表上方,宋体、五号、居中,1.25倍行距,间距段前段后均为0行,图表与上下文之间各空一行。阅后删除此段。}

\textcolor{blue}{\underline{\underline{图-示例:(阅后删除此段)}}}

\begin{figure}[htbp]
  \vspace{13pt} % 调整图片与上文的垂直距离
  \centering
  \includegraphics[]{images/bit_logo.png}
  \caption{标题序号}\label{标题序号} % label 用来在文中索引
\end{figure}

\textcolor{blue}{\underline{\underline{表-示例:(阅后删除此段)}}}

\begin{table}[htbp]
  \linespread{1.5}
  \zihao{5}
  \centering
  \caption{统计表}\label{统计表}
  \begin{tabular}{*{5}{>{\centering\arraybackslash}p{2cm}}}
    \hline
    项目    & 产量    & 销量    & 产值   & 比重    \\ \hline
    手机    & 1000  & 10000 & 500  & 50\%  \\
    计算机   & 5500  & 5000  & 220  & 22\%  \\
    笔记本电脑 & 1100  & 1000  & 280  & 28\%  \\ \hline
    合计    & 17600 & 16000 & 1000 & 100\% \\ \hline
    \end{tabular}
\end{table}

\textcolor{blue}{公式标注应于该公式所在行的最右侧。对于较长的公式只可在符号处(+、-、*、/、$\leqslant$ $\geqslant$ 等)转行。在文中引用公式时,在标号前加“式”,如式(1-2)。阅后删除此
段。}

\textcolor{blue}{公式-示例:(阅后删除此段)}
% 公式上下不要空行,置于同一个段落下即可,否则上下距离会出现高度不一致的问题
\begin{equation}
    LRI=1\ ∕\ \sqrt{1+{\left(\frac{{\mu }_{R}}{{\mu }_{s}}\right)}^{2}{\left(\frac{{\delta }_{R}}{{\delta }_{s}}\right)}^{2}}
\end{equation}

\subsubsection{生僻字}

% 一个可能无法正常显示的生僻字
一个可能无法正常显示的生僻字: 彧。下文注释中,介绍了如何通过自定义字体来显示生僻字。

% 定义一个提供了生僻字的字体,注意要确保你的系统存在该字体
% \setCJKfamilyfont{custom-font}{Noto Serif CJK SC}

% 使用自己定义的字体
% 使用提供了相应字型的字体:\CJKfamily{custom-font}{彧}。


% 在这里添加第二章、第三章……TeX 文件的引用
%%
% The BIThesis Template for Bachelor Graduation Thesis
%
% 北京理工大学毕业设计(论文)第二章节 —— 使用 XeLaTeX 编译
%
% Copyright 2020-2021 BITNP
%
% This work may be distributed and/or modified under the
% conditions of the LaTeX Project Public License, either version 1.3
% of this license or (at your option) any later version.
% The latest version of this license is in
%   http://www.latex-project.org/lppl.txt
% and version 1.3 or later is part of all distributions of LaTeX
% version 2005/12/01 or later.
%
% This work has the LPPL maintenance status `maintained'.
%
% The Current Maintainer of this work is Feng Kaiyu.
%%

\chapter{另一个章节}

\section{代码片段}

\begin{lstlisting}[language=Python, caption={Python Code}, label={lst:pythonfile}]
import numpy as np

def incmatrix(genl1,genl2):
    m = len(genl1)
    n = len(genl2)
    M = None #to become the incidence matrix
    VT = np.zeros((n*m,1), int)  #dummy variable

    #compute the bitwise xor matrix
    M1 = bitxormatrix(genl1)
    M2 = np.triu(bitxormatrix(genl2),1)

    for i in range(m-1):
        for j in range(i+1, m):
            [r,c] = np.where(M2 == M1[i,j])
            for k in range(len(r)):
                VT[(i)*n + r[k]] = 1;
                VT[(i)*n + c[k]] = 1;
                VT[(j)*n + r[k]] = 1;
                VT[(j)*n + c[k]] = 1;

                if M is None:
                    M = np.copy(VT)
                else:
                    M = np.concatenate((M, VT), 1)

                VT = np.zeros((n*m,1), int)

    return M
\end{lstlisting}

% %%
% BIThesis 本科毕业设计论文模板(全英文) —— 使用 XeLaTeX 编译 The BIThesis Template for Undergraduate Thesis
% This file has no copyright assigned and is placed in the Public Domain.
%%

\chapter{Engineering Design}

Although engineering drawing still plays an important role in product design and manufacturing in many industrial sectors around the world, manual sketching for creating drawings has been gradually replaced by CAD (computer-aided design) software using computers. Beginning in the 1980s, CAD software reduced the need for draftsmen significantly, especially in small to mid-sized companies. The software's affordability and ability to run on personal computers in the mid-1990s allowed engineers to do their own drafting and analytic work to some extent \ref{eq:1}.

\begin{equation}
x^n + y^n = z^n
\label{eq:1}
\end{equation}

\section{注意事项}

为保证封面、参考文献等支持中文,模板尽管是英文,仍使用了 \texttt{ctex} 及 \texttt{xeCJK} 宏包。
请注意以下事项。

\subsection{输入标点符号}

% 此节删改自 lshort-zh-cn(GFDL v1.3)。
% https://github.com/CTeX-org/lshort-zh-cn/blob/e95288a697822a6f63e6f2eb1b9160a6f324c566/src/chap/chap.02.text.tex#L160-L178

\LaTeX{} 中,西文的单引号 ` 和\ ' 分别用 \verb|`| 和 \verb|'| 输入;双引号 `` 和\ '' 分别用 \verb|``| 和 \verb|''| 输入。
若用其它方式,引号具体形状和宽度可能错误。
(而中文的标点符号使用中文输入法输入即可,一般不需要过多留意。)

\begin{lstlisting}
``No, sir!'' All the pedant in Dean Hart was aroused.
``In the English language, the word `majority' means `more than half'.
Thirteen out of fourteen is a majority, nothing more.''

“他不回答,对柜里说,‘温两碗酒,要一碟茴香豆。’便排出九文大钱。”
\end{lstlisting}

``No, sir!'' All the pedant in Dean Hart was aroused. ``In the English language, the word `majority' means `more than half'. Thirteen out of fourteen is a majority, nothing more.''

“他不回答,对柜里说,‘温两碗酒,要一碟茴香豆。’便排出九文大钱。”

% 如有特殊需要,可参考 https://github.com/CTeX-org/ctex-kit/issues/389 定制。


\section{常见问题和疑难解答}

如果您遇到\href{https://bithesis.bitnp.net/faq/char-missing.html}{生僻字无法显示}、
\href{https://bithesis.bitnp.net/faq/enumitem-nosep.html}{列表项间距过大}、
\href{https://bithesis.bitnp.net/faq/longtable.html}{三线表需要跨页}等问题,
请参考\textcolor{magenta}{\href{https://bithesis.bitnp.net/faq/}{在线文档的「疑难杂症」部分}}。


% 后置部分
\backmatter

% 结论:在结论相应的 TeX 文件处进行结论部分的撰写
\input{misc/2_conclusion.tex}
% 参考文献:如无特殊需要,参考文献相应的 TeX 文件无需改动,添加参考文献请使用 BibTeX 的格式
%   添加至 misc/ref.bib 中,并在正文的相应位置使用 \cite{xxx} 的格式引用参考文献
\input{misc/3_reference.tex}
% 附录:在附录相应的 TeX 文件处进行附录部分的撰写
%%
% BIThesis 本科毕业设计论文模板 —— 使用 XeLaTeX 编译 The BIThesis Template for Undergraduate Thesis
% This file has no copyright assigned and is placed in the Public Domain.
%%

\begin{appendices}
  附录相关内容…

  % 这里示范一下添加多个附录的方法:
  % 使用 \section 来添加一个附录

  \section{\LaTeX 环境的安装}
  \LaTeX 环境的安装。

  \section{BIThesis 使用说明}
  BIThesis 使用说明。

  \textcolor{blue}{附录是毕业设计(论文)主体的补充项目,为了体现整篇文章的完整性,写入正文又可能有损于论文的条理性、逻辑性和精炼性,这些材料可以写入附录段,但对于每一篇文章并不是必须的。附录依次用大写正体英文字母 A、B、C……编序号,如附录 A、附录 B。阅后删除此段。}

  \textcolor{blue}{附录正文样式与文章正文相同:宋体、小四;行距:22 磅;间距段前段后均为 0 行。阅后删除此段。}

\end{appendices}

% 致谢:在致谢相应的 TeX 文件处进行致谢部分的撰写
%%
% BIThesis 本科毕业设计论文模板 —— 使用 XeLaTeX 编译 The BIThesis Template for Undergraduate Thesis
% This file has no copyright assigned and is placed in the Public Domain.
%%

% 致谢部分尽量不使用 \subsection 二级标题,只使用 \section 一级标题
\begin{acknowledgements}
  值此论文完成之际,首先向我的导师……

  \textcolor{blue}{致谢正文样式与文章正文相同:宋体、小四;行距:22 磅;间距段前段后均为 0 行。阅后删除此段。}
\end{acknowledgements}


\end{document}
