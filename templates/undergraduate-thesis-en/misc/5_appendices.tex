\begin{appendices}
  \section{Design Structure Matrix Example}
  One of the most useful applications of the design structure matrix (DSM) method is to represent well-established, but complex, engineering design processes. This rich process modeling approach facilitates:• Understanding of the existing development process.• Communication of the process to the people involved.• Process improvement.• Visualization of progress during the project.Exhibit 18-14 shows a DSM model of a critical portion of the development process at a major automobile manufacturer. The model includes50 tasks involved in the digital mock-up (DMU) process for the layout of all of the many components in the engine compartment of the vehicle. The process takes place in six phases, depicted by the blocks of activities along the diagonal. The first two of these phases (project planning and CAD data collection) occur  in  parallel,  followed  by  the development  of  the  digital  assembly  model  (DMU preparation). Each of the last three phases involves successively more ac-curate analytical verification that components represented by the digital assembly model actually fit properly within the engine compartment area of the vehicle.In contrast to the simpler DSM model shown in Exhibit 18-3, where the squares on the diagonal identify sets of coupled activities, the DSM in Exhibit 18-14 uses such blocks to show which activities are executed together (in parallel, sequentially, and/or iteratively) within each phase. Arrows and dashed lines represent the major iterations between sets of activities within each phase.
\end{appendices}
