%%
% The BIThesis Template for Undergraduate Thesis
%
% 北京理工大学毕业设计(论文) —— 使用 XeLaTeX 编译
%
% Copyright 2021-2023 BITNP
%
% This work may be distributed and/or modified under the
% conditions of the LaTeX Project Public License, either version 1.3
% of this license or (at your option) any later version.
% The latest version of this license is in
%   https://www.latex-project.org/lppl.txt
% and version 1.3 or later is part of all distributions of LaTeX
% version 2005/12/01 or later.
%
% This work has the LPPL maintenance status `maintained'.
%
% The Current Maintainer of this work is Feng Kaiyu.
%
% Compile with: xelatex -> biber -> xelatex -> xelatex

% !TeX program = xelatex
% !BIB program = biber


% 开启盲审格式 blindPeerReview=true (如:[type=bachelor_english,blindPeerReview=true])

\documentclass[type=bachelor_english]{bithesis}

% 若需中文封面或在经管学院,请参考 ./README.md 中「BITSetup 学院变体」一节。
% 此处仅列出常用的配置。全部配置用法请见「bithesis.pdf」手册。
\BITSetup{
  cover = {
    % 封面需要「北京理工大学」字样图片,如无必要请勿修改该项。
    headerImage = images/header.png,
    % 封面标题需要“华文细黑”,如无必要请勿修改该项。
    xiheiFont = STXIHEI.TTF,
    % 可用以下参数自定义封面日期。
    % date = {November 5, 1955},
  },
  info = {
    % 想要删除某项封面信息,直接删除该项即可。
    % 想要让某项封面信息留空(但是保留下划线),请传入空白符组成的字符串,如"{~}"。
    % 如需要换行,则用 “\\” 符号分割。
    title = 你的论文标题(中文),
    titleEn = {Your Thesis Title},
    school = School of Mechanical Engineering,
    major = Bachelor of Science in Mechanical Engineering,
    author = Feng Kaiyu,
    studentId = 11xxxxxxxx,
    supervisor = Alex Zhang,
    keywords = {北京理工大学;本科生;毕业设计(论文)——请在“main.tex”开头设置},
    keywordsEn = {Computer-Aided Design; FEM; CAM},
    % 如果你的毕设为校外毕设,请将下面这一行语句解除注释(删除第一个百分号字符)并填写你的校外毕设导师名字
    % externalSupervisor = 左偏树,
  },
  % 有些专业名带上“Bachelor of…”太长,学院可能建议把 degree 改成 major。
  % 若要将封面的“Degree:”改为“Major:”,请解除下一行的注释。
  % const/info/major = {Major},
  %
  style = {
    % 页眉若要改为中文,请解除下一行的注释。
    % head = {北京理工大学本科生毕业设计(论文)},
    %
    % 开启该选项后,将用 Times New Roman 的开源字体 TeX Gyre Termes 作为正文字体。
    % 这个选项适用于以下情况:
    % 1. 不想在系统中安装 Times New Roman。
    % 2. 在 Linux/macOS 下遇到 `\textsc` 无法正常显示的问题。
    % betterTimesNewRoman = true,
  },
  misc = {
    % 微调表格行间距
    tabularRowSeparation = 1.25,
  },
}

% 使用 listings 宏包进行代码块使用,并使用了预定义的样式,
% 你也可以选用自己的喜欢的其他宏包,如 minted;
% 然而由于 minted 依赖 Python 的 Pygments 库作为外部依赖,因此出于模板的简洁程度考虑,我们没有提供 minted 进行代码块书写的示例。
\usepackage{listings}


% 大部分关于参考文献样式的修改,都可以通过此处的选项进行配置。
% 详情请搜索「biblatex-gb7714-2015 文档」进行阅读。
\usepackage[
  backend=biber,
  style=gb7714-2015,
  gbalign=gb7714-2015,
  gbnamefmt=lowercase,
  gbpub=false,
  doi=false,
  url=false,
  eprint=false,
  isbn=false,
]{biblatex}

% 参考文献引用文件位于 misc/ref.bib
\addbibresource{misc/ref.bib}

% 如果要按照计算机学院的要求,带有“北京理工大学”水印,请使用此 issue 提供的方法:
% https://github.com/BITNP/BIThesis/issues/350#issuecomment-1565974141

% 文档开始
\begin{document}

% 标题页面:如无特殊需要,本部分无需改动
\MakeCover

% 原创性声明:盲审结束前无需改动
\MakeOriginality
% 后续添加签名时,可考虑先用 Word 制作再导出到 misc/0_originality.pdf,
% 同时注释以上 `\MakeOriginality`,解除注释以下三行代码。
% \begin{blindPeerReview}
%   \includepdf{misc/0_originality.pdf}\newpage
% \end{blindPeerReview}

% 前置内容定义
\frontmatter
% 摘要:在相应 TeX 文件撰写
%%
% BIThesis 本科毕业设计论文模板(全英文) —— 使用 XeLaTeX 编译 The BIThesis Template for Undergraduate Thesis
% This file has no copyright assigned and is placed in the Public Domain.
%%

% 摘要若要按经管学院的要求,先英文再中文,请调换以下 abstract、abstractEn 的顺序。

\begin{abstract}
  Conventional  product  development  employs  a  design-build-test  philosophy.
  The sequentially  executed  development  process  often  results  in  prolonged
  lead  times  and elevated product costs. The proposed e-Design paradigm employs
  IT-enabled technology for product design, including virtual prototyping (VP) to
  support a cross-functional team in analyzing  product  performance,  reliability,
  and  manufacturing costs  early  in  product development, and in making quantitative
  trade-offs for design decision making. Physical prototypes  of  the  product  design
  are  then  produced  using  the  rapid  prototyping  (RP) technique  and  computer
  numerical  control  (CNC)  to  support  design  verification  and functional prototyping, respectively.
\end{abstract}

\begin{abstractEn}
  Conventional  product  development  employs  a  design-build-test  philosophy.
  The sequentially  executed  development  process  often  results  in  prolonged
  lead  times  and elevated product costs. The proposed e-Design paradigm employs
  IT-enabled technology for product design, including virtual prototyping (VP) to
  support a cross-functional team in analyzing  product  performance,  reliability,
  and  manufacturing costs  early  in  product development, and in making quantitative
  trade-offs for design decision making. Physical prototypes  of  the  product  design
  are  then  produced  using  the  rapid  prototyping  (RP) technique  and  computer
  numerical  control  (CNC)  to  support  design  verification  and functional prototyping, respectively.
\end{abstractEn}


\MakeTOC

% 正文开始
\mainmatter

% 在这里引用各章 TeX 文件,按需添加
%%
% The BIThesis Template for Bachelor Graduation Thesis
%
% 北京理工大学毕业设计(论文)第一章节 —— 使用 XeLaTeX 编译
%
% Copyright 2020-2021 BITNP
%
% This work may be distributed and/or modified under the
% conditions of the LaTeX Project Public License, either version 1.3
% of this license or (at your option) any later version.
% The latest version of this license is in
%   http://www.latex-project.org/lppl.txt
% and version 1.3 or later is part of all distributions of LaTeX
% version 2005/12/01 or later.
%
% This work has the LPPL maintenance status `maintained'.
%
% The Current Maintainer of this work is Feng Kaiyu.
%
% 第一章节

\chapter{一级题目}

\section{二级题目}
% 这里插入一个参考文献,仅作参考
正文……\cite{yuFeiJiZongTiDuoXueKeSheJiYouHuaDeXianZhuangYuFaZhanFangXiang2008}

\subsection{三级题目}

正文……\cite{Hajela2012Application}

\textcolor{blue}{正文部分:宋体、小四;正文行距:22磅;间距段前段后均为0行。阅后删除此段。}

\textcolor{blue}{图、表居中,图注标在图下方,表头标在表上方,宋体、五号、居中,1.25倍行距,间距段前段后均为0行,图表与上下文之间各空一行。阅后删除此段。}

\textcolor{blue}{\underline{\underline{图-示例:(阅后删除此段)}}}

\begin{figure}[htbp]
  \vspace{13pt} % 调整图片与上文的垂直距离
  \centering
  \includegraphics[]{images/bit_logo.png}
  \caption{标题序号}\label{标题序号} % label 用来在文中索引
\end{figure}

\textcolor{blue}{\underline{\underline{表-示例:(阅后删除此段)}}}

\begin{table}[htbp]
  \linespread{1.5}
  \zihao{5}
  \centering
  \caption{统计表}\label{统计表}
  \begin{tabular}{*{5}{>{\centering\arraybackslash}p{2cm}}}
    \hline
    项目    & 产量    & 销量    & 产值   & 比重    \\ \hline
    手机    & 1000  & 10000 & 500  & 50\%  \\
    计算机   & 5500  & 5000  & 220  & 22\%  \\
    笔记本电脑 & 1100  & 1000  & 280  & 28\%  \\ \hline
    合计    & 17600 & 16000 & 1000 & 100\% \\ \hline
    \end{tabular}
\end{table}

\textcolor{blue}{公式标注应于该公式所在行的最右侧。对于较长的公式只可在符号处(+、-、*、/、$\leqslant$ $\geqslant$ 等)转行。在文中引用公式时,在标号前加“式”,如式(1-2)。阅后删除此
段。}

\textcolor{blue}{公式-示例:(阅后删除此段)}
% 公式上下不要空行,置于同一个段落下即可,否则上下距离会出现高度不一致的问题
\begin{equation}
    LRI=1\ ∕\ \sqrt{1+{\left(\frac{{\mu }_{R}}{{\mu }_{s}}\right)}^{2}{\left(\frac{{\delta }_{R}}{{\delta }_{s}}\right)}^{2}}
\end{equation}

\subsubsection{生僻字}

% 一个可能无法正常显示的生僻字
一个可能无法正常显示的生僻字: 彧。下文注释中,介绍了如何通过自定义字体来显示生僻字。

% 定义一个提供了生僻字的字体,注意要确保你的系统存在该字体
% \setCJKfamilyfont{custom-font}{Noto Serif CJK SC}

% 使用自己定义的字体
% 使用提供了相应字型的字体:\CJKfamily{custom-font}{彧}。


%%
% The BIThesis Template for Bachelor Graduation Thesis
%
% 北京理工大学毕业设计(论文)第二章节 —— 使用 XeLaTeX 编译
%
% Copyright 2020-2021 BITNP
%
% This work may be distributed and/or modified under the
% conditions of the LaTeX Project Public License, either version 1.3
% of this license or (at your option) any later version.
% The latest version of this license is in
%   http://www.latex-project.org/lppl.txt
% and version 1.3 or later is part of all distributions of LaTeX
% version 2005/12/01 or later.
%
% This work has the LPPL maintenance status `maintained'.
%
% The Current Maintainer of this work is Feng Kaiyu.
%%

\chapter{另一个章节}

\section{代码片段}

\begin{lstlisting}[language=Python, caption={Python Code}, label={lst:pythonfile}]
import numpy as np

def incmatrix(genl1,genl2):
    m = len(genl1)
    n = len(genl2)
    M = None #to become the incidence matrix
    VT = np.zeros((n*m,1), int)  #dummy variable

    #compute the bitwise xor matrix
    M1 = bitxormatrix(genl1)
    M2 = np.triu(bitxormatrix(genl2),1)

    for i in range(m-1):
        for j in range(i+1, m):
            [r,c] = np.where(M2 == M1[i,j])
            for k in range(len(r)):
                VT[(i)*n + r[k]] = 1;
                VT[(i)*n + c[k]] = 1;
                VT[(j)*n + r[k]] = 1;
                VT[(j)*n + c[k]] = 1;

                if M is None:
                    M = np.copy(VT)
                else:
                    M = np.concatenate((M, VT), 1)

                VT = np.zeros((n*m,1), int)

    return M
\end{lstlisting}

%%
% BIThesis 本科毕业设计论文模板(全英文) —— 使用 XeLaTeX 编译 The BIThesis Template for Undergraduate Thesis
% This file has no copyright assigned and is placed in the Public Domain.
%%

\chapter{Engineering Design}

Although engineering drawing still plays an important role in product design and manufacturing in many industrial sectors around the world, manual sketching for creating drawings has been gradually replaced by CAD (computer-aided design) software using computers. Beginning in the 1980s, CAD software reduced the need for draftsmen significantly, especially in small to mid-sized companies. The software's affordability and ability to run on personal computers in the mid-1990s allowed engineers to do their own drafting and analytic work to some extent \ref{eq:1}.

\begin{equation}
x^n + y^n = z^n
\label{eq:1}
\end{equation}

\section{注意事项}

为保证封面、参考文献等支持中文,模板尽管是英文,仍使用了 \texttt{ctex} 及 \texttt{xeCJK} 宏包。
请注意以下事项。

\subsection{输入标点符号}

% 此节删改自 lshort-zh-cn(GFDL v1.3)。
% https://github.com/CTeX-org/lshort-zh-cn/blob/e95288a697822a6f63e6f2eb1b9160a6f324c566/src/chap/chap.02.text.tex#L160-L178

\LaTeX{} 中,西文的单引号 ` 和\ ' 分别用 \verb|`| 和 \verb|'| 输入;双引号 `` 和\ '' 分别用 \verb|``| 和 \verb|''| 输入。
若用其它方式,引号具体形状和宽度可能错误。
(而中文的标点符号使用中文输入法输入即可,一般不需要过多留意。)

\begin{lstlisting}
``No, sir!'' All the pedant in Dean Hart was aroused.
``In the English language, the word `majority' means `more than half'.
Thirteen out of fourteen is a majority, nothing more.''

“他不回答,对柜里说,‘温两碗酒,要一碟茴香豆。’便排出九文大钱。”
\end{lstlisting}

``No, sir!'' All the pedant in Dean Hart was aroused. ``In the English language, the word `majority' means `more than half'. Thirteen out of fourteen is a majority, nothing more.''

“他不回答,对柜里说,‘温两碗酒,要一碟茴香豆。’便排出九文大钱。”

% 如有特殊需要,可参考 https://github.com/CTeX-org/ctex-kit/issues/389 定制。


\section{常见问题和疑难解答}

如果您遇到\href{https://bithesis.bitnp.net/faq/char-missing.html}{生僻字无法显示}、
\href{https://bithesis.bitnp.net/faq/enumitem-nosep.html}{列表项间距过大}、
\href{https://bithesis.bitnp.net/faq/longtable.html}{三线表需要跨页}等问题,
请参考\textcolor{magenta}{\href{https://bithesis.bitnp.net/faq/}{在线文档的「疑难杂症」部分}}。


% 后置内容
\backmatter

% 结论:在相应 TeX 文件撰写
%%
% BIThesis 本科毕业设计论文模板(全英文) —— 使用 XeLaTeX 编译 The BIThesis Template for Undergraduate Thesis
% This file has no copyright assigned and is placed in the Public Domain.
%%

\begin{conclusion}
  During the 1980s, design-for-manufacturing practices were put into place in thousands of firms. Today DFM is an essential part of almost every product development effort. No longer can designers ``throw the design over the wall'' to production engineers. As a result of this emphasis on improved design quality, some manufacturers claim to have reduced production costs of products by up to 50 percent. In fact, comparing current new product designs with earlier generations, one can usually identify fewer parts in the new product, as well as new materials, more integrated and custom parts, higher-volume standard parts and subassemblies, and simpler assembly procedures.
\end{conclusion}

% 参考文献:
% 添加文献时,请按 BibTeX 格式添加至 misc/ref.bib,并在正文所需位置使用 \cite{…} 引用。
% 如无特殊需要,无需改动相应 TeX 文件。
%%
% BIThesis 本科毕业设计论文模板(全英文) —— 使用 XeLaTeX 编译 The BIThesis Template for Undergraduate Thesis
% This file has no copyright assigned and is placed in the Public Domain.
%%

\begin{bibprint}
  \printbibliography[heading=none]
\end{bibprint}

% 附录:在相应 TeX 文件撰写;不需要时可删除
%%
% The BIThesis Template for Graduate Thesis
%
% Copyright 2020-2023 Yang Yating, BITNP
%
% This work may be distributed and/or modified under the
% conditions of the LaTeX Project Public License, either version 1.3
% of this license or (at your option) any later version.
% The latest version of this license is in
%   https://www.latex-project.org/lppl.txt
% and version 1.3 or later is part of all distributions of LaTeX
% version 2005/12/01 or later.
%
% This work has the LPPL maintenance status `maintained'.
%
% The Current Maintainer of this work is Feng Kaiyu.

\begin{appendices}
  \chapter{费马大定理的证明}
  关于此,我确信已发现了一种美妙的证法,可惜这里空白的地方太小,写不下。

  \chapter{Maxwell Equations}
  因为在柱坐标系下,$\overline{\overline\mu}$是对角的,所以Maxwell方程组中电场$\bf
  E$的旋度

  所以$\bf H$的各个分量可以写为:
  \begin{subequations}
    \begin{eqnarray}
      H_r=\frac{1}{\mathbf{i}\omega\mu_r}\frac{1}{r}\frac{\partial
        E_z}{\partial\theta } \\
      H_\theta=-\frac{1}{\mathbf{i}\omega\mu_\theta}\frac{\partial E_z}{\partial r}
    \end{eqnarray}
  \end{subequations}

  同样地,在柱坐标系下,$\overline{\overline\epsilon}$是对角的,所以Maxwell方程组中磁场$\bf
  H$的旋度
  \begin{subequations}
    \begin{eqnarray}
      &&\nabla\times{\bf H}=-\mathbf{i}\omega{\bf D}\\
      &&\left[\frac{1}{r}\frac{\partial}{\partial
          r}(rH_\theta)-\frac{1}{r}\frac{\partial
          H_r}{\partial\theta}\right]{\hat{\bf
          z}}=-\mathbf{i}\omega{\overline{\overline\epsilon}}{\bf
        E}=-\mathbf{i}\omega\epsilon_zE_z{\hat{\bf z}} \\
      &&\frac{1}{r}\frac{\partial}{\partial
        r}(rH_\theta)-\frac{1}{r}\frac{\partial
        H_r}{\partial\theta}=-\mathbf{i}\omega\epsilon_zE_z
    \end{eqnarray}
  \end{subequations}

  由此我们可以得到关于$E_z$的波函数方程:
  \begin{eqnarray}
    \frac{1}{\mu_\theta\epsilon_z}\frac{1}{r}\frac{\partial}{\partial r}
    \left(r\frac{\partial E_z}{\partial r}\right)+
    \frac{1}{\mu_r\epsilon_z}\frac{1}{r^2}\frac{\partial^2E_z}{\partial\theta^2}
    +\omega^2 E_z=0
  \end{eqnarray}

  \chapter{要求}

  \textcolor{blue}{
  有些材料编入文章主体会有损于编排的条理性和逻辑性,或有碍于文章结构的紧凑和突出主题思想等,这些材料可作为附录另页排在参考文献之后,也可以单编成册。下列内容可作为附录:
  }
  \begin{enumerate}
    \item \textcolor{blue}{为了整篇论文材料的完整,但编入正文有损于编排的条理性和逻辑性的材料,这一类材料包括比正文更为详尽的信息、研究方法和技术等更深入的叙述,以及建议可阅读的参考文献题录和对了解正文内容有用的补充信息等;}
    \item \textcolor{blue}{ 由于篇幅过大或取材的复制资料不便于编入正文的材料; }
    \item \textcolor{blue}{ 不便于编入正文的罕见珍贵资料; }
    \item \textcolor{blue}{ 一般读者无须阅读,但对本专业同行有参考价值的资料; }
    \item \textcolor{blue}{ 某些重要的原始数据、推导、计算程序、框图、结构图、注释、统计表、计算机打印输出件等; }
  \end{enumerate}

  \section{一级标题}
  \subsection{二级标题}
\end{appendices}

% 致谢:在相应 TeX 文件撰写
%%
% BIThesis 本科毕业设计论文模板(全英文) —— 使用 XeLaTeX 编译 The BIThesis Template for Undergraduate Thesis
% This file has no copyright assigned and is placed in the Public Domain.
%%

\begin{acknowledgements}
This book contains material developed for use
in the interdisciplinary courses on product development that we teach.
Participants in these courses include graduate students in engineering,
industrial design students, and MBA students. While we
aimed the book at interdisciplinary graduate-level audiences such
as this, many faculty teaching graduate and undergraduate courses
in engineering design have also found the material useful. Product
Design and Development is also for practicing professionals. Indeed,
we could not avoid writing  for  a  professional  audience,
because  most  of  our  students  are  themselves professionals who have worked
either in product development or in closely related functions.

This book blends the perspectives of marketing, design, and manufacturing into
a single approach to product development. As a result, we provide students
of all kinds with an appreciation for the realities of industrial practice
and for the complex and essential roles played by the various members of
product development teams. For industrial practitioners, in  particular,
we  provide  a  set  of  product  development  methods  that  can  be
put  into immediate practice on development projects.
\end{acknowledgements}


\end{document}
