%%
% BIThesis 本科毕业设计论文模板(全英文) —— 使用 XeLaTeX 编译 The BIThesis Template for Undergraduate Thesis
% This file has no copyright assigned and is placed in the Public Domain.
% Compile with: xelatex -> biber -> xelatex -> xelatex
%%

% !TeX program = xelatex
% !BIB program = biber

% 我校英文论文尚无统一规范,目前存在各种学院变体,请参考 README.md 调整。

% 开启盲审格式 blindPeerReview=true (如:[type=bachelor_english,blindPeerReview=true])

\documentclass[type=bachelor_english]{bithesis}

% 若需中文封面或在经管学院,请参考 ./README.md 中「BITSetup 学院变体」一节。
% 此处仅列出常用的配置。全部配置用法请见「bithesis.pdf」手册。
\BITSetup{
  cover = {
    % 封面需要「北京理工大学」字样图片,如无必要请勿修改该项。
    headerImage = images/header.png,
    % 封面标题需要“华文细黑”,如无必要请勿修改该项。
    xiheiFont = STXIHEI.TTF,
    % 可用以下参数自定义封面日期。
    % date = {November 5, 1955},
  },
  info = {
    % 想要删除某项封面信息,直接删除该项即可。
    % 想要让某项封面信息留空(但是保留下划线),请传入空白符组成的字符串,如"{~}"。
    % 如需要换行,则用 “\\” 符号分割。
    title = 你的论文标题(中文),
    titleEn = {Your Thesis Title},
    school = School of Mechanical Engineering,
    major = Bachelor of Science in Mechanical Engineering,
    author = Feng Kaiyu,
    studentId = 11xxxxxxxx,
    supervisor = Alex Zhang,
    keywords = {北京理工大学;本科生;毕业设计(论文)——请在“main.tex”开头设置},
    keywordsEn = {Computer-Aided Design; FEM; CAM},
    % 如果你的毕设为校外毕设,请将下面这一行语句解除注释(删除第一个百分号字符)并填写你的校外毕设导师名字
    % externalSupervisor = 左偏树,
  },
  % 有些专业名带上“Bachelor of…”太长,学院可能建议把 degree 改成 major。
  % 若要将封面的“Degree:”改为“Major:”,请解除下一行的注释。
  % const/info/major = {Major},
  %
  style = {
    % 页眉若要改为中文,请解除下一行的注释。
    % head = {北京理工大学本科生毕业设计(论文)},
    %
    % 开启该选项后,将用 Times New Roman 的开源字体 TeX Gyre Termes 作为正文字体。
    % 这个选项适用于以下情况:
    % 1. 不想在系统中安装 Times New Roman。
    % 2. 在 Linux/macOS 下遇到 `\textsc` 无法正常显示的问题。
    % betterTimesNewRoman = true,
  },
  misc = {
    % 微调表格行间距
    tabularRowSeparation = 1.25,
  },
}

% 这份模板提供了代码块示例,采用 listings 宏包及其预定义样式。
% 使用代码块时,也可选用自己的喜欢的其他宏包,如 minted:https://bithesis.bitnp.net/faq/minted.html
% 如果不需要,直接删除即可。
\usepackage{listings}


% 大部分关于参考文献样式的修改,都可以通过此处的选项进行配置。
% 详情请搜索「biblatex-gb7714-2015 文档」进行阅读。
\usepackage[
  backend=biber,
  style=gb7714-2015,
  gbalign=gb7714-2015,
  gbnamefmt=lowercase,
  gbpub=false,
  doi=false,
  url=false,
  eprint=false,
  isbn=false,
]{biblatex}

% 参考文献引用文件位于 misc/ref.bib
\addbibresource{misc/ref.bib}

% 若要按计算机学院要求,添加“北京理工大学”水印,请参考
% https://bithesis.bitnp.net/faq/watermark.html

% 文档开始
\begin{document}

% 标题页面:如无特殊需要,本部分无需改动
\MakeCover

% 原创性声明:盲审结束前无需改动
\MakeOriginality
% 后续添加签名时,可考虑先用 Word 制作再导出到 misc/0_originality.pdf,
% 同时注释以上 `\MakeOriginality`,解除注释以下三行代码。
% \begin{blindPeerReview}
%   \includepdf{misc/0_originality.pdf}\newpage
% \end{blindPeerReview}

% 前置内容定义
\frontmatter
% 摘要:在相应 TeX 文件撰写
%%
% The BIThesis Template for Bachelor Graduation Thesis
%
% 北京理工大学毕业设计(论文)中英文摘要 —— 使用 XeLaTeX 编译
%
% Copyright 2020-2023 BITNP
%
% This work may be distributed and/or modified under the
% conditions of the LaTeX Project Public License, either version 1.3
% of this license or (at your option) any later version.
% The latest version of this license is in
%   http://www.latex-project.org/lppl.txt
% and version 1.3 or later is part of all distributions of LaTeX
% version 2005/12/01 or later.
%
% This work has the LPPL maintenance status `maintained'.
%
% The Current Maintainer of this work is Feng Kaiyu.

% 中英文摘要章节
\begin{abstract}
% 中文摘要正文从这里开始
本文……。

\textcolor{blue}{摘要正文选用模板中的样式所定义的“正文”,每段落首行缩进 2 个字符;或者手动设置成每段落首行缩进 2 个汉字,字体:宋体,字号:小四,行距:固定值 22 磅,间距:段前、段后均为 0 行。阅后删除此段。}

\textcolor{blue}{摘要是一篇具有独立性和完整性的短文,应概括而扼要地反映出本论文的主要内容。包括研究目的、研究方法、研究结果和结论等,特别要突出研究结果和结论。中文摘要力求语言精炼准确,本科生毕业设计(论文)摘要建议 300-500 字。摘要中不可出现参考文献、图、表、化学结构式、非公知公用的符号和术语。英文摘要与中文摘要的内容应一致。阅后删除此段。}

\end{abstract}

% 一般情况下,超出该行宽度的最后一个单词会被排版引擎尝试(在适合的地方)插入连字符(hyphen)
% 换行。然而,当单词本身比较生僻的情况下,LaTeX 可能会保留完整单词,从而导致该行宽度
% 绕过限制。以“aaaaaabbb” 为例,你可以在以下两个方法中的一种来指定插入连字符的位置:
% 1. 在此处使用 \hyphenation{aaaaaa-bbb}。
% 2. 在文章中使用 aaaaaa\-bbb。
% 以上两种方法都能让引擎在 ab 交界处插入连字符,从而正常换行。

% 英文摘要章节
\begin{abstractEn}
% 英文摘要正文从这里开始
In order to study……

\textcolor{blue}{Abstract 正文设置成每段落首行缩进 2 字符,字体:Times New Roman,字号:小四,行距:固定值 22 磅,间距:段前、段后均为 0 行。阅后删除此段。}
\end{abstractEn}


\MakeTOC

% 正文开始
\mainmatter

% 在这里引用各章 TeX 文件,按需添加
%%
% The BIThesis Template for Bachelor Graduation Thesis
%
% 北京理工大学毕业设计(论文)第一章节 —— 使用 XeLaTeX 编译
%
% Copyright 2020-2023 BITNP
%
% This work may be distributed and/or modified under the
% conditions of the LaTeX Project Public License, either version 1.3
% of this license or (at your option) any later version.
% The latest version of this license is in
%   https://www.latex-project.org/lppl.txt
% and version 1.3 or later is part of all distributions of LaTeX
% version 2005/12/01 or later.
%
% This work has the LPPL maintenance status `maintained'.
%
% The Current Maintainer of this work is Feng Kaiyu.
%
% 第一章节

\chapter{一级题目}

\section{二级题目}
% 这里插入一个参考文献,仅作参考

\subsection{三级题目}

正文……\parencite{yuFeiJiZongTiDuoXueKeSheJiYouHuaDeXianZhuangYuFaZhanFangXiang2008}……\cite{Hajela2012Application}

\textcolor{blue}{正文部分:宋体、小四;正文行距:22磅;间距段前段后均为0行。阅后删除此段。}

\textcolor{blue}{图、表居中,图注标在图下方,表头标在表上方,宋体、五号、居中,1.25倍行距,间距段前段后均为0行,图表与上下文之间各空一行。阅后删除此段。}

\textcolor{blue}{\underline{\underline{图-示例:(阅后删除此段)}}}


\begin{figure}[htbp]
  \centering
  \includegraphics[]{images/bit_logo.png}
  \caption{标题序号}\label{标题序号} % label 用来在文中索引
\end{figure}

\textcolor{blue}{\underline{\underline{表-示例:(阅后删除此段)}}}
% 三线表
\begin{table}[htbp]
  \centering
  \caption{统计表}\label{统计表}
  \begin{tabular}{*{5}{>{\centering\arraybackslash}p{2cm}}} \toprule
    项目    & 产量    & 销量    & 产值   & 比重    \\ \midrule
    手机    & 1000  & 10000 & 500  & 50\%  \\
    计算机   & 5500  & 5000  & 220  & 22\%  \\
    笔记本电脑 & 1100  & 1000  & 280  & 28\%  \\ \midrule
    合计    & 17600 & 16000 & 1000 & 100\% \\ \bottomrule
    \end{tabular}
\end{table}

\textcolor{blue}{公式标注应于该公式所在行的最右侧。对于较长的公式只可在符号处(+、-、*、/、$\leqslant$ $\geqslant$ 等)转行。在文中引用公式时,在标号前加“式”,如式(1-2)。阅后删除此
段。}

\textcolor{blue}{公式-示例:(阅后删除此段)}
% 公式上下不要空行,置于同一个段落下即可,否则上下距离会出现高度不一致的问题
\begin{equation}
    LRI=1\ ∕\ \sqrt{1+{\left(\frac{{\mu }_{R}}{{\mu }_{s}}\right)}^{2}{\left(\frac{{\delta }_{R}}{{\delta }_{s}}\right)}^{2}}
\end{equation}


\section{字体效果表格}

% 列:Regular、Italic、Bold、Bold Italic
% 行:宋体、黑体、楷体、Serif、Sans Serif、Typewriter、Math

\begin{table}[htb]
    \centering
    \caption{字体效果表格}
    \begin{tabular}{@{}lllll@{}}
    \toprule
               & Regular & Bold & Italic & Bold Italic \\ \midrule
      宋体       & 宋体      & \colorbox{orange}{\textbf{宋体粗体}} & \textit{楷体}     &   \colorbox{gray}{\textbf{\textit{楷书粗斜体}}}  \\
      黑体         & {\heiti{}黑体}      & \textbf{\heiti{}黑体粗体} &   \textit{\heiti{}黑体斜体}     & \colorbox{gray}{\textit{\textbf{\heiti{}黑体粗斜体}}}   \\
      楷体         & {\kaishu{}楷书}      & \textbf{\kaishu{}楷书粗体} & \textit{\kaishu{}斜体楷体} &  \colorbox{gray}{\textbf{\textit{\kaishu{}楷书粗斜体}}}    \\
    Serif(Roman/Normal)      &    Regular    &  \textbf{Bold}  &    \textit{Italic}    &     \textbf{\textit{Bold Italic}}    \\
    Sans Serif &  \textsf{Regular}       &  \textbf{\textsf{Bold}}    &  \textit{\textsf{Bold}}   &    \textbf{\textit{\textsf{Bold}}}   \\
    % 有些字体缺少特定变体,LaTeX会抛出警告,同时尽量替换为相近变体。
    % 若编译结果能接受,可忽略之。
    % 例如此处Typewriter代表的Latin Modern Sans Typewriter(lmtt)字体没有bold extended italic(bx/it)变体,
    % 所以`\textbf{\textit{\texttt{…}}}`会被替换为bold slant(b/sl)变体,并引发如下警告。(其中TU指字体编码)
    % Font shape `TU/lmtt/bx/it' in size <…> not available. Font shape `TU/lmtt/b/sl' tried instead.
    Typewriter &  \texttt{Regular}       &  \textbf{\texttt{Bold}}    &  \textit{\texttt{Bold}}   &    \textbf{\textit{\texttt{Bold}}}   \\
    Math       &   $\mathnormal{Regular} \mathrm{Roman}$  & $\mathbf{Bold}$   &    $\mathit{Italic}$    &  $\mathbf{\mathit{Bold Italic}}$    \\ \bottomrule
    \end{tabular}
\end{table}

\begin{itemize}[nosep]
  \item \colorbox{orange}{宋体粗体}在 Windows 下会成为黑体。这是因为 Windows 的中易宋体没有粗体字重而进行的妥协。
    如果想要获得宋体粗体的样式,请在配置中开启伪粗体选项。
  \item \colorbox{gray}{粗斜体}的效果是因操作系统字体而定的,中文写作中不会使用这种字形,可以忽略。
\end{itemize}

\textit{有关公式与上下文间距的一些注意事项:请保证源码中的公式的环境(如}
\\ \verb|\begin{equation}|
  \textit{)与上一段落不要有空行。否则,公式和上文段落之间会有额外的空白。}


\section{常见问题和疑难解答}

如果您遇到\href{https://bithesis.bitnp.net/faq/char-missing.html}{生僻字无法显示}、
\href{https://bithesis.bitnp.net/faq/enumitem-nosep.html}{列表项间距过大}、
\href{https://bithesis.bitnp.net/faq/longtable.html}{三线表需要跨页}等问题,
请参考\textcolor{magenta}{\href{https://bithesis.bitnp.net/faq/}{在线文档的「疑难杂症」部分}}。

\chapter{Identifying Customer Needs}

A successful hand tool manufacturer was exploring the growing market for handheld power tools. After performing initial research, the firm decided to enter the market with a cordless screwdriver. Exhibit 5-1 shows several existing products used to drive screws. After some initial concept work, the manufacturer’s development team fabricated and field-tested several prototypes. The results were discouraging. Although some of the products were liked better than others, each one had some feature that customers objected to in one way or another. The results were quite mystifying since the company had been successful in related consumer products for years. After much discussion, the team decided that its process for identifying customer needs was inadequate.

\section{Method of study}

This chapter presents a method for comprehensively identifying a set of customer needs. The goals of the method are to:
\begin{itemize}
  \item Ensure that the product is focused on customer needs.
  \item Identify latent or hidden needs as well as explicit needs. 
  \item Provide a fact base for justifying the product specifications.
  \item Create an archival record of the needs activity of the development process.
  \item Ensure that no critical customer need is missed or forgotten. 
  \item Develop  a  common  understanding  of  customer  needs  among  members  of  the development team.
\end{itemize}
The philosophy behind the method is to create a high-quality information channel that runs directly between customers in the target market and the developers of the product. This philosophy is built on the premise that those who directly control the details of the product, including  the  engineers  and  industrial  designers,  must  interact  with customers  and experience the use environment of the product. Without this direct experience, technical trade-offs are not likely to be made correctly, innovative solutions to customer needs may never be discovered, and the development team may never develop a deep commitment to meeting customer needs. See table \ref{tab:1}

\begin{table}[htbp]
  \linespread{1.5}
  \centering
  \caption{Range of variables in design of experiment}\label{tab:1}
  \begin{tabular}{*{5}{>{\centering\arraybackslash}p{2cm}}}
    \toprule
       & \textbf{AAA}    & \textbf{BBB}    & \textbf{CCC}   & \textbf{DDD}    \\ \midrule
    Alpha    & 1000  & 10000 & 500  & 50\%  \\
    Beta   & 5500  & 5000  & 220  & 22\%  \\
    Gamma & 1100  & 1000  & 280  & 28\%  \\
    Sum    & 17600 & 16000 & 1000 & 100\% \\ \bottomrule
    \end{tabular}
\end{table}


\subsection{Process of identifying customer needs}

The process of identifying customer needs is an integral part of the larger product development process and is most closely related to concept generation, concept selection, competitive benchmarking, and the establishment of product specifications. The customer-needs  activity  is  shown  in  Exhibit  5-2  in  relation  to  these  other  front-end  product development activities, which collectively can be thought of as the concept development phase.

\subsection{Development process}

The concept development process illustrated in Exhibit 5-2 implies a distinction be-tween customer needs and product specifications. This distinction is subtle but important.

\chapter{Engineering Design}

Although engineering drawing still plays an important role in product design and manufacturing in many industrial sectors around the world, manual sketching for creating drawings has been gradually replaced by CAD (computer-aided design) software using computers. Beginning in the 1980s, CAD software reduced the need for draftsmen significantly, especially in small to mid-sized companies. The software's affordability and ability to run on personal computers in the mid-1990s allowed engineers to do their own drafting and analytic work to some extent \ref{eq:1}.

\begin{equation}
x^n + y^n = z^n
\label{eq:1}
\end{equation}

\section{注意事项}

为保证封面、参考文献等支持中文,模板尽管是英文,仍使用了 \texttt{ctex} 及 \texttt{xeCJK} 宏包。
请注意以下事项。

\subsection{输入标点符号}

% 此节删改自 lshort-zh-cn(GFDL v1.3)。
% https://github.com/CTeX-org/lshort-zh-cn/blob/e95288a697822a6f63e6f2eb1b9160a6f324c566/src/chap/chap.02.text.tex#L160-L178

\LaTeX{} 中,西文的单引号 ` 和\ ' 分别用 \verb|`| 和 \verb|'| 输入;双引号 `` 和\ '' 分别用 \verb|``| 和 \verb|''| 输入。
若用其它方式,引号具体形状和宽度可能错误。
(而中文的标点符号使用中文输入法输入即可,一般不需要过多留意。)

\begin{lstlisting}
``No, sir!'' All the pedant in Dean Hart was aroused.
``In the English language, the word `majority' means `more than half'.
Thirteen out of fourteen is a majority, nothing more.''

“他不回答,对柜里说,‘温两碗酒,要一碟茴香豆。’便排出九文大钱。”
\end{lstlisting}

``No, sir!'' All the pedant in Dean Hart was aroused. ``In the English language, the word `majority' means `more than half'. Thirteen out of fourteen is a majority, nothing more.''

“他不回答,对柜里说,‘温两碗酒,要一碟茴香豆。’便排出九文大钱。”

% 如有特殊需要,可参考 https://github.com/CTeX-org/ctex-kit/issues/389 定制。


% 后置内容
\backmatter

% 结论:在相应 TeX 文件撰写
%%
% The BIThesis Template for Undergraduate Thesis
%
% 北京理工大学毕业设计(论文) —— 使用 XeLaTeX 编译
%
% Copyright 2021-2023 BITNP
%
% This work may be distributed and/or modified under the
% conditions of the LaTeX Project Public License, either version 1.3
% of this license or (at your option) any later version.
% The latest version of this license is in
%   http://www.latex-project.org/lppl.txt
% and version 1.3 or later is part of all distributions of LaTeX
% version 2005/12/01 or later.
%
% This work has the LPPL maintenance status `maintained'.
%
% The Current Maintainer of this work is Feng Kaiyu.
%
% Compile with: xelatex -> biber -> xelatex -> xelatex
%%

\begin{conclusion}
  During the 1980s, design-for-manufacturing practices were put into place in thousands of firms. Today DFM is an essential part of almost every product development effort. No longer can designers ``throw the design over the wall'' to production engineers. As a result of this emphasis on improved design quality, some manufacturers claim to have reduced production costs of products by up to 50 percent. In fact, comparing current new product designs with earlier generations, one can usually identify fewer parts in the new product, as well as new materials, more integrated and custom parts, higher-volume standard parts and subassemblies, and simpler assembly procedures.
\end{conclusion}

% 参考文献:
% 添加文献时,请按 BibTeX 格式添加至 misc/ref.bib,并在正文所需位置使用 \cite{…} 引用。
% 如无特殊需要,无需改动相应 TeX 文件。
%%
% The BIThesis Template for Undergraduate Thesis
%
% 北京理工大学毕业设计(论文) —— 使用 XeLaTeX 编译
%
% Copyright 2021-2023 BITNP
%
% This work may be distributed and/or modified under the
% conditions of the LaTeX Project Public License, either version 1.3
% of this license or (at your option) any later version.
% The latest version of this license is in
%   https://www.latex-project.org/lppl.txt
% and version 1.3 or later is part of all distributions of LaTeX
% version 2005/12/01 or later.
%
% This work has the LPPL maintenance status `maintained'.
%
% The Current Maintainer of this work is Feng Kaiyu.
%
% Compile with: xelatex -> biber -> xelatex -> xelatex
%%

\begin{bibprint}
  \printbibliography[heading=none]
\end{bibprint}

% 附录:在相应 TeX 文件撰写;不需要时可删除
%%
% The BIThesis Template for Undergraduate Thesis
%
% 北京理工大学毕业设计(论文) —— 使用 XeLaTeX 编译
%
% Copyright 2021-2023 BITNP
%
% This work may be distributed and/or modified under the
% conditions of the LaTeX Project Public License, either version 1.3
% of this license or (at your option) any later version.
% The latest version of this license is in
%   http://www.latex-project.org/lppl.txt
% and version 1.3 or later is part of all distributions of LaTeX
% version 2005/12/01 or later.
%
% This work has the LPPL maintenance status `maintained'.
%
% The Current Maintainer of this work is Feng Kaiyu.
%
% Compile with: xelatex -> biber -> xelatex -> xelatex
%%

\begin{appendices}
  \section{Design Structure Matrix Example}
  One of the most useful applications of the design structure matrix (DSM) method is to represent well-established, but complex, engineering design processes. This rich process modeling approach facilitates:
  \begin{itemize}
    \item Understanding of the existing development process.
    \item Communication of the process to the people involved.
    \item Process improvement.
    \item Visualization of progress during the project.
  \end{itemize}

Exhibit 18-14 shows a DSM model of a critical portion of the development process at a major automobile manufacturer.

\section{Image Test}

\begin{figure}[htbp]
  \centering
  \includegraphics[width=0.45\textwidth]{example-image}
  \caption{Hello World}\label{fig:2}
\end{figure}

Some text that ref to the figure~\ref{fig:2}.

\end{appendices}

% 致谢:在相应 TeX 文件撰写
%%
% The BIThesis Template for Undergraduate Thesis
%
% 北京理工大学毕业设计(论文) —— 使用 XeLaTeX 编译
%
% Copyright 2021-2023 BITNP
%
% This work may be distributed and/or modified under the
% conditions of the LaTeX Project Public License, either version 1.3
% of this license or (at your option) any later version.
% The latest version of this license is in
%   http://www.latex-project.org/lppl.txt
% and version 1.3 or later is part of all distributions of LaTeX
% version 2005/12/01 or later.
%
% This work has the LPPL maintenance status `maintained'.
%
% The Current Maintainer of this work is Feng Kaiyu.
%
% Compile with: xelatex -> biber -> xelatex -> xelatex
%%

\begin{acknowledgements}
This book contains material developed for use
in the interdisciplinary courses on product development that we teach.
Participants in these courses include graduate students in engineering,
industrial design students, and MBA students. While we
aimed the book at interdisciplinary graduate-level audiences such
as this, many faculty teaching graduate and undergraduate courses
in engineering design have also found the material useful. Product
Design and Development is also for practicing professionals. Indeed,
we could not avoid writing  for  a  professional  audience,
because  most  of  our  students  are  themselves professionals who have worked
either in product development or in closely related functions.

This book blends the perspectives of marketing, design, and manufacturing into
a single approach to product development. As a result, we provide students
of all kinds with an appreciation for the realities of industrial practice
and for the complex and essential roles played by the various members of
product development teams. For industrial practitioners, in  particular,
we  provide  a  set  of  product  development  methods  that  can  be
put  into immediate practice on development projects.
\end{acknowledgements}


\end{document}
