%%
% The BIThesis Template for Bachelor Graduation Thesis
%
% 北京理工大学毕业设计(论文) —— 使用 XeLaTeX 编译
%
% Copyright 2021-2023 BITNP
%
% This work may be distributed and/or modified under the
% conditions of the LaTeX Project Public License, either version 1.3
% of this license or (at your option) any later version.
% The latest version of this license is in
%   https://www.latex-project.org/lppl.txt
% and version 1.3 or later is part of all distributions of LaTeX
% version 2005/12/01 or later.
%
% This work has the LPPL maintenance status `maintained'.
%
% The Current Maintainer of this work is Feng Kaiyu.
%
% Compile with: xelatex -> biber -> xelatex -> xelatex
%
% https://github.com/BITNP/BIThesis

% 摘要若要按经管学院的要求,先英文再中文,请调换以下 abstract、abstractEn 的顺序。

\begin{abstract}
  Conventional  product  development  employs  a  design-build-test  philosophy.
  The sequentially  executed  development  process  often  results  in  prolonged
  lead  times  and elevated product costs. The proposed e-Design paradigm employs
  IT-enabled technology for product design, including virtual prototyping (VP) to
  support a cross-functional team in analyzing  product  performance,  reliability,
  and  manufacturing costs  early  in  product development, and in making quantitative
  trade-offs for design decision making. Physical prototypes  of  the  product  design
  are  then  produced  using  the  rapid  prototyping  (RP) technique  and  computer
  numerical  control  (CNC)  to  support  design  verification  and functional prototyping, respectively.
\end{abstract}

\begin{abstractEn}
  Conventional  product  development  employs  a  design-build-test  philosophy.
  The sequentially  executed  development  process  often  results  in  prolonged
  lead  times  and elevated product costs. The proposed e-Design paradigm employs
  IT-enabled technology for product design, including virtual prototyping (VP) to
  support a cross-functional team in analyzing  product  performance,  reliability,
  and  manufacturing costs  early  in  product development, and in making quantitative
  trade-offs for design decision making. Physical prototypes  of  the  product  design
  are  then  produced  using  the  rapid  prototyping  (RP) technique  and  computer
  numerical  control  (CNC)  to  support  design  verification  and functional prototyping, respectively.
\end{abstractEn}
