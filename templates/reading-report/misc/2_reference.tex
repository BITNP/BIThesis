%%
% BIThesis 读书报告模板 —— 使用 XeLaTeX 编译 The BIThesis Template for Reading Report
% This file has no copyright assigned and is placed in the Public Domain.
%%

\begin{bibprint}

% \printbibliography[heading=none]
% 正式使用时,请启用上方语句以输出所有的参考文献,并删除/注释下方示例内容。
% 删除后仍可参考 https://bithesis.bitnp.net/faq/bib-entry.html 的在线版本。

% -------------------------------- 示例内容(正式使用时请删除) ------------------------------------- %

% 抑制多次调用 \printbibliography 的 warning,只有示例代码会需要此语句。
\BiblatexSplitbibDefernumbersWarningOff

\textcolor{blue}{参考文献书写规范}

\textcolor{blue}{参考国家标准《信息与文献参考文献著录规则》【GB/T 7714—2015】,参考文献书写规范如下:}

\textcolor{blue}{\textbf{1. 文献类型和标识代码}}

\textcolor{blue}{普通图书:M}\qquad\textcolor{blue}{会议录:C}\qquad\textcolor{blue}{汇编:G}\qquad\textcolor{blue}{报纸:N}

\textcolor{blue}{期刊:J}\qquad\textcolor{blue}{学位论文:D}\qquad\textcolor{blue}{报告:R}\qquad\textcolor{blue}{标准:S}

\textcolor{blue}{专利:P}\qquad\textcolor{blue}{数据库:DB}\qquad\textcolor{blue}{计算机程序:CP}\qquad\textcolor{blue}{电子公告:EB}

\textcolor{blue}{档案:A}\qquad\textcolor{blue}{舆图:CM}\qquad\textcolor{blue}{数据集:DS}\qquad\textcolor{blue}{其他:Z}

\textcolor{blue}{\textbf{2. 不同类别文献书写规范要求}}

\textcolor{blue}{\textbf{期刊}}

\noindent\textcolor{blue}{[序号] 主要责任者. 文献题名[J]. 刊名, 出版年份, 卷号(期号): 起止页码. }
\cite{yuFeiJiZongTiDuoXueKeSheJiYouHuaDeXianZhuangYuFaZhanFangXiang2008, Hajela2012Application}

\printbibliography [type=article,heading=none]

\textcolor{blue}{\textbf{普通图书}}

\noindent\textcolor{blue}{[序号] 主要责任者. 文献题名[M]. 出版地: 出版者, 出版年: 起止页码. }
\cite{张伯伟2002全唐五代诗格会考, OBRIEN1994Aircraft}

\printbibliography [keyword={book},heading=none]

\textcolor{blue}{\textbf{会议论文集}}

\noindent\textcolor{blue}{[序号] 主要责任者.题名:其他题名信息[C]. 出版地: 出版者, 出版年. }
\cite{雷光春2012}

\printbibliography [type=proceedings,heading=none]

\textcolor{blue}{\textbf{专著中析出的文献}}

\noindent\textcolor{blue}{[序号] 析出文献主要责任者. 析出题名[M]//专著主要责任者. 专著题名. 出版地: 出版者, 出版年: 起止页码. }
\cite{白书农}

\printbibliography [type=inbook,heading=none]

\textcolor{blue}{\textbf{学位论文}}

\noindent\textcolor{blue}{[序号] 主要责任者. 文献题名[D]. 保存地: 保存单位, 年份. }
\cite{zhanghesheng, Sobieski}

\printbibliography [keyword={thesis},heading=none]

\textcolor{blue}{\textbf{报告}}

\noindent\textcolor{blue}{[序号] 主要责任者. 文献题名[R]. 报告地: 报告会主办单位, 年份. }
\cite{fengxiqiao, Sobieszczanski}

\printbibliography [keyword={techreport},heading=none]

\textcolor{blue}{\textbf{专利文献}}

\noindent\textcolor{blue}{[序号] 专利所有者. 专利题名:专利号[P]. 公告日期或公开日期[引用日期]. 获取和访问路径. 数字对象唯一标识符.}
\cite{jiangxizhou}

\printbibliography [type=patent,heading=none]

\textcolor{blue}{\textbf{国际、国家标准}}

\noindent\textcolor{blue}{[序号] 主要责任人. 题名: 其他题名信息[S]. 出版地: 出版者, 出版年: 引文页码.}
\cite{GB/T3792.4-2009}

\printbibliography [keyword={standard},heading=none]

\textcolor{blue}{\textbf{报纸文章}}

\noindent\textcolor{blue}{[序号] 主要责任者. 文献题名[N]. 报纸名, 年(期): 页码. }
\cite{xiexide}

\printbibliography [keyword={newspaper},heading=none]

\textcolor{blue}{\textbf{电子文献}}

\noindent\textcolor{blue}{[序号] 主要责任者. 电子文献题名[文献类型/载体类型]. (发表或更新日期) [引用日期]. 获取和访问路径. 数字对象唯一标识符. }
\cite{yaoboyuan}

\printbibliography [keyword={online},heading=none]

\textcolor{blue}{关于参考文献的未尽事项可参考国家标准《信息与文献参考文献著录规则》(GB/T 7714—2015)}

\end{bibprint}
