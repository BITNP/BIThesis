%%
% BIThesis 读书报告模板 —— 使用 XeLaTeX 编译 The BIThesis Template for Reading Report
% This file has no copyright assigned and is placed in the Public Domain.
% Compile with: xelatex -> biber -> xelatex -> xelatex
%%

% 请勿删除下面两行注释,以免影响编译。
% !TeX program = xelatex
% !BIB program = biber


\documentclass[type=bachelor]{bithesis}

% 此处仅列出常用的配置。全部配置用法请见「bithesis.pdf」手册。
\BITSetup{
  cover = {
    % 封面需要「北京理工大学」字样图片,如无必要请勿修改该项。
    headerImage = images/header.png,
    % 封面标题需要“华文细黑”,如无必要请勿修改该项。
    xiheiFont = STXIHEI.TTF,
    % 可用以下参数自定义封面日期。
    % date = 1955年11月,
  },
  info = {
    % 想要删除某项封面信息,直接删除该项即可。
    % 想要让某项封面信息留空(但是保留下划线),请传入空白符组成的字符串,如"{~}"。
    % 如需要换行,则用 “\\” 符号分割。
    title = 北京理工大学本科生读书报告题目,
    semester = {1840-1841学年第一学期},
    school = 信息与电子学院,
    author = 罗辑,
    studentId = 11xxxxxxxx,
    course = 宇宙社会学,
    teacher = 叶文洁,
    % 以下用于隐藏字段
    titleEn = {},
  },
  style = {
    head = 北京理工大学本科生读书报告,
    headline = 本科生读书报告,
    % 开启 Windows 平台下的中易宋体伪粗体。
    % windowsSimSunFakeBold = true,
    %
    % 开启该选项后,将用 Times New Roman 的开源字体 TeX Gyre Termes 作为正文字体。
    % 这个选项适用于以下情况:
    % 1. 不想在系统中安装 Times New Roman。
    % 2. 在 Linux/macOS 下遇到 `\textsc` 无法正常显示的问题。
    % betterTimesNewRoman = true,
  },
  misc = {
    % 微调表格行间距
    tabularRowSeparation = 1.25,
  },
}

% 这份模板提供了代码块示例,采用 listings 宏包及其预定义样式。
% 使用代码块时,也可选用自己的喜欢的其他宏包,如 minted:https://bithesis.bitnp.net/faq/minted.html
% 如果不需要,直接删除即可。
\usepackage{listings}


% 大部分关于参考文献样式的修改,都可以通过此处的选项进行配置。
% 详情请搜索「biblatex-gb7714-2015 文档」进行阅读。
\usepackage[
  backend=biber,
  style=gb7714-2015,
  gbalign=gb7714-2015,
  gbnamefmt=lowercase,
  gbpub=false,
  doi=false,
  url=false,
  eprint=false,
  isbn=false,
]{biblatex}

% 参考文献引用文件位于 misc/ref.bib
\addbibresource{misc/ref.bib}

% 文档开始
\begin{document}

% 标题页面:如无特殊需要,本部分无需改动
\MakeCover

% 前置内容定义
\frontmatter

\MakeTOC

% 正文开始
\mainmatter

% 在这里引用各章 TeX 文件,按需添加
%%
% The BIThesis Template for Bachelor Graduation Thesis
%
% 北京理工大学毕业设计(论文)第一章节 —— 使用 XeLaTeX 编译
%
% Copyright 2020-2023 BITNP
%
% This work may be distributed and/or modified under the
% conditions of the LaTeX Project Public License, either version 1.3
% of this license or (at your option) any later version.
% The latest version of this license is in
%   https://www.latex-project.org/lppl.txt
% and version 1.3 or later is part of all distributions of LaTeX
% version 2005/12/01 or later.
%
% This work has the LPPL maintenance status `maintained'.
%
% The Current Maintainer of this work is Feng Kaiyu.
%
% 第一章节

\chapter{一级题目}

\section{二级题目}
% 这里插入一个参考文献,仅作参考

\subsection{三级题目}

正文……\parencite{yuFeiJiZongTiDuoXueKeSheJiYouHuaDeXianZhuangYuFaZhanFangXiang2008}……\cite{Hajela2012Application}

\textcolor{blue}{正文部分:宋体、小四;正文行距:22磅;间距段前段后均为0行。阅后删除此段。}

\textcolor{blue}{图、表居中,图注标在图下方,表头标在表上方,宋体、五号、居中,1.25倍行距,间距段前段后均为0行,图表与上下文之间各空一行。阅后删除此段。}

\textcolor{blue}{\underline{\underline{图-示例:(阅后删除此段)}}}


\begin{figure}[htbp]
  \centering
  \includegraphics[]{images/bit_logo.png}
  \caption{标题序号}\label{标题序号} % label 用来在文中索引
\end{figure}

\textcolor{blue}{\underline{\underline{表-示例:(阅后删除此段)}}}
% 三线表
\begin{table}[htbp]
  \centering
  \caption{统计表}\label{统计表}
  \begin{tabular}{*{5}{>{\centering\arraybackslash}p{2cm}}} \toprule
    项目    & 产量    & 销量    & 产值   & 比重    \\ \midrule
    手机    & 1000  & 10000 & 500  & 50\%  \\
    计算机   & 5500  & 5000  & 220  & 22\%  \\
    笔记本电脑 & 1100  & 1000  & 280  & 28\%  \\ \midrule
    合计    & 17600 & 16000 & 1000 & 100\% \\ \bottomrule
    \end{tabular}
\end{table}

\textcolor{blue}{公式标注应于该公式所在行的最右侧。对于较长的公式只可在符号处(+、-、*、/、$\leqslant$ $\geqslant$ 等)转行。在文中引用公式时,在标号前加“式”,如式(1-2)。阅后删除此
段。}

\textcolor{blue}{公式-示例:(阅后删除此段)}
% 公式上下不要空行,置于同一个段落下即可,否则上下距离会出现高度不一致的问题
\begin{equation}
    LRI=1\ ∕\ \sqrt{1+{\left(\frac{{\mu }_{R}}{{\mu }_{s}}\right)}^{2}{\left(\frac{{\delta }_{R}}{{\delta }_{s}}\right)}^{2}}
\end{equation}


\section{字体效果表格}

% 列:Regular、Italic、Bold、Bold Italic
% 行:宋体、黑体、楷体、Serif、Sans Serif、Typewriter、Math

\begin{table}[htb]
    \centering
    \caption{字体效果表格}
    \begin{tabular}{@{}lllll@{}}
    \toprule
               & Regular & Bold & Italic & Bold Italic \\ \midrule
      宋体       & 宋体      & \colorbox{orange}{\textbf{宋体粗体}} & \textit{楷体}     &   \colorbox{gray}{\textbf{\textit{楷书粗斜体}}}  \\
      黑体         & {\heiti{}黑体}      & \textbf{\heiti{}黑体粗体} &   \textit{\heiti{}黑体斜体}     & \colorbox{gray}{\textit{\textbf{\heiti{}黑体粗斜体}}}   \\
      楷体         & {\kaishu{}楷书}      & \textbf{\kaishu{}楷书粗体} & \textit{\kaishu{}斜体楷体} &  \colorbox{gray}{\textbf{\textit{\kaishu{}楷书粗斜体}}}    \\
    Serif(Roman/Normal)      &    Regular    &  \textbf{Bold}  &    \textit{Italic}    &     \textbf{\textit{Bold Italic}}    \\
    Sans Serif &  \textsf{Regular}       &  \textbf{\textsf{Bold}}    &  \textit{\textsf{Bold}}   &    \textbf{\textit{\textsf{Bold}}}   \\
    % 有些字体缺少特定变体,LaTeX会抛出警告,同时尽量替换为相近变体。
    % 若编译结果能接受,可忽略之。
    % 例如此处Typewriter代表的Latin Modern Sans Typewriter(lmtt)字体没有bold extended italic(bx/it)变体,
    % 所以`\textbf{\textit{\texttt{…}}}`会被替换为bold slant(b/sl)变体,并引发如下警告。(其中TU指字体编码)
    % Font shape `TU/lmtt/bx/it' in size <…> not available. Font shape `TU/lmtt/b/sl' tried instead.
    Typewriter &  \texttt{Regular}       &  \textbf{\texttt{Bold}}    &  \textit{\texttt{Bold}}   &    \textbf{\textit{\texttt{Bold}}}   \\
    Math       &   $\mathnormal{Regular} \mathrm{Roman}$  & $\mathbf{Bold}$   &    $\mathit{Italic}$    &  $\mathbf{\mathit{Bold Italic}}$    \\ \bottomrule
    \end{tabular}
\end{table}

\begin{itemize}[nosep]
  \item \colorbox{orange}{宋体粗体}在 Windows 下会成为黑体。这是因为 Windows 的中易宋体没有粗体字重而进行的妥协。
    如果想要获得宋体粗体的样式,请在配置中开启伪粗体选项。
  \item \colorbox{gray}{粗斜体}的效果是因操作系统字体而定的,中文写作中不会使用这种字形,可以忽略。
\end{itemize}

\textit{有关公式与上下文间距的一些注意事项:请保证源码中的公式的环境(如}
\\ \verb|\begin{equation}|
  \textit{)与上一段落不要有空行。否则,公式和上文段落之间会有额外的空白。}


\section{常见问题和疑难解答}

如果您遇到\href{https://bithesis.bitnp.net/faq/char-missing.html}{生僻字无法显示}、
\href{https://bithesis.bitnp.net/faq/enumitem-nosep.html}{列表项间距过大}、
\href{https://bithesis.bitnp.net/faq/longtable.html}{三线表需要跨页}等问题,
请参考\textcolor{magenta}{\href{https://bithesis.bitnp.net/faq/}{在线文档的「疑难杂症」部分}}。

\chapter{Identifying Customer Needs}

A successful hand tool manufacturer was exploring the growing market for handheld power tools. After performing initial research, the firm decided to enter the market with a cordless screwdriver. Exhibit 5-1 shows several existing products used to drive screws. After some initial concept work, the manufacturer’s development team fabricated and field-tested several prototypes. The results were discouraging. Although some of the products were liked better than others, each one had some feature that customers objected to in one way or another. The results were quite mystifying since the company had been successful in related consumer products for years. After much discussion, the team decided that its process for identifying customer needs was inadequate.

\section{Method of study}

This chapter presents a method for comprehensively identifying a set of customer needs. The goals of the method are to:
\begin{itemize}
  \item Ensure that the product is focused on customer needs.
  \item Identify latent or hidden needs as well as explicit needs. 
  \item Provide a fact base for justifying the product specifications.
  \item Create an archival record of the needs activity of the development process.
  \item Ensure that no critical customer need is missed or forgotten. 
  \item Develop  a  common  understanding  of  customer  needs  among  members  of  the development team.
\end{itemize}
The philosophy behind the method is to create a high-quality information channel that runs directly between customers in the target market and the developers of the product. This philosophy is built on the premise that those who directly control the details of the product, including  the  engineers  and  industrial  designers,  must  interact  with customers  and experience the use environment of the product. Without this direct experience, technical trade-offs are not likely to be made correctly, innovative solutions to customer needs may never be discovered, and the development team may never develop a deep commitment to meeting customer needs. See table \ref{tab:1}

\begin{table}[htbp]
  \linespread{1.5}
  \centering
  \caption{Range of variables in design of experiment}\label{tab:1}
  \begin{tabular}{*{5}{>{\centering\arraybackslash}p{2cm}}}
    \toprule
       & \textbf{AAA}    & \textbf{BBB}    & \textbf{CCC}   & \textbf{DDD}    \\ \midrule
    Alpha    & 1000  & 10000 & 500  & 50\%  \\
    Beta   & 5500  & 5000  & 220  & 22\%  \\
    Gamma & 1100  & 1000  & 280  & 28\%  \\
    Sum    & 17600 & 16000 & 1000 & 100\% \\ \bottomrule
    \end{tabular}
\end{table}


\subsection{Process of identifying customer needs}

The process of identifying customer needs is an integral part of the larger product development process and is most closely related to concept generation, concept selection, competitive benchmarking, and the establishment of product specifications. The customer-needs  activity  is  shown  in  Exhibit  5-2  in  relation  to  these  other  front-end  product development activities, which collectively can be thought of as the concept development phase.

\subsection{Development process}

The concept development process illustrated in Exhibit 5-2 implies a distinction be-tween customer needs and product specifications. This distinction is subtle but important.


% 后置内容
\backmatter

% 结论:在相应 TeX 文件撰写
%%
% The BIThesis Template for Reading Report
%
% 北京理工大学读书报告结论 —— 使用 XeLaTeX 编译
%
% Copyright 2020-2023 BITNP
%
% This work may be distributed and/or modified under the
% conditions of the LaTeX Project Public License, either version 1.3
% of this license or (at your option) any later version.
% The latest version of this license is in
%   http://www.latex-project.org/lppl.txt
% and version 1.3 or later is part of all distributions of LaTeX
% version 2005/12/01 or later.
%
% This work has the LPPL maintenance status `maintained'.
%
% The Current Maintainer of this work is Feng Kaiyu.
%
% Compile with: xelatex -> biber -> xelatex -> xelatex

\begin{conclusion}
  % 结论部分尽量不使用 \subsection 二级标题,只使用 \section 一级标题

  % 这里插入一个参考文献,仅作参考
  本文结论……\cite{张伯伟2002全唐五代诗格会考}。

  \textcolor{blue}{结论作为正文的最后部分单独排写,但不加章号。阅后删除此段。}

  \textcolor{blue}{结论正文样式与文章正文相同:宋体、小四;行距:22磅;间距段前段后均为0行。阅后删除此段。}
\end{conclusion}

% 参考文献:
% 添加文献时,请按 BibTeX 格式添加至 misc/ref.bib,并在正文所需位置使用 \cite{…} 引用。
% 如无特殊需要,无需改动相应 TeX 文件。
%%
% BIThesis 读书报告模板 —— 使用 XeLaTeX 编译 The BIThesis Template for Reading Report
% This file has no copyright assigned and is placed in the Public Domain.
%%

\begin{bibprint}

% \printbibliography[heading=none]
% 正式使用时,请启用上方语句以输出所有的参考文献,并删除/注释下方示例内容。
% 删除后仍可参考 https://bithesis.bitnp.net/faq/bib-entry.html 的在线版本。

% -------------------------------- 示例内容(正式使用时请删除) ------------------------------------- %

% 抑制多次调用 \printbibliography 的 warning,只有示例代码会需要此语句。
\BiblatexSplitbibDefernumbersWarningOff

\textcolor{blue}{参考文献书写规范}

\textcolor{blue}{参考国家标准《信息与文献参考文献著录规则》【GB/T 7714—2015】,参考文献书写规范如下:}

\textcolor{blue}{\textbf{1. 文献类型和标识代码}}

\textcolor{blue}{普通图书:M}\qquad\textcolor{blue}{会议录:C}\qquad\textcolor{blue}{汇编:G}\qquad\textcolor{blue}{报纸:N}

\textcolor{blue}{期刊:J}\qquad\textcolor{blue}{学位论文:D}\qquad\textcolor{blue}{报告:R}\qquad\textcolor{blue}{标准:S}

\textcolor{blue}{专利:P}\qquad\textcolor{blue}{数据库:DB}\qquad\textcolor{blue}{计算机程序:CP}\qquad\textcolor{blue}{电子公告:EB}

\textcolor{blue}{档案:A}\qquad\textcolor{blue}{舆图:CM}\qquad\textcolor{blue}{数据集:DS}\qquad\textcolor{blue}{其他:Z}

\textcolor{blue}{\textbf{2. 不同类别文献书写规范要求}}

\textcolor{blue}{\textbf{期刊}}

\noindent\textcolor{blue}{[序号] 主要责任者. 文献题名[J]. 刊名, 出版年份, 卷号(期号): 起止页码. }
\cite{yuFeiJiZongTiDuoXueKeSheJiYouHuaDeXianZhuangYuFaZhanFangXiang2008, Hajela2012Application}

\printbibliography [type=article,heading=none]

\textcolor{blue}{\textbf{普通图书}}

\noindent\textcolor{blue}{[序号] 主要责任者. 文献题名[M]. 出版地: 出版者, 出版年: 起止页码. }
\cite{张伯伟2002全唐五代诗格会考, OBRIEN1994Aircraft}

\printbibliography [keyword={book},heading=none]

\textcolor{blue}{\textbf{会议论文集}}

\noindent\textcolor{blue}{[序号] 主要责任者.题名:其他题名信息[C]. 出版地: 出版者, 出版年. }
\cite{雷光春2012}

\printbibliography [type=proceedings,heading=none]

\textcolor{blue}{\textbf{专著中析出的文献}}

\noindent\textcolor{blue}{[序号] 析出文献主要责任者. 析出题名[M]//专著主要责任者. 专著题名. 出版地: 出版者, 出版年: 起止页码. }
\cite{白书农}

\printbibliography [type=inbook,heading=none]

\textcolor{blue}{\textbf{学位论文}}

\noindent\textcolor{blue}{[序号] 主要责任者. 文献题名[D]. 保存地: 保存单位, 年份. }
\cite{zhanghesheng, Sobieski}

\printbibliography [keyword={thesis},heading=none]

\textcolor{blue}{\textbf{报告}}

\noindent\textcolor{blue}{[序号] 主要责任者. 文献题名[R]. 报告地: 报告会主办单位, 年份. }
\cite{fengxiqiao, Sobieszczanski}

\printbibliography [keyword={techreport},heading=none]

\textcolor{blue}{\textbf{专利文献}}

\noindent\textcolor{blue}{[序号] 专利所有者. 专利题名:专利号[P]. 公告日期或公开日期[引用日期]. 获取和访问路径. 数字对象唯一标识符.}
\cite{jiangxizhou}

\printbibliography [type=patent,heading=none]

\textcolor{blue}{\textbf{国际、国家标准}}

\noindent\textcolor{blue}{[序号] 主要责任人. 题名: 其他题名信息[S]. 出版地: 出版者, 出版年: 引文页码.}
\cite{GB/T3792.4-2009}

\printbibliography [keyword={standard},heading=none]

\textcolor{blue}{\textbf{报纸文章}}

\noindent\textcolor{blue}{[序号] 主要责任者. 文献题名[N]. 报纸名, 年(期): 页码. }
\cite{xiexide}

\printbibliography [keyword={newspaper},heading=none]

\textcolor{blue}{\textbf{电子文献}}

\noindent\textcolor{blue}{[序号] 主要责任者. 电子文献题名[文献类型/载体类型]. (发表或更新日期) [引用日期]. 获取和访问路径. 数字对象唯一标识符. }
\cite{yaoboyuan}

\printbibliography [keyword={online},heading=none]

\textcolor{blue}{关于参考文献的未尽事项可参考国家标准《信息与文献参考文献著录规则》(GB/T 7714—2015)}

\end{bibprint}


\end{document}
