%%
% BIThesis 研究生学位论文模板 The BIThesis Template for Graduate Thesis
% This file has no copyright assigned and is placed in the Public Domain.

% 1. 在 `./reference/pub.bib` 中添加数据。
% 2. 在下方 `\addpubs` 添加该文献(参考下方示例)。

% **注意:如果发现渲染出来的文献编号不正确,请同时使用以下两个方式解决:**
% 1. 请清除缓存后重新编译(比如使用 `latexmk -c`)。
% 2. 请确保无编译错误。

\begin{publications}

  % **默认情况下,这里的内容将按照学校要求,以发表时间排序。**
  % - 如果想要按照引用顺序排序,可以开启 `publications/sorting` 选项。
  % - 如果想要微调,详见 https://bithesis.bitnp.net/faq/bib-sort.html#sortkey 。
  % 更多信息请参考「bithesis.pdf」手册。
  \addpubs{myCiteKey,myCiteKey2,dummy:1,dummy:2}

  % 主要针对硕士生
  \printbibliography[heading=none,category=mypub,resetnumbers=true]

  % 如果想要分为多个列表,可以使用以下的命令。
  % 主要针对博士生。
  % \pubsection{文章}
  % \printbibliography[heading=none,type=article,category=mypub,resetnumbers=true]{}
  %
  % \pubsection{一些书}
  % \printbibliography[heading=none,type=book,category=mypub,resetnumbers=true,notkeyword=dummy]{}
  %
  % \pubsection{另一些书}
  % \printbibliography[heading=none,type=book,category=mypub,keyword=dummy,resetnumbers=true]{}
  %
  % 关于 \printbibliography 的筛选参数:
  % 0. 请保留“category=mypub”。(这样只列出成果,不列出正文参考文献。)
  % 1. 设置“type=…”,每次只输出某一类型。
  % 2. 若需继续细分,请在 pub.bib 的条目里记录“keywords = {…, …}”,然后在此用“keyword=…”筛选。
  % 3. 如果还有要求,可用notkeyword、subtype等筛选方法,请参考 biblatex 手册。

  % 如果想绕过 pub.bib 直接记录项目(例如获奖),请参考以下内容,
  % 定义一个能和 \printbibliography 共存的列表。
  % https://bithesis.bitnp.net/faq/pub-manual.html
  % \zihao{5} % 字号改为五号
  % \renewcommand{\labelenumi}{[\theenumi]} % 编号改用中括号
  %
  % \begin{enumerate}[nosep, leftmargin=4ex-2pt, labelsep=1ex]
  %   \setcounter{enumi}{4} % 下一项为 5。
  %   \item 于《新青年》发表论文一篇,本人第一作者。
  %   \item 于\textit{La Jeunesse}发表论文一篇,导师第一作者,本人第二作者。
  % \end{enumerate}
\end{publications}
