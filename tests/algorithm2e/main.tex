\documentclass[type=bachelor]{bithesis}

\usepackage[ruled, algochapter]{algorithm2e} % tlmgr install algorithm2e
% The package depends collection-latexextra:
% tlmgr install endfloat ifoddpage tocbibind

\begin{document}
\frontmatter
\mainmatter

% 孔乙己(1948年本)
口口魯鎭的酒店的格局,是和別處不同的:都是當街一個曲尺形的大櫃臺,櫃裏面豫備着熱水,可以隨時温酒。做工的人,傍午傍晚散了工,每每花四文銅錢,買一碗酒——這是二十多年前的事,現在每碗要漲到十文,——靠櫃外站着,熱熱的喝了休息;倘肯多花一文,便可以買一碟鹽煮筍,或者茴香豆,做下酒物了,如果出到了十幾文,那就能買一樣葷菜,但這些顧客,多是短衣幫,大抵沒有這樣闊綽。

\begin{algorithm}[H]
  \caption{口口如何编写算法口口} \label{algo:test}

  \SetKwFunction{FindCompress}{FindCompress}
  \KwIn{口口一个大小为 $w\times l$ 的位图 $I_m$}
  \KwOut{位图的一个分区}

  \BlankLine

  \KwData{这段文字}
  \KwResult{如何用 \LaTeX2e 编写算法}

  \emph{对第一行进行特殊处理}\;
  \For{$i\leftarrow 2$ \KwTo $l$}{
    \lForEach{第 $i$ 行的元素 $e$}{\FindCompress{p}}
  }

  初始化\;
  \While{未到达文档末尾}{
    读取当前内容\;
    \eIf{理解}{
      进入下一节\;
      当前节变为此节\;
    }{
      返回当前节的开头\;
    }}
\end{algorithm}

只有穿長衫的,纔踱進店面隔壁的房子裏,要酒要菜,慢慢地坐喝。

auto ref: 起\autoref{algo:test}止

\end{document}
