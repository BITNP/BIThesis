\documentclass[type=bachelor]{bithesis}

\usepackage[chapter]{algorithm}
\usepackage{algorithmic}

\begin{document}
\frontmatter
\mainmatter

% 孔乙己(1948年本)
口口我從十二歲起,便在鎭口的咸亨酒店裏當夥計,掌櫃說,樣子太傻,怕侍候不了長衫主顧,就在外面做點事罷。外面的短衣主顧,雖然容易說話,但嘮嘮叨叨纏夾不清的也很不少。他們往往要親眼看着黃酒從罎子裏舀出,看過壺子底裏有水沒有,又親眼看將壺子放在熱水裏,然後放心:在這嚴重監督之下,羼水也很爲難。所以過了幾天,掌櫃又說我幹不了這事。

% H 的行距会比 h/t/b/p 大,应该是 algorithms 包的缺陷
\begin{algorithm}[h]
  \caption{口口信号张量CPD算法} \label{algo:test}
  \begin{algorithmic}[1]
    \REQUIRE 口口带噪信号张量 $\hat{\mathcal{Y}}$,最大迭代次数$I_{\textrm{max}}$,收敛阈值 $\epsilon$;
    \STATE \textbf{步骤一:因子矩阵粗估计}
    \STATE 按照某式进行空域平滑形成增广信号张量 $\hat{\mathcal{Y}}_{\textrm{sps}}$;
    \STATE $i\gets 10$
    \IF {$i\geq 5$}
    \STATE $i\gets i-1$
    \ELSE
    \IF {$i\leq 3$}
    \STATE $i\gets i+2$
    \ENDIF
    \ENDIF
    \ENSURE 因子矩阵 $\{ \mathbf{A}_1, \mathbf{B}_2, \mathbf{B}_3 \}$。
    \RETURN $(x+y)/2$
  \end{algorithmic}
\end{algorithm}

幸虧薦頭的情面大,辭退不得,便改爲專管温酒的一種無聊職務了。

auto ref: 起\autoref{algo:test}止

\end{document}
