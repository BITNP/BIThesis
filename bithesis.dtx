% \iffalse meta-comment
%
% Copyright (C) 2022
% Association of Bit Network Pioneer and any individual authors listed elsewhere in this file.
% -----------------------------------
%
% This file may be distributed and/or modified under the
% conditions of the LaTeX Project Public License, either version 1.3
% of this license or (at your option) any later version.
% The latest version of this license is in:
%
%     http://www.latex-project.org/lppl.txt
%
% and version 1.3 or later is part of all distributions of LaTeX
% version 2020/11/27 or later.
%
% \fi
%
% \iffalse
%<cls>\NeedsTeXFormat{LaTeX2e}[1999/12/01]
%<book>\ProvidesClass{bitbook}
%<article>\ProvidesClass{bitart}
%<graduate>\ProvidesClass{bitgrad}
%<cls> [2022/04/01 v2.0.0 BIT Thesis Templates]
%
%<*driver>
\ProvidesFile{bithesis.dtx}[2022/04/01 2.0.0 BIT Thesis Templates]
\documentclass{ltxdoc}
\usepackage{dtx-style}

\EnableCrossrefs
\CodelineIndex

\RecordChanges
\begin{document}
  \DocInput{\jobname.dtx}
\end{document}
%</driver>
% \fi
%
% \DoNotIndex{\newenvironment,\@bsphack,\@empty,\@esphack,\sfcode}
% \DoNotIndex{\addtocounter,\label,\let,\linewidth,\newcounter}
% \DoNotIndex{\noindent,\normalfont,\par,\parskip,\phantomsection}
% \DoNotIndex{\providecommand,\ProvidesPackage,\refstepcounter}
% \DoNotIndex{\RequirePackage,\setcounter,\setlength,\string,\strut}
% \DoNotIndex{\textbackslash,\texttt,\ttfamily,\usepackage}
% \DoNotIndex{\begin,\end,\begingroup,\endgroup,\par,\\}
% \DoNotIndex{\if,\ifx,\ifdim,\ifnum,\ifcase,\else,\or,\fi}
% \DoNotIndex{\let,\def,\xdef,\edef,\newcommand,\renewcommand}
% \DoNotIndex{\expandafter,\csname,\endcsname,\relax,\protect}
% \DoNotIndex{\Huge,\huge,\LARGE,\Large,\large,\normalsize}
% \DoNotIndex{\small,\footnotesize,\scriptsize,\tiny}
% \DoNotIndex{\normalfont,\bfseries,\slshape,\sffamily,\interlinepenalty}
% \DoNotIndex{\textbf,\textit,\textsf,\textsc}
% \DoNotIndex{\hfil,\par,\hskip,\vskip,\vspace,\quad}
% \DoNotIndex{\centering,\raggedright,\ref}
% \DoNotIndex{\c@secnumdepth,\@startsection,\@setfontsize}
% \DoNotIndex{\ ,\@plus,\@minus,\p@,\z@,\@m,\@M,\@ne,\m@ne}
% \DoNotIndex{\@@par,\DeclareOperation,\RequirePackage,\LoadClass}
% \DoNotIndex{\AtBeginDocument,\AtEndDocument}
%
% \GetFileInfo{\jobname.dtx} %
% 
% \def\indexname{索引}
% \IndexPrologue{\section{\indexname}}
%
% \title{\includegraphics[width=0.3\textwidth]{images/icon.png}
% \\[1cm]
% \bfseries 北京理工大学本科生{\LaTeX}学位论文及报告模板 }
% \author{北京理工大学网络开拓者协会 \\ \texttt{webmaster@bitnp.net}} %
% \date{\zihao{-4} \today\quad \color{RubineRed}{\kaishu {\BIThesis}版本\version}}
% \maketitle\thispagestyle{empty}
%
% \def\abstractname{}
% \begin{abstract}\noindent
%   此宏包旨在建立一个简单易用的北京理工大学学位论文模板,包括本科综合论文训练、硕士
%   论文、博士论文以及博士后出站报告。
% \end{abstract}
%
% \vspace{5mm}
%
% \begin{center}
% \noindent\rule[0.25\baselineskip]{0.5\textwidth}{0.7pt}
% \end{center}
% 
% \def\abstractname{免责声明}
% \begin{abstract}
% \noindent
% \begin{enumerate}
% \item 本模板的发布遵守 \LPPL ,使用前请认真阅读协议内容。
% \item 任何个人或组织以本模板为基础进行修改、扩展而生成的新的专用模板,请严格遵
%   守 \LaTeX{} Project Public License 协议。由于违犯协议而引起的任何纠纷争端均与
%   本模板作者无关。
% \end{enumerate}
% \end{abstract}
%
% \vspace{5mm}
%
% \def\abstractname{简介}
% \begin{abstract}
% \BIThesisLaTeX 是北京理工大学本科生毕业设计开题报告、总论文,以及其他课程报告、实验报告等重要论文、报告的 {\LaTeX} 模板集合。
% 如果你厌烦了 Word 格式的不人性化、参考文献的难以管理、公式输入的差劲体验……那么欢迎来尝试用专业的学术稿件排版利器 —— {\LaTeX},来排版你的论文。
% 专业高端、学界认可、开源免费,{\LaTeX} 是你论文排版的最佳搭档。
%
% \BIThesisLaTeX 目前支持使用 {\hologo{XeLaTeX}} 进行编译,使用以 biber 为后端的 BibLaTeX 进行参考文献的生成,
% 符合《信息与文献参考文献著录规则》
% (\href{http://openstd.samr.gov.cn/bzgk/gb/newGbInfo?hcno=7FA63E9BBA56E60471AEDAEBDE44B14C}{GB/T 7714—2015})的标准。
% 目前主要设计完成了计算机学院本科生毕业论文开题报告、毕业设计毕业论文与通用实验报告的 {\LaTeX} 模板。
%
% \end{abstract}
% \newpage
%
% \tableofcontents
% \clearpage
% \setlength{\parskip}{0.8ex}
%
% \section{项目简介}
% \subsection{历史与贡献者们}
% \begin{itemize}
%   \item 2019 - 2020 年,\BIThesis 最早由 2016 级的武上博、王赞、唐誉铭、牟思睿和詹熠莎等人维护。
%   \begin{itemize}
%     \item 在此期间,\BIThesis 从无到有诞生了,包括使用手册、在线文档和开箱即用的模板。
%     \item 同时,2017 级的赵池等同学完成了一系列 \BIThesisLaTeX 的视频教程。
%   \end{itemize}
%   \item 2020 - 2021 年,2017 级的冯开宇、杨思云、郝正亮和顾骁等人接管了维护开发工作。
%   \begin{itemize}
%     \item 在此期间,冯开宇将原来的 .tex 文件制作成了宏包,并发布到 CTAN 上。
%     \item 项目代码也随之被拆分成了 \BIThesisMacroPackage,\BIThesisWiki 和 \BIThesisScaffold。
%   \end{itemize}
%   \item 2021 - 2022 年,2021 级(硕士研究生)的冯开宇针对 2021、2022 毕业季收到的反馈对该项目进行维护升级(前向兼容)。
%   \begin{itemize}
%      \item 在此期间,冯开宇合入了杨雅婷等人在 2017 年开发的研究生学位论文模板。
%      \item 在项目架构上,BIThesis-scaffold 合入 BIThesis 以便于进一步维护。 
%   \end{itemize}
% \end{itemize}
% \subsection{\BIThesis 是什么?}
% \BIThesis 之名是英文单词 Beijing Institution of Technology(北京理工大学)的首字母缩写“BIT” 与“Thesis”结合而成。在纯文本环境下,该名字应写作“BIThesis”。
%
% \BIThesisLaTeX 是由北京理工大学众多学子发起并维护的开源项目。该项目旨在建立一套简单易用的北京理工大学 \LaTeX 学位论文模板,包括本科综合论文训练。
% \subsubsection{\BIThesisLaTeX 的组成}
% 我们将 \BIThesisLaTeX 划分为了三个主要仓库:
% \begin{table}[H]
% \centering
% \begin{tabular}{@{}l l p{6cm} @{}}
% \toprule
% 项目                & 项目地址 & 主要目的 \\ \midrule
% BIThesis          &   \href{https://github.com/BITNP/BIThesis}{BITNP/BIThesis}   &  主要存储 \BIThesis  宏包 \\
% BIThesis-wiki     &   \href{https://github.com/BITNP/BIThesis-wiki}{BITNP/BIThesis-wiki}  &  存储 \BIThesisLaTeX 项目在线文档   \\
% BIThesis-scaffold &   \href{https://github.com/BITNP/BIThesis-scaffold}{BITNP/BIThesis-scaffold}   &  存储开箱即用的论文模板样式,便于使用者快速开始写作  \\ \bottomrule
% \end{tabular}
% \end{table}
%
% 如果你仅想解决「我如何使用 \BIThesisLaTeX 来帮助我完成实验论文?」这个问题,那么欢迎你访问我们的\href{https://bithesis.bitnp.net}{在线文档}以获得更多信息。 
% 
% \section{使用说明}
% \subsection{\BIThesis 宏包的组成}
% 为了适应用户的不同需求,并符合 CTeX 宏集的设计习惯,我们将 \BIThesisMacroPackage 的主要功能设计安排在两个中文文档类当中,具体的组成见 \ref{tab:classes}。
% \begin{table}[H]
% \centering
% \caption{测试}
% \label{tab:classes}
% \begin{tabular}{@{}lll@{}}
% \toprule
% 类别                   & 文件          & 说明                             \\ \midrule
% \multirow{2}{*}{文档类} & bitart.cls  & 对应 ctxart.cls,提供实验报告模板、开题报告模板。 \\
%                      & bitbook.cls & 对应 ctexbook.cls ,提供本科生毕业设计模板。     \\ \cmidrule(l){2-3} 
%                    & bitgrad.cls & 对应 ctexbook.cls ,提供研究生毕业设计模板。(全体硕士及2020级以前博士适用)     \\ \cmidrule(l){2-3}
% \end{tabular}
% \end{table}
% \subsection{\BIThesis 宏包的安装和更新}
% 最常见的 \TeX 发行版(\hologo{TeX} Live 和 \hologo{MiKTeX})已收录\BIThesisMacroPackage 及其依赖的宏包和宏集。
%
% 如果安装以上发行版的时间较早,可能你本地的环境中不存在 \BIThesisMacroPackage 或者不是最新版本的。那么你需要通过包管理器来安装/更新 \BIThesisMacroPackage:
% \mint{bash}|tlmgr update --self --all|
% \subsection{使用 \BIThesis 文档类}
% 推荐使用 \BIThesisScaffold 来进行具体的项目编写。\BIThesisScaffold 提供了多种最常用的模板,你可以在 \href{https://github.com/BITNP/BIThesis}{主项目的 Releases}中找到它们。 
% \section{致谢}
% \section{软件许可证}
% \begin{itemize}
%   \item 北京理工大学校徽校名图片的版权归北京理工大学所有。
%   \item \BIThesisLaTeX 宏包以及相关文档类使用 \LPPL 授权。
%   \item \BIThesisLaTeX 文档及其他附属文件通过 \LPPL 授权。
% \end{itemize}
% \section{实现细节}
%
%    \begin{macrocode}
%<*package>
%    \end{macrocode}
%
% \begin{macro}{\YOURMACRO}
% Put explanation of |\YOURMACRO|’s implementation here.
%    \begin{macrocode}
\newcommand{\YOURMACRO}{}
%    \end{macrocode}
% \end{macro}
%
% \begin{environment}{YOURENV}
% Put explanation of |YOURENV|’s implementation here.
%    \begin{macrocode}
\newenvironment{YOURENV}{}{}
%    \end{macrocode}
% \end{environment}
%
%    \begin{macrocode}
%</package>
%    \end{macrocode}

%    \begin{macrocode}
%<*book>
%    \end{macrocode}
%    \begin{macrocode}

% 目前只有本科的模板,但仍旧要为可能预留空间。
\newif\if@bit@bachelor
\newif\if@bit@docTranslation
\newif\if@bit@master
\newif\if@bit@docter

\RequirePackage{kvoptions}

\SetupKeyvalOptions{
  family=BIThesis,
  prefix=BIThesis@
}

\DeclareStringOption[14pt]{footskip}
\DeclareBoolOption{titleNumberHeiti}
\ProcessKeyvalOptions*


\DeclareOption{bachelor}{\@bit@bachelortrue}
\DeclareOption{translation}{\@bit@docTranslationtrue}
\DeclareOption*{\PassOptionsToClass{\CurrentOption}{ctexbook}}

\ExecuteOptions{bachelor}

\ProcessOptions\relax

\PassOptionsToPackage{AutoFakeBold,AutoFakeSlant}{xeCJK}
\LoadClass[UTF8,zihao=-4,oneside,openany]{ctexbook}

\RequirePackage[a4paper,left=3cm,right=2.6cm,top=3.5cm,bottom=2.9cm]{geometry}
% 目前 29mm 最接近 Word 排版
\RequirePackage{xeCJK}
\RequirePackage{titletoc}
  % \RequirePackage{fontspec}
\RequirePackage{setspace}
\RequirePackage{graphicx}
\RequirePackage{fancyhdr}
\RequirePackage{pdfpages}
\RequirePackage{setspace}
\RequirePackage{booktabs}
\RequirePackage{multirow}
\RequirePackage{tikz}
\RequirePackage{etoolbox}
\RequirePackage{hyperref}
\RequirePackage{xcolor}
\RequirePackage{caption}
\RequirePackage{array}
\RequirePackage{amsmath}
\RequirePackage{amssymb}
\RequirePackage{pdfpages}
\RequirePackage{listings}

%\PackageError{zzzz}{Main \BIThesis@footskip  hellp}{test}

% 设置参考文献编译后端为 biber,引用格式为 GB/T7714-2015 格式
% 参考文献使用宏包见 https://github.com/hushidong/biblatex-gb7714-2015
\RequirePackage[
  backend=biber,
  style=gb7714-2015,
  gbalign=gb7714-2015,
  gbnamefmt=lowercase,
  gbpub=false,
  doi=false,
  url=false,
  eprint=false,
  isbn=false,
]{biblatex}

% 参考文献引用文件位于 misc/ref.bib
\addbibresource{./misc/ref.bib}

% 西文字体默认为 Times New Roman
\setromanfont{Times New Roman}
% 论文题目字体为华文细黑
\setCJKfamilyfont{xihei}[AutoFakeBold,AutoFakeSlant]{[STXIHEI.TTF]} % 若希望使用本机字体,也可以用 {STXihei} 来调用
\newcommand{\xihei}{\CJKfamily{xihei}}

\ifBIThesis@titleNumberHeiti
  \newcommand{\arabicHeiti}[1]{\xeCJKsetup{CJKspace=true}\xeCJKDeclareCharClass{CJK}{`0 -> `9}{\heiti\raisebox{-0.1ex}{#1}}\normalspacedchars{0,1,2,3,4,5,6,7,8,9}\xeCJKsetup{CJKspace=false}}
\else
  \newcommand{\arabicHeiti}[1]{#1}
\fi



% 主题页面格式:BIThesis
\fancypagestyle{BIThesis}{
  % 页眉高度
  \setlength{\headheight}{20pt}
  % 页码高度(不完美,比规定稍微靠下 2mm)
  \setlength{\footskip}{\BIThesis@footskip}

  \fancyhf{}
  % 定义页眉、页码
  \fancyhead[C]{\zihao{4}\ziju{0.08}\songti{北京理工大学本科生毕业设计(论文)}}
  \fancyfoot[C]{\songti\zihao{5} \thepage}
  % 页眉分割线稍微粗一些
  \renewcommand{\headrulewidth}{0.6pt}
}

\if@bit@docTranslation
% 主题页面格式:BIThesis
\fancypagestyle{BIThesis}{
  % 页眉高度
  \setlength{\headheight}{20pt}
  % 页码高度(不完美,比规定稍微靠下 2mm)
  \setlength{\footskip}{\BIThesis@footskip}

  \fancyhf{}
  % 定义页码
  \fancyfoot[C]{\songti\zihao{5} \thepage}
  % 页眉分割线稍微粗一些
  \renewcommand{\headrulewidth}{0.6pt}

  % 定义页眉
  \fancyhead[C]{\zihao{4}\ziju{0.08}\songti{北京理工大学本科生毕业设计(论文)外文翻译}}
}
\fi
% 设置章节格式
% 一级标题:黑体,三号,加粗;间距:段前 0.5 行,段后 1 行;
\ctexset{chapter={
    name = {第,章},
    number = {\arabicHeiti{ \arabic{chapter} }},
    format = {\heiti \bfseries \centering \zihao{3}},
    aftername = \hspace{9bp},
    pagestyle = BIThesis,
    beforeskip = 8bp,
    afterskip = 32bp,
    fixskip = true,
  }
}

% 二级标题:黑体,四号,加粗;间距:段前 0.5 行,段后 0 行;
\ctexset{section={
    number = {\arabicHeiti{\thechapter.\hspace{1bp}\arabic{section}}},
    format = {\heiti \raggedright \bfseries \zihao{4}},
    aftername = \hspace{8bp},
    beforeskip = 20bp plus 1ex minus .2ex,
    afterskip = 18bp plus .2ex,
    fixskip = true,
  }
}

% 三级标题:黑体、小四、加粗;间距:段前 0.5 行,段后 0 行;
\ctexset{subsection={
    number = {\arabicHeiti{\thechapter.\hspace{1bp}\arabic{section}.\hspace{1bp}\arabic{subsection}}},
    format = {\heiti \bfseries \raggedright \zihao{-4}},
    aftername = \hspace{7bp},
    beforeskip = 17bp plus 1ex minus .2ex,
    afterskip = 14bp plus .2ex,
    fixskip = true,
  }
}

% 设置目录样式
% 添加 PDF 链接
\addtocontents{toc}{\protect\hypersetup{hidelinks}}

% 解决「目录」二字的格式问题
\renewcommand{\contentsname}{
  \fontsize{16pt}{\baselineskip}
  \normalfont\heiti{目~~~~录}
  \vspace{-8pt}
}
% 定义目录样式
\titlecontents{chapter}[0pt]{\songti \zihao{-4}}
{\thecontentslabel\hspace{\ccwd}}{}
{\hspace{.5em}\titlerule*{.}\contentspage}
\titlecontents{section}[1\ccwd]{\songti \zihao{-4}}
{\thecontentslabel\hspace{\ccwd}}{}
{\hspace{.5em}\titlerule*{.}\contentspage}
\titlecontents{subsection}[2\ccwd]{\songti \zihao{-4}}
{\thecontentslabel\hspace{\ccwd}}{}
{\hspace{.5em}\titlerule*{.}\contentspage}

% 前置页面(原创性声明、中英文摘要、目录等)
\renewcommand{\frontmatter}{
  \pagenumbering{Roman}
  \pagestyle{BIThesis}
}

% 正文页面
\renewcommand{\mainmatter}{
  \pagenumbering{arabic}
  \pagestyle{BIThesis}
}

% 设置 caption 与 figure 之间的距离
\setlength{\abovecaptionskip}{11pt}
\setlength{\belowcaptionskip}{9pt}

% 设置图片的 caption 格式
\renewcommand{\thefigure}{\thechapter-\arabic{figure}}
\captionsetup[figure]{font=small,labelsep=space}

% 设置 listings 源代码高亮的格式
\AtBeginDocument{
  \renewcommand{\lstlistingname}{代码}
  \renewcommand{\thelstlisting}{\arabic{chapter}-\arabic{lstlisting}}
}

\definecolor{codegreen}{rgb}{0,0.6,0}
\definecolor{codegray}{rgb}{0.5,0.5,0.5}
\definecolor{codepurple}{rgb}{0.58,0,0.82}
\definecolor{backcolour}{rgb}{0.95,0.95,0.92}
\lstdefinestyle{examplestyle}{
    backgroundcolor=\color{backcolour},
    commentstyle=\color{codegreen},
    keywordstyle=\color{magenta},
    numberstyle=\tiny\color{codegray},
    stringstyle=\color{codepurple},
    basicstyle=\ttfamily\footnotesize,
    breakatwhitespace=false,
    breaklines=true,
    captionpos=b,
    keepspaces=true,
    numbers=left,
    numbersep=5pt,
    showspaces=false,
    showstringspaces=false,
    showtabs=false,
    tabsize=2
}
\lstset{style=examplestyle}


% 设置表格的 caption 格式和 caption 与 table 之间的垂直距离
\renewcommand{\thetable}{\thechapter-\arabic{table}}
\captionsetup[table]{font=small,labelsep=space,skip=2pt}

% 调整底层 TeX 排版引擎参数以保证所有段落能够很好地以两端对齐的方式呈现
\tolerance=1
\emergencystretch=\maxdimen
\hyphenpenalty=10000
\hbadness=10000

% 设置数学公式编号格式
\renewcommand{\theequation}{\arabic{chapter}-\arabic{equation}}

\newcommand{\unnumchapter}[1]{
  \chapter*{\vskip 10bp\textmd{#1} \vskip -6bp}
  \addcontentsline{toc}{chapter}{#1}
  \stepcounter{chapter}
}



%    \end{macrocode}
%    \begin{macrocode}
%</book>
%    \end{macrocode}
%    \begin{macrocode}
%<*article>
%    \end{macrocode}
%    \begin{macrocode}

\newif\if@bit@labreport
\newif\if@bit@proposalreport

\DeclareOption{lab-report}{\@bit@labreporttrue\@bit@proposalreportfalse}
\DeclareOption{proposal-report}{\@bit@labreportfalse\@bit@proposalreporttrue}
\DeclareOption*{\PassOptionsToClass{\CurrentOption}{ctexart}}
\ExecuteOptions{lab-report}
\ProcessOptions\relax

\PassOptionsToPackage{AutoFakeBold,AutoFakeSlant}{xeCJK}
\LoadClass[UTF8,zihao=-4]{ctexart}%

\if@bit@labreport
  \RequirePackage[a4paper,left=3.18cm,right=3.18cm,top=2.54cm,bottom=2.54cm,includeheadfoot]{geometry}%
\else
  \RequirePackage[a4paper,left=3cm,right=2.4cm,top=2.6cm,bottom=2.38cm,includeheadfoot]{geometry}
\fi

\RequirePackage{fontspec}%
\RequirePackage{setspace}%
\RequirePackage{graphicx}%
\RequirePackage{fancyhdr}%
\RequirePackage{pdfpages}%
\RequirePackage{setspace}%
\RequirePackage{booktabs}%
\RequirePackage{multirow}%
\RequirePackage{caption}%

\if@bit@labreport
  \RequirePackage{titlesec}%
  \RequirePackage{float}%
  \RequirePackage{etoolbox}
\fi

\if@bit@proposalreport
  % 设置参考文献编译后端为 biber,引用格式为 GB/T7714-2015 格式
  % 参考文献使用宏包见 https://github.com/hushidong/biblatex-gb7714-2015
  \usepackage[style=gb7714-2015,backend=biber]{biblatex}

\fi


\if@bit@labreport
  % 将西文字体设置为 Times New Roman
  \setromanfont{Times New Roman}%

  % 设置引用位于右上角
  \newcommand{\upcite}[1]{\textsuperscript{\cite{#1}}}%

  % 设置文档标题深度
  \setcounter{tocdepth}{3}%
  \setcounter{secnumdepth}{3}%

  %%
  % 设置一级标题、二级标题格式
  \ctexset{section={%
    format={\raggedright \bfseries \songti \zihao{-3}},%
    name = {,.},%
    number = \chinese{section}%
    }%
  }%
  \ctexset{subsection={%
    format = {\bfseries \songti \raggedright \zihao{-4}},%
    }%
  }%

  % 页眉和页脚(页码)的格式设定
  \fancyhf{}%
  \fancyhead[L]{\fontsize{10.5pt}{10.5pt}\selectfont\kaishu{\reportName}}%
  \fancyfoot[C]{\fontsize{9pt}{9pt}\selectfont\kaishu{\thepage}}%
  \renewcommand{\headrulewidth}{0.5pt}%
  \renewcommand{\footrulewidth}{0pt}%

  \AtBeginDocument{
  }
\fi

\if@bit@proposalreport
  % 定义 caption 字体为楷体
  \DeclareCaptionFont{kaiticaption}{\kaishu \normalsize}

  % 设置图片的 caption 格式
  \renewcommand{\thefigure}{\thesection-\arabic{figure}}
  \captionsetup[figure]{font=small,labelsep=space,skip=10bp,labelfont=bf,font=kaiticaption}

  % 设置表格的 caption 格式
  \renewcommand{\thetable}{\thesection-\arabic{table}}
  \captionsetup[table]{font=small,labelsep=space,skip=10bp,labelfont=bf,font=kaiticaption}

  % 输出大写数字日期
  \CTEXoptions[today=big]

  % 将西文字体设置为 Times New Roman
  \setromanfont{Times New Roman}

  %% 将中文楷体设置为 SIMKAI.TTF(如果需要)
  % \setCJKfamilyfont{zhkai}{[SIMKAI.TTF]}
  % \newcommand*{\kaiti}{\CJKfamily{zhkai}}

  % 设置文档标题深度
  \setcounter{tocdepth}{3}
  \setcounter{secnumdepth}{3}

  %%
  % 设置一级标题、二级标题格式
  % 一级标题:小三,宋体,加粗,段前段后各半行
  \ctexset{section={
    format={\raggedright \bfseries \songti \zihao{-3}},
    beforeskip = 24bp plus 1ex minus .2ex,
    afterskip = 24bp plus .2ex,
    fixskip = true,
    name = {,.\quad}
    }
  }
  % 二级标题:小四,宋体,加粗,段前段后各半行
  \ctexset{subsection={
    format = {\bfseries \songti \raggedright \zihao{4}},
    beforeskip =24bp plus 1ex minus .2ex,
    afterskip = 24bp plus .2ex,
    fixskip = true,
    }
  }
  % 页眉和页脚(页码)的格式设定
  \fancyhf{}
  \fancyhead[R]{\fontsize{10.5pt}{10.5pt}\selectfont{北京理工大学本科生毕业设计(论文)开题报告}}
  \fancyfoot[R]{\fontsize{9pt}{9pt}\selectfont{\thepage}}
  \renewcommand{\headrulewidth}{1pt}
  \renewcommand{\footrulewidth}{0pt}
\fi


\AtBeginDocument{
  \if@bit@labreport
    \topskip=0pt

\begin{titlepage}
  \vspace*{-16mm}
  \centering

  \vspace{23mm}

  \hspace{-6mm}\heiti\fontsize{24pt}{24pt}\selectfont{\reportName}

  \vspace{87mm}

  \flushleft
  \begin{spacing}{2.2}
    \hspace{39mm}\songti\fontsize{16pt}{16pt}\selectfont{\textbf{学\hspace{11mm}院:}\underline{\makebox[51mm][c]{\deptName}}}

    \hspace{39mm}\songti\fontsize{16pt}{16pt}\selectfont{\textbf{专\hspace{11mm}业:}\underline{\makebox[51mm][c]{\majorName}}}

    \hspace{39mm}\songti\fontsize{16pt}{16pt}\selectfont{\textbf{班\hspace{11mm}级:}\underline{\makebox[51mm][c]{\className}}}

    \hspace{39mm}\songti\fontsize{16pt}{16pt}\selectfont{\textbf{姓\hspace{11mm}名:}\underline{\makebox[51mm][c]{\yourName}}}

    \hspace{39mm}\songti\fontsize{16pt}{16pt}\selectfont{\textbf{任课教师:}\underline{\makebox[51mm][c]{\teacherName}}}
  \end{spacing}

  \vspace{33mm}

  \centering
  \hspace{-6mm}\songti\fontsize{12pt}{12pt}\selectfont{\today}
\end{titlepage}

    % 正文开始
    \pagestyle{fancy}
    \setcounter{page}{1}%
  \fi
  \if@bit@proposalreport
    % 报告封面
    %%
% The BIThesis Template for Bachelor Graduation Thesis
%
% 北京理工大学毕业设计开题报告 —— 使用 XeLaTeX 编译
%
% Copyright 2020 Spencer Woo
%
% This work may be distributed and/or modified under the
% conditions of the LaTeX Project Public License, either version 1.3
% of this license or (at your option) any later version.
% The latest version of this license is in
%   http://www.latex-project.org/lppl.txt
% and version 1.3 or later is part of all distributions of LaTeX
% version 2005/12/01 or later.
%
% This work has the LPPL maintenance status `maintained'.
%
% The Current Maintainer of this work is Spencer Woo.
%
% This work consists of the files main.tex, misc/cover.tex and
% the external PDF misc/reviewTable.pdf

% 校名顶部非常细小的空白
\topskip=0pt

\begin{titlepage}
  \vspace*{-16mm}
  \centering
  \hspace{-6mm}\songti\fontsize{22pt}{22pt}\selectfont{北京理工大学}

  \vspace{13mm}

  \hspace{-6mm}\heiti\fontsize{24pt}{24pt}\selectfont{本科生毕业设计(论文)开题报告}

  \vspace{53mm}

  \flushleft
  \begin{spacing}{2.2}
    \hspace{46mm}\songti\fontsize{16pt}{16pt}\selectfont{\textbf{学\hspace{11mm}院:}\underline{\makebox[51mm][c]{\deptName}}}

    \hspace{46mm}\songti\fontsize{16pt}{16pt}\selectfont{\textbf{专\hspace{11mm}业:}\underline{\makebox[51mm][c]{\majorName}}}

    \hspace{46mm}\songti\fontsize{16pt}{16pt}\selectfont{\textbf{班\hspace{11mm}级:}\underline{\makebox[51mm][c]{\className}}}

    \hspace{46mm}\songti\fontsize{16pt}{16pt}\selectfont{\textbf{姓\hspace{11mm}名:}\underline{\makebox[51mm][c]{\yourName}}}

    \hspace{46mm}\songti\fontsize{16pt}{16pt}\selectfont{\textbf{指导教师:}\underline{\makebox[51mm][c]{\mentorName}}}

    \hspace{46mm}\songti\fontsize{16pt}{16pt}\selectfont{\textbf{校外指导教师:}\underline{\makebox[40mm][c]{\offCampusMentorName}}}
  \end{spacing}

  \vspace{47mm}

  \centering
  \hspace{-6mm}\songti\fontsize{12pt}{12pt}\selectfont{\today}
\end{titlepage}

  \fi

}

%    \end{macrocode}
%
%    \begin{macrocode}
%    \end{macrocode}
%
%    \begin{macrocode}
%</article>
%    \end{macrocode}
%    \begin{macrocode}
%<*graduate>
%    \end{macrocode}
%    \begin{macrocode}

%% ==================================================
%% BIT-thesis-grd.cls for BIT Thesis
%% modified by yang yating
%% version: 1.4
%% last update: Mar 25th, 2018
%% ==================================================

%% math packages -- conflict with xunicode
\RequirePackage{amsmath,amsthm,amsfonts,amssymb,bm,mathrsfs}
% 直立希腊字母字体
\RequirePackage{upgreek}

\DeclareOption*{\PassOptionsToClass{\CurrentOption}{ctexbook}}
\newif\ifBIT@master\BIT@masterfalse
\newif\ifBIT@doctor\BIT@doctorfalse
\newif\ifBIT@istwoside\BIT@istwosidefalse
\DeclareOption{twoside}{\BIT@istwosidetrue}
\DeclareOption{master}{\BIT@mastertrue}
\DeclareOption{doctor}{\BIT@doctortrue}

\ProcessOptions\relax
\ifBIT@istwoside
\LoadClass[zihao=-4,a4paper,UTF8,space=auto]{ctexbook}
\else
\LoadClass[zihao=-4,a4paper,oneside,openany,UTF8,space=auto]{ctexbook}
\fi

%%
%% the setup of ctex package
%%
\def\contentsname{目\BITspace 录}
\def\listfigurename{插\BITspace 图}
\def\listtablename{表\BITspace 格}

%%
%% 封面标题
%%
\def\BIT@label@major{学~~~~科~~~~专~~~~业}
\def\BIT@label@title{论文题目}
\def\BIT@label@author{作~~~~者~~~~姓~~~~名}
\def\BIT@label@classification{中图分类号:}
\def\BIT@label@confidential{密级}
\def\BIT@label@UDC{UDC\!分类号:}
\def\BIT@label@serialnumber{编号}
\def\BIT@label@thesis{学位论文}
\def\BIT@label@advisor{指~~~~导~~~~教~~~~师}
\def\BIT@label@degree{申~~~~请~~~~学~~~~位}
\def\BIT@label@submitdate{论文提交日期}
\def\BIT@label@defenddate{论~文~答~辩~日~期}
\def\BIT@label@institute{学~~~~院~~~~名~~~~称}
\def\BIT@label@school{学~位~授~予~单~位}
\def\BIT@label@chairman{答辩委员会主席}

%%
%% 封面内容
%%

\def\BIT@value@classification{}
\def\BIT@value@confidential{}
\def\BIT@value@UDC{}
\def\BIT@value@serialnumber{}
\def\BIT@value@school{}
\def\BIT@value@degree{}
\def\BIT@value@title{~~~~~(论~文~题~目)~~~~~}
\def\BIT@value@vtitle{竖排论文题目}
\def\BIT@value@titlemark{\BIT@value@title}
\def\BIT@value@author{(作~者~姓~名)}
\def\BIT@value@advisor{(姓名、专业技术职务、学位)}
\def\BIT@value@advisorinstitute{(单位)}
\def\BIT@value@major{}
\def\BIT@value@studentnumber{} % _ added by wei.jianwen@gmail.com
\def\BIT@value@submitdate{}
\def\BIT@value@defenddate{}
\def\BIT@value@institute{}
\def\BIT@value@chairman{}
\def\BIT@label@statement{}

%% 设置圆圈的格式 或使用\textcircled
\usepackage{tikz}
\usepackage{etoolbox}
\newcommand{\circled}[2][]{\tikz[baseline=(char.base)]
    {\node[shape = circle, draw, inner sep = 1pt]
    (char) {\phantom{\ifblank{#1}{#2}{#1}}};
    \node at (char.center) {\makebox[0pt][c]{#2}};}}
\robustify{\circled}

%% 论文原创性声明
\def\BIT@label@original{研究成果声明}
\def\BIT@label@authorization{关于学位论文使用权的说明}
\def\BIT@label@authorsign{作者签名:}
\def\BIT@label@Supervisorsign{导师签名:}
\def\BIT@label@originalDate{签字日期:}
\def\BIT@label@originalcontent{\BITspace\BITspace 本人郑重声明:所提交的学位论文是我本人在指导教师的指导下进行的研究工作获得的研究成果。尽我所知,文中除特别标注和致谢的地方外,学位论文中不包含其他人已经发表或撰写过的研究成果,也不包含为获得北京理工大学或其它教育机构的学位或证书所使用过的材料。与我一同工作的合作者对此研究工作所做的任何贡献均已在学位论文中作了明确的说明并表示了谢意。\par 特此申明。}
\def\BIT@label@authorizationcontent{\BITspace\BITspace 本人完全了解北京理工大学有关保管、使用学位论文的规定,其中包括:\circled{1} 学校有权保管、并向有关部门送交学位论文的原件与复印件;\circled{2} 学校可以采用影印、缩印或其它复制手段复制并保存学位论文;\circled{3} 学校可允许学位论文被查阅或借阅;\circled{4} 学校可以学术交流为目的,复制赠送和交换学位论文;\circled{5} 学校可以公布学位论文的全部或部分内容(保密学位论文在解密后遵守此规定)。}

%%
%% 英语封面标题
%%
\def\BIT@label@englishadvisor{Supervisor:}
\def\BIT@label@englishstatement{Submitted in total fulfilment
  of the requirements for the degree of \BIT@value@englishdegree \\
  in \BIT@value@englishmajor}
\def\BIT@label@englishauthor{Candidate Name:}
\def\BIT@label@englishadvisor{Faculty Mentor:}
\def\BIT@label@englishchairman{Chair, Thesis Committee:}
\def\BIT@label@englishinstitute{School or Department:}
\def\BIT@label@englishdegree{Degree Applied:}
\def\BIT@label@englishmajor{Major:}
\def\BIT@label@englishschool{Degree by:}
\def\BIT@label@englishdate{The Date of Defence:}


%%
%% 英语封面内容
%%
\def\BIT@value@englishtitle{(English Title of Thesis)}
\def\BIT@value@englishauthor{(Author Name)}
\def\BIT@value@englishadvisor{(Supervisor Name)}
\def\BIT@value@englishinstitute{(Institute Name)}
\def\BIT@value@englishscholl{(BIT)}
\def\BIT@value@englishchair{(someone)}
\def\BIT@value@englishdate{}
\def\BIT@value@englishdegree{}
\def\BIT@value@englishmajor{}



\def\BIT@label@abstract{摘要}
\def\BIT@label@englishabstract{Abstract}
\def\BIT@label@keywords{关键词:}
\def\BIT@label@englishkeywords{Key Words:~}
\def\BIT@label@conclusion{结论}
\def\BIT@label@appendix{附录}
\def\BIT@label@publications{攻读学位期间发表论文与研究成果清单}
\def\BIT@label@projects{攻读学位期间参与的项目}
\def\BIT@label@resume{作者简介}
\def\BIT@label@reference{参考文献!!!!}
\def\BIT@label@thanks{致谢}
\def\BIT@value@templateversion{v1.2}
%%
%% label in the head 页眉页脚
%%
\def\BIT@label@headschoolname{北京理工大学硕士学位论文}

%% 当前模板的版本
\newcommand{\version}{\BIT@value@templateversion}

%% ==============引用geometry 宏包设置纸张和页面========================
% 设置版面:上3.5cm,下2.5cm,左2.7cm,右2.7cm,页眉2.5cm,页脚1.8cm,装订线0cm
\usepackage[%
paper=a4paper,%
top=3.5cm,% 上 3.5cm %
bottom=2.5cm,% 下 2.5cm %
left=2.7cm,% 左 2.7cm %
right=2.7cm,% 右 2.7cm %
headheight=1.0cm,% 页眉 2.5cm %
footskip=0.7cm% 页脚 1.8cm %
]{geometry} % 页面设置 %

\parskip 0.5ex plus 0.25ex minus 0.25ex
%% Command -- Clear Double Page
\def\cleardoublepage{\clearpage\if@twoside \ifodd\c@page\else
  \thispagestyle{empty}%
  \hbox{}\newpage\if@twocolumn\hbox{}\newpage\fi\fi\fi}
% 设置行距,大概为22榜
\RequirePackage{setspace}
\setstretch{1.523}

%% 设置章节格式, 黑体三号加粗居中,行距22磅,与正文或节标题的间距设定为段后间距1行。章序号与章名间空一格。
\ctexset{chapter={
      name = {第,章},
      number = {\arabic{chapter}},
      format = {\bfseries \sffamily \centering \zihao{3}},
      pagestyle = {BIT@headings},
      beforeskip = 16 bp,
      afterskip = 32 bp,
      fixskip = true,
  }
}
%% 设置一级章节格式
% 黑体四号加粗顶左,行距22磅,与上一节的间距为1行,与下面正文或节标题的段间间距为0.5行。序号与题目间空一格。

\ctexset{section={
  format={\raggedright \bfseries \sffamily \zihao{4}},
  beforeskip = 28bp plus 1ex minus .2ex,
  afterskip = 24bp plus .2ex,
  fixskip = true,
  }
}

% 设置二级标题格式

% 黑体小四加粗顶左,行距22磅,与上一节的间距为1行,与下面正文或节标题的段间间距为0.5行。序号与题目间空一格。 

\ctexset{subsection={
   format = {\bfseries \sffamily \raggedright \zihao{-4}},
   beforeskip =28bp plus 1ex minus .2ex,
   afterskip = 24bp plus .2ex,
   fixskip = true,
   }
}

% 设置三节标题格式

\ctexset{subsubsection={
      format={\heiti \raggedright \zihao{-4}},
      beforeskip=28bp plus 1ex minus .2ex,
      afterskip=24bp plus .2ex,
      fixskip=true,
  }
}

%% 设定目录格式。目录颜色更改黑色
\addtocontents{toc}{\protect\hypersetup{hidelinks}}
\addtocontents{lot}{\protect\hypersetup{hidelinks}}
\addtocontents{lof}{\protect\hypersetup{hidelinks}}

\RequirePackage{titletoc}
\titlecontents{chapter}[0pt]{\songti \zihao{4}}
    {\bf\thecontentslabel\hspace{\ccwd}}{\bf}
    {\hspace{.5em}\titlerule*{.}\contentspage}
\titlecontents{section}[2\ccwd]{\songti \zihao{-4}}
    {\thecontentslabel\hspace{\ccwd}}{}
    {\hspace{.5em}\titlerule*{.}\contentspage}
\titlecontents{subsection}[4\ccwd]{\songti \zihao{-4}}
    {\thecontentslabel\hspace{\ccwd}}{}
    {\hspace{.5em}\titlerule*{.}\contentspage}

\titlecontents{figure}[0pt]{\songti\zihao{-4}}
    {\figurename~\thecontentslabel\quad}{\hspace*{-1.5cm}}
    {\hspace{.5em}\titlerule*{.}\contentspage}

\titlecontents{table}[0pt]{\songti\zihao{-4}}
    {\tablename~\thecontentslabel\quad}{\hspace*{-1.5cm}}
    {\hspace{.5em}\titlerule*{.}\contentspage}

%% 选择编译
\RequirePackage{ifthen}

%% check pdfTeX mode
\RequirePackage{ifpdf}

%% fancyhdr 页眉页脚控制
\RequirePackage{fancyhdr}

% 空 页眉页脚
\fancypagestyle{BIT@empty}{%
  \fancyhf{}}

% ======正文页眉页脚=================
\fancypagestyle{BIT@headings}{%
  \fancyhf{}
  \fancyfoot[C]{\songti\zihao{5} \thepage}
  \fancyhead[C]{\ifBIT@master\zihao{5}{\songti 北京理工大学硕士学位论文}
                \else\zihao{5}{\songti 北京理工大学博士学位论文}\fi}
  }

% ==================================对于openright 选项,必须保证章页右开,且如果前章末页内容须清空其页眉页脚。===================
\let\BIT@cleardoublepage\cleardoublepage
\newcommand{\BIT@clearemptydoublepage}{%
	\clearpage{\pagestyle{BIT@empty}\BIT@cleardoublepage}}
\let\cleardoublepage\BIT@clearemptydoublepage

 % ================修该frontmatter 的页码为大写罗马格式,并调整页面风格===============
\renewcommand{\frontmatter}{
 \if@openright\cleardoublepage\else\clearpage\fi
  \@mainmatterfalse
  \pagenumbering{Roman}
  \pagestyle{BIT@headings}
}
% =======================修改mainmatter 的页码为阿拉伯格式,并调整页面风格========================
\renewcommand{\mainmatter}{
  \if@openright\cleardoublepage\else\clearpage\fi
  \@mainmattertrue
  \pagenumbering{arabic}
  \pagestyle{BIT@headings}
}


%% 复杂表格
\RequirePackage{threeparttable}
\RequirePackage{dcolumn}
\RequirePackage{multirow}
\RequirePackage{booktabs}
\newcolumntype{d}[1]{D{.}{.}{#1}}% or D{.}{,}{#1} or D{.}{\cdot}{#1}


%% 定义几个常用的数学常量符号
\newcommand{\me}{\mathrm{e}} % 定义 对数常数e,虚数符号i,j以及微分算子d为直立体。
\newcommand{\mi}{\mathrm{i}}
\newcommand{\mj}{\mathrm{j}}
\newcommand{\dif}{\,\mathrm{d}} 

\theoremstyle{plain}
  \newtheorem{algo}{算法~}[chapter]
  \newtheorem{thm}{定理~}[chapter]
  \newtheorem{lem}[thm]{引理~}
  \newtheorem{prop}[thm]{命题~}
  \newtheorem{cor}[thm]{推论~}
\theoremstyle{definition}
  \newtheorem{defn}{定义~}[chapter]
  \newtheorem{conj}{猜想~}[chapter]
  \newtheorem{exmp}{例~}[chapter]
  \newtheorem{rem}{注~}
  \newtheorem{case}{情形~}
\renewcommand{\proofname}{\bf 证明}

%% 英文字体使用 Times New Roman
\RequirePackage{xltxtra} % \XeTeX Logo
\setmainfont{Times New Roman}
\setsansfont{Arial}
\setmonofont{Courier New}


%% graphics packages
\RequirePackage{graphicx}
%% 并列子图
\RequirePackage{subfigure}

\RequirePackage{wrapfig}
%% ===========================设置图表标题选项==========================
\RequirePackage{amsmath}
\RequirePackage{caption}
\DeclareCaptionLabelSeparator{zhspace}{\hspace{1\ccwd}}
\DeclareCaptionFont{fontsize}{\zihao{5}}
\captionsetup{
  font = {fontsize},
  labelsep = zhspace,
}
\captionsetup[table]{
  position = top,
  aboveskip = 6bp,
  belowskip = 6bp,
}
\numberwithin{table}{chapter}
\captionsetup[figure]{
  position = bottom,
  aboveskip = 6bp,
  belowskip = 6bp,
}

%% 如果插入的图片没有指定扩展名,那么依次搜索下面的扩展名所对应的文件
\DeclareGraphicsExtensions{.pdf,.eps,.png,.jpg,.jpeg}
% ccaption -- bicaption
% \RequirePackage{ccaption}
% \captiondelim{\ }
% \captionnamefont{\songti\zihao{5}}
% \captiontitlefont{\songti\zihao{5}}


%% sort and compress citations
\RequirePackage[numbers,square,comma,super,sort&compress]{natbib}
% 上标引用
\newcommand{\upcite}[1]{\textsuperscript{\cite{#1}}}


% 将浮动参数设为较宽松的值
\renewcommand{\textfraction}{0.15}
\renewcommand{\topfraction}{0.85}
\renewcommand{\bottomfraction}{0.65}
\renewcommand{\floatpagefraction}{0.60}


% 定公式、图、表编号为"3-1"的形式,即分隔符由.变为短杠
\renewcommand\theequation{\arabic{chapter}.\arabic{equation}}
\renewcommand\thefigure{\arabic{chapter}.\arabic{figure}}
\renewcommand\thetable{\arabic{chapter}.\arabic{table}}

% 颜色宏包
\RequirePackage{xcolor}


% 中文破折号
\newcommand{\cndash}{\rule{0.0em}{0pt}\rule[0.35em]{1.4em}{0.05em}\rule{0.2em}{0pt}}

% listings 源代码显示宏包
\RequirePackage{listings}
\lstset{tabsize=4, %
  frame=shadowbox, % 把代码用带有阴影的框圈起来
  commentstyle=\color{red!50!green!50!blue!50},% 浅灰色的注释
  rulesepcolor=\color{red!20!green!20!blue!20},% 代码块边框为淡青色
  keywordstyle=\color{blue!90}\bfseries, % 代码关键字的颜色为蓝色,粗体
  showstringspaces=false,% 不显示代码字符串中间的空格标记
  stringstyle=\ttfamily, % 代码字符串的特殊格式
  keepspaces=true, %
  breakindent=22pt, %
  numbers=left,% 左侧显示行号
  stepnumber=1,%
  numberstyle=\tiny, % 行号字体用小号
  basicstyle=\footnotesize, %
  showspaces=false, %
  flexiblecolumns=true, %
  breaklines=true, % 对过长的代码自动换行
  breakautoindent=true,%
  breakindent=4em, %
  aboveskip=1em, % 代码块边框
  %% added by http://bbs.ctex.org/viewthread.php?tid=53451
  fontadjust,
  captionpos=t,
  framextopmargin=2pt,framexbottommargin=2pt,abovecaptionskip=-3pt,belowcaptionskip=3pt,
  xleftmargin=4em,xrightmargin=4em, % 设定listing左右的空白
  texcl=true,
  % 设定中文冲突,断行,列模式,数学环境输入,listing数字的样式
  extendedchars=false,columns=flexible,mathescape=true
  numbersep=-1em
}
\renewcommand{\lstlistingname}{代码} %% 重命名Listings标题头

%% hyperref package
\definecolor{navyblue}{RGB}{0,0,128} 
\RequirePackage{hyperref}
\hypersetup{
  bookmarksnumbered,%
  linktoc=all,
  colorlinks=true,
  citecolor=navyblue,
  filecolor=cyan,
  linkcolor=navyblue,
  linkbordercolor=navyblue,
  urlcolor=navyblue,
  plainpages=false,%
  pdfstartview=FitH
}

%% enumerate 列表环境间距调节
\usepackage{enumitem}
% \setenumerate[1]{itemsep=0pt,partopsep=0pt,parsep=\parskip,topsep=5pt}
% \setitemize[1]{itemsep=0pt,partopsep=0pt,parsep=\parskip,topsep=0pt}
% \setdescription{itemsep=0pt,partopsep=0pt,parsep=\parskip,topsep=5pt}

% _  参考文献风格 added by wei.jianwen@gmail.com
\bibliographystyle{GBT7714-2005NLang}

% _ BITspace
% \newcommand\BITspace{\protect\CTEX@spaceChar\protect\CTEX@spaceChar}
\newcommand{\BITspace}[1][1]{\hspace{#1\ccwd}}

\def\BIT@getfileinfo#1 #2 #3\relax#4\relax{%
  \def\BITfiledate{#1}%
  \def\BITfileversion{#2}%
  \def\BITfileinfo{#3}}%
\expandafter\ifx\csname ver@bitmaster-xetex.cls\endcsname\relax
  \edef\reserved@a{\csname ver@ctextemp_bitmaster-xetex.cls\endcsname}
\else
  \edef\reserved@a{\csname ver@bitmaster-xetex.cls\endcsname}
\fi
\expandafter\BIT@getfileinfo\reserved@a\relax? ? \relax\relax
\def\BIT@underline[#1]#2{%
  \underline{\hbox to #1{\hfill#2\hfill}}}
\def\BITunderline{\@ifnextchar[\BIT@underline\underline}

% 中文标题页的可用命令
\newcommand\classification[1]{\def\BIT@value@classification{#1}}
\newcommand\studentnumber[1]{\def\BIT@value@studentnumber{#1}}
\newcommand\confidential[1]{\def\BIT@value@confidential{#1}}
\newcommand\UDC[1]{\def\BIT@value@UDC{#1}}
\newcommand\serialnumber[1]{\def\BIT@value@serialnumber{#1}}
\newcommand\school[1]{\def\BIT@value@school{#1}}
\newcommand\degree[1]{\def\BIT@value@degree{#1}}
\renewcommand\title[2][\BIT@value@title]{%
  \def\BIT@value@title{#2}
  \def\BIT@value@titlemark{\MakeUppercase{#1}}}

\newcommand\vtitle[1]{\def\BIT@value@vtitle{#1}}
\renewcommand\author[1]{\def\BIT@value@author{#1}}
\newcommand\advisor[1]{\def\BIT@value@advisor{#1}}
\newcommand\advisorinstitute[1]{\def\BIT@value@advisorinstitute{#1}}
\newcommand\major[1]{\def\BIT@value@major{#1}}
\newcommand\submitdate[1]{\def\BIT@value@submitdate{#1}}
\newcommand\defenddate[1]{\def\BIT@value@defenddate{#1}}
\newcommand\institute[1]{\def\BIT@value@institute{#1}}
\newcommand\chairman[1]{\def\BIT@value@chairman{#1}}

%% 第一页和第二页
%  “绘制”BIT中文标题页
\renewcommand\maketitle[1]{%
  \cleardoublepage
  \thispagestyle{empty}
  \begin{center}
    \vspace*{60mm}
    {\heiti\zihao{-2} \BIT@value@title}
    \vskip 40mm
    {\heiti \zihao{-3} \BIT@value@author} % 黑体 小三
     \vskip 4mm
    {\heiti \zihao{-3} \BIT@value@defenddate} % 黑体 小三
  \end{center}
  \clearpage
  \if@twoside
    \thispagestyle{empty}
    \cleardoublepage
  \fi
 }

\newcommand\makeInfo[1]%
 {
  \newpage
  \cleardoublepage
  \thispagestyle{empty}

% udc ltz 
{ %
  {\heiti \zihao{5} \noindent \BIT@label@classification} \BIT@value@classification \\
  {\heiti \zihao{5} \BIT@label@UDC}  \BIT@value@UDC
}

   \begin{center}

    \vskip \stretch{1}
       {\heiti\zihao{-2} \BIT@value@title}
    \vskip \stretch{1}

    {\fangsong\zihao{4}}
    \def\tabcolsep{1pt}
    \def\arraystretch{1.5}

	% 黑体 小三
    {\heiti\zihao{-3}
     \begin{tabular}{l p{3mm} c}
      \BIT@label@author & &\BITunderline[180pt]{\BIT@value@author}
    \\
      \BIT@label@institute & & \BITunderline[180pt]{\BIT@value@institute}
    \\
      \BIT@label@advisor & &  \BITunderline[180pt]{\BIT@value@advisor}
    \\
      \BIT@label@chairman & &   \BITunderline[180pt]{\BIT@value@chairman}
    \\
      \BIT@label@degree & &    \BITunderline[180pt]{\BIT@value@degree}
    \\
      \BIT@label@major & &    \BITunderline[180pt]{\BIT@value@major}
    \\
      \BIT@label@school & &   \BITunderline[180pt]{\BIT@value@school}
    \\
      \BIT@label@defenddate & &  \BITunderline[180pt]{\BIT@value@defenddate}
    \end{tabular}}

  \end{center}

  \vskip \stretch{0.5}
  \clearpage
  \if@twoside
    \thispagestyle{empty}
    \cleardoublepage
  \fi
}

% English Title Page
% 英文标题页可用命令
\newcommand\englishtitle[1]{\def\BIT@value@englishtitle{#1}}
\newcommand\englishauthor[1]{\def\BIT@value@englishauthor{#1}}
\newcommand\englishadvisor[1]{\def\BIT@value@englishadvisor{#1}}
\newcommand\englishschool[1]{\def\BIT@value@englishschool{#1}}
\newcommand\englishinstitute[1]{\def\BIT@value@englishinstitute{#1}}
\newcommand\englishdate[1]{\def\BIT@value@englishdate{#1}}
\newcommand\englishdegree[1]{\def\BIT@value@englishdegree{#1}}
\newcommand\englishmajor[1]{\def\BIT@value@englishmajor{#1}}
\newcommand\englishchairman[1]{\def\BIT@value@englishchairman{#1}}

% "绘制"英文标题页
\newcommand\makeEnglishInfo[1]{%
  \cleardoublepage
  \thispagestyle{empty}

   \begin{center}


   \vspace*{10em}
% 论文题目	  Times New Roman 小二 加粗
   {\zihao{-2}\textbf{\BIT@value@englishtitle}}
   % \bfseries
   \vskip \stretch{1}
   
% Times New Roman 小三
   {\zihao{-3}
     \begin{tabular}{ll}
      \BIT@label@englishauthor & \BITunderline[200pt]{\BIT@value@englishauthor}
    \\
      \BIT@label@englishinstitute &  \BITunderline[200pt]{\BIT@value@englishinstitute}
    \\
      \BIT@label@englishadvisor &  \BITunderline[200pt]{\BIT@value@englishadvisor}
    \\
      \BIT@label@englishchairman &   \BITunderline[200pt]{\BIT@value@englishchairman}
    \\
      \BIT@label@englishdegree &    \BITunderline[200pt]{\BIT@value@englishdegree}
    \\
      \BIT@label@englishmajor &     \BITunderline[200pt]{\BIT@value@englishmajor}
    \\
      \BIT@label@englishschool &     \BITunderline[200pt]{\BIT@value@englishschool}
    \\
      \BIT@label@englishdate &   \BITunderline[200pt]{\BIT@value@englishdate}
    \end{tabular}}

  \end{center}

  \vskip \stretch{0.5}
  \clearpage
  \if@twoside
  \thispagestyle{empty}
   \cleardoublepage
  \fi
}

% 绘制树立排放的论文题目和学校名称

\newcommand\makeVerticalTitle{
   \cleardoublepage
   \thispagestyle{empty}
   \vskip 5cm
   \begin{center}
    \setstretch{1.1}
    \begin{minipage}{1em}
        {\heiti\zihao{3}\BIT@value@vtitle}
        \vskip 2cm
        {\heiti\zihao{3}\BIT@value@school}
    \end{minipage}
   \end{center}
    \clearpage
    \if@twoside
        \thispagestyle{empty}
        \cleardoublepage
    \fi

}

% 原创性声明
\newcommand\makeDeclareOriginal{%
  \cleardoublepage
  \pdfbookmark[0]{声明}{statement}
  \thispagestyle{empty}
  \begin{center}
  {\bf\zihao{3} \BIT@label@original}
  \end{center}
  \vskip 10pt
  {\zihao{4}\BIT@label@originalcontent}
  \vskip 10pt
  \hspace{8em}{\zihao{4}\BIT@label@authorsign} \BITunderline[6em]{} \hspace{2em} {\zihao{4}\BIT@label@originalDate} \BITunderline[6em]{}

  \vskip 30mm

  \begin{center}
  {\bf\zihao{3} \BIT@label@authorization}
  \end{center}
  \vskip 10pt
  {\zihao{4} \BIT@label@authorizationcontent}
  \vskip 40pt

  \hspace{8em}{\zihao{4}\BIT@label@authorsign} \BITunderline[6em]{} \hspace{2em} {\zihao{4}\BIT@label@Supervisorsign} \BITunderline[6em]{}
  \vskip 15pt
  \hspace{8em}{\zihao{4}\BIT@label@originalDate} \BITunderline[6em]{} \hspace{2em} {\zihao{4}\BIT@label@originalDate} \BITunderline[6em]{}
  \clearpage
  \if@twoside
    \thispagestyle{empty}
    \cleardoublepage
  \fi

}


% 页眉页脚
\pagestyle{fancy}
\fancyhf{}
\fancyhead[C]{\songti \zihao{5} \BIT@label@headschoolname}  % 奇数页左页眉
\fancyfoot[C]{\songti \zihao{5} {\thepage}}      % 页脚


\fancypagestyle{plain}{% 设置开章页页眉页脚风格(只有页码作为页脚)
  \fancyhf{}%
  \fancyfoot[C]{\songti \zihao{5} \BIT@label@headschoolname}
  \fancyfoot[C]{\songti \zihao{5} ~---~{\thepage}~---~} % 首页页脚格式
}


 % 中文摘要
 \newenvironment{abstract}
 {
  \cleardoublepage
  \chapter{\BIT@label@abstract}
 }
 {}
% 下一页从偶数页开始
 \newcommand\beginatevenpage{
 \clearpage
  \if@twoside
    \thispagestyle{empty}
    \cleardoublepage
  \fi
 }
 % 中文关键词
 \newcommand\keywords[1]{%
   \vspace{2ex}\noindent{\bf \BIT@label@keywords} #1}

 % 英文摘要
 \newenvironment{englishabstract}
 {
  \clearpage
  \chapter{\BIT@label@englishabstract}
 }
 {}

 % 英文摘要
 \newcommand\englishkeywords[1]{%
  \vspace{2ex}\noindent{\bf \BIT@label@englishkeywords} #1}


% 目录
\renewcommand\tableofcontents{%
  \if@twocolumn
  \@restonecoltrue\onecolumn
  \else
  \@restonecolfalse
  \fi
  \chapter*{\contentsname}% 目录里显示“目录”,否则\chapter*
  \@mkboth{\MakeUppercase\contentsname}{\MakeUppercase\contentsname}%
  \pdfbookmark[0]{目录}{bittoc}
  \@starttoc{toc}%
  \if@restonecol\twocolumn\fi
}


% 参考文献环境
\renewenvironment{thebibliography}[1]
     {\zihao{5}
      \chapter*{\bibname}
      \@mkboth{\MakeUppercase\bibname}{\MakeUppercase\bibname}%
      \addcontentsline{toc}{chapter}{参考文献}
      \list{\@biblabel{\@arabic\c@enumiv}}%
           {\settowidth\labelwidth{\@biblabel{#1}}%
            \leftmargin\labelwidth
            \advance\leftmargin\labelsep
            \setlength{\parsep}{1mm}
            \setlength{\labelsep}{0.5em}
            \setlength{\itemsep}{0.05pc}
            \setlength{\listparindent}{0in}
            \setlength{\itemindent}{0in}
            \setlength{\rightmargin}{0in}
            \@openbib@code
            \usecounter{enumiv}%
            \let\p@enumiv\@empty
            \renewcommand\theenumiv{\@arabic\c@enumiv}}%
      \sloppy
      \clubpenalty4000
      \@clubpenalty \clubpenalty
      \widowpenalty4000%
      \sfcode`\.\@m}
     {\def\@noitemerr
       {\@latex@warning{Empty `thebibliography' environment}}%
      \endlist}


\newenvironment{publications}[1]
     {\chapter{\BIT@label@publications}%
      \@mkboth{\MakeUppercase\BIT@label@publications}
              {\MakeUppercase\BIT@label@publications}%
      \list{\@biblabel{\@arabic\c@enumiv}}%
           {\settowidth\labelwidth{\@biblabel{#1}}%
            \leftmargin\labelwidth
            \advance\leftmargin\labelsep
            \setlength{\parsep}{1mm}
            \setlength{\labelsep}{0.5em}
            \setlength{\itemsep}{0.05pc}
            \setlength{\listparindent}{0in}
            \setlength{\itemindent}{0in}
            \setlength{\rightmargin}{0in}
            \@openbib@code
            \usecounter{enumiv}%
            \let\p@enumiv\@empty
            \renewcommand\theenumiv{\@arabic\c@enumiv}}%
      \sloppy
      \clubpenalty4000
      \@clubpenalty \clubpenalty
      \widowpenalty4000%
      \sfcode`\.\@m}
     {\def\@noitemerr
       {\@latex@warning{Empty `publications' environment}}%
      \endlist}


\newenvironment{projects}[1]
     {\chapter{\BIT@label@projects}%
      \@mkboth{\MakeUppercase\BIT@label@projects}
              {\MakeUppercase\BIT@label@projects}%
      \list{\@biblabel{\@arabic\c@enumiv}}%
           {\settowidth\labelwidth{\@biblabel{#1}}%
            \leftmargin\labelwidth
            \advance\leftmargin\labelsep
            \@openbib@code
            \usecounter{enumiv}%
            \let\p@enumiv\@empty
            \renewcommand\theenumiv{\@arabic\c@enumiv}}%
      \sloppy
      \clubpenalty4000
      \@clubpenalty \clubpenalty
      \widowpenalty4000%
      \sfcode`\.\@m}
     {\def\@noitemerr
       {\@latex@warning{Empty `projects' environment}}%
      \endlist}

    \newenvironment{resume}
  {\chapter{\BIT@label@resume}}
  {}

\newenvironment{resumesection}[1]
  {{\noindent\normalfont\bfseries #1}
   \list{}{\labelwidth\z@
           \leftmargin 2\ccwd}
   \item\relax}
   {\endlist}

\newenvironment{resumeli}[1]
  {{\noindent\normalfont\bfseries #1}
   \list{}{\labelwidth\z@
           \leftmargin 4\ccwd
           \itemindent -2\ccwd
           \listparindent\itemindent}
   \item\relax}
   {\endlist}

\newenvironment{conclusion}
  {\chapter*{结论}
    \@mkboth{结论}{结论}%
    \addcontentsline{toc}{chapter}{结论}}
  {}

\renewenvironment{thanks}
  {\chapter{\BIT@label@thanks}
  \fangsong
  }
  {}

\newenvironment{symbolnote}
  {\chapter{\BIT@label@symbolnote}
  \fangsong}
  {}

  %% ===========================术语=====================
  \newcommand{\bit@denotation@name}{主要符号对照表}
  \newenvironment{denotation}[1][2.5cm]{
      \chapter{\bit@denotation@name} % no tocline
      \noindent\begin{list}{}%
      {\vskip-30bp\zihao{-4}
       \renewcommand\makelabel[1]{##1\hfil}
       \setlength{\labelwidth}{#1} % 标签盒子宽度
       \setlength{\labelsep}{0.5cm} % 标签与列表文本距离
       \setlength{\itemindent}{0cm} % 标签缩进量
       \setlength{\leftmargin}{\labelwidth+\labelsep} % 左边界
       \setlength{\rightmargin}{0cm}
       \setlength{\parsep}{0cm} % 段落间距
       \setlength{\itemsep}{0cm} % 标签间距
      \setlength{\listparindent}{0cm} % 段落缩进量
      \setlength{\topsep}{0pt} % 标签与上文的间距
     }}{\end{list}}
% ====增加化学、国际单位宏包
     \RequirePackage[version=4]{mhchem}
     \RequirePackage{siunitx}
\setcounter{secnumdepth}{4}  % 章节编号深度 (part 对应 -1)
\setcounter{tocdepth}{2}     % 目录深度 (part 对应 -1)

%% End of file `bitmaster-xetex.cls'.

%% =========================================================

%    \end{macrocode}
%    \begin{macrocode}
%</graduate>
%    \end{macrocode}
%
% \iffalse
%<*dtx-style>
\ProvidesPackage{dtx-style}
\RequirePackage{hypdoc}
\RequirePackage{ifthen}
\RequirePackage{fontspec}[2017/01/20]
\RequirePackage{amsmath}
\RequirePackage{unicode-math}
\RequirePackage[UTF8,scheme=chinese,heading]{ctex}
\RequirePackage[
  top=2.5cm, bottom=2.5cm,
  left=4cm, right=2cm,
  headsep=3mm]{geometry}
\RequirePackage{graphicx}
\RequirePackage{multirow}
\RequirePackage[ruled,vlined]{algorithm2e}
\RequirePackage{wrapfig}
\RequirePackage{hologo}
\RequirePackage{array,longtable,booktabs}
\RequirePackage{listings}
\RequirePackage{fancyhdr}
\RequirePackage[dvipsnames]{xcolor}
\RequirePackage{awesomebox}
\RequirePackage{etoolbox}
\RequirePackage{dirtree}
\RequirePackage{metalogo}
\RequirePackage[tightLists=false]{markdown}
\RequirePackage{caption}
\RequirePackage{tikz}
\usetikzlibrary{positioning}
\RequirePackage{framed}
\RequirePackage{menukeys}

  % 设置代码高亮
\RequirePackage{minted}
\usemintedstyle{tango}

  % 设置列表无间隔
\usepackage{enumitem}
\setlist{nosep}

\markdownSetup{
  renderers = {
    link = {\href{#2}{#1}},
  }
}

\hypersetup{
  pdflang     = zh-CN,
  pdftitle    = {BIThesis:北京理工大学学位论文及报告模板},
  pdfauthor   = {冯开宇},
  pdfsubject  = {北京理工大学学位论文及报告模板使用说明},
  pdfkeywords = {论文模板; 北京理工大学; 使用说明},
  pdfdisplaydoctitle = true
}%

\newcommand{\BIThesisLaTeX}{{\BIThesis}北京理工大学学位论文及报告{\LaTeX}模板}
\newcommand{\BIThesisMacroPackage}{{\BIThesis}宏集}
\newcommand{\BIThesisWiki}{{\BIThesis}在线文档}
\newcommand{\BIThesisScaffold}{{\BIThesis}模板}
\newcommand{\LPPL}{{\href{https://www.latex-project.org/lppl/lppl-1-3c.txt}{\LaTeX{} Project Public License (1.3.c)}}}
\newcommand{\version}{v2.0 BirthdayCake}

\ctexset{
  today=big,
  abstractname=简介
}

\ctexset{section={
  format={\raggedright \bfseries \zihao{-3}},
  name = {第,章}
  }
}

\ctexset{subsection={
  format = {\bfseries \raggedright \zihao{4}}
  }
}



\ifthenelse{\equal{\@nameuse{g__ctex_fontset_tl}}{mac}}{
  \setmainfont{Palatino}
  \setsansfont[Scale=MatchLowercase]{Helvetica}
  \setmonofont[Scale=MatchLowercase]{Menlo}
  \xeCJKsetwidth{‘’“”}{1em}
}{
  \setmainfont[
    Extension      = .otf,
    UprightFont    = *-regular,
    BoldFont       = *-bold,
    ItalicFont     = *-italic,
    BoldItalicFont = *-bolditalic,
  ]{texgyrepagella}
  \setsansfont[
    Extension      = .otf,
    UprightFont    = *-regular,
    BoldFont       = *-bold,
    ItalicFont     = *-italic,
    BoldItalicFont = *-bolditalic,
  ]{texgyreheros}
  \setmonofont[
    Extension      = .otf,
    UprightFont    = *-regular,
    BoldFont       = *-bold,
    ItalicFont     = *-italic,
    BoldItalicFont = *-bolditalic,
    Scale          = MatchLowercase,
    Ligatures      = CommonOff,
  ]{texgyrecursor}
}
\unimathsetup{
  math-style=ISO,
  bold-style=ISO,
}
\IfFontExistsTF{XITSMath-Regular.otf}{
  \setmathfont[
    Extension    = .otf,
    BoldFont     = XITSMath-Bold,
    StylisticSet = 8,
  ]{XITSMath-Regular}
  \setmathfont[range={cal,bfcal},StylisticSet=1]{XITSMath-Regular.otf}
}{
  \setmathfont[
    Extension    = .otf,
    BoldFont     = *bold,
    StylisticSet = 8,
  ]{xits-math}
  \setmathfont[range={cal,bfcal},StylisticSet=1]{xits-math.otf}
}

\colorlet{bit@macro}{blue!60!black}
\colorlet{bit@env}{blue!70!black}
\colorlet{bit@option}{purple}
\patchcmd{\PrintMacroName}{\MacroFont}{\MacroFont\bfseries\color{bit@macro}}{}{}
\patchcmd{\PrintDescribeMacro}{\MacroFont}{\MacroFont\bfseries\color{bit@macro}}{}{}
\patchcmd{\PrintDescribeEnv}{\MacroFont}{\MacroFont\bfseries\color{bit@env}}{}{}
\patchcmd{\PrintEnvName}{\MacroFont}{\MacroFont\bfseries\color{bit@env}}{}{}

\def\DescribeOption{%
  \leavevmode\@bsphack\begingroup\MakePrivateLetters%
  \Describe@Option}
\def\Describe@Option#1{\endgroup
  \marginpar{\raggedleft\PrintDescribeOption{#1}}%
  \bit@special@index{option}{#1}\@esphack\ignorespaces}
\def\PrintDescribeOption#1{\strut \MacroFont\bfseries\sffamily\color{bit@option} #1\ }
\def\bit@special@index#1#2{\@bsphack
  \begingroup
    \HD@target
    \let\HDorg@encapchar\encapchar
    \edef\encapchar usage{%
      \HDorg@encapchar hdclindex{\the\c@HD@hypercount}{usage}%
    }%
    \index{#2\actualchar{\string\ttfamily\space#2}
           (#1)\encapchar usage}%
    \index{#1:\levelchar#2\actualchar
           {\string\ttfamily\space#2}\encapchar usage}%
  \endgroup
  \@esphack}

\lstdefinestyle{lstStyleBase}{%
   basicstyle=\small\ttfamily,
   aboveskip=\medskipamount,
   belowskip=\medskipamount,
   lineskip=0pt,
   boxpos=c,
   showlines=false,
   extendedchars=true,
   upquote=true,
   tabsize=2,
   showtabs=false,
   showspaces=false,
   showstringspaces=false,
   numbers=none,
   linewidth=\linewidth,
   xleftmargin=4pt,
   xrightmargin=0pt,
   resetmargins=false,
   breaklines=true,
   breakatwhitespace=false,
   breakindent=0pt,
   breakautoindent=true,
   columns=flexible,
   keepspaces=true,
   gobble=4,
   framesep=3pt,
   rulesep=1pt,
   framerule=1pt,
   backgroundcolor=\color{gray!5},
   stringstyle=\color{green!40!black!100},
   keywordstyle=\bfseries\color{blue!50!black},
   commentstyle=\slshape\color{black!60}}

\lstdefinestyle{lstStyleShell}{%
   style=lstStyleBase,
   frame=l,
   rulecolor=\color{purple},
   language=bash}

\lstdefinestyle{lstStyleLaTeX}{%
   style=lstStyleBase,
   frame=l,
   rulecolor=\color{violet},
   language=[LaTeX]TeX}

\lstnewenvironment{latex}{\lstset{style=lstStyleLaTeX}}{}
\lstnewenvironment{shell}{\lstset{style=lstStyleShell}}{}

\setlist{nosep}

\DeclareDocumentCommand{\option}{m}{\textsf{#1}}
\DeclareDocumentCommand{\env}{m}{\texttt{#1}}
\newcommand{\myentry}[1]{%
  \marginpar{\raggedleft\color{purple}\bfseries\strut #1}}
\newcommand{\note}[2][Note]{{%
  \color{magenta}{\bfseries #1}\emph{#2}}}

\DeclareDocumentCommand{\githubuser}{m}{\href{https://github.com/#1}{@#1}}


  % 设置 caption 与 figure 之间的距离
\setlength{\abovecaptionskip}{11pt}
\setlength{\belowcaptionskip}{9pt}

  % 设置图片的 caption 格式
\renewcommand{\thefigure}{\thesection-\arabic{figure}}
\captionsetup[figure]{font=small,labelsep=space}

  % 设置表格的 caption 与 table 之间的垂直距离
\captionsetup[table]{skip=2pt}

  % 设置表格的 caption 格式
\renewcommand{\thetable}{\thesection-\arabic{table}}
\captionsetup[table]{font=small,labelsep=space}

  % 定义 BIThesis \LaTeX 风格的 Logo
\usepackage{relsize}
\makeatletter
\def\matex@ssize{\larger[-1]\scshape}
\DeclareRobustCommand{\BIThesis}{
  \mbox{
    \kern-0.5em{B}\kern-0.05em
    {I}\kern-0.05em
    {T}\kern-0.1em
    \raisebox{-0.38ex}{\matex@ssize {H}}\kern-0.1em
    {\matex@ssize {E}}\kern-0.05em
    \raisebox{-0.38ex}{\matex@ssize {S}}\kern-0.05em
    {\matex@ssize {I}}\kern-0.05em
    \raisebox{-0.35ex}{\matex@ssize {S}}\kern-0.5em
    \kern 1ex
   }
}
\makeatother



%</dtx-style>
% \fi
%
%
% \Finale
\endinput
% \iffalse
%  Local Variables:
%  mode: doctex
%  TeX-master: t
%  End:
% \fi
