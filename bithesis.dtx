% \iffalse meta-comment
%
% Copyright (C) 2022
% Association of Bit Network Pioneer and any individual authors listed elsewhere in this file.
% -----------------------------------
%
% This work may be distributed and/or modified under the
% conditions of the LaTeX Project Public License, either
% version 1.3c of this license or (at your option) any later
% version. This version of this license is in
%    http://www.latex-project.org/lppl/lppl-1-3c.txt
% and the latest version of this license is in
%    http://www.latex-project.org/lppl.txt
% and version 1.3 or later is part of all distributions of
% LaTeX version 2020/11/27 or later.
%
% \fi
%
% \iffalse
%<cls>\RequirePackage{expl3,l3keys2e}
%<thesis>\ProvidesExplClass{bithesis}
%<proposal>\ProvidesExplClass{bitproposal}
%<report>\ProvidesExplClass{bitreport}
%<beamer>\ProvidesExplClass{bitbeamer}
%<cls>{2022-06-10}{3.0.0}{BIT Thesis Templates}
%
%<oldcls>\NeedsTeXFormat{LaTeX2e}[2020/10/01]
%<book>\ProvidesClass{bitbook}
%<article>\ProvidesClass{bitart}
%<graduate>\ProvidesClass{bitgrad}
%<oldcls> [2022/05/09 v2.1.1 BIT Thesis Templates]
%
%<*driver>
\ProvidesFile{bithesis.dtx}[2022/05/09 2.1.1 BIT Thesis Templates]
\documentclass[a4paper,full]{l3doc}
\usepackage{dtx-style}

\EnableCrossrefs
\CodelineIndex

\RecordChanges
\begin{document}
  \DocInput{\jobname.dtx}
  \PrintChanges
  \def\indexname{代码索引}
  \PrintIndex
\end{document}
%</driver>
% \fi
%
% \DoNotIndex{\newenvironment,\@bsphack,\@empty,\@esphack,\sfcode}
% \DoNotIndex{\addtocounter,\label,\let,\linewidth,\newcounter}
% \DoNotIndex{\noindent,\normalfont,\par,\parskip,\phantomsection}
% \DoNotIndex{\providecommand,\ProvidesPackage,\refstepcounter}
% \DoNotIndex{\RequirePackage,\setcounter,\setlength,\string,\strut}
% \DoNotIndex{\textbackslash,\texttt,\ttfamily,\usepackage}
% \DoNotIndex{\begin,\end,\begingroup,\endgroup,\par,\\}
% \DoNotIndex{\if,\ifx,\ifdim,\ifnum,\ifcase,\else,\or,\fi}
% \DoNotIndex{\let,\def,\xdef,\edef,\newcommand,\renewcommand}
% \DoNotIndex{\expandafter,\csname,\endcsname,\relax,\protect}
% \DoNotIndex{\Huge,\huge,\LARGE,\Large,\large,\normalsize}
% \DoNotIndex{\small,\footnotesize,\scriptsize,\tiny}
% \DoNotIndex{\normalfont,\bfseries,\slshape,\sffamily,\interlinepenalty}
% \DoNotIndex{\textbf,\textit,\textsf,\textsc}
% \DoNotIndex{\hfil,\par,\hskip,\vskip,\vspace,\quad}
% \DoNotIndex{\centering,\raggedright,\ref}
% \DoNotIndex{\c@secnumdepth,\@startsection,\@setfontsize}
% \DoNotIndex{\ ,\@plus,\@minus,\p@,\z@,\@m,\@M,\@ne,\m@ne}
% \DoNotIndex{\@@par,\DeclareOperation,\RequirePackage,\LoadClass}
% \DoNotIndex{\AtBeginDocument,\AtEndDocument,\AtBeginEnvironment}
%
% \GetFileInfo{\jobname.dtx} %
% 
% \def\indexname{索引}
% \IndexPrologue{\section{\indexname}}
%
% \title{\includegraphics[width=0.3\textwidth]{images/icon.png}
% \\[1cm]
% \bfseries 北京理工大学{\LaTeX}学位论文及报告模板 }
% \author{北京理工大学网络开拓者协会 \\ \texttt{webmaster@bitnp.net}} %
% \date{\zihao{-4} \today\quad \color{RubineRed}{\kaishu {\BIThesis}版本\version}}
% \maketitle\thispagestyle{empty}
%
% \def\abstractname{}
% \begin{abstract}\noindent
%   此宏包旨在建立一个简单易用的北京理工大学学位论文模板(以及其他模板),包括本科毕业设计与研究生论文。
% \end{abstract}
%
% \vspace{5mm}
%
% \begin{center}
% \noindent\rule[0.25\baselineskip]{0.5\textwidth}{0.7pt}
% \end{center}
% 
% \def\abstractname{免责声明}
% \begin{abstract}
% \noindent
% \begin{enumerate}
% \item 本模板的发布遵守 \LPPL ,使用前请认真阅读协议内容。
% \item 与\BIThesis 相关的文档内容采用 \href{https://github.com/BITNP/BIThesis-wiki/blob/main/LICENSE}{CC0-1.0 协议} 发布。
% \item 任何个人或组织以本模板为基础进行修改、扩展而生成的新的专用模板,请严格遵
%   守 \LaTeX{} Project Public License 协议。由于违犯协议而引起的任何纠纷争端均与
%   本模板作者无关。
% \end{enumerate}
% \end{abstract}
%
% \vspace{5mm}
%
% \def\abstractname{简介}
% \begin{abstract}
% \BIThesisLaTeX 是北京理工大学本科生毕业设计与研究生学位论文,以及其他课程报告、实验报告的 {\LaTeX} 模板集合。
% 如果你厌烦了 Word 格式的不人性化、参考文献的难以管理、公式输入的差劲体验……那么欢迎来尝试用专业的学术稿件排版利器 —— {\LaTeX},来排版你的论文。
% 专业高端、学界认可、开源免费,{\LaTeX} 是你论文排版的最佳搭档。
%
% \BIThesisLaTeX 目前支持使用 {\hologo{XeLaTeX}} 进行编译,使用以 biber 为后端的 BibLaTeX 进行参考文献的生成,
% 符合《信息与文献参考文献著录规则》
% (\href{http://openstd.samr.gov.cn/bzgk/gb/newGbInfo?hcno=7FA63E9BBA56E60471AEDAEBDE44B14C}{GB/T 7714—2015})的标准。
% 目前主要设计完成了计算机学院本科生毕业论文开题报告、毕业设计毕业论文与通用实验报告的 {\LaTeX} 模板。
%
% \end{abstract}
% \newpage
%
% \tableofcontents
% \clearpage
% \setlength{\parskip}{0.8ex}
%
% \section{项目简介}
% \subsection{历史与贡献者们}
% \begin{itemize}
%   \item 2017 - 2018 年,杨雅婷等人受研究生院委托,制作了\href{https://github.com/BIT-thesis/LaTeX-template}{BIT-Thesis} 研究生学位论文模板。
%   \item 2019 - 2020 年,\BIThesis 最早由 2016 级的武上博、王赞、唐誉铭、牟思睿和詹熠莎等人维护。
%   \begin{itemize}
%     \item 在此期间,\BIThesis 从无到有诞生了,包括使用手册、在线文档和开箱即用的模板。
%     \item 同时,2017 级的赵池等同学完成了一系列 \BIThesisLaTeX 的视频教程。
%   \end{itemize}
%   \item 2020 - 2021 年,2017 级的冯开宇、杨思云、郝正亮和顾骁等人接管了维护开发工作。
%   \begin{itemize}
%     \item 在此期间,冯开宇将原来的 .tex 文件制作成了宏包,并发布到 CTAN 上。
%     \item 项目代码也随之被拆分成了 \BIThesisMacroPackage,\BIThesisWiki 和 \BIThesisScaffold。
%   \end{itemize}
%   \item 2021 - 2022 年,2021 级(硕士研究生)的冯开宇针对 2021、2022 毕业季收到的反馈对该项目进行维护升级。
%   \begin{itemize}
%      \item 在此期间,冯开宇合入了杨雅婷等人在 2017 年开发的研究生学位论文模板。
%      \item 在项目架构上,BIThesis-scaffold 合入 BIThesis 以便于进一步维护。 
%      \item 次年暑假期间,冯开宇用 \pkg{expl3} 重构了\LaTeX 样式代码,向用户提供了简易易用的接口。
%      \item 同时,也增加了本科全英文专业的毕设论文模板样式。
%   \end{itemize}
% \end{itemize}
% \subsection{\BIThesis 是什么?}
% \BIThesis 之名是英文单词 Beijing Institution of Technology(北京理工大学)的首字母缩写“BIT” 与“Thesis”结合而成。在纯文本环境下,该名字应写作“BIThesis”。
%
% \BIThesisLaTeX 是由北京理工大学众多学子发起并维护的开源项目。该项目旨在建立一套简单易用的北京理工大学 \LaTeX 学位论文模板,包括本科综合论文训练。
% \subsubsection{\BIThesisLaTeX 的组成}
% 我们将 \BIThesisLaTeX 划分为了两个主要仓库:
% \begin{table}[H]
% \centering
% \begin{tabular}{@{}l l p{6cm} @{}}
% \toprule
% 项目                & 项目地址 & 主要目的 \\ \midrule
% BIThesis          &   \href{https://github.com/BITNP/BIThesis}{BITNP/BIThesis}   &  主要存储 \BIThesis  宏包以及开箱即用的模板样式 \\
% BIThesis-wiki     &   \href{https://github.com/BITNP/BIThesis-wiki}{BITNP/BIThesis-wiki}  &  存储 \BIThesisLaTeX 项目在线文档   \\ \bottomrule
% \end{tabular}
% \end{table}
%
% 如果你仅想解决「我如何使用 \BIThesisLaTeX 来帮助我完成实验论文?」这个问题,那么欢迎你访问我们的\href{https://bithesis.bitnp.net}{在线文档}以获得更多信息。 
%
% 如果你想深入了解 \BIThesisLaTeX 提供的接口的各种选项,那么请继续阅读。
% 
% \section{使用说明}
% \subsection{\BIThesis 宏包的组成}
% 为了适应用户的不同需求,并符合 CTeX 宏集的设计习惯,我们将 \BIThesisMacroPackage 的主要功能设计安排在两个中文文档类当中,具体的组成见 \ref{tab:classes}。
% \begin{table}[H]
% \centering
% \caption{测试}
% \label{tab:classes}
% \begin{tabular}{@{}lll@{}}
% \toprule
% 类别                   & 文件          & 说明                             \\ \midrule
% \multirow{2}{*}{文档类} & \cls{bithesis.cls}  & 封装本科生与研究生的毕业论文样式。 \\
%                      & \cls{bitproposal.cls} & 封装本科生开题报告样式。     \\ \cmidrule(l){2-3} 
%                    & \cls{bitreport.cls} & 对应 封装了一个北理工实验报告样式。     \\ \cmidrule(l){2-3}
%                    & \cls{bitbeamer.cls} & 对应 ctexbeamer.cls ,提供了北理工的 Beamer 模板样式。     \\ \cmidrule(l){2-3}
% \end{tabular}
% \end{table}
% \subsection{\BIThesis 宏包的安装和更新}
% 最常见的 \TeX 发行版(\hologo{TeX} Live 和 \hologo{MiKTeX})已收录\BIThesisMacroPackage 及其依赖的宏包和宏集。
%
% 如果安装以上发行版的时间较早,可能你本地的环境中不存在 \BIThesisMacroPackage 或者不是最新版本的。那么你需要通过包管理器来安装/更新 \BIThesisMacroPackage:
% \begin{shell}[morekeywords={tlmgr,install}]
%   tlmgr install fduthesis
% \end{shell}
% \subsection{使用 \BIThesis 文档类}
% 推荐使用\BIThesisRelease (开箱即用)。\BIThesisRelease 提供了多种最常用的模板,你可以在\href{https://github.com/BITNP/BIThesis/releases}{主项目的 Releases}中找到它们。 
% 
% \subsection{\cls{bithesis} 使用与配置}
% 
% 使用此文档类的模板有:
% \begin{itemize}
%  \item \BIThesisTemplates{UT}
%  \item \BIThesisTemplates{UTE}
%  \item \BIThesisTemplates{PT}
%  \item \BIThesisTemplates{GT}
% \end{itemize}
%
% \subsubsection{最小用例}
% 
% \begin{latex}
%   \documentclass[type=bachelor]{bithesis}
%   \BITSetup{
%     info = {
%       author = FKY,
%       ......
%     }
%   }
%   \begin{document}
%   \end{document}
% \end{latex}
% \subsubsection{模板选项}
% % 所谓“模板选项”,指需要在引入文档类的时候指定的选项:
% \begin{latex}[deletetexcs={\documentclass},morekeywords={\documentclass}]
%   \documentclass(*\oarg{模板选项}*){bithesis}
% \end{latex}
% \begin{function}{type}
%   \begin{bitsyntax}[emph={[1]type}]
%     type = (*<(bachelor)|bachelor_translation|bachelor_english|master|docter>*)
%   \end{bitsyntax}
%   选择论文类型
% \end{function}
% 
% \section{致谢}
% \section{软件许可证}
% \begin{itemize}
%   \item 北京理工大学校徽校名图片的版权归北京理工大学所有。
%   \item \BIThesisLaTeX 宏包以及相关文档类使用 \LPPL 授权。
%   \item \BIThesisLaTeX 文档及其他附属文件通过 CC0-1.0 授权。
% \end{itemize}
% \section{实现细节}
%
%    \begin{macrocode}
%<*package>
%    \end{macrocode}
%
% Identify the internal prefix (\LaTeX3 \pkg{DocStrip} convention).
%    \begin{macrocode}
%<@@=bithesis>
%    \end{macrocode}
%
% \begin{macro}{\YOURMACRO}
% Put explanation of |\YOURMACRO|’s implementation here.
%    \begin{macrocode}
\newcommand{\YOURMACRO}{}
%    \end{macrocode}
% \end{macro}
%
% \begin{environment}{YOURENV}
% Put explanation of |YOURENV|’s implementation here.
%    \begin{macrocode}
\newenvironment{YOURENV}{}{}
%    \end{macrocode}
% \end{environment}
%
%    \begin{macrocode}
%</package>
%    \end{macrocode}

%    \begin{macrocode}
%<*thesis>
%    \end{macrocode}

% Define Global Variables
%    \begin{macrocode}
\int_new:N \g_@@_thesis_type_int
\int_new:N \g_@@_head_zihao_int
\bool_new:N \g_@@_twoside_bool
\bool_new:N \g_@@_thesis_type_english_bool
\bool_new:N \g_@@_blind_mode_bool
\tl_new:N \g_@@_label_devide_char_tl

% \bool_new:N \l_@@_cover_auto_width_bool
% \bool_set_true:N \l_@@_cover_auto_width_bool
%
% Define tmp Variables
\seq_new:N \l_@@_right_seq
\seq_new:N \l_@@_left_seq

% helper functions

\cs_generate_variant:Nn \tl_if_empty:nTF {e}
\cs_generate_variant:Nn \seq_set_split:Nnn {Nne}

\cs_new:Npn \@@_same_page: {
  \let\clearpage\relax
  \let\cleardoublepage\relax
}

% 是否为研究生学位论文
\cs_new:Npn \@@_if_graduate:TF #1#2 {
    \int_compare:nNnTF {3} < {\g_@@_thesis_type_int}
      {#1}
      {#2}
  }

% 是否某一特定模板
\cs_new:Npn \@@_if_thesis_int_type:nTF #1#2#3 {\int_compare:nNnTF {\g_@@_thesis_type_int} = {#1} {#2} {#3}}
\cs_new:Npn \@@_if_thesis_int_type:nT #1#2 {\@@_if_thesis_int_type:nTF {#1} {#2} {}}

% 是否为英文模板,这里包括全英文专业和研究生模板的英文模式。
\cs_new:Npn \@@_if_thesis_english:TF #1#2 {\bool_if:nTF {\g_@@_thesis_type_english_bool} {#1} {#2}}
\cs_new:Npn \@@_if_thesis_english:T #1 {\@@_if_thesis_english:TF {#1}{}}

\cs_new:Npn \@@_if_bachelor_thesis:TF #1#2 {\int_compare:nNnTF {\g_@@_thesis_type_int} < {4} {#1} {#2}}
\cs_new:Npn \@@_if_bachelor_thesis:T #1 {\@@_if_bachelor_thesis:TF {#1} {}}
\cs_new:Npn \@@_if_master_thesis:TF #1#2 {\int_compare:nNnTF {\g_@@_thesis_type_int} = {4} {#1} {#2}}
\cs_new:Npn \@@_if_doctor_thesis:TF #1#2 {\int_compare:nNnTF {\g_@@_thesis_type_int} = {5} {#1} {#2}}

% Define Consts.
\clist_const:Nn \c_@@_thesis_type_clist
    { bachelor, bachelor_translation, bachelor_english, master, docter}

\cs_new_protected:Npn \@@_define_label:nn #1#2
  { \tl_const:cn { c_@@_label_ #1 _tl } {#2} }

\cs_new_protected:Npn \@@_define_label_by_thesis_type:nnn #1#2#3 
  {
    \tl_const:cn { c_@@_ #1 _label_ #2 _tl } {#3}
  }

\cs_new_protected:Npn \@@_define_label:nnn #1#2#3
  {
    \tl_const:cn { c_@@_label_ #1    _tl } {#2}
    \tl_const:cn { c_@@_label_ #1 _en_tl } {#3}
  }

\cs_new_protected:Npn \@@_define_label_by_thesis_type:nnnn #1#2#3#4
  {
    \tl_const:cn { c_@@_ #1 _label_ #2 _tl } {#3}
    \tl_const:cn { c_@@_ #1 _label_ #2 _en_tl } {#4}
  }

\clist_map_inline:nn
  {
    {code} {代码},
    {udc} {UDC分类号:},
    {classification} {中国分类号:},
    {classified_level} {密级},
    {type} {种类},
  }
  {\@@_define_label:nn #1}

% bachelor
\clist_map_inline:nn
  {
    {title} {本科生毕业设计(论文)},
    {originality} {原创性声明},
    {originality_clause} {本人郑重声明:所呈交的毕业设计(论文),是本人在指导老师的指导下独立进行研究所取得的成果。除文中已经注明引用的内容外,本文不包含任何其他个人或集体已经发表或撰写过的研究成果。对本文的研究做出重要贡献的个人和集体,均已在文中以明确方式标明。\par~特此申明。},
    {authorization} {关于使用授权的声明},
    {authorization_clause} {本人完全了解北京理工大学有关保管、使用毕业设计(论文)的规定,其中包括:\circled{1}~学校有权保管、并向有关部门送交本毕业设计(论文)的原件与复印件;\circled{2}~学校可以采用影印、缩印或其它复制手段复制并保存本毕业设计(论文);\circled{3}~学校可允许本毕业设计(论文)被查阅或借阅;\circled{4}~学校可以学术交流为目的,复制赠送和交换本毕业设计(论文);\circled{5}~学校可以公布本毕业设计(论文)的全部或部分内容。},
    {originality_author_signature} {本人签名:\hspace{40mm}日\hspace{2.5mm}期:\hspace{13mm}年\hspace{8mm}月\hspace{8mm}日},
    {originality_supervisor_signature} {指导老师签名:\hspace{40mm}日\hspace{2.5mm}期:\hspace{13mm}年\hspace{8mm}月\hspace{8mm}日},
  } {\@@_define_label_by_thesis_type:nnn {bachelor} #1}

% % bachelor english
\clist_map_inline:nn
  {
    {title} {},
    {originality} {原创性声明~Statement~of~Originality},
    {originality_clause} {
        本人郑重声明:所呈交的毕业设计(论文),是本人在指导老师的指导下独立进行研究所取得的成果。除文中已经注明引用的内容外,本文不包含任何其他个人或集体已经发表或撰写过的研究成果。对本文的研究做出重要贡献的个人和集体,均已在文中以明确方式标明。特此申明。\par 
        \arialfamily I,\dunderline[-1pt]{1pt}{\makebox[18mm]{}},~solemnly~
        declare:~the~submitted~graduation~design~(thesis),~is~the~research~achievement~completed~independently~by~myself~
        under~the~guidance~of~the~supervisor.~This~article~does~not~contain~
        any~research~published~or~written~by~any~other~individual~or~group,~
        except~as~already~referenced~in~this~paper.~Individuals~and~groups~
        that~have~made~important~contributions~to~the~study~of~this~paper~
        are~clearly~indicated~and~cited~in~the~paper.\par
    },
    {authorization} {关于使用授权的声明~State~of~Use~Authorization},
    {authorization_clause} {
      本人完全了解北京理工大学有关保管、使用毕业设计(论文)的规定,其中包括:\circled{1}学校有权保管、并向有关部门送交本毕业设计(论文)的原件与复印件;\circled{2}学校可以采用影印、缩印或其它复制手段复制并保存本毕业设计(论文);\circled{3}学校可允许本毕业设计(论文)被查阅或借阅;\circled{4}学校可以学术交流为目的,复制赠送和交换本毕业设计(论文);\circled{5}学校可以公布本毕业设计(论文)的全部或部分内容。\par
  I~fully~understand~the~regulations~on~the~storage,~use~of~graduation~design~(thesis)~in~Beijing~Institute~of~Technology.~Beijing~Institute~of~Technology~has~the~right~to~(1)~keep,~and~to~the~relevant~departments~to~send~the~original~or~copy~of~this~graduation~design~(thesis);~(2)~copy~and~preserve~this~graduation~design~(thesis)~by~photocopying,~miniature~or~other~means~of~reproduction;~(3)~allow~this~graduation~design~(thesis)~to~be~read~or~borrowed;~(4)~for~the~purpose~of~academic~exchange,~copy,~give~and~exchange~this~graduation~design~(thesis);~(5)~publish~all~or~part~of~the~contents~of~this~graduation~design~(thesis).~
    },
  } {\@@_define_label_by_thesis_type:nnn {bachelor_english} #1}

\cs_new:Npn \smallgap: {
  \hspace{0.45ex}
}

\cs_new:Npn \label_space: {
  \@@_if_bachelor_thesis:T {
    \quad
  }
}

% graduate
\clist_map_inline:nn
  {
    {originality} {研究成果声明},
    {originality_clause} {本人郑重声明:所提交的学位论文是我本人在指导教师的指导下进行的研究工作获得的研究成果。尽我所知,文中除特别标注和致谢的地方外,学位论文中不包含其他人已经发表或撰写过的研究成果,也不包含为获得北京理工大学或其它教育机构的学位或证书所使用过的材料。与我一同工作的合作者对此研究工作所做的任何贡献均已在学位论文中作了明确的说明并表示了谢意。\par~特此申明。},
    {authorization} {关于学位论文使用权的说明},
    {authorization_clause} {本人完全了解北京理工大学有关保管、使用学位论文的规定,其中包括:\circled{1}~学校有权保管、并向有关部门送交学位论文的原件与复印件;\circled{2}~学校可以采用影印、缩印或其它复制手段复制并保存学位论文;\circled{3}~学校可允许学位论文被查阅或借阅;\circled{4}~学校可以学术交流为目的,复制赠送和交换学位论文;\circled{5}~学校可以公布学位论文的全部或部分内容(保密学位论文在解密后遵守此规定)。},
    {originality_author_signature} {签\qquad 名:\hspace{40mm}日\hspace{2.5mm}期:\hspace{30mm}\quad},
    {originality_supervisor_signature} {指导老师签名:\hspace{40mm}日\hspace{2.5mm}期:\hspace{30mm}\quad},
  } {\@@_define_label_by_thesis_type:nnn {graduate} #1}

\clist_map_inline:nn
  {
    {author} {作\quad 者\quad 姓\quad 名} {Candiate~Name},
    {school} {学\quad 院\quad 名\quad 称} {School~or~Department},
    {supervisor} {指\quad 导\quad 教\quad 师} {Faculty~Mentor},
    {chairman} {答辩委员会主席} {Chair,~Thesis~Committee},
    {degree} {申\smallgap: 请\smallgap: 学\smallgap: 位\smallgap: 级\smallgap: 别} {Degree~Applied},
    {major} {学\quad 科\quad 专\quad 业} {Major},
    {institute} {学\smallgap: 位\smallgap: 授\smallgap: 予\smallgap: 单\smallgap: 位} {Degree~by},
    {defense_date} {论\smallgap: 文\smallgap: 答\smallgap: 辩\smallgap: 日\smallgap: 期} {The~Date~of~Defence},
  } {\@@_define_label_by_thesis_type:nnnn {graduate} #1}

\clist_map_inline:nn 
  {
    {school} {学\qquad 院} {School},
    {major} {专\qquad 业} {Degree},
    {author} {学生姓名} {Author},
    {student_id} {学\qquad 号} {Student~ID},
    {supervisor} {指导教师} {Supervisor},
    {co_supervisor} {校外指导教师} {Co-Supervisor},
    {keywords} {关键词:} {Key~Words:~},
    {toc} {目\label_space: 录} {Table~of~Contents},
    {abstract} {摘\label_space: 要} {Abstract},
    {conclusion} {结\label_space: 论} {Conclusions},
    {appendix} {附\label_space: 录} {Appendices},
    {ack} {致\label_space: 谢} {Acknowledgement},
    {figure} {插\quad 图} {Illustrations},
    {table} {表\quad 格} {Tables},
    {appendix_prefix} {附录} {Appendix},
    {reference} {参考文献} {References},
    {university} {北京理工大学} {Beijing~Institute~of~Technology},
    {publications} {攻读学位期间发表论文与研究成果清单} {Publications~During~Studies},
    % TODO: Not so sure about the translation.
    {resume} {作者简介} {},
    {symbols} {主要符号对照表} {},
  }
  {\@@_define_label:nnn #1}

% TODO: \clist_item:Nn
\clist_const:Nn \c_@@_bachelor_thesis_header_clist
  {北京理工大学本科生毕业设计(论文), 北京理工大学本科生毕业设计(论文)外文翻译, Beijing~Institute~of~Technology~Bachelor's~Thesis }
\clist_const:Nn \c_@@_bachelor_thesis_cover_title_clist
  {
    本科生毕业设计(论文),
    本科生毕业设计(论文)外文翻译,
    Beijing\nobreak{~}Institute\nobreak{~}of\nobreak{~}Technology~Bachelor's~Thesis,
  }


% key-value interface definition.
\keys_define:nn { bithesis }
{
  info .meta:nn = { bithesis / info } {#1},
  misc .meta:nn = { bithesis / misc } {#1},
  cover .meta:nn = { bithesis / cover } {#1},
  style .meta:nn = { bithesis / style } {#1},
  option .meta:nn = { bithesis / option } {#1},
}

\keys_define:nn { bithesis / option }
{
  type .choice:,
  type .value_required:n = true,
  type .choices:Vn =
    \c_@@_thesis_type_clist
    { 
      \int_set_eq:NN \g_@@_thesis_type_int \l_keys_choice_int 
      \int_case:nn {\l_keys_choice_int} {
        % 本科全英文也是英文模板。
        {3} {\bool_set_true:N \g_@@_thesis_type_english_bool}
      }
    },
  type .initial:n = bachelor,
  twoside .bool_gset:N = \g_@@_twoside_bool,
  blindPeerReview .bool_gset:N = \g_@@_blind_mode_bool,
}

\keys_define:nn { bithesis / cover }
  {
    date .tl_set:N = \l_bit_coverdate_tl,
    headerImage .tl_set:N = \l_bit_coverheaderimage_tl,
    xiheiFont .tl_set:N = \l_@@_cover_xihei_font_path_tl,
    xiheiFont .default:n = {STXihei},
    %% cover entry
    dilimiter .tl_set:N = \l_@@_cover_dilimiter_tl,
    labelAlign .tl_set:N = \l_@@_cover_label_align_tl,
    labelAlign .initial:n = {r},
    valueAlign .tl_set:N = \l_@@_cover_value_align_tl,
    valueAlign .initial:n = {c},
    labelMaxWidth .dim_set:N = \l_@@_cover_label_max_width_dim,
    valueMaxWidth .dim_set:N = \l_@@_cover_value_max_width_dim,
    autoWidth .bool_set:N = \l_@@_cover_auto_width_bool,
    autoWidth .initial:n = {true},
    underlineThickness .dim_set:N = \l_@@_cover_underline_thickness_dim,
    underlineThickness .initial:n = {1pt},
    underlineOffset .dim_set:N = \l_@@_cover_underline_offset_dim,
    underlineOffset .initial:n = { -10pt },
  }

\keys_define:nn { bithesis / info }
  {
    title .tl_set:N = \l_@@_value_title_tl,
    titleEn .tl_set:N = \l_@@_value_title_en_tl,
    school .tl_set:N = \l_@@_value_school_tl,
    major .tl_set:N = \l_@@_value_major_tl,
    author .tl_set:N = \l_@@_value_author_tl,
    studentId .tl_set:N = \l_@@_value_student_id_tl,
    supervisor .tl_set:N = \l_@@_value_supervisor_tl,
    externalSupervisor .tl_set:N = \l_@@_value_external_supervisor_tl,
    keywords .tl_set:N = \l_@@_value_keywords_tl,
    keywordsEn .tl_set:N = \l_@@_value_keywords_en_tl,
    translationTitle .tl_set:N = \l_@@_value_trans_title_tl,
    translationOriginTitle .tl_set:N = \l_@@_value_trans_origin_title_tl,
    % 中国分类号,研究生学位论文使用
    classification .tl_set:N = \l_@@_value_classification_tl,
    % UDC 分类号,研究生学位论文使用
    UDC .tl_set:N = \l_@@_value_udc_tl,
    chairman .tl_set:N = \l_@@_value_chairman_tl,
    degree .tl_set:N = \l_@@_value_degree_tl,
    degreeEn .tl_set:N = \l_@@_value_degree_en_tl,
    institute .tl_set:N = \l_@@_value_institute_tl,
    institute .initial:n = {\c_@@_label_university_tl},
    defenseDate .tl_set:N = \l_@@_value_defense_date_tl,
    authorEn .tl_set:N = \l_@@_value_author_en_tl,
    schoolEn .tl_set:N = \l_@@_value_school_en_tl,
    supervisorEn .tl_set:N = \l_@@_value_supervisor_en_tl,
    chairmanEn .tl_set:N = \l_@@_value_chairman_en_tl,
    majorEn .tl_set:N = \l_@@_value_major_en_tl,
    instituteEn .tl_set:N = \l_@@_value_institute_en_tl,
    instituteEn .initial:n = {\c_@@_label_university_en_tl},
    defenseDateEn .tl_set:N = \l_@@_value_defense_date_en_tl,
    classified_level .tl_set:N = \l_@@_value_classified_level_tl,
  }

\keys_define:nn { bithesis / misc }
  {
    date .tl_set:N = \l_bit_date_tl,
    arialFont .tl_set:N = \l_@@_misc_arial_font_path_tl,
  }

\keys_define:nn { bithesis / style }
{
  head .tl_set:N = \l_@@_style_head_tl,
  head .initial:n = {
    \int_case:nn {\g_@@_thesis_type_int}
    {
      {1} {北京理工大学本科生毕业设计(论文)}
      {2} {北京理工大学本科生毕业设计(论文)外文翻译}
      {3} {Beijing~Institute~of~Technology~Bachelor's~Thesis}
      {4} {北京理工大学硕士学位论文}
      {5} {北京理工大学博士学位论文}
    }
  }
}

\ProcessKeysOptions { bithesis / option }

\@@_if_thesis_english:T {
  \PassOptionsToClass{scheme=plain}{ctexbook}
}

\bool_if:NTF \g_@@_twoside_bool {} {
  \PassOptionsToClass{oneside,openany}{ctexbook}
}

% Any extra option passed by user will be passed to ctexbook.
\DeclareOption*{
  \PassOptionsToClass{\CurrentOption}{ctexbook}
}
% Executes the code for each option.
\ProcessOptions\relax
% Load
\LoadClass[zihao=-4,]{ctexbook}

\RequirePackage{geometry}
\RequirePackage{xeCJK}
\RequirePackage{titletoc}
\RequirePackage{setspace}
\RequirePackage{graphicx}
\RequirePackage{fancyhdr}
\RequirePackage{pdfpages}
\RequirePackage{setspace}
\RequirePackage{booktabs}
\RequirePackage{multirow}
\RequirePackage{tikz}
\RequirePackage{etoolbox}
\RequirePackage{hyperref}
\RequirePackage{xcolor}
\RequirePackage{caption}
\RequirePackage{array}
\RequirePackage{amsmath}
\RequirePackage{amssymb}
\RequirePackage{pdfpages}
\RequirePackage{listings}
\RequirePackage{enumitem}
\RequirePackage{environ}

\@@_if_graduate:TF {
  \int_set:Nn \g_@@_head_zihao_int {5}
  \geometry{
    a4paper,
    left=2.7cm,
    bottom=2.5cm + 7bp,
    top=3.5cm + 7bp,
    right=2.7cm,
    headsep = 3.5cm + 7bp - 2.5cm - 15bp,
    headheight = 15 bp,
    footskip = 2.5cm + 7bp - 1.8cm,
  }
} {
  \int_set:Nn \g_@@_head_zihao_int {4}
  \geometry{
    a4paper,
    left=3cm,
    bottom=2.6cm + 7bp,
    top=3.5cm + 7bp,
    right=2.6cm,
    headsep = 3.5cm + 7bp - 2.4cm - 20bp,
    headheight = 20 bp,
    footskip = 2.6cm + 7bp - 2cm,
  }
}

% One blank line before the figure and after the caption.
\setlength{\intextsep}{2\baselineskip plus 0.2\baselineskip minus 0.2\baselineskip}

\setromanfont{Times~New~Roman}

\ctex_at_end_preamble:n {
  \@@_if_thesis_english:TF {
    \@@_if_thesis_int_type:nT {3} {
      % font Arial needed
      \newfontfamily\arialfamily{Arial}
    }

  } {
    \tl_if_blank:VTF \l_@@_cover_xihei_font_path_tl {} 
    {
      \setCJKfamilyfont{xihei}[AutoFakeBold,AutoFakeSlant]{\l_@@_cover_xihei_font_path_tl}
    }
  }

  \@@_if_thesis_int_type:nT {3} {
    \RequirePackage[en-US]{datetime2}
    \RequirePackage{indentfirst}
    \DTMlangsetup[en-US]{dayyearsep={\space}}
  }

  % Define biblatex category if it was imported.
  \cs_if_exist:NT \DeclareBibliographyCategory {
    \DeclareBibliographyCategory{mypub}
  }
}

\cs_new:Npn \xihei:n #1 {
  \xeCJK_family_if_exist:nTF {xihei} {
    \CJKfamily{xihei} #1
  }{
    \heiti #1
  }
}

\cs_new:Npn \l_@@_title_font_cs:n #1 {
  \int_compare:nNnTF {\g_@@_thesis_type_int} = {3}
  {
    \arialfamily #1
  } {
    \heiti #1
  }
}

\cs_new:Npn \l_@@_unnumchapter_style_cs:n #1 {
  % 本科全英文、研究生学位论文需要加粗
  \int_compare:nNnTF {\g_@@_thesis_type_int} > {2}
  {
    \bfseries #1
  } {
    \mdseries #1
  }
}

\cs_set:Npn \arabicHeiti #1 {#1}

% TODO: custom title
\fancypagestyle{BIThesis}{
  \fancyhf{}
  % 定义页眉、页码
  \fancyhead[C]{\zihao{\int_use:N \g_@@_head_zihao_int}\ziju{0.08}\songti{\tl_use:N \l_@@_style_head_tl}}
  \fancyfoot[C]{\songti\zihao{5} \thepage}
  % 页眉分割线稍微粗一些
  \RenewDocumentCommand \headrulewidth {} {0.6pt}
}

\ctexset{chapter={
    number = {\arabicHeiti{ \arabic{chapter} }},
    format = { \l_@@_title_font_cs:n \bfseries \centering \zihao{3}},
    nameformat = {},
    titleformat = {},
    aftername = \hspace{9bp},
    pagestyle = BIThesis,
    beforeskip = 8bp,
    afterskip = 32bp,
    fixskip = true,
  }
}

\ctexset{section={
    number = {\arabicHeiti{\thechapter.\hspace{1bp}\arabic{section}}},
    format = {\l_@@_title_font_cs:n \raggedright \bfseries \zihao{4}},
    nameformat = {},
    titleformat = {},
    aftername = \hspace{8bp},
    beforeskip = 20bp plus 1ex minus .2ex,
    afterskip = 18bp plus .2ex,
    fixskip = true,
  }
}

\ctexset{subsection={
    number = {\arabicHeiti{\thechapter.\hspace{1bp}\arabic{section}.\hspace{1bp}\arabic{subsection}}},
    format = {\l_@@_title_font_cs:n \bfseries \raggedright \zihao{-4}},
    nameformat = {},
    titleformat = {},
    aftername = \hspace{7bp},
    beforeskip = 17bp plus 1ex minus .2ex,
    afterskip = 14bp plus .2ex,
    fixskip = true,
  }
}

\ctexset{
  secnumdepth = 3,
  subsubsection={
    numbering = true,
    number = {\arabicHeiti{\arabic{chapter}.\hspace{1bp}\arabic{section}.\hspace{1bp}\arabic{subsection}.\hspace{1bp}\arabic{subsubsection}}},
    format={\l_@@_title_font_cs:n \bfseries \raggedright \zihao{-4}},
    nameformat = {},
    titleformat = {},
    beforeskip=28bp plus 1ex minus .2ex,
    afterskip=24bp plus .2ex,
    fixskip=true,
  }
}

% TOC
\addtocontents{toc}{\protect\hypersetup{hidelinks}}

% \RenewDocumentCommand \contentsname {} {
%   \fontsize{16pt}{\baselineskip}
%   \l_@@_unnumchapter_style_cs:n\l_@@_title_font_cs:n{\l_@@_toc_title_tl}
%   \vspace{-8pt}
% }

\@@_if_graduate:TF {
  % 各章标题,宋体四号
  \titlecontents{chapter}[0pt]{\songti \zihao{4}}
  {\thecontentslabel\hspace{\ccwd}}{}
  {\hspace{.5em}\titlerule*{.}\contentspage}
} {
  \titlecontents{chapter}[0pt]{\songti \zihao{-4}}
  {\thecontentslabel\hspace{\ccwd}}{}
  {\hspace{.5em}\titlerule*{.}\contentspage}
}
\titlecontents{section}[1\ccwd]{\songti \zihao{-4}}
{\thecontentslabel\hspace{\ccwd}}{}
{\hspace{.5em}\titlerule*{.}\contentspage}
\titlecontents{subsection}[2\ccwd]{\songti \zihao{-4}}
{\thecontentslabel\hspace{\ccwd}}{}
{\hspace{.5em}\titlerule*{.}\contentspage}


\bool_new:N \l_@@_add_to_toc_bool
\bool_set_true:N \l_@@_add_to_toc_bool

\keys_define:nn { bit }
  {
    abstract .meta:nn = { bit / abstract } {#1},
    abstract_en .meta:nn = { bit / abstract_en } {#1},
  }

\keys_define:nn { bit / abstract }
  {
    addTOC .bool_set:N = \l_@@_add_to_toc_bool,
  }

\keys_define:nn { bit / abstract_en }
  {
    addTOC .bool_set:N = \l_@@_add_to_toc_bool,
  }

\keys_define:nn { bit / symbols }
  {
    addTOC .bool_set:N = \l_@@_add_to_toc_bool,
  }

\RenewDocumentCommand \frontmatter {} {

  \int_compare:nNnTF {\g_@@_thesis_type_int} = {3}
  {
    \pagenumbering{roman}
  } {
    \pagenumbering{Roman}
  }
  \ctexset{
    chapter = {
      numbering = false,
    }
  }
  \pagestyle{BIThesis}
}

\RenewDocumentCommand \mainmatter {} {
  \cleardoublepage

  \ctexset{
    chapter = {
      numbering = true,
    }
  }
  \pagenumbering{arabic}
  \pagestyle{BIThesis}
  % 正文 22 磅的行距
  \setlength{\parskip}{0em}
  \setstretch{1.53}
  % 修复脚注出现跨页的问题
  \interfootnotelinepenalty=10000
}

\RenewDocumentCommand \backmatter {} {
  \setcounter{section}{0}
  \setcounter{subsection}{0}
  \setcounter{subsubsection}{0}
  \ctexset{
    chapter = {
      numbering = false,
      beforeskip = 18bp,
      format = {\l_@@_title_font_cs:n \l_@@_unnumchapter_style_cs:n \centering \zihao{3}},
      afterskip = 26bp,
    }
  }
}

\setlength{\abovecaptionskip}{11pt}
\setlength{\belowcaptionskip}{9pt}

\@@_if_graduate:TF {
  \tl_set:Nn \g_@@_label_devide_char_tl {.}
} {
  \tl_set:Nn \g_@@_label_devide_char_tl {-}
}

% figure
\cs_set:Npn \thefigure {\thechapter\g_@@_label_devide_char_tl\arabic{figure}}
\captionsetup[figure]{font=small,labelsep=space}

% table
\cs_set:Npn \thetable {\thechapter\g_@@_label_devide_char_tl\arabic{table}}
\captionsetup[table]{font=small,labelsep=space,skip=2pt}

% equation
\cs_set:Npn \theequation {\thechapter\g_@@_label_devide_char_tl\arabic{equation}}

% code snippet
\cs_set:Npn \thelstlisting {\thechapter\g_@@_label_devide_char_tl\arabic{lstlisting}}
\cs_set:Npn \lstlistingname {\c_@@_label_code_tl}


% 调整底层 TeX 排版引擎参数以保证所有段落能够很好地以两端对齐的方式呈现
\tolerance=1
\emergencystretch=\maxdimen
\hyphenpenalty=10000
\hbadness=10000

\definecolor{codegreen}{rgb}{0,0.6,0}
\definecolor{codegray}{rgb}{0.5,0.5,0.5}
\definecolor{codepurple}{rgb}{0.58,0,0.82}
\definecolor{backcolour}{rgb}{0.95,0.95,0.92}
\lstdefinestyle{examplestyle}{
    backgroundcolor=\color{backcolour},
    commentstyle=\color{codegreen},
    keywordstyle=\color{magenta},
    numberstyle=\tiny\color{codegray},
    stringstyle=\color{codepurple},
    basicstyle=\ttfamily\footnotesize,
    breakatwhitespace=false,
    breaklines=true,
    captionpos=b,
    keepspaces=true,
    numbers=left,
    numbersep=5pt,
    showspaces=false,
    showstringspaces=false,
    showtabs=false,
    tabsize=2
}
% TODO: optional
\lstset{style=examplestyle}

% 调整插图目录与表格目录的标题
\cs_set:Npn \listfigurename {\c_@@_label_figure_tl}
\cs_set:Npn \listtablename {\c_@@_label_table_tl}

%%%%%%%%%%%%%%%%%%%%%%%%%%%%%%%%%%%%

% user interface.
\DeclareDocumentCommand \BITSetup { m }
  { \keys_set:nn { bithesis } { #1 }}

\cs_new:Npn \@@_render_cover_entry:nn #1#2 {
  \makebox[\l_@@_cover_label_max_width_dim][\l_@@_cover_label_align_tl]{
    \tl_if_blank:VTF #1 {} {#1\l_@@_cover_dilimiter_tl}
  }
  \hspace{1ex}
  \@@_dunderline:nnn{\l_@@_cover_underline_offset_dim}{\l_@@_cover_underline_thickness_dim}{
    \makebox[\l_@@_cover_value_max_width_dim][\l_@@_cover_value_align_tl]{#2}
  }\par
}

% Get text with from #2, then set to #1.
\cs_new:Npn \@@_get_text_width:Nn #1#2
  {
    \hbox_set:Nn \l_tmpa_box {#2}
    \dim_set:Nn #1 { \box_wd:N \l_tmpa_box }
  }
\cs_generate_variant:Nn \@@_get_text_width:Nn { NV }

% Get max text width from seq #2, then set to #1.
\cs_new:Npn \@@_get_max_text_width:NN #1#2
  {
% 这里用 |group| 确保局部变量不会被污染。
    \group_begin:
      \seq_set_eq:NN \l_@@_tmpa_seq #2
      \dim_zero_new:N \l_@@_tmpa_dim
      \bool_until_do:nn { \seq_if_empty_p:N \l_@@_tmpa_seq }
        {
          \seq_pop_left:NN \l_@@_tmpa_seq \l_@@_tmpa_tl
          \@@_get_text_width:NV \l_@@_tmpa_dim \l_@@_tmpa_tl
          \dim_gset:Nn #1 { \dim_max:nn {#1} { \l_@@_tmpa_dim } }
        }
    \group_end:
  }

% process label (#1) and value #2 seperately
\cs_new:Npn \@@_parse_entry #1 #2 {
  \seq_set_split:Nne \l_@@_tmp_right_seq {//} {#2}
  \seq_clear:N \l_@@_tmp_left_seq
  \seq_map_inline:Nn \l_@@_tmp_right_seq {
    \seq_put_right:Nn \l_@@_tmp_left_seq {}
  }
  \seq_put_left:Nn \l_@@_tmp_left_seq {#1}
  \seq_pop_right:NN \l_@@_tmp_left_seq \g_@@_trashcan_tl
}

\cs_new:Npn \@@_render_cover_entry:n #1 {
  \seq_set_from_clist:NN \l_@@_input_seq #1
  % parse newline //
  \seq_map_inline:Nn \l_@@_input_seq {
    \@@_parse_entry ##1
    \seq_concat:NNN \l_@@_right_seq \l_@@_right_seq \l_@@_tmp_right_seq
    \seq_concat:NNN \l_@@_left_seq \l_@@_left_seq \l_@@_tmp_left_seq
  }

  \bool_if:NT \l_@@_cover_auto_width_bool {
    \@@_get_max_text_width:NN \l_@@_cover_label_max_width_dim \l_@@_left_seq
    \@@_get_max_text_width:NN \l_@@_cover_value_max_width_dim \l_@@_right_seq
  }
  

  \group_begin:
    \bool_until_do:nn { \seq_if_empty_p:N \l_@@_left_seq }
      {
        \seq_pop_left:NN \l_@@_left_seq \l_@@_tmpa_tl
        \seq_pop_left:NN \l_@@_right_seq \l_@@_tmpb_tl
        \tl_if_empty:eTF \l_@@_tmpb_tl {} {
          \@@_render_cover_entry:nn {\l_@@_tmpa_tl} {\l_@@_tmpb_tl}
        }
      }
  \group_end:
}

% #1: position
% #2: line_thickness
% #3: token list
\cs_new:Npn \@@_dunderline:nnn #1#2#3 {
  {\setbox0=\hbox{#3}\ooalign{\copy0\cr\rule[\dimexpr#1-#2\relax]{\wd0}{#2}}}
}

\cs_new:Npn \@@_dunderline:nn #1#2 {
  \@@_dunderline:nnn {#1} {1pt} {#2}
}

\cs_new:Npn \@@_dunderline:n #1 {
  \@@_dunderline:nnn {-10pt} {1pt} {#1}
}

\newcommand\dunderline[3][-1pt]{{%
  \setbox0=\hbox{#3}
  \ooalign{\copy0\cr\rule[\dimexpr#1-#2\relax]{\wd0}{#2}}}}

\NewEnviron{blindPeerReview}{
  \bool_if:NTF \g_@@_blind_mode_bool {} {
    \BODY
  }
}

% 重定义 \tn{cleardoublepage},使得偶数页面在没有内容时也不显示页眉页脚,见
% \url{https://tex.stackexchange.com/a/1683}。
\RenewDocumentCommand \cleardoublepage { }
  {
    \clearpage
    \bool_if:NT \g_@@_twoside_bool
      {
        \int_if_odd:nF \c@page
          { \hbox:n { } \thispagestyle { empty } \newpage }
      }
  }

\cs_new:Npn \make_graduate_cover: {
  \cleardoublepage
  \begin{titlepage}
    {
      \heiti\zihao{5}
      \tl_if_blank:VTF \l_@@_value_classified_level_tl {} {
        \flushright
        \c_@@_label_classified_level_tl:~
        \l_@@_value_classified_level_tl \par
      }
    }
    \centering
    \vspace*{65mm}
    {\heiti\zihao{-2} \l_@@_value_title_tl}
    \vskip 60mm
    {\heiti \zihao{-3} \l_@@_value_author_tl} % 黑体 小三
    \vskip 10mm
    {\heiti \zihao{-3} \l_bit_coverdate_tl} % 黑体 小三
  \end{titlepage}
}

\cs_new:Npn \make_paper_back: {
  \cleardoublepage
  \begin{titlepage}
   \vskip 5cm
   \begin{center}
    \setstretch{1.1}
    \begin{minipage}[t][19.7cm]{2em}
      \begin{center}
        {\heiti\zihao{3}\l_@@_value_title_tl}
          \vfill
        {\heiti\zihao{3}\l_@@_value_author_tl}
          \vfill
        {\heiti\zihao{3}\c_@@_label_university_tl}
      \end{center}
    \end{minipage}
   \end{center}
   % \vskip 5cm
  \end{titlepage}
}

\cs_new:Npn \@@_make_chinese_title_page: {
  \cleardoublepage
  \begin{titlepage}
      { %
        {\heiti \zihao{5} \noindent \c_@@_label_classification_tl} \l_@@_value_classification_tl\\
        {\heiti \zihao{5} \c_@@_label_udc_tl}  \l_@@_value_udc_tl
      }
     \begin{center}

      \vskip \stretch{1}
         {\heiti\zihao{-2} \l_@@_value_title_tl}
      \vskip \stretch{1}

      % TODO: delete this?
      {\fangsong\zihao{4}}
      \def\tabcolsep{1pt}
      \def\arraystretch{1.5}

      {
      \renewcommand{\baselinestretch}{2}

        \tl_if_empty:NT \l_@@_cover_dilimiter_tl {
          \tl_set:Nn \l_@@_cover_dilimiter_tl {\qquad}
        }
        \tl_set:Nn \l_@@_cover_underline_offset_dim {-5pt}
        % if not auto width, try override width
        \bool_if:NF \l_@@_cover_auto_width_bool {
          \dim_compare:nNnT {\l_@@_cover_label_max_width_dim} = {0pt} {
            \dim_set:Nn \l_@@_cover_label_max_width_dim {45mm}
          } 
          \dim_compare:nNnT {\l_@@_cover_value_max_width_dim} = {0pt} {
            \dim_set:Nn \l_@@_cover_value_max_width_dim {60mm}
          } 
        }

      \clist_set:Nn \l_@@_input_clist {
          {\c_@@_graduate_label_author_tl} {\l_@@_value_author_tl},
          {\c_@@_graduate_label_school_tl} {\l_@@_value_school_tl},
          {\c_@@_graduate_label_supervisor_tl} {\l_@@_value_supervisor_tl},
          {\c_@@_graduate_label_chairman_tl} {\l_@@_value_chairman_tl},
          {\c_@@_graduate_label_degree_tl} {\l_@@_value_degree_tl},
          {\c_@@_graduate_label_major_tl} {\l_@@_value_major_tl},
          {\c_@@_graduate_label_institute_tl} {\l_@@_value_institute_tl},
          {\c_@@_graduate_label_defense_date_tl} {\l_@@_value_defense_date_tl},
       }

      \heiti\zihao{-3}
      \@@_render_cover_entry:n \l_@@_input_clist
      }
    \end{center}
    \vskip \stretch{0.5}
  \end{titlepage}
}

\cs_new:Npn \@@_make_english_title_page: {
  \begin{titlepage}
    \begin{center}

    \vspace*{10em}
    {\zihao{-2}\textbf{\l_@@_value_title_en_tl}}
    % \bfseries
    \vskip \stretch{1}

    {
      \tl_if_empty:NT \l_@@_cover_dilimiter_tl {
        \tl_set:Nn \l_@@_cover_dilimiter_tl {:~}
      }

      \tl_set:Nn \l_@@_cover_label_align_tl {l}
      \tl_set:Nn \l_@@_cover_underline_offset_dim {-5pt}

      % if not auto width, try override width
      \bool_if:NF \l_@@_cover_auto_width_bool {
        \dim_compare:nNnT {\l_@@_cover_label_max_width_dim} = {0pt} {
          \dim_set:Nn \l_@@_cover_label_max_width_dim {55mm}
        } 
        \dim_compare:nNnT {\l_@@_cover_value_max_width_dim} = {0pt} {
          \dim_set:Nn \l_@@_cover_value_max_width_dim {85mm}
        } 
      }

      \clist_set:Nn \l_@@_input_clist {
          {\c_@@_graduate_label_author_en_tl} {\l_@@_value_author_en_tl},
          {\c_@@_graduate_label_school_en_tl} {\l_@@_value_school_en_tl},
          {\c_@@_graduate_label_supervisor_en_tl} {\l_@@_value_supervisor_en_tl},
          {\c_@@_graduate_label_chairman_en_tl} {\l_@@_value_chairman_en_tl},
          {\c_@@_graduate_label_degree_en_tl} {\l_@@_value_degree_en_tl},
          {\c_@@_graduate_label_major_en_tl} {\l_@@_value_major_en_tl},
          {\c_@@_graduate_label_institute_en_tl} {\l_@@_value_institute_en_tl},
          {\c_@@_graduate_label_defense_date_en_tl} {\l_@@_value_defense_date_en_tl},
       }

      \zihao{-3}
      \@@_render_cover_entry:n \l_@@_input_clist
    }

    \end{center}

    \vskip \stretch{0.5}
  \end{titlepage}
}

\DeclareDocumentCommand \MakeCover {}
  {
    \begin{blindPeerReview}
    \group_begin:

    \int_case:nn {\g_@@_thesis_type_int}
    {
      {1}
      {
        \begin{titlepage}
          \vspace*{16mm}

          \centering

          \tl_if_blank:VTF \l_bit_coverheaderimage_tl {} {
            \includegraphics[width=9.87cm]{\l_bit_coverheaderimage_tl}\\
          }

          \vspace*{-3mm}

          \zihao{-0}\textbf{\ziju{0.12}\songti{\c_@@_bachelor_label_title_tl}}\par

          \vspace{16mm}

          \zihao{2}\textbf{\xihei:n \l_@@_value_title_tl}\par

          \vspace{3mm}

          \begin{spacing}{1.2}
            \zihao{3}\selectfont{\textbf{\l_@@_value_title_en_tl}}\par
          \end{spacing}

          \vspace{15mm}


          \begin{spacing}{1.8}
            \begin{center}
            \tl_if_empty:NT \l_@@_cover_dilimiter_tl {
              \tl_set:Nn \l_@@_cover_dilimiter_tl {:}
            }
            % if not auto width, try override width
            \bool_if:NF \l_@@_cover_auto_width_bool {
              \dim_compare:nNnT {\l_@@_cover_label_max_width_dim} = {0pt} {
                \dim_set:Nn \l_@@_cover_label_max_width_dim {35mm}
              } 
              \dim_compare:nNnT {\l_@@_cover_value_max_width_dim} = {0pt} {
                \dim_set:Nn \l_@@_cover_value_max_width_dim {78mm}
              } 
            }

            \clist_set:Nn \l_@@_input_clist {
              {\c_@@_label_school_tl} {\l_@@_value_school_tl},
              {\c_@@_label_major_tl} {\l_@@_value_major_tl},
              {\c_@@_label_author_tl} {\l_@@_value_author_tl},
              {\c_@@_label_student_id_tl} {\l_@@_value_student_id_tl},
              {\c_@@_label_supervisor_tl} {\l_@@_value_supervisor_tl},
              {\c_@@_label_co_supervisor_tl} {\l_@@_value_external_supervisor_tl},
            }

            \zihao{3}

            \@@_render_cover_entry:n \l_@@_input_clist
            
            \end{center}
          \end{spacing}

          \vspace*{\fill}
          \centering
          \zihao{3}\ziju{0.5}\songti{\today}
        \end{titlepage}
      }
      {2}
      {
        \begin{titlepage}
          \centering

          \tl_if_blank:VTF \l_bit_coverheaderimage_tl {} {
            \includegraphics[width=6.87cm]{\l_bit_coverheaderimage_tl}\\
          }

          \vspace{1mm}

          \zihao{2}\textbf{\songti{本科生毕业设计(论文)外文翻译}}

          \vspace{8mm}

          {

          \begin{spacing}{1.8}
            
            \tl_set:Nn \l_@@_cover_dilimiter_tl {\textbf{:}}
            \bool_set_false:N \l_@@_cover_auto_width_bool
            \dim_set:Nn \l_@@_cover_label_max_width_dim {35mm}
            \dim_set:Nn \l_@@_cover_value_max_width_dim {115mm}

            \clist_set:Nn \l_@@_input_clist {
              {\zihao{4}\textbf{外文原文题目}} {\l_@@_value_trans_origin_title_tl},
              {\zihao{4}\textbf{中文翻译题目}} {\l_@@_value_trans_title_tl},
            }

            \zihao{-3}
            \centering

            \@@_render_cover_entry:n \l_@@_input_clist

          \end{spacing}

          }

          \vspace{14mm}

          \zihao{2}\textbf{\xihei:n \l_@@_value_title_tl}\par

          \vspace{3mm}

          \begin{spacing}{1.2}
            \zihao{3}\selectfont{\textbf{\l_@@_value_title_en_tl}}\par
          \end{spacing}

          \vspace{19mm}

          \begin{spacing}{1.8}
            \tl_if_empty:NT \l_@@_cover_dilimiter_tl {
              \tl_set:Nn \l_@@_cover_dilimiter_tl {:}
            }

            % if not auto width, try override width
            \bool_if:NF \l_@@_cover_auto_width_bool {
              \dim_compare:nNnT {\l_@@_cover_label_max_width_dim} = {0pt} {
                \dim_set:Nn \l_@@_cover_label_max_width_dim {35mm}
              } 
              \dim_compare:nNnT {\l_@@_cover_value_max_width_dim} = {0pt} {
                \dim_set:Nn \l_@@_cover_value_max_width_dim {78mm}
              } 
            }

            \zihao{3}

            \clist_set:Nn \l_@@_input_clist {
              {\c_@@_label_school_tl} {\l_@@_value_school_tl},
              {\c_@@_label_major_tl} {\l_@@_value_major_tl},
              {\c_@@_label_author_tl} {\l_@@_value_author_tl},
              {\c_@@_label_student_id_tl} {\l_@@_value_student_id_tl},
              {\c_@@_label_supervisor_tl} {\l_@@_value_supervisor_tl},
              {\c_@@_label_co_supervisor_tl} {\l_@@_value_external_supervisor_tl},
            }

            \@@_render_cover_entry:n \l_@@_input_clist
            
          \end{spacing}

          \vspace*{\fill}
        \end{titlepage}
      }
      {3} {
        \begin{titlepage}
          \vspace*{16mm}

          \centering

          \tl_if_blank:VTF \l_bit_coverheaderimage_tl {} {
            \includegraphics[width=9.87cm]{\l_bit_coverheaderimage_tl}\\
          }

          \vspace*{-3mm}

          \zihao{1}\textbf{\ziju{0.12}Beijing\nobreak{~}Institute\nobreak{~}of\nobreak{~}Technology~Bachelor's~Thesis}\par

          \vspace{18mm}

          \zihao{2}\textbf{\xihei:n \l_@@_value_title_en_tl}\par

          \vspace{10mm}


          \begin{spacing}{1.8}
            \begin{center}
            \tl_if_empty:NT \l_@@_cover_dilimiter_tl {
              \tl_set:Nn \l_@@_cover_dilimiter_tl {:}
            }

            % if not auto width, try override width
            \bool_if:NF \l_@@_cover_auto_width_bool {
              \dim_compare:nNnT {\l_@@_cover_label_max_width_dim} = {0pt} {
                \dim_set:Nn \l_@@_cover_label_max_width_dim {20mm}
              } 
              \dim_compare:nNnT {\l_@@_cover_value_max_width_dim} = {0pt} {
                \dim_set:Nn \l_@@_cover_value_max_width_dim {105mm}
              } 
            }

            \zihao{4}

            \clist_set:Nn \l_@@_input_clist {
              {\c_@@_label_school_en_tl} {\l_@@_value_school_tl},
              {\c_@@_label_major_en_tl} {\l_@@_value_major_tl},
              {\c_@@_label_author_en_tl} {\l_@@_value_author_tl},
              {\c_@@_label_student_id_en_tl} {\l_@@_value_student_id_tl},
              {\c_@@_label_supervisor_en_tl} {\l_@@_value_supervisor_tl},
              {\c_@@_label_co_supervisor_en_tl} {\l_@@_value_external_supervisor_tl},
            }

            \@@_render_cover_entry:n \l_@@_input_clist

            \end{center}
          \end{spacing}

          \vspace*{\fill}
          \centering
          \zihao{3}\ziju{0.5}\songti{\today}
        \end{titlepage}
      }
      {4} {
        \make_graduate_cover:
      }
      {5} {
        \make_graduate_cover:
      }
    }
    \group_end:
    \end{blindPeerReview}
  }

% 圆形数字编号定义
\newcommand{\circled}[2][]{\tikz[baseline=(char.base)]
  {\node[shape = circle, draw, inner~sep = 1pt]
  (char) {\phantom{\ifblank{#1}{#2}{#1}}};
  \node at (char.center) {\makebox[0pt][c]{#2}};}}
\robustify{\circled}

\cs_new:Npn \@@_graduate_originality: {
  \ctexset {
    chapter / pagestyle = plain,
  }

  \begin{titlepage}
    \pagenumbering{gobble}

    % 原创性声明部分
    \begin{center}
      \@@_same_page:
      \chapter*{\heiti\zihao{3}\c_@@_graduate_label_originality_tl}
    \end{center}~\par

    % 本部分字号为小三
    \zihao{4}
    \c_@@_graduate_label_originality_clause_tl

    \vspace{17mm}

    \begin{flushright}
      \c_@@_graduate_label_originality_author_signature_tl\par
    \end{flushright}

    \vspace{16mm}

    % 使用授权声明部分
    \begin{center}
      \@@_same_page:
      \chapter*{\heiti\zihao{3}\c_@@_graduate_label_authorization_tl}
    \end{center}~\par

    \c_@@_graduate_label_authorization_clause_tl

    \vspace*{15mm}

    \begin{flushright}
      \begin{spacing}{1.65}
        \zihao{4}
        % \hspace{5mm}\raisebox{-2ex}{\includegraphics[width=30mm]{example-image}}\hspace{5mm}
        \c_@@_graduate_label_originality_author_signature_tl\par
        \c_@@_graduate_label_originality_supervisor_signature_tl\par
      \end{spacing}
    \end{flushright}
  \end{titlepage}
  \cleardoublepage
}

\NewDocumentCommand \MakeOriginality {} 
{
  \group_begin:
    \begin{blindPeerReview}
    \int_case:nn {\g_@@_thesis_type_int}
    {
      {1} 
      {
        \pagestyle{BIThesis}
        \pagenumbering{gobble}

        % 原创性声明部分
        \begin{center}
          \vspace*{-2bp}
          \@@_same_page:
          \chapter*{\heiti\zihao{2}\c_@@_bachelor_label_originality_tl}
        \end{center}~\par

        % 本部分字号为小三
        \zihao{-3}
        \c_@@_bachelor_label_originality_clause_tl

        \vspace{17mm}

        \begin{flushright}
          \c_@@_bachelor_label_originality_author_signature_tl\par
        \end{flushright}

        \vspace{16mm}

        % 使用授权声明部分
        \begin{center}
          \@@_same_page:
          \chapter*{\heiti\zihao{2}\c_@@_bachelor_label_authorization_tl}
        \end{center}~\par

        \c_@@_bachelor_label_authorization_clause_tl

        \vspace*{3mm}

        \begin{flushright}
          \begin{spacing}{1.65}
            \zihao{-3}
            % \hspace{5mm}\raisebox{-2ex}{\includegraphics[width=30mm]{example-image}}\hspace{5mm}
            \c_@@_bachelor_label_originality_author_signature_tl\par
            \c_@@_bachelor_label_originality_supervisor_signature_tl\par
          \end{spacing}
        \end{flushright}

        \newpage
      }
      {3} {
        \setstretch{1.26}
        % 原创性声明部分
        \begin{center}
          \vspace*{-2bp}
          \@@_same_page:
          \chapter*{\heiti\zihao{-2}\c_@@_bachelor_english_label_originality_tl}
        \end{center}~\par

        % 本部分字号为小三
        \zihao{-4}
        \c_@@_bachelor_english_label_originality_clause_tl

        \bigbreak

        Student~(Signature):~\dunderline[-1pt]{1pt}{\makebox[18mm]{}}~Date:\par

        \vspace{6mm}

        % 使用授权声明部分
        \begin{center}
          \@@_same_page:
          \chapter*{\heiti\zihao{-2}\c_@@_bachelor_english_label_authorization_tl}
        \end{center}~\par

        \c_@@_bachelor_english_label_authorization_clause_tl

        \bigbreak
        Student~(Signature):~\dunderline[-1pt]{1pt}{\makebox[18mm + 16bp]{}}~\hspace{2mm}Date:\par
        Supervisor~(Signature):~\dunderline[-1pt]{1pt}{\makebox[18mm]{}}~\hspace{2mm}Date:\par
      }
      {4} {\@@_graduate_originality:}
      {5} {\@@_graduate_originality:}
    }
  \end{blindPeerReview}
  \group_end:
}

\NewDocumentCommand \MakePaperBack {}
  {
    \begin{blindPeerReview}
      \make_paper_back:
    \end{blindPeerReview}
  }

\NewDocumentCommand \MakeTitle {}
  {
    \begin{blindPeerReview}
      \@@_make_chinese_title_page:
      \@@_make_english_title_page:
    \end{blindPeerReview}
  }

\DeclareDocumentCommand \MakeTOC {}
  {
    {
      \@@_if_bachelor_thesis:TF {
        \renewcommand{\baselinestretch}{1.35}
      } {
        \renewcommand{\baselinestretch}{1.56}
      }

      \@@_if_thesis_english:TF {
        \tl_set:Nn \l_@@_toc_title_tl {\c_@@_label_toc_en_tl}
      } {
        \tl_set:Nn \l_@@_toc_title_tl {\c_@@_label_toc_tl}
      }

      % 自定义目录样式
      \cs_set:Npn \contentsname {
        \fontsize{16pt}{\baselineskip}
        \l_@@_unnumchapter_style_cs:n\l_@@_title_font_cs:n{\l_@@_toc_title_tl}
        \vspace{-8pt}
      }

      % 制作目录
      \tableofcontents

      % 在本科生全英文模板中,添加「目录」本身到目录中。
      \@@_if_thesis_int_type:nT {3} {
        \addcontentsline{toc}{chapter}{\c_@@_label_toc_en_tl}
      }
    }
  }

  % TODO:
  \NewDocumentEnvironment {abstract} {o}
  {

    \IfValueT {#1} {
      \keys_set:nn { bit / abstract } {#1}
    }

    \cleardoublepage
    \setstretch{1.53}

    \@@_if_bachelor_thesis:T {
      \begin{center}
        \vspace*{-17bp}
        \heiti\zihao{-2}\textbf{
          \int_case:nn {\g_@@_thesis_type_int}
          {
            {1} {\l_@@_value_title_tl}
            {2} {\l_@@_value_trans_title_tl}
            {3} {\l_@@_value_title_tl}
          }
        }
      \end{center}

      \vspace*{2mm}
    }

    \ctexset{
      chapter/numbering = false,
    }

    \@@_if_bachelor_thesis:T {
      \ctexset{
        chapter/titleformat = {\textmd}
      }
    }

    {
      \@@_same_page:
      \bool_if:NTF \l_@@_add_to_toc_bool {
        \chapter{\c_@@_label_abstract_tl}
      } {
        \chapter*{\c_@@_label_abstract_tl}
      }
    }
    \vspace*{1mm}
    \par
  }
  {
    \par
    \vspace{4ex}\noindent\textbf{\heiti \c_@@_label_keywords_tl \l_@@_value_keywords_tl}\par
    \newpage
  }

  \NewDocumentEnvironment {abstract*} {o}
  {
    \IfValueT {#1} {
      \keys_set:nn { bit / abstract_en } {#1}
    }

    \cleardoublepage
    \setstretch{1.53}

    \@@_if_bachelor_thesis:T {
      \begin{spacing}{0.95}
        \centering
        \vspace*{-2bp}

        \@@_if_thesis_int_type:nTF {3} {
          \arialfamily\zihao{-2}\textbf\l_@@_value_title_en_tl\\
        } {
          \heiti\zihao{3}\textbf\l_@@_value_title_en_tl\\
        }
      \end{spacing}
      \vspace*{10mm}
    }
    
    \ctexset{
      chapter/numbering = false,
    }

    \@@_if_bachelor_thesis:TF {
      \int_compare:nNnTF {\g_@@_thesis_type_int} = {3}
      {
        \ctexset{
          chapter = {
            titleformat = {\heiti\zihao{3}\centering\textbf},
          }
        }
      } {
        \ctexset{
          chapter = {
            titleformat = {\heiti\zihao{-3}\centering\textmd},
          }
        }
      }
    } {
      \ctexset {
        chapter/titleformat = {\heiti\zihao{3}\centering\textbf} 
      }
    }

    {
      \@@_same_page:
      \bool_if:nTF {\l_@@_add_to_toc_bool} {
        \chapter{\c_@@_label_abstract_en_tl}
      } {
        \chapter*{\c_@@_label_abstract_en_tl}
      }
    }
  }
  {
    \par\vspace{3ex}\noindent\textbf{\c_@@_label_keywords_en_tl \l_@@_value_keywords_en_tl}
    \newpage
  }

% after \backmatter
  \NewDocumentEnvironment {conclusion} {}
  {
    \ctexset{
      section/number = \arabic{section}
    }

    \@@_if_thesis_english:TF {
      \chapter{\c_@@_label_conclusion_en_tl}
    } {
      \chapter{\c_@@_label_conclusion_tl}
    }
  }
  {}

% after backmatter
  \NewDocumentEnvironment {bibprint} {}
  {
    % 设置参考文献字号为 5 号
    \renewcommand*{\bibfont}{\zihao{5}}
    % 设置参考文献各个项目之间的垂直距离为 0
    \setlength{\bibitemsep}{0ex}
    \setlength{\bibnamesep}{0ex}
    \setlength{\bibinitsep}{0ex}
    \@@_if_graduate:TF {
    } {
      % 「本科生」设置单倍行距
      \renewcommand{\baselinestretch}{1.2}
    }
    % 设置参考文献顺序标签 `[1]` 与文献内容 `作者. 文献标题...` 的间距
    \setlength{\biblabelsep}{1.7mm}
    % 设置参考文献后文缩进为 0(与 Word 模板保持一致)
    \RenewDocumentCommand \itemcmd {} {
      \addvspace{\bibitemsep} % 恢复 \bibitemsep 的作用
      \mkgbnumlabel{\printfield{labelnumber}}
      \hspace{\biblabelsep}
    }
    \@@_if_thesis_english:TF {
      \chapter{\c_@@_label_reference_en_tl}
    } {
      \chapter{\c_@@_label_reference_tl}
    }
  }
  {}

  % #1: The name that used as chapter title
  % #2: The name that used in ToC.
  \NewDocumentEnvironment {appendices} { oo }
  {
    % Used in chapter, ToC.
    \tl_new:N \l_@@_appendix_plain_label_tl
    % Used before reference label.
    \tl_new:N \l_@@_appendix_title_tl

    \int_compare:nNnTF {\g_@@_thesis_type_int} = {3} 
    {
      \tl_set:Nn \l_@@_appendix_plain_label_tl {\c_@@_label_appendix_prefix_en_tl}
      \tl_set:Nn \l_@@_appendix_title_tl {\c_@@_label_appendix_en_tl}
    } {
      \tl_set:Nn \l_@@_appendix_plain_label_tl {\c_@@_label_appendix_prefix_tl}
      \tl_set:Nn \l_@@_appendix_title_tl {\c_@@_label_appendix_tl}
    }

    \ctexset{
      section/number = \l_@@_appendix_plain_label_tl\hspace{1ex}\Alph{section},
      subsection/number = \Alph{section}. \arabic{subsection},
    }

    \IfValueTF {#1} {
      \chapter*{#1}
      \stepcounter{chapter}
      \IfValueTF {#2} {
        \addcontentsline{toc}{chapter}{#2}
      } {
        \addcontentsline{toc}{chapter}{\l_@@_appendix_title_tl}
      }
    } {
      \chapter{\l_@@_appendix_title_tl}
    }

    \cs_set:Npn \thechapter {
      \Alph{section}
    }
  }
  {}

  \NewDocumentEnvironment {acknowledgements} {+b}
  {
    \begin{blindPeerReview}
      \ctexset{
        section/number = \arabic{section},
        subsection/number = \arabic{section}. \arabic{subsection},
      }

      \@@_if_thesis_english:TF {
        \chapter{\c_@@_label_ack_en_tl}
      } {
        \chapter{\c_@@_label_ack_tl}
      }
      #1
    \end{blindPeerReview}
  } {}

  \NewDocumentEnvironment {publications} {+b}
  {
    \begin{blindPeerReview}
      % 设置参考文献字号为 5 号
      \renewcommand*{\bibfont}{\zihao{5}}
      % 设置参考文献各个项目之间的垂直距离为 0
      \setlength{\bibitemsep}{0ex}
      \setlength{\bibnamesep}{0ex}
      \setlength{\bibinitsep}{0ex}
      % 设置单倍行距
      \renewcommand{\baselinestretch}{1.2}
      % 设置参考文献顺序标签 `[1]` 与文献内容 `作者. 文献标题...` 的间距
      \setlength{\biblabelsep}{1.7mm}
      % 设置参考文献后文缩进为 0(与 Word 模板保持一致)
      \RenewDocumentCommand \itemcmd {} {
        \addvspace{\bibitemsep} % 恢复 \bibitemsep 的作用
        \mkgbnumlabel{\printfield{labelnumber}}
        \hspace{\biblabelsep}
      }

      % ===== 上方定义与「参考文献」部分相同
      \cs_set:Npn \mkbibnamegiven ##1 {
        \ifitemannotation{myself}{\textbf{##1}}{##1}
      }

      \cs_set:Npn \mkbibnamefamily ##1 {
        \ifitemannotation{myself}{\textbf{##1}}{##1}
      }

      % Sorting by year, name, type.
      \newrefcontext[sorting=ynt]
      \chapter{\c_@@_label_publications_tl}
      #1
    \end{blindPeerReview}
  }
  {}

  \NewDocumentEnvironment {resume} {+b}
  {
    \begin{blindPeerReview}
    \chapter{\c_@@_label_resume_tl}
    #1
    \end{blindPeerReview}
  }
  {
  }

  \NewDocumentEnvironment {symbols} {o}
  {
    \IfValueT {#1} {
      \keys_set:nn { bit / symbols } {#1}
    }

    \bool_if:NTF \l_@@_add_to_toc_bool {
      \chapter{\c_@@_label_symbols_tl}
    } {
      \chapter*{\c_@@_label_symbols_tl}
    }
    \zihao{-4}
    \begin{itemize}[labelwidth=2.5cm,labelsep=0.5cm,leftmargin=3cm,itemindent=0cm,itemsep=0cm]
    \cs_set:Npn \makelabel ##1 {##1\hfil}
  }
  {
    \end{itemize}
  }
%    \end{macrocode}
%    \begin{macrocode}
%</thesis>
%    \end{macrocode}

%    \begin{macrocode}
%<*proposal>
%    \end{macrocode}
%
%    \begin{macrocode}
% Define global
\int_new:N \g_@@_report_type_int

% Define Consts.
\clist_const:Nn \c_@@_report_type_clist
    { common, undergraduate_proposal}

% Define tmp Variables
\seq_new:N \l_@@_right_seq
\seq_new:N \l_@@_left_seq

\PassOptionsToPackage{AutoFakeBold,AutoFakeSlant}{xeCJK}

% Pass every option not explicitly defined to `ctexbeamer`.
\DeclareOption*{
  \PassOptionsToClass{\CurrentOption}{ctexart}
}
% Executes the code for each option.
\ProcessOptions\relax
% Load
\LoadClass[zihao=-4]{ctexart}

\RequirePackage[a4paper,left=3cm,right=2.4cm,top=2.6cm,bottom=2.38cm,includeheadfoot]{geometry}
\RequirePackage{fancyhdr}%
\RequirePackage{setspace}%
\RequirePackage{caption}%
\RequirePackage{booktabs}%

% \RequirePackage{fontspec}%
% \RequirePackage{setspace}%
% \RequirePackage{graphicx}%
% \RequirePackage{fancyhdr}%
% \RequirePackage{pdfpages}%
% \RequirePackage{setspace}%
% \RequirePackage{caption}%

% TODO: to be deleted
% \RequirePackage{fontspec}%

\keys_define:nn { bitproposal }
  {
    option .meta:nn = {bitproposal / option } {#1},
    cover .meta:nn = { bitproposal / cover  } {#1},
    info .meta:nn = { bitproposal / info } {#1},
    misc .meta:nn = { bitproposal / misc } {#1}
  }

\keys_define:nn { bitproposal / cover }
  {
    date .tl_set:N = \l_bit_coverdate_tl,
    %% cover entry
    dilimiter .tl_set:N = \l_@@_cover_dilimiter_tl,
    labelAlign .tl_set:N = \l_@@_cover_label_align_tl,
    labelAlign .initial:n = {r},
    valueAlign .tl_set:N = \l_@@_cover_value_align_tl,
    valueAlign .initial:n = {c},
    labelMaxWidth .dim_set:N = \l_@@_cover_label_max_width_dim,
    valueMaxWidth .dim_set:N = \l_@@_cover_value_max_width_dim,
    autoWidth .bool_set:N = \l_@@_cover_auto_width_bool,
    autoWidth .initial:n = {true},
    underlineThickness .dim_set:N = \l_@@_cover_underline_thickness_dim,
    underlineThickness .initial:n = {1pt},
    underlineOffset .dim_set:N = \l_@@_cover_underline_offset_dim,
    underlineOffset .initial:n = { -10pt },
  }

\keys_define:nn { bitproposal / info }
  {
    dept .tl_set:N = \l_@@_value_school_tl,
    major .tl_set:N = \l_@@_value_major_tl,
    class .tl_set:N = \l_@@_value_class_tl,
    name .tl_set:N = \l_@@_value_author_tl,
    mentor .tl_set:N = \l_@@_value_supervisor_tl,
    offCampusMentor .tl_set:N = \l_@@_value_external_supervisor_tl,
  }

\keys_define:nn { bitproposal / option }
{
  type .choice:,
  type .value_required:n = true,
  type .choices:Vn =
    \c_@@_report_type_clist
    { 
      \int_set_eq:NN \g_@@_report_type_int \l_keys_choice_int 
    },
  type .initial:n = common,

}

\keys_define:nn { bitproposal / misc }
  {
    reviewTable .tl_set:N = \l_bit_reviewtable_tl,
  }

\ProcessKeysOptions { bitproposal / option }

\cs_generate_variant:Nn \tl_if_empty:nTF {e}
\cs_generate_variant:Nn \seq_set_split:Nnn {Nne}

% #1: position
% #2: line_thickness
% #3: token list
\cs_new:Npn \@@_dunderline:nnn #1#2#3 {
  {\setbox0=\hbox{#3}\ooalign{\copy0\cr\rule[\dimexpr#1-#2\relax]{\wd0}{#2}}}
}

\cs_new:Npn \@@_render_cover_entry:nn #1#2 {
  \makebox[\l_@@_cover_label_max_width_dim][\l_@@_cover_label_align_tl]{
    \tl_if_blank:VTF #1 {} {#1\l_@@_cover_dilimiter_tl}
  }
  \hspace{1ex}
  \@@_dunderline:nnn{\l_@@_cover_underline_offset_dim}{\l_@@_cover_underline_thickness_dim}{
    \makebox[\l_@@_cover_value_max_width_dim][\l_@@_cover_value_align_tl]{#2}
  }\par
}

% Get text with from #2, then set to #1.
\cs_new:Npn \@@_get_text_width:Nn #1#2
  {
    \hbox_set:Nn \l_tmpa_box {#2}
    \dim_set:Nn #1 { \box_wd:N \l_tmpa_box }
  }
\cs_generate_variant:Nn \@@_get_text_width:Nn { NV }

% Get max text width from seq #2, then set to #1.
\cs_new:Npn \@@_get_max_text_width:NN #1#2
  {
% 这里用 |group| 确保局部变量不会被污染。
    \group_begin:
      \seq_set_eq:NN \l_@@_tmpa_seq #2
      \dim_zero_new:N \l_@@_tmpa_dim
      \bool_until_do:nn { \seq_if_empty_p:N \l_@@_tmpa_seq }
        {
          \seq_pop_left:NN \l_@@_tmpa_seq \l_@@_tmpa_tl
          \@@_get_text_width:NV \l_@@_tmpa_dim \l_@@_tmpa_tl
          \dim_gset:Nn #1 { \dim_max:nn {#1} { \l_@@_tmpa_dim } }
        }
    \group_end:
  }

% process label (#1) and value #2 seperately
\cs_new:Npn \@@_parse_entry #1 #2 {
  \seq_set_split:Nne \l_@@_tmp_right_seq {//} {#2}
  \seq_clear:N \l_@@_tmp_left_seq
  \seq_map_inline:Nn \l_@@_tmp_right_seq {
    \seq_put_right:Nn \l_@@_tmp_left_seq {}
  }
  \seq_put_left:Nn \l_@@_tmp_left_seq {#1}
  \seq_pop_right:NN \l_@@_tmp_left_seq \g_@@_trashcan_tl
}

\cs_new:Npn \@@_render_cover_entry:n #1 {
  \seq_set_from_clist:NN \l_@@_input_seq #1
  % parse newline //
  \seq_map_inline:Nn \l_@@_input_seq {
    \@@_parse_entry ##1
    \seq_concat:NNN \l_@@_right_seq \l_@@_right_seq \l_@@_tmp_right_seq
    \seq_concat:NNN \l_@@_left_seq \l_@@_left_seq \l_@@_tmp_left_seq
  }

  \bool_if:NT \l_@@_cover_auto_width_bool {
    \@@_get_max_text_width:NN \l_@@_cover_label_max_width_dim \l_@@_left_seq
    \@@_get_max_text_width:NN \l_@@_cover_value_max_width_dim \l_@@_right_seq
  }
  

  \group_begin:
    \bool_until_do:nn { \seq_if_empty_p:N \l_@@_left_seq }
      {
        \seq_pop_left:NN \l_@@_left_seq \l_@@_tmpa_tl
        \seq_pop_left:NN \l_@@_right_seq \l_@@_tmpb_tl
        \tl_if_empty:eTF \l_@@_tmpb_tl {} {
          \@@_render_cover_entry:nn {\l_@@_tmpa_tl} {\l_@@_tmpb_tl}
        }
      }
  \group_end:
}

\DeclareDocumentCommand \BITProposalSetup { m }
  { \keys_set:nn { bitproposal } { #1 }}
\DeclareDocumentCommand \MakeCover {}
  {
    \group_begin:
      % Main code for \MakeCover
      \begin{titlepage}
        \topskip=0pt
        \vspace*{-16mm}
        \centering
        \hspace{-6mm}\songti\fontsize{22pt}{22pt}\selectfont{北京理工大学}\par

        \vspace{13mm}

        \hspace{-6mm}\heiti\fontsize{24pt}{24pt}\selectfont{本科生毕业设计(论文)开题报告}\par

        \vspace{53mm}

        \begin{spacing}{2.2}
          \songti\zihao{3}
          \clist_set:Nn \l_@@_input_clist {
              {\textbf{学\qquad 院:}} {\l_@@_value_school_tl},
              {\textbf{专\qquad 业:}} {\l_@@_value_major_tl},
              {\textbf{班\qquad 级:}} {\l_@@_value_class_tl},
              {\textbf{姓\qquad 名:}} {\l_@@_value_author_tl},
              {\textbf{指导教师:}} {\l_@@_value_supervisor_tl},
              {\textbf{校外指导教师:}} {\l_@@_value_external_supervisor_tl},
           }

        \@@_render_cover_entry:n \l_@@_input_clist

          % \hspace{46mm}\songti\fontsize{16pt}{16pt}\selectfont{\textbf{学\hspace{11mm}院:}\underline{\makebox[51mm][c]{\l_bit_dept_tl}}}\par
          %
          % \hspace{46mm}\songti\fontsize{16pt}{16pt}\selectfont{\textbf{专\hspace{11mm}业:}\underline{\makebox[51mm][c]{\l_bit_major_tl}}}\par
          %
          % \hspace{46mm}\songti\fontsize{16pt}{16pt}\selectfont{\textbf{班\hspace{11mm}级:}\underline{\makebox[51mm][c]{\l_bit_class_tl}}}\par
          %
          % \hspace{46mm}\songti\fontsize{16pt}{16pt}\selectfont{\textbf{姓\hspace{11mm}名:}\underline{\makebox[51mm][c]{\l_bit_name_tl}}}\par
          %
          % \hspace{46mm}\songti\fontsize{16pt}{16pt}\selectfont{\textbf{指导教师:}\underline{\makebox[51mm][c]{\l_bit_mentor_tl}}}\par
          %
          % \hspace{46mm}\songti\fontsize{16pt}{16pt}\selectfont{\textbf{校外指导教师:}\underline{\makebox[40mm][c]{\l_bit_offcampusmentor_tl}}}\par
        \end{spacing}

        \vspace*{\fill}

        \centering
        \hspace{-6mm}\songti\fontsize{12pt}{12pt}\selectfont{\today}
      \end{titlepage}
    \group_end:
  }

\DeclareDocumentCommand \MakeReviewTable {} 
  {
    \group_begin:
      \begin{titlepage}
        \includepdf[pages=-]{\l_bit_reviewtable_tl}
      \end{titlepage}
    \group_end:
  }

% 定义 caption 字体为楷体
\DeclareCaptionFont{kaiticaption}{\kaishu \normalsize}

% 设置图片的 caption 格式
\renewcommand{\thefigure}{\thesection-\arabic{figure}}
\captionsetup[figure]{font=small,labelsep=space,skip=10bp,labelfont=bf,font=kaiticaption}

% 设置表格的 caption 格式
\renewcommand{\thetable}{\thesection-\arabic{table}}
\captionsetup[table]{font=small,labelsep=space,skip=10bp,labelfont=bf,font=kaiticaption}

% 输出大写数字日期
\ctexset{today=big}

% 将西文字体设置为 Times New Roman
\setromanfont{Times~New~Roman}

%% 将中文楷体设置为 SIMKAI.TTF(如果需要)
% \setCJKfamilyfont{zhkai}{[SIMKAI.TTF]}
% \newcommand*{\kaiti}{\CJKfamily{zhkai}}

% 设置文档标题深度
\setcounter{tocdepth}{3}
\setcounter{secnumdepth}{3}

%%
% 设置一级标题、二级标题格式
% 一级标题:小三,宋体,加粗,段前段后各半行
\ctexset{section={
  format={\raggedright \bfseries \songti \zihao{-3}},
  beforeskip = 24bp plus 1ex minus .2ex,
  afterskip = 24bp plus .2ex,
  fixskip = true,
  name = {,.\quad}
  }
}
% 二级标题:小四,宋体,加粗,段前段后各半行
\ctexset{subsection={
  format = {\bfseries \songti \raggedright \zihao{4}},
  beforeskip = 24bp plus 1ex minus .2ex,
  afterskip = 24bp plus .2ex,
  fixskip = true,
  }
}
% 页眉和页脚(页码)的格式设定
\fancyhf{}
\fancyhead[R]{\fontsize{10.5pt}{10.5pt}\selectfont{北京理工大学本科生毕业设计(论文)开题报告}}
\fancyfoot[R]{\fontsize{9pt}{9pt}\selectfont{\thepage}}
\renewcommand{\headrulewidth}{1pt}
\renewcommand{\footrulewidth}{0pt}

% 正文开始
\pagestyle{fancy}
\setcounter{page}{1}

% 正文 22 磅的行距,段前段后间距为 0
% \setlength{\parskip}{0em}
% \renewcommand{\baselinestretch}{1.53}
% 正文首行悬挂 1.02cm
% \setlength{\parindent}{1.02cm}
%    \end{macrocode}
%
%    \begin{macrocode}
%</proposal>
%    \end{macrocode}
%    \begin{macrocode}
%<*report>
%    \end{macrocode}
%
%    \begin{macrocode}
\PassOptionsToPackage{AutoFakeBold,AutoFakeSlant}{xeCJK}
% Pass every option not explicitly defined to `ctexbeamer`.
\DeclareOption*{
  \PassOptionsToClass{\CurrentOption}{ctexart}
}
% Executes the code for each option.
\ProcessOptions\relax
% Load
\LoadClass[zihao=-4]{ctexart}

\RequirePackage{fancyhdr}%

\RequirePackage{titlesec}%
\RequirePackage{fontspec}%
\RequirePackage{setspace}%

\RequirePackage[a4paper,left=3.18cm,right=3.18cm,top=2.54cm,bottom=2.54cm,includeheadfoot]{geometry}%

\keys_define:nn { bitreport }
  {
    cover .meta:nn = { bitreport / cover  } {#1},
    info .meta:nn = { bitreport / info } {#1}
  }

\keys_define:nn { bitreport / cover }
  {
    imagePath .tl_set:N = \l_bit_coverimagepath_tl,
    date .tl_set:N = \l_bit_coverdate_tl,
  }

\keys_define:nn { bitreport / info }
  {
    title .tl_set:N = \l_bit_title_tl,
    dept .tl_set:N = \l_bit_depart_tl,
    major .tl_set:N = \l_bit_major_tl,
    classNumber .tl_set:N = \l_bit_classnumber_tl,
    studentNumber .tl_set:N = \l_bit_studentnumber_tl,
    name .tl_set:N = \l_bit_name_tl,
    teacherName .tl_set:N = \l_bit_teachername_tl,
  }

\DeclareDocumentCommand \BITLabReportSetup { m }
  { \keys_set:nn { bitreport } { #1 }}
\DeclareDocumentCommand \MakeCover {}
  {
    \group_begin:
      % Main code for \MakeCover
      \begin{titlepage}
        \centering
        \vspace{23mm}
        \tl_if_empty:NF \l_bit_coverimagepath_tl {
          \includegraphics[width=.5\textwidth]{\l_bit_coverimagepath_tl}\\
        }
        \vspace{10mm}
        \heiti\fontsize{24pt}{24pt}\selectfont{\l_bit_title_tl}\\
        \vspace{77mm}
          \begin{spacing}{2.2}
            \tl_if_empty:NF \l_bit_depart_tl {
              \songti\fontsize{16pt}{16pt}\selectfont{\textbf{学\hspace{11mm}院:}\underline{\makebox[51mm][c]{\l_bit_depart_tl}}}\\
            }

            \tl_if_empty:NF \l_bit_major_tl {
              \songti\fontsize{16pt}{16pt}\selectfont{\textbf{专\hspace{11mm}业:}\underline{\makebox[51mm][c]{\l_bit_major_tl}}}\\
            }

            \tl_if_empty:NF \l_bit_classnumber_tl {
              \songti\fontsize{16pt}{16pt}\selectfont{\textbf{班\hspace{11mm}级:}\underline{\makebox[51mm][c]{\l_bit_classnumber_tl}}}\\
            }

            \tl_if_empty:NF \l_bit_name_tl {
              \songti\fontsize{16pt}{16pt}\selectfont{\textbf{姓\hspace{11mm}名:}\underline{\makebox[51mm][c]{\l_bit_name_tl}}}\\
            }

            \tl_if_empty:NF \l_bit_teachername_tl {
              \songti\fontsize{16pt}{16pt}\selectfont{\textbf{任课教师:}\underline{\makebox[51mm][c]{\l_bit_teachername_tl}}}\\
            }
          \end{spacing}
        \vspace*{\fill}
        \centering
        \songti\fontsize{12pt}{12pt}\selectfont{
          \tl_if_empty:NTF \l_bit_coverdate_tl {
            \today
          } {
            \l_bit_coverdate_tl
          }
        }
      \end{titlepage}
    \group_end:
  }


% 将西文字体设置为 Times New Roman
% \setromanfont{Times~New~Roman}%

% 设置文档标题深度
\setcounter{tocdepth}{3}%
\setcounter{secnumdepth}{3}%

%%
% 设置一级标题、二级标题格式
\ctexset{section={%
  format={\raggedright \bfseries \songti \zihao{-3}},%
  name = {,.},%
  number = \chinese{section}%
  }%
}%
\ctexset{subsection={%
  format = {\bfseries \songti \raggedright \zihao{-4}},%
  }%
}%

% 页眉和页脚(页码)的格式设定
\fancyhf{}%
\fancyhead[L]{\fontsize{10.5pt}{10.5pt}\selectfont\kaishu{\l_bit_title_tl}}%
\fancyfoot[C]{\fontsize{9pt}{9pt}\selectfont\kaishu{\thepage}}%
\renewcommand{\headrulewidth}{0.5pt}%
\renewcommand{\footrulewidth}{0pt}%

% 正文 pagestyle
\pagestyle{fancy}
\setcounter{page}{1}%
%    \end{macrocode}
%
%    \begin{macrocode}
%</report>
%    \end{macrocode}
%    \begin{macrocode}
%<*beamer>
%    \end{macrocode}
%
%    \begin{macrocode}
% Define our keyvalues
\keys_define:nn { bitbeamer }
  {
    titlegraphic .tl_set:N = \l_bit_titlegraphic_tl,
    framelogo .tl_set:N = \l_bit_framelogo_tl,
  }
\ProcessKeysOptions { bitbeamer }

% Pass every option not explicitly defined to `ctexbeamer`.
\DeclareOption*{
  \PassOptionsToClass{\CurrentOption}{ctexbeamer}
}
% Executes the code for each option.
\ProcessOptions\relax

% Load
\LoadClass{ctexbeamer}

\RequirePackage{xeCJKfntef}
\RequirePackage{tikz}

\usetheme{Madrid}
% 设置主题色
\colorlet{beamer@blendedblue}{green!40!black}

\cs_new:Npn \CJKhl:nn #1 #2
  { \CJKsout*[thickness=2.5ex, format=\color{#1}]{#2} }

% Set header if logo path is provided. 
\tl_if_empty:NF \l_bit_titlegraphic_tl {
  % BIT Logo
  \titlegraphic{
      \includegraphics[width=2cm]{\l_bit_titlegraphic_tl}
  }
}

% Set title logo if logo path is provided.
\tl_if_empty:NF \l_bit_framelogo_tl {
  \addtobeamertemplate{frametitle}{}{%
    \begin{tikzpicture}[remember~picture,overlay]
      \node[anchor=north~east,yshift=2pt] at (current~page.north~east) {\includegraphics[height=0.8cm]{\tl_use:N \l_bit_framelogo_tl}};
    \end{tikzpicture}
  }
}

% Expose command to user.
\cs_new_eq:NN \CJKhl \CJKhl:nn
%    \end{macrocode}
%    \begin{macrocode}
%</beamer>
%    \end{macrocode}

%    \begin{macrocode}
%<*book>
%    \end{macrocode}
%    \begin{macrocode}

% 目前只有本科的模板,但仍旧要为可能预留空间。
\newif\if@bit@bachelor
\newif\if@bit@docTranslation
\newif\if@bit@master
\newif\if@bit@docter

\RequirePackage{kvoptions}

\SetupKeyvalOptions{
  family=BIThesis,
  prefix=BIThesis@
}

\DeclareStringOption[14pt]{footskip}
\DeclareBoolOption{titleNumberHeiti}
\ProcessKeyvalOptions*


\DeclareOption{bachelor}{\@bit@bachelortrue}
\DeclareOption{translation}{\@bit@docTranslationtrue}
\DeclareOption*{\PassOptionsToClass{\CurrentOption}{ctexbook}}

\ExecuteOptions{bachelor}

\ProcessOptions\relax

\PassOptionsToPackage{AutoFakeBold,AutoFakeSlant}{xeCJK}
\LoadClass[UTF8,zihao=-4,oneside,openany]{ctexbook}

% \RequirePackage[a4paper,left=3cm,right=2.6cm,top=3.5cm,bottom=2.9cm]{geometry}
% 目前 29mm 最接近 Word 排版
\RequirePackage{xeCJK}
\RequirePackage{titletoc}
  % \RequirePackage{fontspec}
\RequirePackage{setspace}
\RequirePackage{graphicx}
\RequirePackage{fancyhdr}
\RequirePackage{pdfpages}
\RequirePackage{setspace}
\RequirePackage{booktabs}
\RequirePackage{multirow}
\RequirePackage{tikz}
\RequirePackage{etoolbox}
\RequirePackage{hyperref}
\RequirePackage{xcolor}
\RequirePackage{caption}
\RequirePackage{array}
\RequirePackage{amsmath}
\RequirePackage{amssymb}
\RequirePackage{pdfpages}
\RequirePackage{listings}

%\PackageError{zzzz}{Main \BIThesis@footskip  hellp}{test}

% 设置参考文献编译后端为 biber,引用格式为 GB/T7714-2015 格式
% 参考文献使用宏包见 https://github.com/hushidong/biblatex-gb7714-2015
\RequirePackage[
  backend=biber,
  style=gb7714-2015,
  gbalign=gb7714-2015,
  gbnamefmt=lowercase,
  gbpub=false,
  doi=false,
  url=false,
  eprint=false,
  isbn=false,
]{biblatex}

% 参考文献引用文件位于 misc/ref.bib
\addbibresource{./misc/ref.bib}

% 西文字体默认为 Times New Roman
\setromanfont{Times New Roman}
% 论文题目字体为华文细黑
\setCJKfamilyfont{xihei}[AutoFakeBold,AutoFakeSlant]{[STXIHEI.TTF]} % 若希望使用本机字体,也可以用 {STXihei} 来调用
\newcommand{\xihei}{\CJKfamily{xihei}}

\ifBIThesis@titleNumberHeiti
  \newcommand{\arabicHeiti}[1]{\xeCJKsetup{CJKspace=true}\xeCJKDeclareCharClass{CJK}{`0 -> `9}{\heiti\raisebox{-0.1ex}{#1}}\normalspacedchars{0,1,2,3,4,5,6,7,8,9}\xeCJKsetup{CJKspace=false}}
\else
  \newcommand{\arabicHeiti}[1]{#1}
\fi



% 主题页面格式:BIThesis
\fancypagestyle{BIThesis}{
  % 页眉高度
  \setlength{\headheight}{20pt}
  % 页码高度(不完美,比规定稍微靠下 2mm)
  \setlength{\footskip}{\BIThesis@footskip}

  \fancyhf{}
  % 定义页眉、页码
  \fancyhead[C]{\zihao{4}\ziju{0.08}\songti{北京理工大学本科生毕业设计(论文)}}
  \fancyfoot[C]{\songti\zihao{5} \thepage}
  % 页眉分割线稍微粗一些
  \renewcommand{\headrulewidth}{0.6pt}
}

\if@bit@docTranslation
% 主题页面格式:BIThesis
\fancypagestyle{BIThesis}{
  % 页眉高度
  \setlength{\headheight}{20pt}
  % 页码高度(不完美,比规定稍微靠下 2mm)
  \setlength{\footskip}{\BIThesis@footskip}

  \fancyhf{}
  % 定义页码
  \fancyfoot[C]{\songti\zihao{5} \thepage}
  % 页眉分割线稍微粗一些
  \renewcommand{\headrulewidth}{0.6pt}

  % 定义页眉
  \fancyhead[C]{\zihao{4}\ziju{0.08}\songti{北京理工大学本科生毕业设计(论文)外文翻译}}
}
\fi
% 设置章节格式
% 一级标题:黑体,三号,加粗;间距:段前 0.5 行,段后 1 行;
\ctexset{chapter={
    name = {第,章},
    number = {\arabicHeiti{ \arabic{chapter} }},
    format = {\heiti \bfseries \centering \zihao{3}},
    aftername = \hspace{9bp},
    pagestyle = BIThesis,
    beforeskip = 8bp,
    afterskip = 32bp,
    fixskip = true,
  }
}

% 二级标题:黑体,四号,加粗;间距:段前 0.5 行,段后 0 行;
\ctexset{section={
    number = {\arabicHeiti{\thechapter.\hspace{1bp}\arabic{section}}},
    format = {\heiti \raggedright \bfseries \zihao{4}},
    aftername = \hspace{8bp},
    beforeskip = 20bp plus 1ex minus .2ex,
    afterskip = 18bp plus .2ex,
    fixskip = true,
  }
}

% 三级标题:黑体、小四、加粗;间距:段前 0.5 行,段后 0 行;
\ctexset{subsection={
    number = {\arabicHeiti{\thechapter.\hspace{1bp}\arabic{section}.\hspace{1bp}\arabic{subsection}}},
    format = {\heiti \bfseries \raggedright \zihao{-4}},
    aftername = \hspace{7bp},
    beforeskip = 17bp plus 1ex minus .2ex,
    afterskip = 14bp plus .2ex,
    fixskip = true,
  }
}

% 设置目录样式
% 添加 PDF 链接
\addtocontents{toc}{\protect\hypersetup{hidelinks}}

% 解决「目录」二字的格式问题
\renewcommand{\contentsname}{
  \fontsize{16pt}{\baselineskip}
  \normalfont\heiti{目~~~~录}
  \vspace{-8pt}
}
% 定义目录样式
\titlecontents{chapter}[0pt]{\songti \zihao{-4}}
{\thecontentslabel\hspace{\ccwd}}{}
{\hspace{.5em}\titlerule*{.}\contentspage}
\titlecontents{section}[1\ccwd]{\songti \zihao{-4}}
{\thecontentslabel\hspace{\ccwd}}{}
{\hspace{.5em}\titlerule*{.}\contentspage}
\titlecontents{subsection}[2\ccwd]{\songti \zihao{-4}}
{\thecontentslabel\hspace{\ccwd}}{}
{\hspace{.5em}\titlerule*{.}\contentspage}

% 前置页面(原创性声明、中英文摘要、目录等)
\renewcommand{\frontmatter}{
  \pagenumbering{Roman}
  \pagestyle{BIThesis}
}

% 正文页面
\renewcommand{\mainmatter}{
  \pagenumbering{arabic}
  \pagestyle{BIThesis}
}

% 设置 caption 与 figure 之间的距离
\setlength{\abovecaptionskip}{11pt}
\setlength{\belowcaptionskip}{9pt}

% 设置图片的 caption 格式
\renewcommand{\thefigure}{\thechapter-\arabic{figure}}
\captionsetup[figure]{font=small,labelsep=space}

% 设置 listings 源代码高亮的格式
\AtBeginDocument{
  \renewcommand{\lstlistingname}{代码}
  \renewcommand{\thelstlisting}{\arabic{chapter}-\arabic{lstlisting}}
}

\definecolor{codegreen}{rgb}{0,0.6,0}
\definecolor{codegray}{rgb}{0.5,0.5,0.5}
\definecolor{codepurple}{rgb}{0.58,0,0.82}
\definecolor{backcolour}{rgb}{0.95,0.95,0.92}
\lstdefinestyle{examplestyle}{
    backgroundcolor=\color{backcolour},
    commentstyle=\color{codegreen},
    keywordstyle=\color{magenta},
    numberstyle=\tiny\color{codegray},
    stringstyle=\color{codepurple},
    basicstyle=\ttfamily\footnotesize,
    breakatwhitespace=false,
    breaklines=true,
    captionpos=b,
    keepspaces=true,
    numbers=left,
    numbersep=5pt,
    showspaces=false,
    showstringspaces=false,
    showtabs=false,
    tabsize=2
}
\lstset{style=examplestyle}


% 设置表格的 caption 格式和 caption 与 table 之间的垂直距离
\renewcommand{\thetable}{\thechapter-\arabic{table}}
\captionsetup[table]{font=small,labelsep=space,skip=2pt}

% 调整底层 TeX 排版引擎参数以保证所有段落能够很好地以两端对齐的方式呈现
\tolerance=1
\emergencystretch=\maxdimen
\hyphenpenalty=10000
\hbadness=10000

% 设置数学公式编号格式
\renewcommand{\theequation}{\arabic{chapter}-\arabic{equation}}

\newcommand{\unnumchapter}[1]{
  \chapter*{\vskip 10bp\textmd{#1} \vskip -6bp}
  \addcontentsline{toc}{chapter}{#1}
  \stepcounter{chapter}
}



%    \end{macrocode}
%    \begin{macrocode}
%</book>
%    \end{macrocode}
%    \begin{macrocode}
%<*article>
%    \end{macrocode}
%    \begin{macrocode}

\newif\if@bit@labreport
\newif\if@bit@proposalreport

\DeclareOption{lab-report}{\@bit@labreporttrue\@bit@proposalreportfalse}
\DeclareOption{proposal-report}{\@bit@labreportfalse\@bit@proposalreporttrue}
\DeclareOption*{\PassOptionsToClass{\CurrentOption}{ctexart}}
\ExecuteOptions{lab-report}
\ProcessOptions\relax

\PassOptionsToPackage{AutoFakeBold,AutoFakeSlant}{xeCJK}
\LoadClass[UTF8,zihao=-4]{ctexart}%

\if@bit@labreport
  \RequirePackage[a4paper,left=3.18cm,right=3.18cm,top=2.54cm,bottom=2.54cm,includeheadfoot]{geometry}%
\else
  \RequirePackage[a4paper,left=3cm,right=2.4cm,top=2.6cm,bottom=2.38cm,includeheadfoot]{geometry}
\fi

\RequirePackage{fontspec}%
\RequirePackage{setspace}%
\RequirePackage{graphicx}%
\RequirePackage{fancyhdr}%
\RequirePackage{pdfpages}%
\RequirePackage{setspace}%
\RequirePackage{booktabs}%
\RequirePackage{multirow}%
\RequirePackage{caption}%

\if@bit@labreport
  \RequirePackage{titlesec}%
  \RequirePackage{float}%
  \RequirePackage{etoolbox}
\fi

% 设置参考文献编译后端为 biber,引用格式为 GB/T7714-2015 格式
% 参考文献使用宏包见 https://github.com/hushidong/biblatex-gb7714-2015
\RequirePackage[style=gb7714-2015,backend=biber]{biblatex}

\if@bit@labreport
  % 将西文字体设置为 Times New Roman
  \setromanfont{Times New Roman}%

  % 设置文档标题深度
  \setcounter{tocdepth}{3}%
  \setcounter{secnumdepth}{3}%

  %%
  % 设置一级标题、二级标题格式
  \ctexset{section={%
    format={\raggedright \bfseries \songti \zihao{-3}},%
    name = {,.},%
    number = \chinese{section}%
    }%
  }%
  \ctexset{subsection={%
    format = {\bfseries \songti \raggedright \zihao{-4}},%
    }%
  }%

  % 页眉和页脚(页码)的格式设定
  \fancyhf{}%
  \fancyhead[L]{\fontsize{10.5pt}{10.5pt}\selectfont\kaishu{\reportName}}%
  \fancyfoot[C]{\fontsize{9pt}{9pt}\selectfont\kaishu{\thepage}}%
  \renewcommand{\headrulewidth}{0.5pt}%
  \renewcommand{\footrulewidth}{0pt}%

  \AtBeginDocument{
  }
\fi

\if@bit@proposalreport
  % 定义 caption 字体为楷体
  \DeclareCaptionFont{kaiticaption}{\kaishu \normalsize}

  % 设置图片的 caption 格式
  \renewcommand{\thefigure}{\thesection-\arabic{figure}}
  \captionsetup[figure]{font=small,labelsep=space,skip=10bp,labelfont=bf,font=kaiticaption}

  % 设置表格的 caption 格式
  \renewcommand{\thetable}{\thesection-\arabic{table}}
  \captionsetup[table]{font=small,labelsep=space,skip=10bp,labelfont=bf,font=kaiticaption}

  % 输出大写数字日期
  \ctexset{today=big}

  % 将西文字体设置为 Times New Roman
  \setromanfont{Times New Roman}

  %% 将中文楷体设置为 SIMKAI.TTF(如果需要)
  % \setCJKfamilyfont{zhkai}{[SIMKAI.TTF]}
  % \newcommand*{\kaiti}{\CJKfamily{zhkai}}

  % 设置文档标题深度
  \setcounter{tocdepth}{3}
  \setcounter{secnumdepth}{3}

  %%
  % 设置一级标题、二级标题格式
  % 一级标题:小三,宋体,加粗,段前段后各半行
  \ctexset{section={
    format={\raggedright \bfseries \songti \zihao{-3}},
    beforeskip = 24bp plus 1ex minus .2ex,
    afterskip = 24bp plus .2ex,
    fixskip = true,
    name = {,.\quad}
    }
  }
  % 二级标题:小四,宋体,加粗,段前段后各半行
  \ctexset{subsection={
    format = {\bfseries \songti \raggedright \zihao{4}},
    beforeskip =24bp plus 1ex minus .2ex,
    afterskip = 24bp plus .2ex,
    fixskip = true,
    }
  }
  % 页眉和页脚(页码)的格式设定
  \fancyhf{}
  \fancyhead[R]{\fontsize{10.5pt}{10.5pt}\selectfont{北京理工大学本科生毕业设计(论文)开题报告}}
  \fancyfoot[R]{\fontsize{9pt}{9pt}\selectfont{\thepage}}
  \renewcommand{\headrulewidth}{1pt}
  \renewcommand{\footrulewidth}{0pt}
\fi


\AtBeginDocument{
  \if@bit@labreport
    \topskip=0pt

\begin{titlepage}
  \vspace*{-16mm}
  \centering

  \vspace{23mm}

  \hspace{-6mm}\heiti\fontsize{24pt}{24pt}\selectfont{\reportName}

  \vspace{87mm}

  \flushleft
  \begin{spacing}{2.2}
    \hspace{39mm}\songti\fontsize{16pt}{16pt}\selectfont{\textbf{学\hspace{11mm}院:}\underline{\makebox[51mm][c]{\deptName}}}

    \hspace{39mm}\songti\fontsize{16pt}{16pt}\selectfont{\textbf{专\hspace{11mm}业:}\underline{\makebox[51mm][c]{\majorName}}}

    \hspace{39mm}\songti\fontsize{16pt}{16pt}\selectfont{\textbf{班\hspace{11mm}级:}\underline{\makebox[51mm][c]{\className}}}

    \hspace{39mm}\songti\fontsize{16pt}{16pt}\selectfont{\textbf{姓\hspace{11mm}名:}\underline{\makebox[51mm][c]{\yourName}}}

    \hspace{39mm}\songti\fontsize{16pt}{16pt}\selectfont{\textbf{任课教师:}\underline{\makebox[51mm][c]{\teacherName}}}
  \end{spacing}

  \vspace{33mm}

  \centering
  \hspace{-6mm}\songti\fontsize{12pt}{12pt}\selectfont{\today}
\end{titlepage}

    % 正文开始
    \pagestyle{fancy}
    \setcounter{page}{1}%
  \fi
  \if@bit@proposalreport
    % 报告封面
    %%
% The BIThesis Template for Bachelor Graduation Thesis
%
% 北京理工大学毕业设计开题报告 —— 使用 XeLaTeX 编译
%
% Copyright 2020 Spencer Woo
%
% This work may be distributed and/or modified under the
% conditions of the LaTeX Project Public License, either version 1.3
% of this license or (at your option) any later version.
% The latest version of this license is in
%   http://www.latex-project.org/lppl.txt
% and version 1.3 or later is part of all distributions of LaTeX
% version 2005/12/01 or later.
%
% This work has the LPPL maintenance status `maintained'.
%
% The Current Maintainer of this work is Spencer Woo.
%
% This work consists of the files main.tex, misc/cover.tex and
% the external PDF misc/reviewTable.pdf

% 校名顶部非常细小的空白
\topskip=0pt

\begin{titlepage}
  \vspace*{-16mm}
  \centering
  \hspace{-6mm}\songti\fontsize{22pt}{22pt}\selectfont{北京理工大学}

  \vspace{13mm}

  \hspace{-6mm}\heiti\fontsize{24pt}{24pt}\selectfont{本科生毕业设计(论文)开题报告}

  \vspace{53mm}

  \flushleft
  \begin{spacing}{2.2}
    \hspace{46mm}\songti\fontsize{16pt}{16pt}\selectfont{\textbf{学\hspace{11mm}院:}\underline{\makebox[51mm][c]{\deptName}}}

    \hspace{46mm}\songti\fontsize{16pt}{16pt}\selectfont{\textbf{专\hspace{11mm}业:}\underline{\makebox[51mm][c]{\majorName}}}

    \hspace{46mm}\songti\fontsize{16pt}{16pt}\selectfont{\textbf{班\hspace{11mm}级:}\underline{\makebox[51mm][c]{\className}}}

    \hspace{46mm}\songti\fontsize{16pt}{16pt}\selectfont{\textbf{姓\hspace{11mm}名:}\underline{\makebox[51mm][c]{\yourName}}}

    \hspace{46mm}\songti\fontsize{16pt}{16pt}\selectfont{\textbf{指导教师:}\underline{\makebox[51mm][c]{\mentorName}}}

    \hspace{46mm}\songti\fontsize{16pt}{16pt}\selectfont{\textbf{校外指导教师:}\underline{\makebox[40mm][c]{\offCampusMentorName}}}
  \end{spacing}

  \vspace{47mm}

  \centering
  \hspace{-6mm}\songti\fontsize{12pt}{12pt}\selectfont{\today}
\end{titlepage}

  \fi

}

%    \end{macrocode}
%
%    \begin{macrocode}
%    \end{macrocode}
%
%    \begin{macrocode}
%</article>
%    \end{macrocode}
%    \begin{macrocode}
%<*graduate>
%    \end{macrocode}
%    \begin{macrocode}

%% ==================================================
%% BIT-thesis-grd.cls for BIT Thesis
%% modified by yang yating
%% version: 1.4
%% last update: Mar 25th, 2018
%% ==================================================

%% math packages -- conflict with xunicode
\RequirePackage{amsmath,amsthm,amsfonts,amssymb,bm,mathrsfs}
% 直立希腊字母字体
\RequirePackage{upgreek}

\DeclareOption*{\PassOptionsToClass{\CurrentOption}{ctexbook}}
\newif\ifBIT@master\BIT@masterfalse
\newif\ifBIT@doctor\BIT@doctorfalse
\newif\ifBIT@istwoside\BIT@istwosidefalse
\DeclareOption{twoside}{\BIT@istwosidetrue}
\DeclareOption{master}{\BIT@mastertrue}
\DeclareOption{doctor}{\BIT@doctortrue}

\ProcessOptions\relax
\ifBIT@istwoside
\LoadClass[zihao=-4,a4paper,UTF8,space=auto]{ctexbook}
\else
\LoadClass[zihao=-4,a4paper,oneside,openany,UTF8,space=auto]{ctexbook}
\fi

%%
%% the setup of ctex package
%%
\def\contentsname{目\BITspace 录}
\def\listfigurename{插\BITspace 图}
\def\listtablename{表\BITspace 格}

%%
%% 封面标题
%%
\def\BIT@label@major{学~~~~科~~~~专~~~~业}
\def\BIT@label@title{论文题目}
\def\BIT@label@author{作~~~~者~~~~姓~~~~名}
\def\BIT@label@classification{中图分类号:}
\def\BIT@label@confidential{密级}
\def\BIT@label@UDC{UDC\!分类号:}
\def\BIT@label@serialnumber{编号}
\def\BIT@label@thesis{学位论文}
\def\BIT@label@advisor{指~~~~导~~~~教~~~~师}
\def\BIT@label@degree{申~~~~请~~~~学~~~~位}
\def\BIT@label@submitdate{论文提交日期}
\def\BIT@label@defenddate{论~文~答~辩~日~期}
\def\BIT@label@institute{学~~~~院~~~~名~~~~称}
\def\BIT@label@school{学~位~授~予~单~位}
\def\BIT@label@chairman{答辩委员会主席}

%%
%% 封面内容
%%

\def\BIT@value@classification{}
\def\BIT@value@confidential{}
\def\BIT@value@UDC{}
\def\BIT@value@serialnumber{}
\def\BIT@value@school{}
\def\BIT@value@degree{}
\def\BIT@value@title{~~~~~(论~文~题~目)~~~~~}
\def\BIT@value@vtitle{竖排论文题目}
\def\BIT@value@titlemark{\BIT@value@title}
\def\BIT@value@author{(作~者~姓~名)}
\def\BIT@value@advisor{(姓名、专业技术职务、学位)}
\def\BIT@value@advisorinstitute{(单位)}
\def\BIT@value@major{}
\def\BIT@value@studentnumber{} % _ added by wei.jianwen@gmail.com
\def\BIT@value@submitdate{}
\def\BIT@value@defenddate{}
\def\BIT@value@institute{}
\def\BIT@value@chairman{}
\def\BIT@label@statement{}

%% 设置圆圈的格式 或使用\textcircled
\usepackage{tikz}
\usepackage{etoolbox}
\newcommand{\circled}[2][]{\tikz[baseline=(char.base)]
    {\node[shape = circle, draw, inner sep = 1pt]
    (char) {\phantom{\ifblank{#1}{#2}{#1}}};
    \node at (char.center) {\makebox[0pt][c]{#2}};}}
\robustify{\circled}

%% 论文原创性声明
\def\BIT@label@original{研究成果声明}
\def\BIT@label@authorization{关于学位论文使用权的说明}
\def\BIT@label@authorsign{作者签名:}
\def\BIT@label@Supervisorsign{导师签名:}
\def\BIT@label@originalDate{签字日期:}
\def\BIT@label@originalcontent{\BITspace\BITspace 本人郑重声明:所提交的学位论文是我本人在指导教师的指导下进行的研究工作获得的研究成果。尽我所知,文中除特别标注和致谢的地方外,学位论文中不包含其他人已经发表或撰写过的研究成果,也不包含为获得北京理工大学或其它教育机构的学位或证书所使用过的材料。与我一同工作的合作者对此研究工作所做的任何贡献均已在学位论文中作了明确的说明并表示了谢意。\par 特此申明。}
\def\BIT@label@authorizationcontent{\BITspace\BITspace 本人完全了解北京理工大学有关保管、使用学位论文的规定,其中包括:\circled{1} 学校有权保管、并向有关部门送交学位论文的原件与复印件;\circled{2} 学校可以采用影印、缩印或其它复制手段复制并保存学位论文;\circled{3} 学校可允许学位论文被查阅或借阅;\circled{4} 学校可以学术交流为目的,复制赠送和交换学位论文;\circled{5} 学校可以公布学位论文的全部或部分内容(保密学位论文在解密后遵守此规定)。}

%%
%% 英语封面标题
%%
\def\BIT@label@englishadvisor{Supervisor:}
\def\BIT@label@englishstatement{Submitted in total fulfilment
  of the requirements for the degree of \BIT@value@englishdegree \\
  in \BIT@value@englishmajor}
\def\BIT@label@englishauthor{Candidate Name:}
\def\BIT@label@englishadvisor{Faculty Mentor:}
\def\BIT@label@englishchairman{Chair, Thesis Committee:}
\def\BIT@label@englishinstitute{School or Department:}
\def\BIT@label@englishdegree{Degree Applied:}
\def\BIT@label@englishmajor{Major:}
\def\BIT@label@englishschool{Degree by:}
\def\BIT@label@englishdate{The Date of Defence:}


%%
%% 英语封面内容
%%
\def\BIT@value@englishtitle{(English Title of Thesis)}
\def\BIT@value@englishauthor{(Author Name)}
\def\BIT@value@englishadvisor{(Supervisor Name)}
\def\BIT@value@englishinstitute{(Institute Name)}
\def\BIT@value@englishscholl{(BIT)}
\def\BIT@value@englishchair{(someone)}
\def\BIT@value@englishdate{}
\def\BIT@value@englishdegree{}
\def\BIT@value@englishmajor{}



\def\BIT@label@abstract{摘要}
\def\BIT@label@englishabstract{Abstract}
\def\BIT@label@keywords{关键词:}
\def\BIT@label@englishkeywords{Key Words:~}
\def\BIT@label@conclusion{结论}
\def\BIT@label@appendix{附录}
\def\BIT@label@publications{攻读学位期间发表论文与研究成果清单}
\def\BIT@label@projects{攻读学位期间参与的项目}
\def\BIT@label@resume{作者简介}
\def\BIT@label@reference{参考文献!!!!}
\def\BIT@label@thanks{致谢}
\def\BIT@value@templateversion{v1.2}
%%
%% label in the head 页眉页脚
%%
\def\BIT@label@headschoolname{北京理工大学硕士学位论文}

%% 当前模板的版本
\newcommand{\version}{\BIT@value@templateversion}

%% ==============引用geometry 宏包设置纸张和页面========================
% 设置版面:上3.5cm,下2.5cm,左2.7cm,右2.7cm,页眉2.5cm,页脚1.8cm,装订线0cm
\usepackage[%
paper=a4paper,%
top=3.5cm,% 上 3.5cm %
bottom=2.5cm,% 下 2.5cm %
left=2.7cm,% 左 2.7cm %
right=2.7cm,% 右 2.7cm %
headheight=1.0cm,% 页眉 2.5cm %
footskip=0.7cm% 页脚 1.8cm %
]{geometry} % 页面设置 %

\parskip 0.5ex plus 0.25ex minus 0.25ex
%% Command -- Clear Double Page
\def\cleardoublepage{\clearpage\if@twoside \ifodd\c@page\else
  \thispagestyle{empty}%
  \hbox{}\newpage\if@twocolumn\hbox{}\newpage\fi\fi\fi}
% 设置行距,大概为22榜
\RequirePackage{setspace}
\setstretch{1.523}

%% 设置章节格式, 黑体三号加粗居中,行距22磅,与正文或节标题的间距设定为段后间距1行。章序号与章名间空一格。
\ctexset{chapter={
      name = {第,章},
      number = {\arabic{chapter}},
      format = {\bfseries \sffamily \centering \zihao{3}},
      pagestyle = {BIT@headings},
      beforeskip = 16 bp,
      afterskip = 32 bp,
      fixskip = true,
  }
}
%% 设置一级章节格式
% 黑体四号加粗顶左,行距22磅,与上一节的间距为1行,与下面正文或节标题的段间间距为0.5行。序号与题目间空一格。

\ctexset{section={
  format={\raggedright \bfseries \sffamily \zihao{4}},
  beforeskip = 28bp plus 1ex minus .2ex,
  afterskip = 24bp plus .2ex,
  fixskip = true,
  }
}

% 设置二级标题格式

% 黑体小四加粗顶左,行距22磅,与上一节的间距为1行,与下面正文或节标题的段间间距为0.5行。序号与题目间空一格。 

\ctexset{subsection={
   format = {\bfseries \sffamily \raggedright \zihao{-4}},
   beforeskip =28bp plus 1ex minus .2ex,
   afterskip = 24bp plus .2ex,
   fixskip = true,
   }
}

% 设置三节标题格式

\ctexset{subsubsection={
      format={\heiti \raggedright \zihao{-4}},
      beforeskip=28bp plus 1ex minus .2ex,
      afterskip=24bp plus .2ex,
      fixskip=true,
  }
}

%% 设定目录格式。目录颜色更改黑色
\addtocontents{toc}{\protect\hypersetup{hidelinks}}
\addtocontents{lot}{\protect\hypersetup{hidelinks}}
\addtocontents{lof}{\protect\hypersetup{hidelinks}}

\RequirePackage{titletoc}
\titlecontents{chapter}[0pt]{\songti \zihao{4}}
    {\bf\thecontentslabel\hspace{\ccwd}}{\bf}
    {\hspace{.5em}\titlerule*{.}\contentspage}
\titlecontents{section}[2\ccwd]{\songti \zihao{-4}}
    {\thecontentslabel\hspace{\ccwd}}{}
    {\hspace{.5em}\titlerule*{.}\contentspage}
\titlecontents{subsection}[4\ccwd]{\songti \zihao{-4}}
    {\thecontentslabel\hspace{\ccwd}}{}
    {\hspace{.5em}\titlerule*{.}\contentspage}

\titlecontents{figure}[0pt]{\songti\zihao{-4}}
    {\figurename~\thecontentslabel\quad}{\hspace*{-1.5cm}}
    {\hspace{.5em}\titlerule*{.}\contentspage}

\titlecontents{table}[0pt]{\songti\zihao{-4}}
    {\tablename~\thecontentslabel\quad}{\hspace*{-1.5cm}}
    {\hspace{.5em}\titlerule*{.}\contentspage}

%% 选择编译
\RequirePackage{ifthen}

%% check pdfTeX mode
\RequirePackage{ifpdf}

%% fancyhdr 页眉页脚控制
\RequirePackage{fancyhdr}

% 空 页眉页脚
\fancypagestyle{BIT@empty}{%
  \fancyhf{}}

% ======正文页眉页脚=================
\fancypagestyle{BIT@headings}{%
  \fancyhf{}
  \fancyfoot[C]{\songti\zihao{5} \thepage}
  \fancyhead[C]{\ifBIT@master\zihao{5}{\songti 北京理工大学硕士学位论文}
                \else\zihao{5}{\songti 北京理工大学博士学位论文}\fi}
  }

% ==================================对于openright 选项,必须保证章页右开,且如果前章末页内容须清空其页眉页脚。===================
\let\BIT@cleardoublepage\cleardoublepage
\newcommand{\BIT@clearemptydoublepage}{%
	\clearpage{\pagestyle{BIT@empty}\BIT@cleardoublepage}}
\let\cleardoublepage\BIT@clearemptydoublepage

 % ================修该frontmatter 的页码为大写罗马格式,并调整页面风格===============
\renewcommand{\frontmatter}{
 \if@openright\cleardoublepage\else\clearpage\fi
  \@mainmatterfalse
  \pagenumbering{Roman}
  \pagestyle{BIT@headings}
}
% =======================修改mainmatter 的页码为阿拉伯格式,并调整页面风格========================
\renewcommand{\mainmatter}{
  \if@openright\cleardoublepage\else\clearpage\fi
  \@mainmattertrue
  \pagenumbering{arabic}
  \pagestyle{BIT@headings}
}


%% 复杂表格
\RequirePackage{threeparttable}
\RequirePackage{dcolumn}
\RequirePackage{multirow}
\RequirePackage{booktabs}
\newcolumntype{d}[1]{D{.}{.}{#1}}% or D{.}{,}{#1} or D{.}{\cdot}{#1}


%% 定义几个常用的数学常量符号
\newcommand{\me}{\mathrm{e}} % 定义 对数常数e,虚数符号i,j以及微分算子d为直立体。
\newcommand{\mi}{\mathrm{i}}
\newcommand{\mj}{\mathrm{j}}
\newcommand{\dif}{\,\mathrm{d}} 

\theoremstyle{plain}
  \newtheorem{algo}{算法~}[chapter]
  \newtheorem{thm}{定理~}[chapter]
  \newtheorem{lem}[thm]{引理~}
  \newtheorem{prop}[thm]{命题~}
  \newtheorem{cor}[thm]{推论~}
\theoremstyle{definition}
  \newtheorem{defn}{定义~}[chapter]
  \newtheorem{conj}{猜想~}[chapter]
  \newtheorem{exmp}{例~}[chapter]
  \newtheorem{rem}{注~}
  \newtheorem{case}{情形~}
\renewcommand{\proofname}{\bf 证明}

%% 英文字体使用 Times New Roman
\RequirePackage{xltxtra} % \XeTeX Logo
\setmainfont{Times New Roman}
\setsansfont{Arial}
\setmonofont{Courier New}


%% graphics packages
\RequirePackage{graphicx}
%% 并列子图
\RequirePackage{subfigure}

\RequirePackage{wrapfig}
%% ===========================设置图表标题选项==========================
\RequirePackage{amsmath}
\RequirePackage{caption}
\DeclareCaptionLabelSeparator{zhspace}{\hspace{1\ccwd}}
\DeclareCaptionFont{fontsize}{\zihao{5}}
\captionsetup{
  font = {fontsize},
  labelsep = zhspace,
}
\captionsetup[table]{
  position = top,
  aboveskip = 6bp,
  belowskip = 6bp,
}
\numberwithin{table}{chapter}
\captionsetup[figure]{
  position = bottom,
  aboveskip = 6bp,
  belowskip = 6bp,
}

%% 如果插入的图片没有指定扩展名,那么依次搜索下面的扩展名所对应的文件
\DeclareGraphicsExtensions{.pdf,.eps,.png,.jpg,.jpeg}
% ccaption -- bicaption
% \RequirePackage{ccaption}
% \captiondelim{\ }
% \captionnamefont{\songti\zihao{5}}
% \captiontitlefont{\songti\zihao{5}}

\RequirePackage[
  backend=biber,
  style=gb7714-2015,
  gbalign=gb7714-2015,
  gbnamefmt=lowercase,
  gbpub=false,
  doi=false,
  url=false,
  eprint=false,
  isbn=false,
]{biblatex}

% 将浮动参数设为较宽松的值
\renewcommand{\textfraction}{0.15}
\renewcommand{\topfraction}{0.85}
\renewcommand{\bottomfraction}{0.65}
\renewcommand{\floatpagefraction}{0.60}


% 定公式、图、表编号为"3-1"的形式,即分隔符由.变为短杠
\renewcommand\theequation{\arabic{chapter}.\arabic{equation}}
\renewcommand\thefigure{\arabic{chapter}.\arabic{figure}}
\renewcommand\thetable{\arabic{chapter}.\arabic{table}}

% 颜色宏包
\RequirePackage{xcolor}


% 中文破折号
\newcommand{\cndash}{\rule{0.0em}{0pt}\rule[0.35em]{1.4em}{0.05em}\rule{0.2em}{0pt}}

% listings 源代码显示宏包
\RequirePackage{listings}
\lstset{tabsize=4, %
  frame=shadowbox, % 把代码用带有阴影的框圈起来
  commentstyle=\color{red!50!green!50!blue!50},% 浅灰色的注释
  rulesepcolor=\color{red!20!green!20!blue!20},% 代码块边框为淡青色
  keywordstyle=\color{blue!90}\bfseries, % 代码关键字的颜色为蓝色,粗体
  showstringspaces=false,% 不显示代码字符串中间的空格标记
  stringstyle=\ttfamily, % 代码字符串的特殊格式
  keepspaces=true, %
  breakindent=22pt, %
  numbers=left,% 左侧显示行号
  stepnumber=1,%
  numberstyle=\tiny, % 行号字体用小号
  basicstyle=\footnotesize, %
  showspaces=false, %
  flexiblecolumns=true, %
  breaklines=true, % 对过长的代码自动换行
  breakautoindent=true,%
  breakindent=4em, %
  aboveskip=1em, % 代码块边框
  %% added by http://bbs.ctex.org/viewthread.php?tid=53451
  fontadjust,
  captionpos=t,
  framextopmargin=2pt,framexbottommargin=2pt,abovecaptionskip=-3pt,belowcaptionskip=3pt,
  xleftmargin=4em,xrightmargin=4em, % 设定listing左右的空白
  texcl=true,
  % 设定中文冲突,断行,列模式,数学环境输入,listing数字的样式
  extendedchars=false,columns=flexible,mathescape=true
  numbersep=-1em
}
\renewcommand{\lstlistingname}{代码} %% 重命名Listings标题头

%% hyperref package
\definecolor{navyblue}{RGB}{0,0,128} 
\RequirePackage{hyperref}
\hypersetup{
  bookmarksnumbered,%
  linktoc=all,
  colorlinks=true,
  citecolor=navyblue,
  filecolor=cyan,
  linkcolor=navyblue,
  linkbordercolor=navyblue,
  urlcolor=navyblue,
  plainpages=false,%
  pdfstartview=FitH
}

%% enumerate 列表环境间距调节
\usepackage{enumitem}
% \setenumerate[1]{itemsep=0pt,partopsep=0pt,parsep=\parskip,topsep=5pt}
% \setitemize[1]{itemsep=0pt,partopsep=0pt,parsep=\parskip,topsep=0pt}
% \setdescription{itemsep=0pt,partopsep=0pt,parsep=\parskip,topsep=5pt}

% _ BITspace
% \newcommand\BITspace{\protect\CTEX@spaceChar\protect\CTEX@spaceChar}
\newcommand{\BITspace}[1][1]{\hspace{#1\ccwd}}

\def\BIT@getfileinfo#1 #2 #3\relax#4\relax{%
  \def\BITfiledate{#1}%
  \def\BITfileversion{#2}%
  \def\BITfileinfo{#3}}%
\expandafter\ifx\csname ver@bitmaster-xetex.cls\endcsname\relax
  \edef\reserved@a{\csname ver@ctextemp_bitmaster-xetex.cls\endcsname}
\else
  \edef\reserved@a{\csname ver@bitmaster-xetex.cls\endcsname}
\fi
\expandafter\BIT@getfileinfo\reserved@a\relax? ? \relax\relax
\def\BIT@underline[#1]#2{%
  \underline{\hbox to #1{\hfill#2\hfill}}}
\def\BITunderline{\@ifnextchar[\BIT@underline\underline}

% 中文标题页的可用命令
\newcommand\classification[1]{\def\BIT@value@classification{#1}}
\newcommand\studentnumber[1]{\def\BIT@value@studentnumber{#1}}
\newcommand\confidential[1]{\def\BIT@value@confidential{#1}}
\newcommand\UDC[1]{\def\BIT@value@UDC{#1}}
\newcommand\serialnumber[1]{\def\BIT@value@serialnumber{#1}}
\newcommand\school[1]{\def\BIT@value@school{#1}}
\newcommand\degree[1]{\def\BIT@value@degree{#1}}
\renewcommand\title[2][\BIT@value@title]{%
  \def\BIT@value@title{#2}
  \def\BIT@value@titlemark{\MakeUppercase{#1}}}

\newcommand\vtitle[1]{\def\BIT@value@vtitle{#1}}
\renewcommand\author[1]{\def\BIT@value@author{#1}}
\newcommand\advisor[1]{\def\BIT@value@advisor{#1}}
\newcommand\advisorinstitute[1]{\def\BIT@value@advisorinstitute{#1}}
\newcommand\major[1]{\def\BIT@value@major{#1}}
\newcommand\submitdate[1]{\def\BIT@value@submitdate{#1}}
\newcommand\defenddate[1]{\def\BIT@value@defenddate{#1}}
\newcommand\institute[1]{\def\BIT@value@institute{#1}}
\newcommand\chairman[1]{\def\BIT@value@chairman{#1}}

%% 第一页和第二页
%  “绘制”BIT中文标题页
\renewcommand\maketitle[1]{%
  \cleardoublepage
  \thispagestyle{empty}
  \begin{center}
    \vspace*{60mm}
    {\heiti\zihao{-2} \BIT@value@title}
    \vskip 40mm
    {\heiti \zihao{-3} \BIT@value@author} % 黑体 小三
     \vskip 4mm
    {\heiti \zihao{-3} \BIT@value@defenddate} % 黑体 小三
  \end{center}
  \clearpage
  \if@twoside
    \thispagestyle{empty}
    \cleardoublepage
  \fi
 }

\newcommand\makeInfo[1]%
 {
  \newpage
  \cleardoublepage
  \thispagestyle{empty}

% udc ltz 
{ %
  {\heiti \zihao{5} \noindent \BIT@label@classification} \BIT@value@classification \\
  {\heiti \zihao{5} \BIT@label@UDC}  \BIT@value@UDC
}

   \begin{center}

    \vskip \stretch{1}
       {\heiti\zihao{-2} \BIT@value@title}
    \vskip \stretch{1}

    {\fangsong\zihao{4}}
    \def\tabcolsep{1pt}
    \def\arraystretch{1.5}

	% 黑体 小三
    {\heiti\zihao{-3}
     \begin{tabular}{l p{3mm} c}
      \BIT@label@author & &\BITunderline[180pt]{\BIT@value@author}
    \\
      \BIT@label@institute & & \BITunderline[180pt]{\BIT@value@institute}
    \\
      \BIT@label@advisor & &  \BITunderline[180pt]{\BIT@value@advisor}
    \\
      \BIT@label@chairman & &   \BITunderline[180pt]{\BIT@value@chairman}
    \\
      \BIT@label@degree & &    \BITunderline[180pt]{\BIT@value@degree}
    \\
      \BIT@label@major & &    \BITunderline[180pt]{\BIT@value@major}
    \\
      \BIT@label@school & &   \BITunderline[180pt]{\BIT@value@school}
    \\
      \BIT@label@defenddate & &  \BITunderline[180pt]{\BIT@value@defenddate}
    \end{tabular}}

  \end{center}

  \vskip \stretch{0.5}
  \clearpage
  \if@twoside
    \thispagestyle{empty}
    \cleardoublepage
  \fi
}

% English Title Page
% 英文标题页可用命令
\newcommand\englishtitle[1]{\def\BIT@value@englishtitle{#1}}
\newcommand\englishauthor[1]{\def\BIT@value@englishauthor{#1}}
\newcommand\englishadvisor[1]{\def\BIT@value@englishadvisor{#1}}
\newcommand\englishschool[1]{\def\BIT@value@englishschool{#1}}
\newcommand\englishinstitute[1]{\def\BIT@value@englishinstitute{#1}}
\newcommand\englishdate[1]{\def\BIT@value@englishdate{#1}}
\newcommand\englishdegree[1]{\def\BIT@value@englishdegree{#1}}
\newcommand\englishmajor[1]{\def\BIT@value@englishmajor{#1}}
\newcommand\englishchairman[1]{\def\BIT@value@englishchairman{#1}}

% "绘制"英文标题页
\newcommand\makeEnglishInfo[1]{%
  \cleardoublepage
  \thispagestyle{empty}

   \begin{center}


   \vspace*{10em}
% 论文题目	  Times New Roman 小二 加粗
   {\zihao{-2}\textbf{\BIT@value@englishtitle}}
   % \bfseries
   \vskip \stretch{1}
   
% Times New Roman 小三
   {\zihao{-3}
     \begin{tabular}{ll}
      \BIT@label@englishauthor & \BITunderline[200pt]{\BIT@value@englishauthor}
    \\
      \BIT@label@englishinstitute &  \BITunderline[200pt]{\BIT@value@englishinstitute}
    \\
      \BIT@label@englishadvisor &  \BITunderline[200pt]{\BIT@value@englishadvisor}
    \\
      \BIT@label@englishchairman &   \BITunderline[200pt]{\BIT@value@englishchairman}
    \\
      \BIT@label@englishdegree &    \BITunderline[200pt]{\BIT@value@englishdegree}
    \\
      \BIT@label@englishmajor &     \BITunderline[200pt]{\BIT@value@englishmajor}
    \\
      \BIT@label@englishschool &     \BITunderline[200pt]{\BIT@value@englishschool}
    \\
      \BIT@label@englishdate &   \BITunderline[200pt]{\BIT@value@englishdate}
    \end{tabular}}

  \end{center}

  \vskip \stretch{0.5}
  \clearpage
  \if@twoside
  \thispagestyle{empty}
   \cleardoublepage
  \fi
}

% 绘制树立排放的论文题目和学校名称

\newcommand\makeVerticalTitle{
   \cleardoublepage
   \thispagestyle{empty}
   \vskip 5cm
   \begin{center}
    \setstretch{1.1}
    \begin{minipage}{2em}
      \begin{center}
        {\heiti\zihao{3}\BIT@value@vtitle}
          \vskip 2em
        {\heiti\zihao{3}\BIT@value@school}
      \end{center}
    \end{minipage}
   \end{center}
    \clearpage
    \if@twoside
        \thispagestyle{empty}
        \cleardoublepage
    \fi
}

% 原创性声明
\newcommand\makeDeclareOriginal{%
  \cleardoublepage
  \pdfbookmark[0]{声明}{statement}
  \thispagestyle{empty}
  \begin{center}
  {\bf\zihao{3} \BIT@label@original}
  \end{center}
  \vskip 10pt
  {\zihao{4}\BIT@label@originalcontent}
  \vskip 10pt
  \hspace{8em}{\zihao{4}\BIT@label@authorsign} \BITunderline[6em]{} \hspace{2em} {\zihao{4}\BIT@label@originalDate} \BITunderline[6em]{}

  \vskip 30mm

  \begin{center}
  {\bf\zihao{3} \BIT@label@authorization}
  \end{center}
  \vskip 10pt
  {\zihao{4} \BIT@label@authorizationcontent}
  \vskip 40pt

  \hspace{8em}{\zihao{4}\BIT@label@authorsign} \BITunderline[6em]{} \hspace{2em} {\zihao{4}\BIT@label@Supervisorsign} \BITunderline[6em]{}
  \vskip 15pt
  \hspace{8em}{\zihao{4}\BIT@label@originalDate} \BITunderline[6em]{} \hspace{2em} {\zihao{4}\BIT@label@originalDate} \BITunderline[6em]{}
  \clearpage
  \if@twoside
    \thispagestyle{empty}
    \cleardoublepage
  \fi

}


% 页眉页脚
\pagestyle{fancy}
\fancyhf{}
\fancyhead[C]{\songti \zihao{5} \BIT@label@headschoolname}  % 奇数页左页眉
\fancyfoot[C]{\songti \zihao{5} {\thepage}}      % 页脚


\fancypagestyle{plain}{% 设置开章页页眉页脚风格(只有页码作为页脚)
  \fancyhf{}%
  \fancyfoot[C]{\songti \zihao{5} \BIT@label@headschoolname}
  \fancyfoot[C]{\songti \zihao{5} ~---~{\thepage}~---~} % 首页页脚格式
}


 % 中文摘要
 \newenvironment{abstract}
 {
  \cleardoublepage
  \chapter{\BIT@label@abstract}
 }
 {}
% 下一页从偶数页开始
 \newcommand\beginatevenpage{
 \clearpage
  \if@twoside
    \thispagestyle{empty}
    \cleardoublepage
  \fi
 }
 % 中文关键词
 \newcommand\keywords[1]{%
   \vspace{2ex}\noindent{\bf \BIT@label@keywords} #1}

 % 英文摘要
 \newenvironment{englishabstract}
 {
  \clearpage
  \chapter{\BIT@label@englishabstract}
 }
 {}

 % 英文摘要
 \newcommand\englishkeywords[1]{%
  \vspace{2ex}\noindent{\bf \BIT@label@englishkeywords} #1}


% 目录
\renewcommand\tableofcontents{%
  \if@twocolumn
  \@restonecoltrue\onecolumn
  \else
  \@restonecolfalse
  \fi
  \chapter*{\contentsname}% 目录里显示“目录”,否则\chapter*
  \@mkboth{\MakeUppercase\contentsname}{\MakeUppercase\contentsname}%
  \pdfbookmark[0]{目录}{bittoc}
  \@starttoc{toc}%
  \if@restonecol\twocolumn\fi
}


% 参考文献环境
\renewenvironment{thebibliography}[1]
     {\zihao{5}
      \chapter*{\bibname}
      \@mkboth{\MakeUppercase\bibname}{\MakeUppercase\bibname}%
      \addcontentsline{toc}{chapter}{参考文献}
      \list{\@biblabel{\@arabic\c@enumiv}}%
           {\settowidth\labelwidth{\@biblabel{#1}}%
            \leftmargin\labelwidth
            \advance\leftmargin\labelsep
            \setlength{\parsep}{1mm}
            \setlength{\labelsep}{0.5em}
            \setlength{\itemsep}{0.05pc}
            \setlength{\listparindent}{0in}
            \setlength{\itemindent}{0in}
            \setlength{\rightmargin}{0in}
            \@openbib@code
            \usecounter{enumiv}%
            \let\p@enumiv\@empty
            \renewcommand\theenumiv{\@arabic\c@enumiv}}%
      \sloppy
      \clubpenalty4000
      \@clubpenalty \clubpenalty
      \widowpenalty4000%
      \sfcode`\.\@m}
     {\def\@noitemerr
       {\@latex@warning{Empty `thebibliography' environment}}%
      \endlist}


\newenvironment{publications}[1]
     {\chapter{\BIT@label@publications}%
      \@mkboth{\MakeUppercase\BIT@label@publications}
              {\MakeUppercase\BIT@label@publications}%
      \list{\@biblabel{\@arabic\c@enumiv}}%
           {\settowidth\labelwidth{\@biblabel{#1}}%
            \leftmargin\labelwidth
            \advance\leftmargin\labelsep
            \setlength{\parsep}{1mm}
            \setlength{\labelsep}{0.5em}
            \setlength{\itemsep}{0.05pc}
            \setlength{\listparindent}{0in}
            \setlength{\itemindent}{0in}
            \setlength{\rightmargin}{0in}
            \@openbib@code
            \usecounter{enumiv}%
            \let\p@enumiv\@empty
            \renewcommand\theenumiv{\@arabic\c@enumiv}}%
      \sloppy
      \clubpenalty4000
      \@clubpenalty \clubpenalty
      \widowpenalty4000%
      \sfcode`\.\@m}
     {\def\@noitemerr
       {\@latex@warning{Empty `publications' environment}}%
      \endlist}


\newenvironment{projects}[1]
     {\chapter{\BIT@label@projects}%
      \@mkboth{\MakeUppercase\BIT@label@projects}
              {\MakeUppercase\BIT@label@projects}%
      \list{\@biblabel{\@arabic\c@enumiv}}%
           {\settowidth\labelwidth{\@biblabel{#1}}%
            \leftmargin\labelwidth
            \advance\leftmargin\labelsep
            \@openbib@code
            \usecounter{enumiv}%
            \let\p@enumiv\@empty
            \renewcommand\theenumiv{\@arabic\c@enumiv}}%
      \sloppy
      \clubpenalty4000
      \@clubpenalty \clubpenalty
      \widowpenalty4000%
      \sfcode`\.\@m}
     {\def\@noitemerr
       {\@latex@warning{Empty `projects' environment}}%
      \endlist}

    \newenvironment{resume}
  {\chapter{\BIT@label@resume}}
  {}

\newenvironment{resumesection}[1]
  {{\noindent\normalfont\bfseries #1}
   \list{}{\labelwidth\z@
           \leftmargin 2\ccwd}
   \item\relax}
   {\endlist}

\newenvironment{resumeli}[1]
  {{\noindent\normalfont\bfseries #1}
   \list{}{\labelwidth\z@
           \leftmargin 4\ccwd
           \itemindent -2\ccwd
           \listparindent\itemindent}
   \item\relax}
   {\endlist}

\newenvironment{conclusion}
  {\chapter*{结论}
    \@mkboth{结论}{结论}%
    \addcontentsline{toc}{chapter}{结论}}
  {}

\renewenvironment{thanks}
  {\chapter{\BIT@label@thanks}
  \fangsong
  }
  {}

\newenvironment{symbolnote}
  {\chapter{\BIT@label@symbolnote}
  \fangsong}
  {}

  %% ===========================术语=====================
  \newcommand{\bit@denotation@name}{主要符号对照表}
  \newenvironment{denotation}[1][2.5cm]{
      \chapter{\bit@denotation@name} % no tocline
      \noindent\begin{list}{}%
      {\vskip-30bp\zihao{-4}
       \renewcommand\makelabel[1]{##1\hfil}
       \setlength{\labelwidth}{#1} % 标签盒子宽度
       \setlength{\labelsep}{0.5cm} % 标签与列表文本距离
       \setlength{\itemindent}{0cm} % 标签缩进量
       \setlength{\leftmargin}{\labelwidth+\labelsep} % 左边界
       \setlength{\rightmargin}{0cm}
       \setlength{\parsep}{0cm} % 段落间距
       \setlength{\itemsep}{0cm} % 标签间距
      \setlength{\listparindent}{0cm} % 段落缩进量
      \setlength{\topsep}{0pt} % 标签与上文的间距
     }}{\end{list}}
% ====增加化学、国际单位宏包
     \RequirePackage[version=4]{mhchem}
     \RequirePackage{siunitx}
\setcounter{secnumdepth}{4}  % 章节编号深度 (part 对应 -1)
\setcounter{tocdepth}{2}     % 目录深度 (part 对应 -1)

%% End of file `bitmaster-xetex.cls'.

%% =========================================================

%    \end{macrocode}
%    \begin{macrocode}
%</graduate>
%    \end{macrocode}
%
% \iffalse
%<*dtx-style>
\ProvidesPackage{dtx-style}
\RequirePackage{hypdoc}
\RequirePackage{ifthen}
\RequirePackage{fontspec}
\RequirePackage{amsmath}
\RequirePackage{unicode-math}
\RequirePackage[UTF8,scheme=chinese,heading]{ctex}
\RequirePackage[
  top=2.5cm, bottom=2.5cm,
  left=4cm, right=2cm,
  headsep=3mm]{geometry}
\RequirePackage{graphicx}
\RequirePackage{multirow}
\RequirePackage{wrapfig}
\RequirePackage{hologo}
\RequirePackage{array,longtable,booktabs}
\RequirePackage{listings}
\RequirePackage{fancyhdr}
\RequirePackage[dvipsnames]{xcolor}
\RequirePackage{awesomebox}
% \RequirePackage{etoolbox}
\RequirePackage{dirtree}
\RequirePackage{metalogo}
\RequirePackage[tightLists=false]{markdown}
\RequirePackage{caption}
\RequirePackage{tikz}
\usetikzlibrary{positioning}
\RequirePackage{framed}
\RequirePackage{menukeys}
\RequirePackage{float}

 % 设置列表无间隔
\usepackage{enumitem}
\setlist{nosep}

\markdownSetup{
  renderers = {
    link = {\href{#2}{#1}},
  }
}

\hypersetup{
  pdflang     = zh-CN,
  pdftitle    = {BIThesis:北京理工大学学位论文及报告模板},
  pdfauthor   = {冯开宇},
  pdfsubject  = {北京理工大学学位论文及报告模板使用说明},
  pdfkeywords = {论文模板; 北京理工大学; 使用说明},
  pdfdisplaydoctitle = true
}%

\newcommand{\BIThesisLaTeX}{{\BIThesis}北京理工大学学位论文及报告{\LaTeX}模板}
\newcommand{\BIThesisMacroPackage}{{\BIThesis}宏集}
\newcommand{\BIThesisWiki}{{\BIThesis}在线文档}
\newcommand{\BIThesisScaffold}{{\BIThesis}模板}
\newcommand{\BIThesisRelease}{{\BIThesis}模板}
\newcommand{\LPPL}{{\href{https://www.latex-project.org/lppl/lppl-1-3c.txt}{\LaTeX{} Project Public License (1.3.c)}}}
\newcommand{\versionold}{v2.0 BirthdayCake}
\newcommand{\version}{v3 Summer Time}
\ExplSyntaxOn

\AtBeginEnvironment { bitsyntax } {
  \cs_set:Npn \lparen { \textup { ( } }
  \cs_set:Npn \rparen { \textup { ) } }
  \char_set_catcode_active:N |
  \char_set_catcode_active:N <
  \char_set_catcode_active:N (
  \char_set_active_eq:NN | \orbar
  \char_set_active_eq:NN < \syntaxopt@aux
  \char_set_active_eq:NN ( \defaultval@aux
}

\NewDocumentCommand \BIThesisTemplates {m} {
  \str_case:nn {#1} {
    {UT}{本科生毕业论文模板(undergraduate-thesis)}
    {UTE}{本科生全英文专业毕业论文模板(undergraduate-thesis-en)}
    {GT}{研究生学位论文模板(graduate-thesis)}
    {LR}{简易使用报告模板(lab-report)}
    {PT}{本科生毕业设计外文翻译模板(paper-translation)}
    {PS}{北理工主题的 Beamer 模板(presentation-slide)}
    {UP}{本科生毕业设计开题报告(undergraduate-proposal)}
  }
}

\ExplSyntaxOff

\ctexset{
  today=big,
  abstractname=简介
}

\ctexset{section={
  format={\raggedright \bfseries \zihao{-3}},
  name = {第,章}
  }
}

\ctexset{subsection={
  format = {\bfseries \raggedright \zihao{4}}
  }
}



\ifthenelse{\equal{\@nameuse{g__ctex_fontset_tl}}{mac}}{
  \setmainfont{Palatino}
  \setsansfont[Scale=MatchLowercase]{Helvetica}
  \setmonofont[Scale=MatchLowercase]{Menlo}
  \xeCJKsetwidth{‘’“”}{1em}
}{
  \setmainfont[
    Extension      = .otf,
    UprightFont    = *-regular,
    BoldFont       = *-bold,
    ItalicFont     = *-italic,
    BoldItalicFont = *-bolditalic,
  ]{texgyrepagella}
  \setsansfont[
    Extension      = .otf,
    UprightFont    = *-regular,
    BoldFont       = *-bold,
    ItalicFont     = *-italic,
    BoldItalicFont = *-bolditalic,
  ]{texgyreheros}
  \setmonofont[
    Extension      = .otf,
    UprightFont    = *-regular,
    BoldFont       = *-bold,
    ItalicFont     = *-italic,
    BoldItalicFont = *-bolditalic,
    Scale          = MatchLowercase,
    Ligatures      = CommonOff,
  ]{texgyrecursor}
}
\unimathsetup{
  math-style=ISO,
  bold-style=ISO,
}
\IfFontExistsTF{XITSMath-Regular.otf}{
  \setmathfont[
    Extension    = .otf,
    BoldFont     = XITSMath-Bold,
    StylisticSet = 8,
  ]{XITSMath-Regular}
  \setmathfont[range={cal,bfcal},StylisticSet=1]{XITSMath-Regular.otf}
}{
  \setmathfont[
    Extension    = .otf,
    BoldFont     = *bold,
    StylisticSet = 8,
  ]{xits-math}
  \setmathfont[range={cal,bfcal},StylisticSet=1]{xits-math.otf}
}

\colorlet{bit@macro}{blue!60!black}
\colorlet{bit@env}{blue!70!black}
\colorlet{bit@option}{purple}
\patchcmd{\PrintMacroName}{\MacroFont}{\MacroFont\bfseries\color{bit@macro}}{}{}
\patchcmd{\PrintDescribeMacro}{\MacroFont}{\MacroFont\bfseries\color{bit@macro}}{}{}
\patchcmd{\PrintDescribeEnv}{\MacroFont}{\MacroFont\bfseries\color{bit@env}}{}{}
\patchcmd{\PrintEnvName}{\MacroFont}{\MacroFont\bfseries\color{bit@env}}{}{}

\def\DescribeOption{%
  \leavevmode\@bsphack\begingroup\MakePrivateLetters%
  \Describe@Option}
\def\Describe@Option#1{\endgroup
  \marginpar{\raggedleft\PrintDescribeOption{#1}}%
  \bit@special@index{option}{#1}\@esphack\ignorespaces}
\def\PrintDescribeOption#1{\strut \MacroFont\bfseries\sffamily\color{bit@option} #1\ }
\def\bit@special@index#1#2{\@bsphack
  \begingroup
    \HD@target
    \let\HDorg@encapchar\encapchar
    \edef\encapchar usage{%
      \HDorg@encapchar hdclindex{\the\c@HD@hypercount}{usage}%
    }%
    \index{#2\actualchar{\string\ttfamily\space#2}
           (#1)\encapchar usage}%
    \index{#1:\levelchar#2\actualchar
           {\string\ttfamily\space#2}\encapchar usage}%
  \endgroup
  \@esphack}

\lstdefinestyle{lstStyleBase}{%
   basicstyle=\small\ttfamily,
   aboveskip=\medskipamount,
   belowskip=\medskipamount,
   lineskip=0pt,
   boxpos=c,
   showlines=false,
   extendedchars=true,
   escapeinside  = {(*}{*)},
   upquote=true,
   tabsize=2,
   showtabs=false,
   showspaces=false,
   showstringspaces=false,
   numbers=none,
   linewidth=\linewidth,
   xleftmargin=4pt,
   xrightmargin=0pt,
   resetmargins=false,
   breaklines=true,
   breakatwhitespace=false,
   breakindent=0pt,
   breakautoindent=true,
   columns=flexible,
   keepspaces=true,
   gobble=4,
   framesep=3pt,
   rulesep=1pt,
   framerule=1pt,
   backgroundcolor=\color{gray!5},
   stringstyle=\color{green!40!black!100},
   keywordstyle=\bfseries\color{blue!50!black},
   commentstyle=\slshape\color{black!60}}

\lstdefinestyle{lstStyleShell}{%
   style=lstStyleBase,
   frame=l,
   rulecolor=\color{purple},
   language=bash}

\lstdefinestyle{lstStyleLaTeX}{%
   style=lstStyleBase,
   frame=l,
   rulecolor=\color{violet},
   language=[LaTeX]TeX}

\lstdefinestyle{lstStyleSyntax}{%
   style=lstStyleBase,
   frame=l,
   rulecolor=\color{violet},
   language=[LaTeX]TeX,
   emphstyle=[1]\color{teal},
}

\lstnewenvironment{latex}{\lstset{style=lstStyleLaTeX}}{}
\lstnewenvironment{shell}{\lstset{style=lstStyleShell}}{}
\lstnewenvironment{bitsyntax}[1][]{\lstset{style=lstStyleSyntax, #1}}{}

\def\orbar{\textup{\textbar}}
\def\syntaxopt#1{\textit{#1}}
\def\defaultval#1{\textbf{\textup{#1}}}
\def\syntaxopt@aux#1>{\syntaxopt{#1}}
\def\defaultval@aux#1){\defaultval{#1}}


\setlist{nosep}

\DeclareDocumentCommand{\option}{m}{\textsf{#1}}
\DeclareDocumentCommand{\env}{m}{\texttt{#1}}
\newcommand{\myentry}[1]{%
  \marginpar{\raggedleft\color{purple}\bfseries\strut #1}}
\newcommand{\note}[2][Note]{{%
  \color{magenta}{\bfseries #1}\emph{#2}}}

\DeclareDocumentCommand{\githubuser}{m}{\href{https://github.com/#1}{@#1}}


  % 设置 caption 与 figure 之间的距离
\setlength{\abovecaptionskip}{11pt}
\setlength{\belowcaptionskip}{9pt}

  % 设置图片的 caption 格式
\renewcommand{\thefigure}{\thesection-\arabic{figure}}
\captionsetup[figure]{font=small,labelsep=space}

  % 设置表格的 caption 与 table 之间的垂直距离
\captionsetup[table]{skip=2pt}

  % 设置表格的 caption 格式
\renewcommand{\thetable}{\thesection-\arabic{table}}
\captionsetup[table]{font=small,labelsep=space}

  % 定义 BIThesis \LaTeX 风格的 Logo
\usepackage{relsize}
\makeatletter
\def\matex@ssize{\larger[-1]\scshape}
\DeclareRobustCommand{\BIThesis}{
  \mbox{
    \kern-0.5em{B}\kern-0.05em
    {I}\kern-0.05em
    {T}\kern-0.1em
    \raisebox{-0.38ex}{\matex@ssize {H}}\kern-0.1em
    {\matex@ssize {E}}\kern-0.05em
    \raisebox{-0.38ex}{\matex@ssize {S}}\kern-0.05em
    {\matex@ssize {I}}\kern-0.05em
    \raisebox{-0.35ex}{\matex@ssize {S}}\kern-0.5em
    \kern 1ex
   }
}

\makeatother

%</dtx-style>
% \fi
%
%
% \Finale
\endinput
% \iffalse
%  Local Variables:
%  mode: doctex
%  TeX-master: t
%  End:
% \fi
