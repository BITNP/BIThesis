% \iffalse meta-comment
%
% Copyright (C) 2021
% Association of Bit Network Pioneer and any individual authors listed elsewhere in this file.
% -----------------------------------
%
% This file may be distributed and/or modified under the
% conditions of the LaTeX Project Public License, either version 1.3
% of this license or (at your option) any later version.
% The latest version of this license is in:
%
%     http://www.latex-project.org/lppl.txt
%
% and version 1.3 or later is part of all distributions of LaTeX
% version 2020/11/27 or later.
%
% \fi
%
% \iffalse
%<cls>\NeedsTeXFormat{LaTeX2e}[1999/12/01]
%<book>\ProvidesClass{bitbook}
%<article>\ProvidesClass{bitart}
%<cls> [2021/01/06 v1.0.1 BIT Thesis Templates]
%
%<*driver>
\ProvidesFile{bithesis.dtx}[2021/01/06 1.0.1 BIT Thesis Templates]
\documentclass{ltxdoc}
\usepackage{dtx-style}

\EnableCrossrefs
\CodelineIndex

\RecordChanges
\begin{document}
  \DocInput{\jobname.dtx}
\end{document}
%</driver>
% \fi
%
% \DoNotIndex{\newenvironment,\@bsphack,\@empty,\@esphack,\sfcode}
% \DoNotIndex{\addtocounter,\label,\let,\linewidth,\newcounter}
% \DoNotIndex{\noindent,\normalfont,\par,\parskip,\phantomsection}
% \DoNotIndex{\providecommand,\ProvidesPackage,\refstepcounter}
% \DoNotIndex{\RequirePackage,\setcounter,\setlength,\string,\strut}
% \DoNotIndex{\textbackslash,\texttt,\ttfamily,\usepackage}
% \DoNotIndex{\begin,\end,\begingroup,\endgroup,\par,\\}
% \DoNotIndex{\if,\ifx,\ifdim,\ifnum,\ifcase,\else,\or,\fi}
% \DoNotIndex{\let,\def,\xdef,\edef,\newcommand,\renewcommand}
% \DoNotIndex{\expandafter,\csname,\endcsname,\relax,\protect}
% \DoNotIndex{\Huge,\huge,\LARGE,\Large,\large,\normalsize}
% \DoNotIndex{\small,\footnotesize,\scriptsize,\tiny}
% \DoNotIndex{\normalfont,\bfseries,\slshape,\sffamily,\interlinepenalty}
% \DoNotIndex{\textbf,\textit,\textsf,\textsc}
% \DoNotIndex{\hfil,\par,\hskip,\vskip,\vspace,\quad}
% \DoNotIndex{\centering,\raggedright,\ref}
% \DoNotIndex{\c@secnumdepth,\@startsection,\@setfontsize}
% \DoNotIndex{\ ,\@plus,\@minus,\p@,\z@,\@m,\@M,\@ne,\m@ne}
% \DoNotIndex{\@@par,\DeclareOperation,\RequirePackage,\LoadClass}
% \DoNotIndex{\AtBeginDocument,\AtEndDocument}
%
% \GetFileInfo{\jobname.dtx} %
% 
% \def\indexname{索引}
% \IndexPrologue{\section{\indexname}}
%
% \title{\includegraphics[width=0.3\textwidth]{images/icon.png}
% \\[1cm]
% \bfseries 北京理工大学本科生{\LaTeX}学位论文及报告模板 }
% \author{北京理工大学网络开拓者协会 \\ \texttt{webmaster@bitnp.net}} %
% \date{\zihao{-4} \today\quad \color{RubineRed}{\kaishu {\BIThesis}版本\version}}
% \maketitle\thispagestyle{empty}
%
% \def\abstractname{}
% \begin{abstract}\noindent
%   此宏包旨在建立一个简单易用的北京理工大学学位论文模板,包括本科综合论文训练、硕士
%   论文、博士论文以及博士后出站报告。
% \end{abstract}
%
% \vspace{5mm}
%
% \begin{center}
% \noindent\rule[0.25\baselineskip]{0.5\textwidth}{0.7pt}
% \end{center}
% 
% \def\abstractname{免责声明}
% \begin{abstract}
% \noindent
% \begin{enumerate}
% \item 本模板的发布遵守 \LPPL ,使用前请认真阅读协议内容。
% \item 任何个人或组织以本模板为基础进行修改、扩展而生成的新的专用模板,请严格遵
%   守 \LaTeX{} Project Public License 协议。由于违犯协议而引起的任何纠纷争端均与
%   本模板作者无关。
% \end{enumerate}
% \end{abstract}
%
% \vspace{5mm}
%
% \def\abstractname{简介}
% \begin{abstract}
% \BIThesisLaTeX 是北京理工大学本科生毕业设计开题报告、总论文,以及其他课程报告、实验报告等重要论文、报告的 {\LaTeX} 模板集合。
% 如果你厌烦了 Word 格式的不人性化、参考文献的难以管理、公式输入的差劲体验……那么欢迎来尝试用专业的学术稿件排版利器 —— {\LaTeX},来排版你的论文。
% 专业高端、学界认可、开源免费,{\LaTeX} 是你论文排版的最佳搭档。
%
% \BIThesisLaTeX 目前支持使用 {\hologo{XeLaTeX}} 进行编译,使用以 biber 为后端的 BibLaTeX 进行参考文献的生成,
% 符合《信息与文献参考文献著录规则》
% (\href{http://openstd.samr.gov.cn/bzgk/gb/newGbInfo?hcno=7FA63E9BBA56E60471AEDAEBDE44B14C}{GB/T 7714—2015})的标准。
% 目前主要设计完成了计算机学院本科生毕业论文开题报告、毕业设计毕业论文与通用实验报告的 {\LaTeX} 模板。
%
% \end{abstract}
% \newpage
%
% \tableofcontents
% \clearpage
% \setlength{\parskip}{0.8ex}
%
% \section{项目简介}
% \subsection{历史与贡献者们}
% \begin{itemize}
% \item 2019 - 2020 年,\BIThesis 最早由 2016 级的武上博、王赞、唐誉铭、牟思睿和詹熠莎等人维护。
%   \begin{itemize}
%     \item 在此期间,\BIThesis 从无到有诞生了,包括使用手册、在线文档和开箱即用的模板。
%     \item 同时,2017 级的赵池等同学完成了一系列 \BIThesisLaTeX 的视频教程。
%   \end{itemize}
% \item 2020 - 2021 年,2017 级的冯开宇、杨思云、郝正亮和顾骁等人接管了维护开发工作。
%   \begin{itemize}
%     \item 在此期间,冯开宇将原来的 .tex 文件制作成了宏包,并发布到 CTAN 上。
%     \item 项目代码也随之被拆分成了 \BIThesisMacroPackage,\BIThesisWiki 和 \BIThesisScaffold。
%   \end{itemize}
% \end{itemize}
% \subsection{\BIThesis 是什么?}
% \BIThesis 之名是英文单词 Beijing Institution of Technology(北京理工大学)的首字母缩写“BIT” 与“Thesis”结合而成。在纯文本环境下,该名字应写作“BIThesis”。
%
% \BIThesisLaTeX 是由北京理工大学众多学子发起并维护的开源项目。该项目旨在建立一套简单易用的北京理工大学 \LaTeX 学位论文模板,包括本科综合论文训练。
% \subsubsection{\BIThesisLaTeX 的组成}
% 我们将 \BIThesisLaTeX 划分为了三个主要仓库:
% \begin{table}[H]
% \centering
% \begin{tabular}{@{}l l p{6cm} @{}}
% \toprule
% 项目                & 项目地址 & 主要目的 \\ \midrule
% BIThesis          &   \href{https://github.com/BITNP/BIThesis}{BITNP/BIThesis}   &  主要存储 \BIThesis  宏包 \\
% BIThesis-wiki     &   \href{https://github.com/BITNP/BIThesis-wiki}{BITNP/BIThesis-wiki}  &  存储 \BIThesisLaTeX 项目在线文档   \\
% BIThesis-scaffold &   \href{https://github.com/BITNP/BIThesis-scaffold}{BITNP/BIThesis-scaffold}   &  存储开箱即用的论文模板样式,便于使用者快速开始写作  \\ \bottomrule
% \end{tabular}
% \end{table}
%
% 如果你仅想解决「我如何使用 \BIThesisLaTeX 来帮助我完成实验论文?」这个问题,那么欢迎你访问我们的\href{https://bithesis.bitnp.net}{在线文档}以获得更多信息。 
% 
% \section{使用说明}
% \subsection{\BIThesis 宏包的组成}
% 为了适应用户的不同需求,并符合 CTeX 宏集的设计习惯,我们将 \BIThesisMacroPackage 的主要功能设计安排在两个中文文档类当中,具体的组成见 \ref{tab:classes}。
% \begin{table}[H]
% \centering
% \caption{测试}
% \label{tab:classes}
% \begin{tabular}{@{}lll@{}}
% \toprule
% 类别                   & 文件          & 说明                             \\ \midrule
% \multirow{2}{*}{文档类} & bitart.cls  & 对应 ctxart.cls,提供实验报告模板、开题报告模板。 \\
%                      & bitbook.cls & 对应 ctexbook.cls ,提供本科毕业模板。     \\ \cmidrule(l){2-3} 
% \end{tabular}
% \end{table}
% \subsection{\BIThesis 宏包的安装和更新}
% 最常见的 \TeX 发行版(\hologo{TeX} Live 和 \hologo{MiKTeX})已收录\BIThesisMacroPackage 及其依赖的宏包和宏集。
%
% 如果安装以上发行版的时间较早,可能你本地的环境中不存在 \BIThesisMacroPackage 或者不是最新版本的。那么你需要通过包管理器来安装/更新 \BIThesisMacroPackage:
% \mint{bash}|tlmgr update --self --all|
% \subsection{使用 \BIThesis 文档类}
% 推荐使用 \BIThesisScaffold 来进行具体的项目编写。\BIThesisScaffold 提供了多种最常用的模板,你可以在 \href{https://github.com/BITNP/BIThesis}{主项目的 Releases}中找到它们。 
% \section{致谢}
% \section{软件许可证}
% \begin{itemize}
%   \item 北京理工大学校徽校名图片的版权归北京理工大学所有。
%   \item \BIThesisLaTeX 宏包以及相关文档类使用 \LPPL 授权。
%   \item \BIThesisLaTeX 文档及其他附属文件通过 \LPPL 授权。
% \end{itemize}
% \section{实现细节}
%
%    \begin{macrocode}
%<*package>
%    \end{macrocode}
%
% \begin{macro}{\YOURMACRO}
% Put explanation of |\YOURMACRO|’s implementation here.
%    \begin{macrocode}
\newcommand{\YOURMACRO}{}
%    \end{macrocode}
% \end{macro}
%
% \begin{environment}{YOURENV}
% Put explanation of |YOURENV|’s implementation here.
%    \begin{macrocode}
\newenvironment{YOURENV}{}{}
%    \end{macrocode}
% \end{environment}
%
%    \begin{macrocode}
%</package>
%    \end{macrocode}

%    \begin{macrocode}
%<*book>
%    \end{macrocode}
%    \begin{macrocode}

\newif\if@bit@bachelor
\newif\if@bit@master
\newif\if@bit@docter
\newif\if@bit@docTranslation

\DeclareOption{bachelor}{\@bit@bachelortrue}
\DeclareOption{translation}{\@bit@docTranslationtrue}
\DeclareOption*{\PassOptionsToClass{\CurrentOption}{ctexbook}}
\ExecuteOptions{bachelor}

\ProcessOptions\relax

\LoadClass[UTF8,AutoFakeBold,AutoFakeSlant,zihao=-4,oneside,openany]{ctexbook}

\RequirePackage[a4paper,left=3cm,right=2.6cm,top=3.5cm,bottom=2.9cm]{geometry}
% 目前 29mm 最接近 Word 排版
\RequirePackage{xeCJK}
\RequirePackage{titletoc}
  % \RequirePackage{fontspec}
\RequirePackage{setspace}
\RequirePackage{graphicx}
\RequirePackage{fancyhdr}
\RequirePackage{pdfpages}
\RequirePackage{setspace}
\RequirePackage{booktabs}
\RequirePackage{multirow}
\RequirePackage{tikz}
\RequirePackage{etoolbox}
\RequirePackage{hyperref}
\RequirePackage{xcolor}
\RequirePackage{caption}
\RequirePackage{array}
\RequirePackage{amsmath}
\RequirePackage{amssymb}
\RequirePackage{pdfpages}

% 设置参考文献编译后端为 biber,引用格式为 GB/T7714-2015 格式
% 参考文献使用宏包见 https://github.com/hushidong/biblatex-gb7714-2015
\RequirePackage[
  backend=biber,
  style=gb7714-2015,
  gbalign=gb7714-2015,
  gbnamefmt=lowercase,
  gbpub=false,
  doi=false,
  url=false,
  eprint=false,
  isbn=false,
]{biblatex}

% 参考文献引用文件位于 misc/ref.bib
\addbibresource{./misc/ref.bib}

% 西文字体默认为 Times New Roman
\setromanfont{Times New Roman}
% 论文题目字体为华文细黑
\setCJKfamilyfont{xihei}[AutoFakeBold,AutoFakeSlant]{[STXIHEI.TTF]} % 若希望使用本机字体,也可以用 {STXihei} 来调用
\newcommand{\xihei}{\CJKfamily{xihei}}


% 主题页面格式:BIThesis
\fancypagestyle{BIThesis}{
  % 页眉高度
  \setlength{\headheight}{20pt}
  % 页码高度(不完美,比规定稍微靠下 2mm)
  \setlength{\footskip}{14pt}

  \fancyhf{}
  % 定义页眉、页码
  \fancyhead[C]{\zihao{4}\ziju{0.08}\songti{北京理工大学本科生毕业设计(论文)}}
  \fancyfoot[C]{\songti\zihao{5} \thepage}
  % 页眉分割线稍微粗一些
  \renewcommand{\headrulewidth}{0.6pt}
}

\if@bit@docTranslation
% 主题页面格式:BIThesis
\fancypagestyle{BIThesis}{
  % 页眉高度
  \setlength{\headheight}{20pt}
  % 页码高度(不完美,比规定稍微靠下 2mm)
  \setlength{\footskip}{14pt}

  \fancyhf{}
  % 定义页码
  \fancyfoot[C]{\songti\zihao{5} \thepage}
  % 页眉分割线稍微粗一些
  \renewcommand{\headrulewidth}{0.6pt}

  % 定义页眉
  \fancyhead[C]{\zihao{4}\ziju{0.08}\songti{北京理工大学本科生毕业设计(论文)外文翻译}}
}
\fi
% 设置章节格式
% 一级标题:黑体,三号,加粗;间距:段前 0.5 行,段后 1 行;
\ctexset{chapter={
    name = {第,章},
    number = {\arabic{chapter}},
    format = {\heiti \bfseries \centering \zihao{3}},
    aftername = \hspace{9bp},
    pagestyle = BIThesis,
    beforeskip = 8bp,
    afterskip = 32bp,
    fixskip = true,
  }
}

% 二级标题:黑体,四号,加粗;间距:段前 0.5 行,段后 0 行;
\ctexset{section={
    number = {\thechapter.\hspace{4bp}\arabic{section}},
    format = {\heiti \raggedright \bfseries \zihao{4}},
    aftername = \hspace{8bp},
    beforeskip = 20bp plus 1ex minus .2ex,
    afterskip = 18bp plus .2ex,
    fixskip = true,
  }
}

% 三级标题:黑体、小四、加粗;间距:段前 0.5 行,段后 0 行;
\ctexset{subsection={
    number = {\thechapter.\hspace{3bp}\arabic{section}.\hspace{3bp}\arabic{subsection}},
    format = {\heiti \bfseries \raggedright \zihao{-4}},
    aftername = \hspace{7bp},
    beforeskip = 17bp plus 1ex minus .2ex,
    afterskip = 14bp plus .2ex,
    fixskip = true,
  }
}

% 设置目录样式
% 添加 PDF 链接
\addtocontents{toc}{\protect\hypersetup{hidelinks}}

% 解决「目录」二字的格式问题
\renewcommand{\contentsname}{
  \fontsize{16pt}{\baselineskip}
  \normalfont\heiti{目~~~~录}
  \vspace{-8pt}
}
% 定义目录样式
\titlecontents{chapter}[0pt]{\songti \zihao{-4}}
{\thecontentslabel\hspace{\ccwd}}{}
{\hspace{.5em}\titlerule*{.}\contentspage}
\titlecontents{section}[1\ccwd]{\songti \zihao{-4}}
{\thecontentslabel\hspace{\ccwd}}{}
{\hspace{.5em}\titlerule*{.}\contentspage}
\titlecontents{subsection}[2\ccwd]{\songti \zihao{-4}}
{\thecontentslabel\hspace{\ccwd}}{}
{\hspace{.5em}\titlerule*{.}\contentspage}

% 前置页面(原创性声明、中英文摘要、目录等)
\renewcommand{\frontmatter}{
  \pagenumbering{Roman}
  \pagestyle{BIThesis}
}

% 正文页面
\renewcommand{\mainmatter}{
  \pagenumbering{arabic}
  \pagestyle{BIThesis}
}

% 设置 caption 与 figure 之间的距离
\setlength{\abovecaptionskip}{11pt}
\setlength{\belowcaptionskip}{9pt}

% 设置图片的 caption 格式
\renewcommand{\thefigure}{\thechapter-\arabic{figure}}
\captionsetup[figure]{font=small,labelsep=space}

% 设置表格的 caption 格式和 caption 与 table 之间的垂直距离
\renewcommand{\thetable}{\thechapter-\arabic{table}}
\captionsetup[table]{font=small,labelsep=space,skip=2pt}

% 调整底层 TeX 排版引擎参数以保证所有段落能够很好地以两端对齐的方式呈现
\tolerance=1
\emergencystretch=\maxdimen
\hyphenpenalty=10000
\hbadness=10000

% 设置数学公式编号格式
\renewcommand{\theequation}{\arabic{chapter}-\arabic{equation}}

\newcommand{\unnumchapter}[1]{
  \chapter*{\vskip 10bp\textmd{#1} \vskip -6bp}
  \addcontentsline{toc}{chapter}{#1}
  \stepcounter{chapter}
}



%    \end{macrocode}
%    \begin{macrocode}
%</book>
%    \end{macrocode}
%    \begin{macrocode}
%<*article>
%    \end{macrocode}
%    \begin{macrocode}

\newif\if@bit@labreport
\newif\if@bit@proposalreport

\DeclareOption{lab-report}{\@bit@labreporttrue\@bit@proposalreportfalse}
\DeclareOption{proposal-report}{\@bit@labreportfalse\@bit@proposalreporttrue}
\DeclareOption*{\PassOptionsToClass{\CurrentOption}{ctexart}}
\ExecuteOptions{lab-report}
\ProcessOptions\relax

\LoadClass[UTF8,AutoFakeBold,AutoFakeSlant,zihao=-4]{ctexart}%

\if@bit@labreport
  \RequirePackage[a4paper,left=3.18cm,right=3.18cm,top=2.54cm,bottom=2.54cm,includeheadfoot]{geometry}%
\else
  \RequirePackage[a4paper,left=3cm,right=2.4cm,top=2.6cm,bottom=2.38cm,includeheadfoot]{geometry}
\fi

\RequirePackage{fontspec}%
\RequirePackage{setspace}%
\RequirePackage{graphicx}%
\RequirePackage{fancyhdr}%
\RequirePackage{pdfpages}%
\RequirePackage{setspace}%
\RequirePackage{booktabs}%
\RequirePackage{multirow}%
\RequirePackage{caption}%

\if@bit@labreport
  \RequirePackage{titlesec}%
  \RequirePackage{float}%
  \RequirePackage{etoolbox}
\fi

\if@bit@proposalreport
  % 设置参考文献编译后端为 biber,引用格式为 GB/T7714-2015 格式
  % 参考文献使用宏包见 https://github.com/hushidong/biblatex-gb7714-2015
  \usepackage[style=gb7714-2015,backend=biber]{biblatex}

\fi


\if@bit@labreport
  % 将西文字体设置为 Times New Roman
  \setromanfont{Times New Roman}%

  % 设置引用位于右上角
  \newcommand{\upcite}[1]{\textsuperscript{\cite{#1}}}%

  % 设置文档标题深度
  \setcounter{tocdepth}{3}%
  \setcounter{secnumdepth}{3}%

  %%
  % 设置一级标题、二级标题格式
  \ctexset{section={%
    format={\raggedright \bfseries \songti \zihao{-3}},%
    name = {,.},%
    number = \chinese{section}%
    }%
  }%
  \ctexset{subsection={%
    format = {\bfseries \songti \raggedright \zihao{-4}},%
    }%
  }%

  % 页眉和页脚(页码)的格式设定
  \fancyhf{}%
  \fancyhead[L]{\fontsize{10.5pt}{10.5pt}\selectfont\kaishu{\reportName}}%
  \fancyfoot[C]{\fontsize{9pt}{9pt}\selectfont\kaishu{\thepage}}%
  \renewcommand{\headrulewidth}{0.5pt}%
  \renewcommand{\footrulewidth}{0pt}%

  \AtBeginDocument{
  }
\fi

\if@bit@proposalreport
  % 定义 caption 字体为楷体
  \DeclareCaptionFont{kaiticaption}{\kaishu \normalsize}

  % 设置图片的 caption 格式
  \renewcommand{\thefigure}{\thesection-\arabic{figure}}
  \captionsetup[figure]{font=small,labelsep=space,skip=10bp,labelfont=bf,font=kaiticaption}

  % 设置表格的 caption 格式
  \renewcommand{\thetable}{\thesection-\arabic{table}}
  \captionsetup[table]{font=small,labelsep=space,skip=10bp,labelfont=bf,font=kaiticaption}

  % 输出大写数字日期
  \CTEXoptions[today=big]

  % 将西文字体设置为 Times New Roman
  \setromanfont{Times New Roman}

  %% 将中文楷体设置为 SIMKAI.TTF(如果需要)
  % \setCJKfamilyfont{zhkai}{[SIMKAI.TTF]}
  % \newcommand*{\kaiti}{\CJKfamily{zhkai}}

  % 设置文档标题深度
  \setcounter{tocdepth}{3}
  \setcounter{secnumdepth}{3}

  %%
  % 设置一级标题、二级标题格式
  % 一级标题:小三,宋体,加粗,段前段后各半行
  \ctexset{section={
    format={\raggedright \bfseries \songti \zihao{-3}},
    beforeskip = 24bp plus 1ex minus .2ex,
    afterskip = 24bp plus .2ex,
    fixskip = true,
    name = {,.\quad}
    }
  }
  % 二级标题:小四,宋体,加粗,段前段后各半行
  \ctexset{subsection={
    format = {\bfseries \songti \raggedright \zihao{4}},
    beforeskip =24bp plus 1ex minus .2ex,
    afterskip = 24bp plus .2ex,
    fixskip = true,
    }
  }
  % 页眉和页脚(页码)的格式设定
  \fancyhf{}
  \fancyhead[R]{\fontsize{10.5pt}{10.5pt}\selectfont{北京理工大学本科生毕业设计(论文)开题报告}}
  \fancyfoot[R]{\fontsize{9pt}{9pt}\selectfont{\thepage}}
  \renewcommand{\headrulewidth}{1pt}
  \renewcommand{\footrulewidth}{0pt}
\fi


\AtBeginDocument{
  \if@bit@labreport
    \topskip=0pt

\begin{titlepage}
  \vspace*{-16mm}
  \centering

  \vspace{23mm}

  \hspace{-6mm}\heiti\fontsize{24pt}{24pt}\selectfont{\reportName}

  \vspace{87mm}

  \flushleft
  \begin{spacing}{2.2}
    \hspace{39mm}\songti\fontsize{16pt}{16pt}\selectfont{\textbf{学\hspace{11mm}院:}\underline{\makebox[51mm][c]{\deptName}}}

    \hspace{39mm}\songti\fontsize{16pt}{16pt}\selectfont{\textbf{专\hspace{11mm}业:}\underline{\makebox[51mm][c]{\majorName}}}

    \hspace{39mm}\songti\fontsize{16pt}{16pt}\selectfont{\textbf{班\hspace{11mm}级:}\underline{\makebox[51mm][c]{\className}}}

    \hspace{39mm}\songti\fontsize{16pt}{16pt}\selectfont{\textbf{姓\hspace{11mm}名:}\underline{\makebox[51mm][c]{\yourName}}}

    \hspace{39mm}\songti\fontsize{16pt}{16pt}\selectfont{\textbf{任课教师:}\underline{\makebox[51mm][c]{\teacherName}}}
  \end{spacing}

  \vspace{33mm}

  \centering
  \hspace{-6mm}\songti\fontsize{12pt}{12pt}\selectfont{\today}
\end{titlepage}

    % 正文开始
    \pagestyle{fancy}
    \setcounter{page}{1}%
  \fi
  \if@bit@proposalreport
    % 报告封面
    %%
% The BIThesis Template for Bachelor Graduation Thesis
%
% 北京理工大学毕业设计开题报告 —— 使用 XeLaTeX 编译
%
% Copyright 2020 Spencer Woo
%
% This work may be distributed and/or modified under the
% conditions of the LaTeX Project Public License, either version 1.3
% of this license or (at your option) any later version.
% The latest version of this license is in
%   http://www.latex-project.org/lppl.txt
% and version 1.3 or later is part of all distributions of LaTeX
% version 2005/12/01 or later.
%
% This work has the LPPL maintenance status `maintained'.
%
% The Current Maintainer of this work is Spencer Woo.
%
% This work consists of the files main.tex, misc/cover.tex and
% the external PDF misc/reviewTable.pdf

% 校名顶部非常细小的空白
\topskip=0pt

\begin{titlepage}
  \vspace*{-16mm}
  \centering
  \hspace{-6mm}\songti\fontsize{22pt}{22pt}\selectfont{北京理工大学}

  \vspace{13mm}

  \hspace{-6mm}\heiti\fontsize{24pt}{24pt}\selectfont{本科生毕业设计(论文)开题报告}

  \vspace{53mm}

  \flushleft
  \begin{spacing}{2.2}
    \hspace{46mm}\songti\fontsize{16pt}{16pt}\selectfont{\textbf{学\hspace{11mm}院:}\underline{\makebox[51mm][c]{\deptName}}}

    \hspace{46mm}\songti\fontsize{16pt}{16pt}\selectfont{\textbf{专\hspace{11mm}业:}\underline{\makebox[51mm][c]{\majorName}}}

    \hspace{46mm}\songti\fontsize{16pt}{16pt}\selectfont{\textbf{班\hspace{11mm}级:}\underline{\makebox[51mm][c]{\className}}}

    \hspace{46mm}\songti\fontsize{16pt}{16pt}\selectfont{\textbf{姓\hspace{11mm}名:}\underline{\makebox[51mm][c]{\yourName}}}

    \hspace{46mm}\songti\fontsize{16pt}{16pt}\selectfont{\textbf{指导教师:}\underline{\makebox[51mm][c]{\mentorName}}}

    \hspace{46mm}\songti\fontsize{16pt}{16pt}\selectfont{\textbf{校外指导教师:}\underline{\makebox[40mm][c]{\offCampusMentorName}}}
  \end{spacing}

  \vspace{47mm}

  \centering
  \hspace{-6mm}\songti\fontsize{12pt}{12pt}\selectfont{\today}
\end{titlepage}

  \fi

}

%    \end{macrocode}
%
%    \begin{macrocode}
%    \end{macrocode}
%
%    \begin{macrocode}
%</article>
%    \end{macrocode}
%
% \iffalse
%<*dtx-style>
\ProvidesPackage{dtx-style}
\RequirePackage{hypdoc}
\RequirePackage{ifthen}
\RequirePackage{fontspec}[2017/01/20]
\RequirePackage{amsmath}
\RequirePackage{unicode-math}
\RequirePackage[UTF8,scheme=chinese,heading]{ctex}
\RequirePackage[
  top=2.5cm, bottom=2.5cm,
  left=4cm, right=2cm,
  headsep=3mm]{geometry}
\RequirePackage{graphicx}
\RequirePackage{multirow}
\RequirePackage[ruled,vlined]{algorithm2e}
\RequirePackage{wrapfig}
\RequirePackage{hologo}
\RequirePackage{array,longtable,booktabs}
\RequirePackage{listings}
\RequirePackage{fancyhdr}
\RequirePackage[dvipsnames]{xcolor}
\RequirePackage{awesomebox}
\RequirePackage{etoolbox}
\RequirePackage{dirtree}
\RequirePackage{metalogo}
\RequirePackage[tightLists=false]{markdown}
\RequirePackage{caption}
\RequirePackage{tikz}
\usetikzlibrary{positioning}
\RequirePackage{framed}
\RequirePackage{menukeys}

  % 设置代码高亮
\RequirePackage{minted}
\usemintedstyle{tango}

  % 设置列表无间隔
\usepackage{enumitem}
\setlist{nosep}

\markdownSetup{
  renderers = {
    link = {\href{#2}{#1}},
  }
}

\hypersetup{
  pdflang     = zh-CN,
  pdftitle    = {BIThesis:北京理工大学学位论文及报告模板},
  pdfauthor   = {冯开宇},
  pdfsubject  = {北京理工大学学位论文及报告模板使用说明},
  pdfkeywords = {论文模板; 北京理工大学; 使用说明},
  pdfdisplaydoctitle = true
}%

\newcommand{\BIThesisLaTeX}{{\BIThesis}北京理工大学学位论文及报告{\LaTeX}模板}
\newcommand{\BIThesisMacroPackage}{{\BIThesis}宏集}
\newcommand{\BIThesisWiki}{{\BIThesis}在线文档}
\newcommand{\BIThesisScaffold}{{\BIThesis}模板}
\newcommand{\LPPL}{{\href{https://www.latex-project.org/lppl/lppl-1-3c.txt}{\LaTeX{} Project Public License (1.3.c)}}}
\newcommand{\version}{v2.0 BirthdayCake}

\ctexset{
  today=big,
  abstractname=简介
}

\ctexset{section={
  format={\raggedright \bfseries \zihao{-3}},
  name = {第,章}
  }
}

\ctexset{subsection={
  format = {\bfseries \raggedright \zihao{4}}
  }
}



\ifthenelse{\equal{\@nameuse{g__ctex_fontset_tl}}{mac}}{
  \setmainfont{Palatino}
  \setsansfont[Scale=MatchLowercase]{Helvetica}
  \setmonofont[Scale=MatchLowercase]{Menlo}
  \xeCJKsetwidth{‘’“”}{1em}
}{
  \setmainfont[
    Extension      = .otf,
    UprightFont    = *-regular,
    BoldFont       = *-bold,
    ItalicFont     = *-italic,
    BoldItalicFont = *-bolditalic,
  ]{texgyrepagella}
  \setsansfont[
    Extension      = .otf,
    UprightFont    = *-regular,
    BoldFont       = *-bold,
    ItalicFont     = *-italic,
    BoldItalicFont = *-bolditalic,
  ]{texgyreheros}
  \setmonofont[
    Extension      = .otf,
    UprightFont    = *-regular,
    BoldFont       = *-bold,
    ItalicFont     = *-italic,
    BoldItalicFont = *-bolditalic,
    Scale          = MatchLowercase,
    Ligatures      = CommonOff,
  ]{texgyrecursor}
}
\unimathsetup{
  math-style=ISO,
  bold-style=ISO,
}
\IfFontExistsTF{XITSMath-Regular.otf}{
  \setmathfont[
    Extension    = .otf,
    BoldFont     = XITSMath-Bold,
    StylisticSet = 8,
  ]{XITSMath-Regular}
  \setmathfont[range={cal,bfcal},StylisticSet=1]{XITSMath-Regular.otf}
}{
  \setmathfont[
    Extension    = .otf,
    BoldFont     = *bold,
    StylisticSet = 8,
  ]{xits-math}
  \setmathfont[range={cal,bfcal},StylisticSet=1]{xits-math.otf}
}

\colorlet{bit@macro}{blue!60!black}
\colorlet{bit@env}{blue!70!black}
\colorlet{bit@option}{purple}
\patchcmd{\PrintMacroName}{\MacroFont}{\MacroFont\bfseries\color{bit@macro}}{}{}
\patchcmd{\PrintDescribeMacro}{\MacroFont}{\MacroFont\bfseries\color{bit@macro}}{}{}
\patchcmd{\PrintDescribeEnv}{\MacroFont}{\MacroFont\bfseries\color{bit@env}}{}{}
\patchcmd{\PrintEnvName}{\MacroFont}{\MacroFont\bfseries\color{bit@env}}{}{}

\def\DescribeOption{%
  \leavevmode\@bsphack\begingroup\MakePrivateLetters%
  \Describe@Option}
\def\Describe@Option#1{\endgroup
  \marginpar{\raggedleft\PrintDescribeOption{#1}}%
  \bit@special@index{option}{#1}\@esphack\ignorespaces}
\def\PrintDescribeOption#1{\strut \MacroFont\bfseries\sffamily\color{bit@option} #1\ }
\def\bit@special@index#1#2{\@bsphack
  \begingroup
    \HD@target
    \let\HDorg@encapchar\encapchar
    \edef\encapchar usage{%
      \HDorg@encapchar hdclindex{\the\c@HD@hypercount}{usage}%
    }%
    \index{#2\actualchar{\string\ttfamily\space#2}
           (#1)\encapchar usage}%
    \index{#1:\levelchar#2\actualchar
           {\string\ttfamily\space#2}\encapchar usage}%
  \endgroup
  \@esphack}

\lstdefinestyle{lstStyleBase}{%
   basicstyle=\small\ttfamily,
   aboveskip=\medskipamount,
   belowskip=\medskipamount,
   lineskip=0pt,
   boxpos=c,
   showlines=false,
   extendedchars=true,
   upquote=true,
   tabsize=2,
   showtabs=false,
   showspaces=false,
   showstringspaces=false,
   numbers=none,
   linewidth=\linewidth,
   xleftmargin=4pt,
   xrightmargin=0pt,
   resetmargins=false,
   breaklines=true,
   breakatwhitespace=false,
   breakindent=0pt,
   breakautoindent=true,
   columns=flexible,
   keepspaces=true,
   gobble=4,
   framesep=3pt,
   rulesep=1pt,
   framerule=1pt,
   backgroundcolor=\color{gray!5},
   stringstyle=\color{green!40!black!100},
   keywordstyle=\bfseries\color{blue!50!black},
   commentstyle=\slshape\color{black!60}}

\lstdefinestyle{lstStyleShell}{%
   style=lstStyleBase,
   frame=l,
   rulecolor=\color{purple},
   language=bash}

\lstdefinestyle{lstStyleLaTeX}{%
   style=lstStyleBase,
   frame=l,
   rulecolor=\color{violet},
   language=[LaTeX]TeX}

\lstnewenvironment{latex}{\lstset{style=lstStyleLaTeX}}{}
\lstnewenvironment{shell}{\lstset{style=lstStyleShell}}{}

\setlist{nosep}

\DeclareDocumentCommand{\option}{m}{\textsf{#1}}
\DeclareDocumentCommand{\env}{m}{\texttt{#1}}
\newcommand{\myentry}[1]{%
  \marginpar{\raggedleft\color{purple}\bfseries\strut #1}}
\newcommand{\note}[2][Note]{{%
  \color{magenta}{\bfseries #1}\emph{#2}}}

\DeclareDocumentCommand{\githubuser}{m}{\href{https://github.com/#1}{@#1}}


  % 设置 caption 与 figure 之间的距离
\setlength{\abovecaptionskip}{11pt}
\setlength{\belowcaptionskip}{9pt}

  % 设置图片的 caption 格式
\renewcommand{\thefigure}{\thesection-\arabic{figure}}
\captionsetup[figure]{font=small,labelsep=space}

  % 设置表格的 caption 与 table 之间的垂直距离
\captionsetup[table]{skip=2pt}

  % 设置表格的 caption 格式
\renewcommand{\thetable}{\thesection-\arabic{table}}
\captionsetup[table]{font=small,labelsep=space}

  % 定义 BIThesis \LaTeX 风格的 Logo
\usepackage{relsize}
\makeatletter
\def\matex@ssize{\larger[-1]\scshape}
\DeclareRobustCommand{\BIThesis}{
  \mbox{
    \kern-0.5em{B}\kern-0.05em
    {I}\kern-0.05em
    {T}\kern-0.1em
    \raisebox{-0.38ex}{\matex@ssize {H}}\kern-0.1em
    {\matex@ssize {E}}\kern-0.05em
    \raisebox{-0.38ex}{\matex@ssize {S}}\kern-0.05em
    {\matex@ssize {I}}\kern-0.05em
    \raisebox{-0.35ex}{\matex@ssize {S}}\kern-0.5em
    \kern 1ex
   }
}
\makeatother



%</dtx-style>
% \fi
%
%
% \Finale
\endinput
% \iffalse
%  Local Variables:
%  mode: doctex
%  TeX-master: t
%  End:
% \fi
