\section{通用北京理工大学本科生实验报告模板使用指南} \label{lab-report}

\tipbox{说明:这个实验报告模板是一个通用的报告模板,不适用所有实验报告要求。实验课程未提供实验报告模板时可以使用该模板。当前本实验报告模板只包含一个封面,欢迎大家往项目仓库 PR 制作更多的封面。}

\subsection{熟悉项目}
\dirtree{%
  .1 /.
  .2 README.md.
  .2 main.pdf.
  .2 main.tex.
  .2 misc.
  .3 cover\_v1.tex.
  .2 assets.
  .3 …….
}

\begin{itemize}
  \item[\color{RubineRed}\textbf{\texttt{main.tex}}] tex 源文件,本实验报告模板的主体文件,所有需要添加的内容都在该文件里进行修改即可
  \item[\color{RubineRed}\textbf{\texttt{main.pdf}}] 编译项目生成的 pdf 文件
  \item[\color{RubineRed}\textbf{\texttt{./misc}}] 杂项(包括实验报告封面等):
        \begin{itemize}
          \item[\color{RoyalBlue}\texttt{cover\_v1.tex}] 这是一个示范性的报告封面,该文件无需修改
        \end{itemize}
  \item[\color{RubineRed}\textbf{\texttt{./asset}}] 一些图片资源存放文件夹
\end{itemize}

\subsection{编译方式与使用}

由于实验报告模板没有涉及到参考文献的使用,因此我们只需要使用 \hologo{XeLaTeX} 即可进行全文编译。

整个项目的编译工具链的顺序为:
\begin{center}
  \begin{tikzpicture}[
      bib/.style={rectangle, draw=ForestGreen!60, fill=ForestGreen!5, very thick, minimum size=8mm},
      xe/.style={rectangle, draw=RubineRed!60, fill=RubineRed!5, very thick, minimum size=8mm},
    ]
    % Nodes
    \node[xe] (xelatex1) {xelatex};
    \node[xe] (xelatex2) [right=of xelatex1] {xelatex};
    \node[xe] (xelatex3) [right=of xelatex2] {xelatex};

    % Arrows
    \draw[->] (xelatex1.east) -- (xelatex2.west);
    \draw[->] (xelatex2.east) -- (xelatex3.west);
  \end{tikzpicture}
\end{center}

其中,按照 VS Code 的 LaTeX Workshop 设置格式:{\hologo{XeLaTeX}} 的编译命令为:

\begin{minted}[frame=single]{json}
  {
    "name": "xelatex",
    "command": "xelatex",
    "args": [
      "-synctex=1",
      "-interaction=nonstopmode",
      "-file-line-error",
      "-pdf",
      "-outdir=%OUTDIR%",
      "-cd",
      "%DOC%"
    ],
    "env": {}
  }
\end{minted}

整个编译的 recipe 为:

\begin{minted}[frame=single]{json}
  {
    "name": "xelatex * 3",
    "tools": [
        "xelatex",
        "xelatex",
        "xelatex"
    ]
  }
\end{minted}

\paragraph{使用}

各种内容的插入请参考源文件。
