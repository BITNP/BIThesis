\section{计算机学院本科生开题报告使用指南}

\subsection{熟悉项目}
% 输出文件数,第一行百分号 % 不能删,详见:http://tug.ctan.org/macros/generic/dirtree/dirtree.pdf
\dirtree{%
.1 / 计算机学院本科生开题报告项目包含内容:.
.2 README.md.
.2 main.pdf.
.2 main.tex.
.2 merge-sort-recursion-tree.png.
.2 misc.
.3 cover.tex.
.3 refs.bib.
.3 reviewTableBlank.pdf.
}

本项目由四个主要文件编译而成:\texttt{main.tex}、\texttt{cover.tex}、\texttt{refs.bib} 与\\ \texttt{reviewTableBlank.pdf}(也包括文档中所涉及到的图片等素材文件,比如:\\ \texttt{merge-sort-recursion-tree.png})。请大家重点关注这四个文件的功能与作用:

\begin{itemize}
  \item[\color{RubineRed}\textbf{\texttt{main.tex}}] 开题报告的开始文件(主文件),你的报告内容应该从此文件开始撰写。\texttt{main.tex} 中有详细的注释,介绍了每一部分内容都有什么作用,请仔细阅读后进行相应的修改、
  \item[\color{RubineRed}\textbf{\texttt{main.pdf}}] 开题报告编译得到的 PDF 文件
  \item[\color{RubineRed}\textbf{\texttt{./misc}}] 开题报告中所需要的杂项所在文件夹,其中包含有:
  \begin{itemize}
    \item[\color{RoyalBlue}\texttt{cover.tex}] 开题报告封面,按照教务部提供的封面设计,如无特殊需要请不要修改
    \item[\color{RoyalBlue}\texttt{reviewTableBlank.pdf}] 开题报告 PDF 格式的“评审表”,由于考虑到评审表后期由评委老师填写,因此本部分如无需要也无需改动
    \item[\color{RoyalBlue}\texttt{refs.bib}] 开题报告的参考文献 {\hologo{BibTeX}} 数据库,你应该向其中加入开题报告中所需要的所有参考文献的 {\hologo{BibTeX}} 格式引用(详见下文)
  \end{itemize}
\end{itemize}

\subsection{使用与编译方式}
\subsubsection{使用 Overleaf 直接打开}

\subsubsection{在本地撰写}
\paragraph{使用 {\hologo{XeLaTeX}} 编译}
整个项目的编译工具链的顺序为:
\begin{center}
  \begin{tikzpicture}[
    bib/.style={rectangle, draw=green!60, fill=green!5, very thick, minimum size=8mm},
    xe/.style={rectangle, draw=red!60, fill=red!5, very thick, minimum size=8mm},
    ]
  % Nodes
  \node[xe] (xelatex1) {xelatex};
  \node[bib] (biber) [right=of xelatex1] {biber};
  \node[xe] (xelatex2) [right=of biber] {xelatex};
  \node[xe] (xelatex3) [right=of xelatex2] {xelatex};

  % Arrows
  \draw[->] (xelatex1.east) -- (biber.west);
  \draw[->] (biber.east) -- (xelatex2.west);
  \draw[->] (xelatex2.east) -- (xelatex3.west);
  \end{tikzpicture}
\end{center}


其中,按照 VS Code 的 LaTeX Workshop 设置格式:

\begin{itemize}
  \item {\hologo{XeLaTeX}} 的编译命令为:
  \begin{minted}[frame=single]{json}
  {
    "name": "xelatex",
    "command": "xelatex",
    "args": [
      "-synctex=1",
      "-interaction=nonstopmode",
      "-file-line-error",
      "-pdf",
      "-outdir=%OUTDIR%",
      "-cd",
      "%DOC%"
    ],
    "env": {}
  }
  \end{minted}
  \item {\hologo{biber}} 的编译命令为:
  \begin{minted}[frame=single]{json}
  {
    "name": "biber",
    "command": "biber",
    "args": [
        "%DOCFILE%"
    ],
    "env": {}
  }
  \end{minted}
\end{itemize}

那么,整个编译的 recipe 即为:

\begin{minted}[frame=single]{json}
{
  "name": "xelatex -> biber -> xelatex * 2",
  "tools": [
      "xelatex",
      "biber",
      "xelatex",
      "xelatex"
  ]
}
\end{minted}

\paragraph{使用 latexmk 编译}

\subsection{你的内容从哪里开始?}

\subsection{其他注意事项}
