\documentclass[UTF8,AutoFakeBold,AutoFakeSlant,12pt]{ctexart}
\usepackage[a4paper,left=3.18cm,right=3.18cm,top=2.54cm,bottom=2.54cm]{geometry}
\usepackage{graphicx}
\usepackage{hyperref}
\usepackage{hologo}
\usepackage{fontspec}
\usepackage{booktabs}
\usepackage{multirow}
\usepackage{dirtree}
\usepackage[ruled,vlined]{algorithm2e}
\usepackage[dvipsnames]{xcolor}
\usepackage{awesomebox}
\usepackage{wrapfig}
\usepackage{caption}
\usepackage{menukeys}
\usepackage{tikz}
\usetikzlibrary{positioning}
\usepackage{framed}

% 设置代码高亮
\usepackage{minted}
\usemintedstyle{tango}

% 设置定理环境
\usepackage{ntheorem}
\theoremstyle{plain}
\newtheorem{theorem}{\hspace{2em}定理}[section]
\newtheorem{corollary}{\hspace{2em}推论}[theorem]

\theoremstyle{plain}
\theoremheaderfont{\heiti} 
\theorembodyfont{\songti}
\newtheorem{theo}{\hspace{2em}定理}[section]
\newtheorem{coro}{\hspace{2em}推论}[theo]

% 设置列表无间隔
\usepackage{enumitem}
\setlist{nosep}

\newcommand{\version}{v1.0}

\ctexset{
  today=big,
  abstractname=简介
}

\ctexset{section={
  format={\raggedright \bfseries \zihao{-3}},
  name = {第,章}
  }
}

\ctexset{subsection={
  format = {\bfseries \raggedright \zihao{4}}
  }
}

% 设置 caption 与 figure 之间的距离
\setlength{\abovecaptionskip}{11pt}
\setlength{\belowcaptionskip}{9pt}

% 设置图片的 caption 格式
\renewcommand{\thefigure}{\thesection-\arabic{figure}}
\captionsetup[figure]{font=small,labelsep=space}

% 设置表格的 caption 与 table 之间的垂直距离
\captionsetup[table]{skip=2pt}

% 设置表格的 caption 格式
\renewcommand{\thetable}{\thesection-\arabic{table}}
\captionsetup[table]{font=small,labelsep=space}

% 定义 BIThesis \LaTeX 风格的 Logo
\usepackage{relsize}
\makeatletter
\def\matex@ssize{\larger[-1]\scshape}
\DeclareRobustCommand{\BIThesis}{
  \mbox{
    \kern-0.5em{B}\kern-0.05em
    {I}\kern-0.05em
    {T}\kern-0.1em
    \raisebox{-0.38ex}{\matex@ssize {H}}\kern-0.1em
    {\matex@ssize {E}}\kern-0.05em
    \raisebox{-0.38ex}{\matex@ssize {S}}\kern-0.05em
    {\matex@ssize {I}}\kern-0.05em
    \raisebox{-0.35ex}{\matex@ssize {S}}\kern-0.5em
   }
}
\makeatother

\begin{document}
\title{
\includegraphics[width=0.3\textwidth]{images/icon.png}
\\[1cm]
\bfseries 北京理工大学本科生{\LaTeX}模板使用手册
}
\author{
  \href{https://github.com/BITNP/BIThesis}{\color{RoyalBlue}{BITNP/BIThesis}}
  \\
  \zihao{5}{\kaishu 主编:北京理工大学 2016 级计算机学院 \href{https://github.com/spencerwooo}{武上博} \href{https://github.com/Silverster98}{王赞}}
}
\date{\zihao{-4} \today\quad \color{RubineRed}{\kaishu {\BIThesis} 版本 \version}}
\maketitle

\begin{abstract}
  {\BIThesis} 北京理工大学本科生 {\LaTeX} 模板是北京理工大学本科生毕业设计开题报告、总论文,以及其他课程报告、实验报告等重要论文、报告的 {\LaTeX} 模板集合。如果你厌烦了 Word 格式的不人性化、参考文献的难以管理、公式输入的差劲体验……那么欢迎来尝试用专业的学术稿件排版利器 —— {\LaTeX},来排版你的论文。专业高端、学界认可、开源免费,{\LaTeX} 是你论文排版的最佳搭档。

  {\BIThesis} 北京理工大学本科生 {\LaTeX} 模板目前支持使用 {\hologo{XeLaTeX}} 进行编译,使用以 biber 为后端的 BibLaTeX 进行参考文献的生成,符合《信息与文献参考文献著录规则》(\href{http://openstd.samr.gov.cn/bzgk/gb/newGbInfo?hcno=7FA63E9BBA56E60471AEDAEBDE44B14C}{GB/T 7714—2015})的标准。目前主要设计完成了计算机学院本科生毕业论文开题报告、毕业设计毕业论文与通用实验报告的 {\LaTeX} 模板。
\end{abstract}

\tableofcontents
\clearpage
\setlength{\parskip}{0.8ex}

\section{如何开始}
{\BIThesis} 为各位在北京理工大学就读的本科同学提供了基于北京理工大学计算机学院教务部给出的“北京理工大学计算机学院本科生毕业论文:开题报告”与北京理工大学教务部提供的“北京理工大学本科生毕业设计:论文模板(目前是 2019 届版本)”的 \LaTeX 样版。借助于 {\BIThesis} 的 \LaTeX 模板,你可以在保证论文格式整齐、完美、符合要求的前提下,专注于学术研究、项目实现,从而顺利完成你的学术项目。

本“使用手册”希望为大家全面的介绍 {\LaTeX} 环境的搭建方法、{\BIThesis} 的使用方法,从而快速掌握使用 {\LaTeX} 排版引擎进行基本的论文撰写的方法,完成符合学校要求的学位论文。{\BIThesis} 目前使用 GitHub 进行维护,官方项目地址位于:

\begin{center}
\color{ForestGreen}\href{https://github.com/spencerwooo/BIThesis}{\texttt{https://github.com/spencerwooo/BIThesis}}
\end{center}

\subsection{{\BIThesis} 在线说明文档}
和本手册的目标类似,{\BIThesis} 项目同样维护了一个在线版本的说明文档,位于:{\href{https://github.com/spencerwooo/BIThesis/wiki}{BIThesis - wiki}},二者的目的、内容、功能类似,且会随着模板的开发与维护同步更新。

{\BIThesis} 在线说明文档目前拥有如下模块:

\begin{enumerate}
\item \href{https://github.com/spencerwooo/BIThesis/wiki}{主页:Home}
\item \href{https://github.com/spencerwooo/BIThesis/wiki/First-things-first}{如何开始:First things first }
\item \href{https://github.com/spencerwooo/BIThesis/wiki/Using-one-of-the-templates}{使用其中一个模板:Using one of the templates}
\item \href{https://github.com/spencerwooo/BIThesis/wiki/Proposal-Report}{本科生开题报告:Proposal report}
\item \href{https://github.com/spencerwooo/BIThesis/wiki/Final-Graduation-Thesis}{本科生毕业论文:Graduation thesis}
\item \href{https://github.com/spencerwooo/BIThesis/wiki/Lab-Report}{本科生实验报告:Lab report}
\item \href{https://github.com/spencerwooo/BIThesis/wiki/Converting-to-Word}{将 LaTeX 文档转换为 Word:Converting to Word}
\end{enumerate}

接下来,我们正式开始介绍 {\LaTeX} 与 {\BIThesis} 的使用方法。

\subsection{准备工作}

第一章节:{\BIThesis}

\subsection{下载合适的 \LaTeX 发行版}
\subsection{挑选合适的 \LaTeX 编辑器}

\clearpage
\section{使用一个模板}
{\BIThesis} 整个项目中包含多个模板,每个模板各自位于独立的文件夹中。

\subsection{熟悉简单 \LaTeX 语法}
如果你之前没有接触过 {\LaTeX},请前往 Overleaf 的“\href{https://www.overleaf.com/learn/latex/Learn_LaTeX_in_30_minutes}{30 分钟学习 {\LaTeX}}”文档进行阅读,从而对 {\LaTeX} 有大致的印象。

一些常用的 {\LaTeX} 格式与使用技巧:

\begin{itemize}
  \item \href{https://www.overleaf.com/learn/latex/Sections_and_chapters}{{\LaTeX} 章节设定:Sections and chapters}
  \item \href{https://www.overleaf.com/learn/latex/Paragraphs_and_new_lines}{{\LaTeX} 段落格式:Paragraphs and new lines}
  \item \href{https://www.overleaf.com/learn/latex/Bold,_italics_and_underlining}{{\LaTeX} 粗体、斜体与下划线:Bold, italics and underlining}
  \item \href{https://www.overleaf.com/learn/latex/Lists}{{\LaTeX} 有序列表、无序列表:Lists}
  \item \href{https://www.overleaf.com/learn/latex/Inserting_Images}{{\LaTeX} 插入图片:Inserting Images}
  \item \href{https://www.overleaf.com/learn/latex/Tables}{{\LaTeX} 构建表格:Tables}
  \item \href{https://www.overleaf.com/learn/latex/Mathematical_expressions}{{\LaTeX} 插入数学公式:Mathematical expressions}
  \item \href{https://www.overleaf.com/learn/latex/Code_Highlighting_with_minted}{{\LaTeX} 插入代码与代码高亮:Code Highlighting with minted}
  \item \href{https://www.overleaf.com/learn/latex/algorithms}{{\LaTeX} 插入算法伪代码描述:Algorithms}
  \item \href{https://www.overleaf.com/learn/latex/Bibliography_management_in_LaTeX}{使用 {Bib\LaTeX} 管理参考文献:Bibliography management in LaTeX}
\end{itemize}

有关 {\LaTeX} 使用的更多技巧,请直接前往 \href{https://www.overleaf.com/learn/latex/Main_Page}{Overleaf 官方文档}进行查看。准备就绪之后,你就可以前往下载 {\BIThesis} 模板啦。

\subsection{在项目的 Release 页面下载你希望使用的模板}
为了方便各位同学使用,项目按照 Release 发布的流程,将每个模板进行打包,并在每次发版后用 GitHub Release 进行模板分发。也就是,你可以直接前本项目的 GitHub Release 页面,直接下载你所希望使用的模板压缩包,并解压到本地进行使用。

你可以点击这个链接前往最新的 Release 版本进行模板下载:

\begin{center}
  \color{ForestGreen}\href{https://github.com/spencerwooo/BIThesis/releases/latest}{https://github.com/spencerwooo/BIThesis/releases/latest}
\end{center}

在 Release 页面,你会看到:

\dirtree{%
.1 /.
.2 proposal-report.zip \ldots{} \color{RoyalBlue}{本科生毕业设计开题报告模板压缩包}.
.2 graduation-thesis.zip \ldots{} \color{RoyalBlue}{本科生毕业设计毕业论文模板压缩包}.
.2 lab-report.zip \ldots{} \color{RoyalBlue}{本科生实验报告模板压缩包}.
}

\begin{figure}[H]
  \centering
  \includegraphics[width=\textwidth]{images/release.png}
  \caption{{\BIThesis} 的 Release 页面}
\end{figure}

根据你的选择,下载其中你所要使用的模板即可。(当然,你也可以直接用 Git 将本项目完整克隆至本地,使用最新版本的模板。)

\clearpage
\section{计算机学院本科生开题报告使用指南} \label{proposal}

本模板已经发布在 Overleaf 上,你可以打开直接使用(点击下图 \ref{overleaf-proposal} 所示中的 Open as Template 即可:

\begin{center}
  \color{ForestGreen}\href{https://www.overleaf.com/latex/templates/bei-jing-li-gong-da-xue-ben-ke-sheng-bi-ye-lun-wen-kai-ti-bao-gao-mo-ban/dgqdjptfqtrn}{https://www.overleaf.com/latex/templates/bei-jing-li-gong-da-xue-ben-ke-sheng-bi-ye-lun-wen-kai-ti-bao-gao-mo-ban/dgqdjptfqtrn}
\end{center}

\begin{figure}[H]
  \centering
  \includegraphics[width=\textwidth]{images/overleaf.png}
  \caption{Overleaf 在线版本的开题报告模板}
  \label{overleaf-proposal}
\end{figure}

Overleaf 缺少一些微软版权字体(比如宋体、黑体等),\textbf{因此如果你希望格式完全准确,请使用本机进行编辑。}

\subsection{熟悉项目}
% 输出文件数,第一行百分号 % 不能删,详见:http://tug.ctan.org/macros/generic/dirtree/dirtree.pdf
\dirtree{%
.1 /.
.2 README.md.
.2 main.pdf.
.2 main.tex.
.2 merge-sort-recursion-tree.png.
.2 misc.
.3 cover.tex.
.3 refs.bib.
.3 reviewTableBlank.pdf.
}

本项目由四个主要文件编译而成:\texttt{main.tex}、\texttt{cover.tex}、\texttt{refs.bib} 与\\ \texttt{reviewTableBlank.pdf}(也包括文档中所涉及到的图片等素材文件,比如:\\ \texttt{merge-sort-recursion-tree.png})。请大家重点关注这四个文件的功能与作用:

\begin{itemize}
  \item[\color{RubineRed}\textbf{\texttt{main.tex}}] 开题报告的开始文件(主文件),你的报告内容应该从此文件开始撰写。\texttt{main.tex} 中有详细的注释,介绍了每一部分内容都有什么作用,请仔细阅读后进行相应的修改、
  \item[\color{RubineRed}\textbf{\texttt{main.pdf}}] 开题报告编译得到的 PDF 文件
  \item[\color{RubineRed}\textbf{\texttt{./misc}}] 开题报告中所需要的杂项所在文件夹,其中包含有:
  \begin{itemize}
    \item[\color{RoyalBlue}\texttt{cover.tex}] 开题报告封面,按照教务部提供的封面设计,如无特殊需要请不要修改
    \item[\color{RoyalBlue}\texttt{reviewTableBlank.pdf}] 开题报告 PDF 格式的“评审表”,由于考虑到评审表后期由评委老师填写,因此本部分如无需要也无需改动
    \item[\color{RoyalBlue}\texttt{refs.bib}] 开题报告的参考文献 {\hologo{BibTeX}} 数据库,你应该向其中加入开题报告中所需要的所有参考文献的 {\hologo{BibTeX}} 格式引用(详见下文)
  \end{itemize}
\end{itemize}

\subsection{你的内容从哪里开始?}
开题报告项目结构相对来说比较简单,因此你只需要重点关注 \texttt{main.tex} 这一文件 —— 项目的主文件。你的内容应该从 \texttt{main.tex} 第 127 行 的 \texttt{\%内容开始} 开始。你需要重点关注的部分有:

\begin{table}[H]
\centering
\caption{开题报告内容概要}
\label{tab:proposalreport}
\resizebox{\textwidth}{!}{%
\begin{tabular}{@{}rll@{}}
\toprule
\textbf{文章部分} & \textbf{内容主旨}      & \textbf{对应 {\LaTeX} 模板 section}    \\ \midrule
第一部分 & 选题内容      & \texttt{\textbackslash{}section\{毕业设计(论文)\}}        \\ \midrule
第二部分 & 研究方案      & \texttt{\textbackslash{}section\{研究方案\}}            \\ \midrule
2-1  & 主要任务      & \texttt{\textbackslash{}subsection\{本选题的主要任务\}}     \\
2-2  & 技术方案      & \texttt{\textbackslash{}subsection\{技术方案的分析、选择\}}   \\
2-3  & 实施方案所需环境  & \texttt{\textbackslash{}subsection\{实施技术方案所需的条件\}}  \\
2-4  & 存在问题与技术关键 & \texttt{\textbackslash{}subsection\{存在的主要问题和技术关键\}} \\
2-5  & 预期研究目标    & \texttt{\textbackslash{}subsection\{预期能够达到的研究目标\}}  \\ \midrule
第三部分 & 课题计划进度表   & \texttt{\textbackslash{}section\{课题计划进度表\}}         \\ \bottomrule
\end{tabular}%
}
\end{table}

以及最后的“参考文献”。你应该将参考文献的 {\hologo{BibTeX}} 引用复制进入 \texttt{./misc/refs.bib},并在正文中用 \verb|\cite{}| 的方法进行引用。其中 {\hologo{BibTeX}} 格式的引用内容可以在谷歌学术中搜索文章直接复制得到,也可以考虑使用 Zotero 等文献管理工具批量生成。

\subsection{其他注意事项}
\tipbox{有关具体的 {\LaTeX} 语法,请参考前文中《第二章 \ref{subsec:latex-grammar}》给出的参考链接与学习文档。以下是模板中提供的一些示例性代码。}

\subsubsection{插入图片}
如果你希望加入图片,可以将图片直接放在根目录(比如此处的\\ \texttt{merge-sort-recursion-tree.png}),或者统一将图片安置在一个文件夹下,在正文里按照相对路径进行引用。模板中有一处插入图片的参考样例,位于 \texttt{main.tex} 的 \href{https://github.com/BITNP/BIThesis/blob/master/proposal-report/main.tex#L138}{第 138 行},可以进行参考。比如,我填入一个放在 \texttt{images/BIT\_Name.jpg} 处的图片:

\begin{minted}[frame=single,linenos,breaklines]{latex}
  \begin{figure}[!ht]
    \centering
    \includegraphics[width=0.6\linewidth]{images/BIT_Name.jpg}
    \caption{北京理工大学(一张示意图)}
    \label{fig:BITName}
  \end{figure}
\end{minted}

这样就会渲染如图 \ref{fig:BITName} 的效果:

\begin{figure}[!ht]
  \centering
  \includegraphics[width=0.6\linewidth]{images/BIT_Name.jpg}
  \caption{北京理工大学(一张示意图)}
  \label{fig:BITName}
\end{figure}

\subsubsection{插入表格}
如果你希望插入表格,可以统一使用 \href{https://www.tablesgenerator.com/}{LaTeX Tables Generator} 进行生成,再粘贴进入模板之中。模板中有两处表格的参考样例,分别位于 \href{https://github.com/BITNP/BIThesis/blob/master/proposal-report/main.tex#L151}{第 151 行} 和 \href{https://github.com/BITNP/BIThesis/blob/master/proposal-report/main.tex#L176}{第 176 行},可以进行参考。比如:

\begin{minted}[frame=single,linenos,breaklines]{latex}
  \begin{table}[!ht]
    \centering
    \caption{硬件、软件环境}
    \label{tab:soft-hardware}
    \begin{tabular}{@{}lcl@{}}
      \toprule
                                & 指标     & \multicolumn{1}{c}{版本参数} \\ \midrule
      \multirow{2}{*}{硬件环境} & CPU      & Intel i7-6500U               \\ \cmidrule(l){2-3}
                                & RAM      & 8 GB                         \\ \midrule
      \multirow{2}{*}{软件环境} & 操作系统 & \begin{tabular}[c]{@{}l@{}}Windows 10 Pro x86\_64\\  Ubuntu 18.04.3 LTS\end{tabular}    \\ \cmidrule(l){2-3}
                                & Python   & Python 3.7.6                 \\ \bottomrule
    \end{tabular}
  \end{table}
\end{minted}

渲染效果如表 \ref{tab:proposalreport} 所示:

\begin{table}[!ht]
  \centering
  \caption{硬件、软件环境}
  \label{tab:soft-hardware}
  \begin{tabular}{@{}lcl@{}}
    \toprule
                              & 指标     & \multicolumn{1}{c}{版本参数} \\ \midrule
    \multirow{2}{*}{硬件环境} & CPU      & Intel i7-6500U               \\ \cmidrule(l){2-3}
                              & RAM      & 8 GB                         \\ \midrule
    \multirow{2}{*}{软件环境} & 操作系统 & \begin{tabular}[c]{@{}l@{}}Windows 10 Pro x86\_64\\  Ubuntu 18.04.3 LTS\end{tabular}    \\ \cmidrule(l){2-3}
                              & Python   & Python 3.7.6                 \\ \bottomrule
  \end{tabular}
\end{table}

\clearpage
\section{北京理工大学本科生毕业设计论文模板使用指南}

\tipbox{注意:目前版本的毕业设计论文是按照北京理工大学计算机学院 2015 级毕业论文模板进行的设计与排版,如果 2016 级毕业论文模板有任何格式更新,我们会及时在这里更新。}

\subsection{熟悉项目}

\dirtree{%
  .1 /.
  .2 README.md.
  .2 main.tex.
  .2 main.pdf.
  .2 chapters.
  .3 0\_abstract.tex.
  .3 1\_chapter1.tex.
  .2 images.
  .3 bit\_logo.png.
  .3 header.png.
  .2 misc.
  .3 0\_cover.tex.
  .3 1\_originality.tex.
  .3 2\_toc.tex.
  .3 3\_conclusion.tex.
  .3 4\_reference.tex.
  .3 5\_appendix.tex.
  .3 6\_acknowledgements.tex.
  .3 ref.bib.
}

本项目由一个主文件和与之并存的几个辅助文件夹中的文件构成:

\begin{itemize}
  \item[\color{RubineRed}\textbf{\texttt{main.tex}}] 毕业论文模板的主文件
  \item[\color{RubineRed}\textbf{\texttt{./chapters}}] 文件夹:包含有整个毕业论文的“摘要”和正文的全部“章节”
  \begin{itemize}
    \item[\color{RoyalBlue}\texttt{0\_abstract.tex}] 毕业论文的“摘要”(中文摘要与英文摘要)
    \item[\color{RoyalBlue}\texttt{1\_chapter1.tex}] 毕业论文正文“第一章”(示例章节)
    \item[\color{RoyalBlue}\texttt{...}] (你可以继续添加第二章 \texttt{2\_chapter2.tex}、第三章 \texttt{3\_chapter3.tex}……,并在主文件 \texttt{main.tex} 中引用(详见下文)
  \end{itemize}
  \item[\color{RubineRed}\textbf{\texttt{./misc}}] 文件夹:包含有毕业论文模板中的封面、后置章节与参考文献
  \begin{itemize}
    \item[\color{RoyalBlue}\textbf{\texttt{0\_cover.tex}}] 毕业论文的“封面”,一般情况无需更改
    \item[\color{RoyalBlue}\textbf{\texttt{1\_originality.tex}}] 毕业论文的“原创性声明”,一般情况无需更改(签字和日期后期手动添加)
    \item[\color{RoyalBlue}\textbf{\texttt{2\_toc.tex}}] 毕业论文的“目录”,一般情况无需更改(由 {\LaTeX} 自动生成)
    \item[\color{RoyalBlue}\textbf{\texttt{3\_conclusion.tex}}] 毕业论文的“结论”,按照一般章节文件对待
    \item[\color{RoyalBlue}\textbf{\texttt{4\_reference.tex}}] 毕业论文的“参考文献”,一般情况无需更改(由 {\LaTeX} 根据你文档中的 \verb|\cite{}| 自动生成)
    \item[\color{RoyalBlue}\textbf{\texttt{5\_appendix.tex}}] 毕业论文的“附录”,按照一般章节文件对待
    \item[\color{RoyalBlue}\textbf{\texttt{6\_acknow...ments.tex}}] 毕业论文的“致谢”,按照一般章节文件对待
    \item[\color{RoyalBlue}\textbf{\texttt{ref.bib}}] 参考文献 \hologo{BibTeX} 数据库
  \end{itemize}
\end{itemize}

主文件与其余文件之间的引用关系大致如下图 \ref{grad_thesis_main_submodule} 所示:

\begin{figure}[H]
  \center
  \includegraphics[width=\textwidth]{images/grad_thesis.png}
  \caption{毕业论文模板主模块与各个分支之间的关系}
  \label{grad_thesis_main_submodule}
\end{figure}

具体编译和使用,请见下文详细描述。

\subsection{使用与编译方式}

\subsubsection{使用 Overleaf 直接打开}

本模板已经发布在 Overleaf 上,你可以打开直接使用:

\begin{center}
  \color{ForestGreen}\href{https://www.overleaf.com/latex/templates/bei-jing-li-gong-da-xue-ben-ke-sheng-bi-ye-she-ji-lun-wen-mo-ban/mwhjgqsncxxg}{https://www.overleaf.com/latex/templates/bei-jing-li-gong-da-xue-ben-ke-sheng-bi-ye-she-ji-lun-wen-mo-ban/mwhjgqsncxxg}
\end{center}

\begin{figure}[H]
  \centering
  \includegraphics[width=\textwidth]{images/overleaf_grad_thesis.png}
  \caption{Overleaf 在线版本的毕业论文模板}
\end{figure}

Overleaf 版本的毕业论文模板中由于没有微软版权字体“华文细黑”,导致封面的毕业论文中文大标题无法用 Word 模板中规定的字体渲染,使得最终呈现样式与要求有些出入,如果希望保证 {\LaTeX} 模板输出和学校模板一致,那么还是推荐在本地进行撰写和编译。

\subsubsection{在本地撰写}

由于:

\begin{itemize}
  \item {\BIThesis} 文章主体部分是中文,使用了 \texttt{ctex} 宏包,因此需要使用 \texttt{xelatex} 进行全文编译
  \item 参考文献部分使用了 {Bib\LaTeX},因此需要使用 \hologo{biber} 进行参考文献的编译
\end{itemize}

\paragraph{使用 {\hologo{XeLaTeX}} 编译}
整个项目的编译工具链的顺序为:

\begin{center}
  \begin{tikzpicture}[
    bib/.style={rectangle, draw=ForestGreen!60, fill=ForestGreen!5, very thick, minimum size=8mm},
    xe/.style={rectangle, draw=RubineRed!60, fill=RubineRed!5, very thick, minimum size=8mm},
    ]
  % Nodes
  \node[xe] (xelatex1) {xelatex};
  \node[bib] (biber) [right=of xelatex1] {biber};
  \node[xe] (xelatex2) [right=of biber] {xelatex};
  \node[xe] (xelatex3) [right=of xelatex2] {xelatex};

  % Arrows
  \draw[->] (xelatex1.east) -- (biber.west);
  \draw[->] (biber.east) -- (xelatex2.west);
  \draw[->] (xelatex2.east) -- (xelatex3.west);
  \end{tikzpicture}
\end{center}


其中,按照 VS Code 的 LaTeX Workshop 设置格式:

\begin{itemize}
  \item {\hologo{XeLaTeX}} 的编译命令为:
  \begin{minted}[frame=single]{json}
  {
    "name": "xelatex",
    "command": "xelatex",
    "args": [
      "-synctex=1",
      "-interaction=nonstopmode",
      "-file-line-error",
      "-pdf",
      "-outdir=%OUTDIR%",
      "-cd",
      "%DOC%"
    ],
    "env": {}
  }
  \end{minted}
  \item {\hologo{biber}} 的编译命令为:
  \begin{minted}[frame=single]{json}
  {
    "name": "biber",
    "command": "biber",
    "args": [
        "%DOCFILE%"
    ],
    "env": {}
  }
  \end{minted}
\end{itemize}

那么,整个编译的 recipe 即为:

\begin{minted}[frame=single]{json}
  {
    "name": "xelatex -> biber -> xelatex * 2",
    "tools": [
        "xelatex",
        "biber",
        "xelatex",
        "xelatex"
    ]
  }
\end{minted}

\paragraph{使用 latexmk 编译}
如果你使用 \texttt{latexmk},也可以使用如下的编译方法:

\begin{itemize}
  \item \texttt{latexmk} 的编译命令:
  \begin{minted}[frame=single]{json}
  {
    "name": "latexmk",
    "command": "latexmk",
    "args": [
        "-synctex=1",
        "-interaction=nonstopmode",
        "-file-line-error",
        "-xelatex",
        "-outdir=%OUTDIR%",
        "-cd",
        "%DOC%"
    ],
    "env": {}
  }
  \end{minted}
\end{itemize}

那么,整个编译的 recipe 即为:
\begin{minted}[frame=single]{json}
  {
    "name": "latexmk 🔃",
    "tools": [
        "latexmk"
    ]
  }
\end{minted}

\subsection{你的内容从哪里开始?}

\subsubsection{开始}

\subsubsection{中英摘要}

\subsubsection{正文}

\subsubsection{后续模块}

\subsection{参考文献管理}

\subsection{图片素材}

\subsection{表格插入}

\subsection{公式插入}

\subsection{其他}

\subsubsection{代码高亮}

\subsubsection{算法模块}

\clearpage
\section{通用北京理工大学本科生实验报告模板使用指南} \label{lab-report}

\tipbox{说明:这个实验报告模板是一个通用的报告模板,不适用所有实验报告要求。实验课程未提供实验报告模板时可以使用该模板。当前本实验报告模板只包含一个封面,欢迎大家往项目仓库 PR 制作更多的封面。}

\subsection{熟悉项目}
\dirtree{%
  .1 /.
  .2 README.md.
  .2 main.pdf.
  .2 main.tex.
  .2 misc.
  .3 cover\_v1.tex.
  .2 assets.
  .3 …….
}

\begin{itemize}
  \item[\color{RubineRed}\textbf{\texttt{main.tex}}] tex 源文件,本实验报告模板的主体文件,所有需要添加的内容都在该文件里进行修改即可
  \item[\color{RubineRed}\textbf{\texttt{main.pdf}}] 编译项目生成的 pdf 文件
  \item[\color{RubineRed}\textbf{\texttt{./misc}}] 杂项(包括实验报告封面等):
        \begin{itemize}
          \item[\color{RoyalBlue}\texttt{cover\_v1.tex}] 这是一个示范性的报告封面,该文件无需修改
        \end{itemize}
  \item[\color{RubineRed}\textbf{\texttt{./asset}}] 一些图片资源存放文件夹
\end{itemize}

\subsection{编译方式与使用}

由于实验报告模板没有涉及到参考文献的使用,因此我们只需要使用 \hologo{XeLaTeX} 即可进行全文编译。

整个项目的编译工具链的顺序为:
\begin{center}
  \begin{tikzpicture}[
      bib/.style={rectangle, draw=ForestGreen!60, fill=ForestGreen!5, very thick, minimum size=8mm},
      xe/.style={rectangle, draw=RubineRed!60, fill=RubineRed!5, very thick, minimum size=8mm},
    ]
    % Nodes
    \node[xe] (xelatex1) {xelatex};
    \node[xe] (xelatex2) [right=of xelatex1] {xelatex};
    \node[xe] (xelatex3) [right=of xelatex2] {xelatex};

    % Arrows
    \draw[->] (xelatex1.east) -- (xelatex2.west);
    \draw[->] (xelatex2.east) -- (xelatex3.west);
  \end{tikzpicture}
\end{center}

其中,按照 VS Code 的 LaTeX Workshop 设置格式:{\hologo{XeLaTeX}} 的编译命令为:

\begin{minted}[frame=single]{json}
  {
    "name": "xelatex",
    "command": "xelatex",
    "args": [
      "-synctex=1",
      "-interaction=nonstopmode",
      "-file-line-error",
      "-pdf",
      "-outdir=%OUTDIR%",
      "-cd",
      "%DOC%"
    ],
    "env": {}
  }
\end{minted}

整个编译的 recipe 为:

\begin{minted}[frame=single]{json}
  {
    "name": "xelatex * 3",
    "tools": [
        "xelatex",
        "xelatex",
        "xelatex"
    ]
  }
\end{minted}

\paragraph{使用}

各种内容的插入请参考源文件。

\clearpage
\section{如何将 {\LaTeX} 文档转换为 Word}

\clearpage
\section{疑难杂症}

\clearpage
\section{致谢}

{\BIThesis} 模板是一个团队项目。从一个开题报告的启发,到最终完成程度接近 100\%,格式几乎完全匹配的毕业论文;从一个简陋的 README,到近一万两千字的在线版本 Wiki 文档和近一万字的 {\LaTeX} PDF 文档(本文档)……这些离不开团队成员(2016 级计算机学院本科生武上博、2016 级计算机学院本科生王赞)两位同学近一个月的努力与维护,他们见证了 {\BIThesis} 的从无到有。

作为 {\BIThesis} 的创造者,我(武上博)由衷的感谢北京理工大学教务部、北京理工大学计算机学院为我们项目提供的强大背书,没有支持我们的领导和老师们的助力,本项目将毫无意义;感谢最初打开、阅读、与分享这一项目的介绍公众号文章的各位老师与同学,没有愿意分享这一项目的你们,本项目将无人问津;感谢愿意为这一项目在 GitHub 上做出贡献、提出 pull request 的诸位同学,没有你们的贡献,本项目也不会有这么多的特色;感谢因为各种问题在 GitHub 的 issue 上进行询问的同学,没有你们的提问,本项目同样也不会这样完善。当然,最终,也要感谢使用 {\BIThesis} 的同学们,你们为我们的工作赋予了意义,为我们的后续维护提供了动力,感谢你们,感谢所有人。

本文档是 {\BIThesis} 的官方参考手册,由 2016 级计算机学院武上博同学、2016 级计算机学院王赞同学撰写,感谢 2016 级计算机学院唐誉铭同学、2017 级经济与管理学院詹熠莎同学提供的宝贵意见。


\end{document}
