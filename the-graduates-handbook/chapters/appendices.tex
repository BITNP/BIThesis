\begin{appendices}
  \chapter{学习资料} \label{resources}

  \section{\LaTeX 学习资料推荐}
  \begin{itemize}[nosep]
    \item \href{https://www.overleaf.com/learn/latex/Tutorials}{《Overleaf 在线文档》(英文)} 提供了非常好的在线学习资源。
    \item \href{https://texdoc.org/serve/lshort-zh-cn.pdf/0}{《一份(不太)简短的 LATEX 2ε 介绍》} 可以作为更详尽的语法手册。
  \end{itemize}

  更多可参考 \href{https://bithesis.bitnp.net/guide/resources.html}{\LaTeX 学习与使用资源 | BIThesis}。

  \section{\BIThesis 模板配置使用手册}
  \BIThesis{} 使用手册位于项目文件夹的 \verb|./bithesis.pdf|。它包括了关于 \BIThesis{} 的详细使用说明,
      对于每一个配置选项都有详细的说明和示例。
      
  \chapter{\BIThesis 与北理工历代\LaTeX{}模板项目简介}
  
\begin{itemize}[nosep]
  \item 在 2017 年之前,网络上已经出现一些北京理工大学学位论文 \LaTeX 模板。
    它们是“2012大眼小蚂蚁版”和“2016汪卫版”,均以上海交通大学的模板为基础。
  \item 2017 - 2018 年,计算机学院 2016 级研究生杨雅婷等人受研究生院委托,
    制作了\href{https://github.com/BIT-thesis/LaTeX-template}{BIT-Thesis} 
    研究生学位论文模板。
  \item 2019 - 2020 年,\BIThesis 最早由 2016 级的
    武上博、王赞、唐誉铭、牟思睿和詹熠莎等人维护。
    \begin{itemize}[nosep]
    \item 此时,\BIThesis 仅支持本科生毕业论文的排版。
    \item 在此期间,\BIThesis 从无到有诞生了,包括使用手册、
      在线文档和开箱即用的模板。
    \item 同时,2017 级的赵池等同学完成了一系列 \BIThesis 
      的视频教程。
    \item 武上博推进了教务部对 \BIThesis{} 的认可工作。
  \end{itemize}
  \item 2020 - 2021 年,2017 级的冯开宇、杨思云、郝正亮和顾骁等人
      接管了维护开发工作。
  \begin{itemize}[nosep]
    \item 在此期间,冯开宇将原来的 .tex 文件制作成了宏包,并发布到 CTAN 上。
    \item 此版本是 V2 版本,代号为 Birthday Cake.
  \end{itemize}
  \item 2021 - 2022 年,2021 级(硕士研究生)的冯开宇针对 2021、
      2022 毕业季收到的反馈对该项目进行维护升级。
  \begin{itemize}[nosep]
    \item 在此期间,冯开宇合入了杨雅婷等人在 2017 年开发的研究生学位论文模板。
    \item 次年暑假期间,冯开宇用 \verb|expl3| 重构了\LaTeX 样式代码,
      向用户提供了简易易用的接口。同时,也增加了本科全英文专业的毕设论文模板样式。
    \item 此版本是 V3 版本,代号为 Summer Time.
  \end{itemize}
  \item 2023 年,冯开宇在此版本上增加了多种新的功能,并修复了一些已知的问题。
  并推进了官方(教务部、研究生院)对 \BIThesis 的认可工作。
\end{itemize}

\end{appendices}
