%%
% The BIThesis Template for Bachelor Graduation Thesis
%
% 北京理工大学毕业设计(论文) —— 使用 XeLaTeX 编译
%
% Copyright 2020 Spencer Woo
%
% This work may be distributed and/or modified under the
% conditions of the LaTeX Project Public License, either version 1.3
% of this license or (at your option) any later version.
% The latest version of this license is in
%   http://www.latex-project.org/lppl.txt
% and version 1.3 or later is part of all distributions of LaTeX
% version 2005/12/01 or later.
%
% This work has the LPPL maintenance status `maintained'.
%
% The Current Maintainer of this work is Spencer Woo.
%
% This work consists of the files main.tex, misc/cover.tex and
% the external PDF misc/reviewTable.pdf
%
% Compile with: xelatex -> biber -> xelatex -> xelatex

% 章节支持、单面打印:ctexbook
\documentclass[UTF8,AutoFakeBold,AutoFakeSlant,zihao=-4,oneside,openany]{ctexbook}
\usepackage[a4paper,left=3cm,right=2.6cm,top=3.5cm,bottom=2.6cm]{geometry}
\usepackage{xeCJK}
\usepackage{titletoc}
\usepackage{fontspec}
\usepackage{setspace}
\usepackage{graphicx}
\usepackage{fancyhdr}
\usepackage{titlesec}
\usepackage{pdfpages}
\usepackage{setspace}
\usepackage{booktabs}
\usepackage{multirow}
\usepackage{caption}
\usepackage{tikz}
\usepackage{etoolbox}
\usepackage{hyperref}

% 设置参考文献编译后端为 biber,引用格式为 IEEE 格式
% \usepackage[style=ieee,backend=biber]{biblatex}

% 参考文献引用文件 refs.bib
% \addbibresource{misc/refs.bib}

% 西文字体默认为 Times New Roman
\setromanfont{Times New Roman}
% 论文题目字体为华文细黑
\setCJKfamilyfont{xihei}{STXihei}
\newcommand{\xihei}{\CJKfamily{xihei}}

\newcommand{\thesisTitle}{北京理工大学本科生毕业设计(论文)题目}
\newcommand{\thesisTitleEN}{The Subject of Undergraduate Graduation Project (Thesis) of Beijing Institute of Technology}

\newcommand{\deptName}{计算机学院}
\newcommand{\majorName}{计算机科学与技术}
\newcommand{\yourName}{惠计算}
\newcommand{\yourStudentID}{11xxxxxxxx}
\newcommand{\mentorName}{张哈希}

\fancypagestyle{BIThesis}{%
  % 页眉高度
  \setlength{\headheight}{20pt}
  \setlength{\footskip}{14pt}
  \fancyhf{}
  \fancyhead[C]{\zihao{4}\ziju{0.08}\songti{北京理工大学本科生毕业设计(论文)}}
  \fancyfoot[C]{\songti\zihao{5} \thepage}
  % 页眉分割线稍微粗一些
  \renewcommand{\headrulewidth}{0.6pt}
}

%% 设置章节格式, 黑体三号加粗居中,行距22磅,与正文或节标题的间距设定为段后间距1行。章序号与章名间空一格。
\ctexset{chapter={
    name = {第,章},
    number = {\arabic{chapter}},
    format = {\bfseries \centering \zihao{3} },
    pagestyle = BIThesis,
    beforeskip = 16 bp,
    afterskip = 32 bp,
    fixskip = true,
  }
}

% 设置一级标题格式
% 黑体四号加粗顶左,行距22磅,与上一节的间距为1行,与下面正文或节标题的段间间距为0.5行。序号与题目间空一格。
\ctexset{section={
  format={\heiti \raggedright \bfseries \zihao{4}},
  beforeskip = 28bp plus 1ex minus .2ex,
  afterskip = 24bp plus .2ex,
  fixskip = true,
  }
}

% 设置二级标题格式
% 黑体小四加粗顶左,行距22磅,与上一节的间距为1行,与下面正文或节标题的段间间距为0.5行。序号与题目间空一格。
\ctexset{subsection={
   format = {\heiti \bfseries \raggedright \zihao{-4}},
   beforeskip =28bp plus 1ex minus .2ex,
   afterskip = 24bp plus .2ex,
   fixskip = true,
   }
}


% 设置三级标题格式
\ctexset{subsubsection={
      format={\heiti \raggedright \zihao{-4}},
      beforeskip=28bp plus 1ex minus .2ex,
      afterskip=24bp plus .2ex,
      fixskip=true,
  }
}

\begin{document}

% 标题页面
%%
% The BIThesis Template for Bachelor Graduation Thesis
%
% 北京理工大学毕业设计(论文)封面页 —— 使用 XeLaTeX 编译
%
% Copyright 2020-2021 BITNP
%
% This work may be distributed and/or modified under the
% conditions of the LaTeX Project Public License, either version 1.3
% of this license or (at your option) any later version.
% The latest version of this license is in
%   http://www.latex-project.org/lppl.txt
% and version 1.3 or later is part of all distributions of LaTeX
% version 2005/12/01 or later.
%
% This work has the LPPL maintenance status `maintained'.
%
% The Current Maintainer of this work is Feng Kaiyu.
%
% 封面
%
% 如无特殊需要,本页面无需更改

% Underline new command for student information
% Usage: \dunderline[<offset>]{<line_thickness>}
\newcommand\dunderline[3][-1pt]{{%
  \setbox0=\hbox{#3}
  \ooalign{\copy0\cr\rule[\dimexpr#1-#2\relax]{\wd0}{#2}}}}

% Cover Page
\begin{titlepage}
  \makeatletter
  \@ifundefined{externalMentorName}{
    % 校内毕设封面顶部间距
    \vspace*{19mm}
  }{
    % 校外毕设封面顶部间距
    \vspace*{13mm}
  }
  \centering

  \includegraphics[width=9.87cm]{images/header.png}

  \vspace*{-3mm}

  \zihao{-0}\textbf{\ziju{0.12}\songti{本科生毕业设计(论文)}}

  \vspace{16mm}

  \zihao{2}\textbf{\xihei\thesisTitle}

  \vspace{3mm}

  \begin{spacing}{1.2}
    \zihao{3}\selectfont{\textbf{\thesisTitleEN}}
  \end{spacing}

  \vspace{15mm}

  \flushleft

  \makeatletter
  \@ifundefined{externalMentorName}{
    % 生成校内毕设封面字段
    \makeatother
    \begin{spacing}{1.8}
      \hspace{27mm}\songti\zihao{3}\selectfont{学\hspace{11mm}院:\dunderline[-10pt]{1pt}{\makebox[78mm][c]{\deptName}}}

      \hspace{27mm}\songti\zihao{3}\selectfont{专\hspace{11mm}业:\dunderline[-10pt]{1pt}{\makebox[78mm][c]{\majorName}}}

      \hspace{27mm}\songti\zihao{3}\selectfont{学生姓名:\dunderline[-10pt]{1pt}{\makebox[78mm][c]{\yourName}}}

      \hspace{27mm}\songti\zihao{3}\selectfont{学\hspace{11mm}号:\dunderline[-10pt]{1pt}{\makebox[78mm][c]{\yourStudentID}}}

      \hspace{27mm}\songti\zihao{3}\selectfont{指导教师:\dunderline[-10pt]{1pt}{\makebox[78mm][c]{\mentorName}}}
    \end{spacing}
  }{
    % 生成校外毕设封面字段
    \makeatother
    \begin{spacing}{1.8}
      \hspace{19.4mm}\songti\zihao{3}\selectfont{学\hspace{19.6mm}院\hspace{3mm}:\dunderline[-10pt]{1pt}{\makebox[77.4mm][c]{\deptName}}}

      \hspace{19.4mm}\songti\zihao{3}\selectfont{专\hspace{19.6mm}业\hspace{3mm}:\dunderline[-10pt]{1pt}{\makebox[77.4mm][c]{\majorName}}}

      \hspace{19.4mm}\songti\zihao{3}\selectfont{学\hspace{2.8mm}生\hspace{2.8mm}姓\hspace{2.8mm}名\hspace{3mm}:\dunderline[-10pt]{1pt}{\makebox[77.4mm][c]{\yourName}}}

      \hspace{19.4mm}\songti\zihao{3}\selectfont{学\hspace{19.6mm}号\hspace{3mm}:\dunderline[-10pt]{1pt}{\makebox[77.4mm][c]{\yourStudentID}}}

      \hspace{19.4mm}\songti\zihao{3}\selectfont{指\hspace{2.8mm}导\hspace{2.8mm}教\hspace{2.8mm}师\hspace{3mm}:\dunderline[-10pt]{1pt}{\makebox[77.4mm][c]{\mentorName}}}

      \hspace{19.4mm}\songti\zihao{3}\selectfont{校外指导教师:\dunderline[-10pt]{1pt}{\makebox[77.4mm][c]{\externalMentorName}}}
    \end{spacing}
  }

  \vspace*{\fill}
  \centering
  \zihao{3}\ziju{0.5}\songti{\today}
\end{titlepage}


\renewcommand{\frontmatter}{
  \pagenumbering{Roman}
  \pagestyle{BIThesis}
}

\renewcommand{\mainmatter}{
  \pagenumbering{arabic}
  \pagestyle{BIThesis}
}

% 前置页面
\frontmatter
% 原创性声明
%%
% The BIThesis Template for Bachelor Graduation Thesis
%
% 北京理工大学毕业设计(论文)原创性声明页 —— 使用 XeLaTeX 编译
%
% Copyright 2020 Spencer Woo
%
% This work may be distributed and/or modified under the
% conditions of the LaTeX Project Public License, either version 1.3
% of this license or (at your option) any later version.
% The latest version of this license is in
%   http://www.latex-project.org/lppl.txt
% and version 1.3 or later is part of all distributions of LaTeX
% version 2005/12/01 or later.
%
% This work has the LPPL maintenance status `maintained'.
%
% The Current Maintainer of this work is Spencer Woo.
%
% 如无特殊需要,本页面无需更改

% 原创性声明页无页码页面格式
\fancypagestyle{originality}{
  % 页眉高度
  \setlength{\headheight}{20pt}

  % 页眉和页脚(页码)的格式设定
  \fancyhf{}
  \fancyhead[C]{\zihao{4}\ziju{0.08}\songti{北京理工大学本科生毕业设计(论文)}}

  % 页眉分割线稍微粗一些
  \renewcommand{\headrulewidth}{0.6pt}
}

\pagestyle{originality}
\topskip=0pt

% 圆形数字编号定义
\newcommand{\circled}[2][]{\tikz[baseline=(char.base)]
  {\node[shape = circle, draw, inner sep = 1pt]
  (char) {\phantom{\ifblank{#1}{#2}{#1}}};
  \node at (char.center) {\makebox[0pt][c]{#2}};}}
\robustify{\circled}

% 设置行间距
\setlength{\parskip}{0.4em}
\renewcommand{\baselinestretch}{1.41}

% 顶部空白
\vspace*{-6mm}

% 原创性声明部分
\begin{center}
  \heiti\zihao{2}\textbf{原创性声明}
\end{center}

% 本部分字号为小三
\zihao{-3}

本人郑重声明:所呈交的毕业设计(论文),是本人在指导老师的指导下独立进行研究所取得的成果。除文中已经注明引用的内容外,本文不包含任何其他个人或集体已经发表或撰写过的研究成果。对本文的研究做出重要贡献的个人和集体,均已在文中以明确方式标明。

特此申明。

\vspace{13mm}

\begin{flushright}
  本人签名:\hspace{40mm}日\hspace{2.5mm}期:\hspace{13mm}年\hspace{8mm}月\hspace{8mm}日
\end{flushright}

\vspace{17mm}

% 使用授权声明部分
\begin{center}
  \heiti\zihao{2}\textbf{关于使用授权的声明}
\end{center}

本人完全了解北京理工大学有关保管、使用毕业设计(论文)的规定,其中包括:\circled{1}学校有权保管、并向有关部门送交本毕业设计(论文)的原件与复印件;\circled{2}学校可以采用影印、缩印或其它复制手段复制并保存本毕业设计(论文);\circled{3}学校可允许本毕业设计(论文)被查阅或借阅;\circled{4}学校可以学术交流为目的,复制赠送和交换本毕业设计(论文);\circled{5}学校可以公布本毕业设计(论文)的全部或部分内容。

\vspace*{1mm}

\begin{flushright}
  \begin{spacing}{1.65}
    \zihao{-3}
    本人签名:\hspace{40mm}日\hspace{2.5mm}期:\hspace{13mm}年\hspace{8mm}月\hspace{8mm}日\\
    指导老师签名:\hspace{40mm}日\hspace{2.5mm}期:\hspace{13mm}年\hspace{8mm}月\hspace{8mm}日
  \end{spacing}
\end{flushright}

\newpage

% 摘要
%%
% The BIThesis Template for Bachelor Graduation Thesis
%
% 北京理工大学毕业设计(论文)中英文摘要 —— 使用 XeLaTeX 编译
%
% Copyright 2020 Spencer Woo
%
% This work may be distributed and/or modified under the
% conditions of the LaTeX Project Public License, either version 1.3
% of this license or (at your option) any later version.
% The latest version of this license is in
%   http://www.latex-project.org/lppl.txt
% and version 1.3 or later is part of all distributions of LaTeX
% version 2005/12/01 or later.
%
% This work has the LPPL maintenance status `maintained'.
%
% The Current Maintainer of this work is Spencer Woo.
%
% 中英文摘要章节

\topskip=0pt
\zihao{-4}

\vspace*{-7mm}

\begin{center}
  \heiti\zihao{-2}\textbf{\thesisTitle}
\end{center}

\addcontentsline{toc}{chapter}{摘要}
{\let\clearpage\relax \chapter*{摘\quad 要}}
\setcounter{page}{1}

\vspace*{1mm}

\setstretch{1.53}
\setlength{\parskip}{0em}
本文……。

摘要正文选用模板中的样式所定义的“正文”,每段落首行缩进2个字符;或者手动设置成每段落首行缩进2个汉字,字体:宋体,字号:小四,行距:固定值22磅,间距:段前、段后均为0行。阅后删除此段。

摘要是一篇具有独立性和完整性的短文,应概括而扼要地反映出本论文的主要内容。包括研究目的、研究方法、研究结果和结论等,特别要突出研究结果和结论。中文摘要力求语言精炼准确,本科生毕业设计(论文)摘要建议300-500字。摘要中不可出现参考文献、图、表、化学结构式、非公知公用的符号和术语。英文摘要与中文摘要的内容应一致。阅后删除此段。

关键词:北京理工大学;本科生;毕业设计(论文)
\newpage

% 英文摘要章节
\topskip=0pt

\vspace*{0mm}

\begin{spacing}{1}
  \centering
  \heiti\zihao{3}\textbf{\thesisTitleEN}
\end{spacing}

\vspace*{12mm}

\addcontentsline{toc}{chapter}{Abstract}
{\let\clearpage\relax \chapter*{Abstract}}
\setcounter{page}{2}

\setstretch{1.53}
\setlength{\parskip}{0em}
In order to study……

Abstract正文设置成每段落首行缩进2字符,字体:Times New Roman,字号:小四,行距:固定值22磅,间距:段前、段后均为0行。阅后删除此段。

Key Words: BIT; Undergraduate; Graduation Project (Thesis)
\newpage

% 目录
\input{chapters/3_toc.tex}

% 正文开始
\mainmatter
% 正文 22 磅的行距
\setstretch{1.53}
\setlength{\parskip}{0em}

% 第一章
%%
% The BIThesis Template for Bachelor Graduation Thesis
%
% 北京理工大学毕业设计(论文)中英文摘要 —— 使用 XeLaTeX 编译
%
% Copyright 2020 Spencer Woo
%
% This work may be distributed and/or modified under the
% conditions of the LaTeX Project Public License, either version 1.3
% of this license or (at your option) any later version.
% The latest version of this license is in
%   http://www.latex-project.org/lppl.txt
% and version 1.3 or later is part of all distributions of LaTeX
% version 2005/12/01 or later.
%
% This work has the LPPL maintenance status `maintained'.
%
% The Current Maintainer of this work is Spencer Woo.
%
% 第一章节

\chapter{一级题目}

\section{二级标题}

\subsection{三级标题}

正文……

正文部分:宋体、小四;正文行距:22磅;间距段前段后均为0行。阅后删除此段。
图、表居中,图注标在图下方,表头标在表上方,宋体、五号、居中,1.25倍行距,间距段前段后均为0行,图表与上下文之间各空一行。阅后删除此段。

图-示例:(阅后删除此段)


\end{document}
