%%
% The BIThesis Template for Bachelor Graduation Thesis
%
% 北京理工大学毕业设计(论文) —— 使用 XeLaTeX 编译
%
% Copyright 2020 Spencer Woo
%
% This work may be distributed and/or modified under the
% conditions of the LaTeX Project Public License, either version 1.3
% of this license or (at your option) any later version.
% The latest version of this license is in
%   http://www.latex-project.org/lppl.txt
% and version 1.3 or later is part of all distributions of LaTeX
% version 2005/12/01 or later.
%
% This work has the LPPL maintenance status `maintained'.
%
% The Current Maintainer of this work is Spencer Woo.
%
% Compile with: xelatex -> biber -> xelatex -> xelatex

% 章节支持、单面打印:ctexbook
\documentclass[UTF8,AutoFakeBold,AutoFakeSlant,zihao=-4,oneside,openany]{ctexbook}
\usepackage[a4paper,left=3cm,right=2.6cm,top=3.5cm,bottom=2.6cm]{geometry}
\usepackage{xeCJK}
\usepackage{titletoc}
\usepackage{fontspec}
\usepackage{setspace}
\usepackage{graphicx}
\usepackage{fancyhdr}
\usepackage{pdfpages}
\usepackage{setspace}
\usepackage{booktabs}
\usepackage{multirow}
\usepackage{caption}
\usepackage{tikz}
\usepackage{etoolbox}
\usepackage{hyperref}
\usepackage{xcolor}

% 设置参考文献编译后端为 biber,引用格式为 IEEE 格式
% \usepackage[style=ieee,backend=biber]{biblatex}

% 参考文献引用文件 refs.bib
% \addbibresource{misc/refs.bib}

% 西文字体默认为 Times New Roman
\setromanfont{Times New Roman}
% 论文题目字体为华文细黑
\setCJKfamilyfont{xihei}{STXihei}
\newcommand{\xihei}{\CJKfamily{xihei}}

\newcommand{\thesisTitle}{北京理工大学本科生毕业设计(论文)题目}
\newcommand{\thesisTitleEN}{The Subject of Undergraduate Graduation Project (Thesis) of Beijing Institute of Technology}

\newcommand{\deptName}{计算机学院}
\newcommand{\majorName}{计算机科学与技术}
\newcommand{\yourName}{惠计算}
\newcommand{\yourStudentID}{11xxxxxxxx}
\newcommand{\mentorName}{张哈希}

% 主题页面格式:BIThesis
\fancypagestyle{BIThesis}{%
  % 页眉高度
  \setlength{\headheight}{20pt}
  % 页码高度(不完美,比规定稍微靠下 2mm)
  \setlength{\footskip}{14pt}

  \fancyhf{}
  % 定义页眉、页码
  \fancyhead[C]{\zihao{4}\ziju{0.08}\songti{北京理工大学本科生毕业设计(论文)}}
  \fancyfoot[C]{\songti\zihao{5} \thepage}
  % 页眉分割线稍微粗一些
  \renewcommand{\headrulewidth}{0.6pt}
}

% 设置章节格式
% 一级标题:黑体,三号,加粗;间距:段前 0.5 行,段后 1 行;
\ctexset{chapter={
    name = {第,章},
    number = {\arabic{chapter}},
    format = {\heiti \bfseries \centering \zihao{3} },
    pagestyle = BIThesis,
    beforeskip = 8bp,
    afterskip = 32bp,
    fixskip = true,
  }
}

% 二级标题:黑体,四号,加粗;间距:段前 0.5 行,段后 0 行;
\ctexset{section={
    format = {\heiti \raggedright \bfseries \zihao{4}},
    beforeskip = 20bp plus 1ex minus .2ex,
    afterskip = 17bp plus .2ex,
    fixskip = true,
  }
}

% 三级标题:黑体、小四、加粗;间距:段前 0.5 行,段后 0 行;
\ctexset{subsection={
      format = {\heiti \bfseries \raggedright \zihao{-4}},
      beforeskip = 18bp plus 1ex minus .2ex,
      afterskip = 16bp plus .2ex,
      fixskip = true,
    }
}

\begin{document}

% 标题页面
%%
% The BIThesis Template for Bachelor Graduation Thesis
%
% 北京理工大学毕业设计(论文)封面页 —— 使用 XeLaTeX 编译
%
% Copyright 2020-2021 BITNP
%
% This work may be distributed and/or modified under the
% conditions of the LaTeX Project Public License, either version 1.3
% of this license or (at your option) any later version.
% The latest version of this license is in
%   http://www.latex-project.org/lppl.txt
% and version 1.3 or later is part of all distributions of LaTeX
% version 2005/12/01 or later.
%
% This work has the LPPL maintenance status `maintained'.
%
% The Current Maintainer of this work is Feng Kaiyu.
%
% 封面
%
% 如无特殊需要,本页面无需更改

% Underline new command for student information
% Usage: \dunderline[<offset>]{<line_thickness>}
\newcommand\dunderline[3][-1pt]{{%
  \setbox0=\hbox{#3}
  \ooalign{\copy0\cr\rule[\dimexpr#1-#2\relax]{\wd0}{#2}}}}

% Cover Page
\begin{titlepage}
  \makeatletter
  \@ifundefined{externalMentorName}{
    % 校内毕设封面顶部间距
    \vspace*{19mm}
  }{
    % 校外毕设封面顶部间距
    \vspace*{13mm}
  }
  \centering

  \includegraphics[width=9.87cm]{images/header.png}

  \vspace*{-3mm}

  \zihao{-0}\textbf{\ziju{0.12}\songti{本科生毕业设计(论文)}}

  \vspace{16mm}

  \zihao{2}\textbf{\xihei\thesisTitle}

  \vspace{3mm}

  \begin{spacing}{1.2}
    \zihao{3}\selectfont{\textbf{\thesisTitleEN}}
  \end{spacing}

  \vspace{15mm}

  \flushleft

  \makeatletter
  \@ifundefined{externalMentorName}{
    % 生成校内毕设封面字段
    \makeatother
    \begin{spacing}{1.8}
      \hspace{27mm}\songti\zihao{3}\selectfont{学\hspace{11mm}院:\dunderline[-10pt]{1pt}{\makebox[78mm][c]{\deptName}}}

      \hspace{27mm}\songti\zihao{3}\selectfont{专\hspace{11mm}业:\dunderline[-10pt]{1pt}{\makebox[78mm][c]{\majorName}}}

      \hspace{27mm}\songti\zihao{3}\selectfont{学生姓名:\dunderline[-10pt]{1pt}{\makebox[78mm][c]{\yourName}}}

      \hspace{27mm}\songti\zihao{3}\selectfont{学\hspace{11mm}号:\dunderline[-10pt]{1pt}{\makebox[78mm][c]{\yourStudentID}}}

      \hspace{27mm}\songti\zihao{3}\selectfont{指导教师:\dunderline[-10pt]{1pt}{\makebox[78mm][c]{\mentorName}}}
    \end{spacing}
  }{
    % 生成校外毕设封面字段
    \makeatother
    \begin{spacing}{1.8}
      \hspace{19.4mm}\songti\zihao{3}\selectfont{学\hspace{19.6mm}院\hspace{3mm}:\dunderline[-10pt]{1pt}{\makebox[77.4mm][c]{\deptName}}}

      \hspace{19.4mm}\songti\zihao{3}\selectfont{专\hspace{19.6mm}业\hspace{3mm}:\dunderline[-10pt]{1pt}{\makebox[77.4mm][c]{\majorName}}}

      \hspace{19.4mm}\songti\zihao{3}\selectfont{学\hspace{2.8mm}生\hspace{2.8mm}姓\hspace{2.8mm}名\hspace{3mm}:\dunderline[-10pt]{1pt}{\makebox[77.4mm][c]{\yourName}}}

      \hspace{19.4mm}\songti\zihao{3}\selectfont{学\hspace{19.6mm}号\hspace{3mm}:\dunderline[-10pt]{1pt}{\makebox[77.4mm][c]{\yourStudentID}}}

      \hspace{19.4mm}\songti\zihao{3}\selectfont{指\hspace{2.8mm}导\hspace{2.8mm}教\hspace{2.8mm}师\hspace{3mm}:\dunderline[-10pt]{1pt}{\makebox[77.4mm][c]{\mentorName}}}

      \hspace{19.4mm}\songti\zihao{3}\selectfont{校外指导教师:\dunderline[-10pt]{1pt}{\makebox[77.4mm][c]{\externalMentorName}}}
    \end{spacing}
  }

  \vspace*{\fill}
  \centering
  \zihao{3}\ziju{0.5}\songti{\today}
\end{titlepage}


% 前置页面(原创性声明、中英文摘要、目录等)
\renewcommand{\frontmatter}{
  \pagenumbering{Roman}
  \pagestyle{BIThesis}
}

% 正文页面
\renewcommand{\mainmatter}{
  \pagenumbering{arabic}
  \pagestyle{BIThesis}
}

% 前置页面定义
\frontmatter
% 原创性声明
%%
% The BIThesis Template for Bachelor Graduation Thesis
%
% 北京理工大学毕业设计(论文)原创性声明页 —— 使用 XeLaTeX 编译
%
% Copyright 2020 Spencer Woo
%
% This work may be distributed and/or modified under the
% conditions of the LaTeX Project Public License, either version 1.3
% of this license or (at your option) any later version.
% The latest version of this license is in
%   http://www.latex-project.org/lppl.txt
% and version 1.3 or later is part of all distributions of LaTeX
% version 2005/12/01 or later.
%
% This work has the LPPL maintenance status `maintained'.
%
% The Current Maintainer of this work is Spencer Woo.
%
% 如无特殊需要,本页面无需更改

% 原创性声明页无页码页面格式
\fancypagestyle{originality}{
  % 页眉高度
  \setlength{\headheight}{20pt}

  % 页眉和页脚(页码)的格式设定
  \fancyhf{}
  \fancyhead[C]{\zihao{4}\ziju{0.08}\songti{北京理工大学本科生毕业设计(论文)}}

  % 页眉分割线稍微粗一些
  \renewcommand{\headrulewidth}{0.6pt}
}

\pagestyle{originality}
\topskip=0pt

% 圆形数字编号定义
\newcommand{\circled}[2][]{\tikz[baseline=(char.base)]
  {\node[shape = circle, draw, inner sep = 1pt]
  (char) {\phantom{\ifblank{#1}{#2}{#1}}};
  \node at (char.center) {\makebox[0pt][c]{#2}};}}
\robustify{\circled}

% 设置行间距
\setlength{\parskip}{0.4em}
\renewcommand{\baselinestretch}{1.41}

% 顶部空白
\vspace*{-6mm}

% 原创性声明部分
\begin{center}
  \heiti\zihao{2}\textbf{原创性声明}
\end{center}

% 本部分字号为小三
\zihao{-3}

本人郑重声明:所呈交的毕业设计(论文),是本人在指导老师的指导下独立进行研究所取得的成果。除文中已经注明引用的内容外,本文不包含任何其他个人或集体已经发表或撰写过的研究成果。对本文的研究做出重要贡献的个人和集体,均已在文中以明确方式标明。

特此申明。

\vspace{13mm}

\begin{flushright}
  本人签名:\hspace{40mm}日\hspace{2.5mm}期:\hspace{13mm}年\hspace{8mm}月\hspace{8mm}日
\end{flushright}

\vspace{17mm}

% 使用授权声明部分
\begin{center}
  \heiti\zihao{2}\textbf{关于使用授权的声明}
\end{center}

本人完全了解北京理工大学有关保管、使用毕业设计(论文)的规定,其中包括:\circled{1}学校有权保管、并向有关部门送交本毕业设计(论文)的原件与复印件;\circled{2}学校可以采用影印、缩印或其它复制手段复制并保存本毕业设计(论文);\circled{3}学校可允许本毕业设计(论文)被查阅或借阅;\circled{4}学校可以学术交流为目的,复制赠送和交换本毕业设计(论文);\circled{5}学校可以公布本毕业设计(论文)的全部或部分内容。

\vspace*{1mm}

\begin{flushright}
  \begin{spacing}{1.65}
    \zihao{-3}
    本人签名:\hspace{40mm}日\hspace{2.5mm}期:\hspace{13mm}年\hspace{8mm}月\hspace{8mm}日\\
    指导老师签名:\hspace{40mm}日\hspace{2.5mm}期:\hspace{13mm}年\hspace{8mm}月\hspace{8mm}日
  \end{spacing}
\end{flushright}

\newpage

% 摘要
%%
% BIThesis 本科毕业设计论文模板(全英文) —— 使用 XeLaTeX 编译 The BIThesis Template for Undergraduate Thesis
% This file has no copyright assigned and is placed in the Public Domain.
%%

% 摘要若要按经管学院的要求,先英文再中文,请调换以下 abstract、abstractEn 的顺序。

\begin{abstract}
  Conventional  product  development  employs  a  design-build-test  philosophy.
  The sequentially  executed  development  process  often  results  in  prolonged
  lead  times  and elevated product costs. The proposed e-Design paradigm employs
  IT-enabled technology for product design, including virtual prototyping (VP) to
  support a cross-functional team in analyzing  product  performance,  reliability,
  and  manufacturing costs  early  in  product development, and in making quantitative
  trade-offs for design decision making. Physical prototypes  of  the  product  design
  are  then  produced  using  the  rapid  prototyping  (RP) technique  and  computer
  numerical  control  (CNC)  to  support  design  verification  and functional prototyping, respectively.
\end{abstract}

\begin{abstractEn}
  Conventional  product  development  employs  a  design-build-test  philosophy.
  The sequentially  executed  development  process  often  results  in  prolonged
  lead  times  and elevated product costs. The proposed e-Design paradigm employs
  IT-enabled technology for product design, including virtual prototyping (VP) to
  support a cross-functional team in analyzing  product  performance,  reliability,
  and  manufacturing costs  early  in  product development, and in making quantitative
  trade-offs for design decision making. Physical prototypes  of  the  product  design
  are  then  produced  using  the  rapid  prototyping  (RP) technique  and  computer
  numerical  control  (CNC)  to  support  design  verification  and functional prototyping, respectively.
\end{abstractEn}

% 目录
%%
% The BIThesis Template for Bachelor Graduation Thesis
%
% 北京理工大学毕业设计(论文)目录 —— 使用 XeLaTeX 编译
%
% Copyright 2020 Spencer Woo
%
% This work may be distributed and/or modified under the
% conditions of the LaTeX Project Public License, either version 1.3
% of this license or (at your option) any later version.
% The latest version of this license is in
%   http://www.latex-project.org/lppl.txt
% and version 1.3 or later is part of all distributions of LaTeX
% version 2005/12/01 or later.
%
% This work has the LPPL maintenance status `maintained'.
%
% The Current Maintainer of this work is Spencer Woo.
%
% 如无特殊需要,本页面无需更改

% 目录开始

% 调整目录行间距
\renewcommand{\baselinestretch}{1.35}
% 目录
\tableofcontents
\newpage


% 正文开始
\mainmatter
% 正文 22 磅的行距
\setlength{\parskip}{0em}
\renewcommand{\baselinestretch}{1.53}

% 第一章
%%
% The BIThesis Template for Bachelor Graduation Thesis
%
% 北京理工大学毕业设计(论文)第一章节 —— 使用 XeLaTeX 编译
%
% Copyright 2020-2021 BITNP
%
% This work may be distributed and/or modified under the
% conditions of the LaTeX Project Public License, either version 1.3
% of this license or (at your option) any later version.
% The latest version of this license is in
%   http://www.latex-project.org/lppl.txt
% and version 1.3 or later is part of all distributions of LaTeX
% version 2005/12/01 or later.
%
% This work has the LPPL maintenance status `maintained'.
%
% The Current Maintainer of this work is Feng Kaiyu.
%
% 第一章节

\chapter{一级题目}

\section{二级题目}
% 这里插入一个参考文献,仅作参考
正文……\cite{yuFeiJiZongTiDuoXueKeSheJiYouHuaDeXianZhuangYuFaZhanFangXiang2008}

\subsection{三级题目}

正文……\cite{Hajela2012Application}

\textcolor{blue}{正文部分:宋体、小四;正文行距:22磅;间距段前段后均为0行。阅后删除此段。}

\textcolor{blue}{图、表居中,图注标在图下方,表头标在表上方,宋体、五号、居中,1.25倍行距,间距段前段后均为0行,图表与上下文之间各空一行。阅后删除此段。}

\textcolor{blue}{\underline{\underline{图-示例:(阅后删除此段)}}}

\begin{figure}[htbp]
  \vspace{13pt} % 调整图片与上文的垂直距离
  \centering
  \includegraphics[]{images/bit_logo.png}
  \caption{标题序号}\label{标题序号} % label 用来在文中索引
\end{figure}

\textcolor{blue}{\underline{\underline{表-示例:(阅后删除此段)}}}

\begin{table}[htbp]
  \linespread{1.5}
  \zihao{5}
  \centering
  \caption{统计表}\label{统计表}
  \begin{tabular}{*{5}{>{\centering\arraybackslash}p{2cm}}}
    \hline
    项目    & 产量    & 销量    & 产值   & 比重    \\ \hline
    手机    & 1000  & 10000 & 500  & 50\%  \\
    计算机   & 5500  & 5000  & 220  & 22\%  \\
    笔记本电脑 & 1100  & 1000  & 280  & 28\%  \\ \hline
    合计    & 17600 & 16000 & 1000 & 100\% \\ \hline
    \end{tabular}
\end{table}

\textcolor{blue}{公式标注应于该公式所在行的最右侧。对于较长的公式只可在符号处(+、-、*、/、$\leqslant$ $\geqslant$ 等)转行。在文中引用公式时,在标号前加“式”,如式(1-2)。阅后删除此
段。}

\textcolor{blue}{公式-示例:(阅后删除此段)}
% 公式上下不要空行,置于同一个段落下即可,否则上下距离会出现高度不一致的问题
\begin{equation}
    LRI=1\ ∕\ \sqrt{1+{\left(\frac{{\mu }_{R}}{{\mu }_{s}}\right)}^{2}{\left(\frac{{\delta }_{R}}{{\delta }_{s}}\right)}^{2}}
\end{equation}

\subsubsection{生僻字}

% 一个可能无法正常显示的生僻字
一个可能无法正常显示的生僻字: 彧。下文注释中,介绍了如何通过自定义字体来显示生僻字。

% 定义一个提供了生僻字的字体,注意要确保你的系统存在该字体
% \setCJKfamilyfont{custom-font}{Noto Serif CJK SC}

% 使用自己定义的字体
% 使用提供了相应字型的字体:\CJKfamily{custom-font}{彧}。



\end{document}
